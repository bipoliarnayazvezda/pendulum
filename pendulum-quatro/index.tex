% Options for packages loaded elsewhere
\PassOptionsToPackage{unicode}{hyperref}
\PassOptionsToPackage{hyphens}{url}
\PassOptionsToPackage{dvipsnames,svgnames,x11names}{xcolor}
%
\documentclass[
  a5paperpaper,
  DIV=11,
  numbers=noendperiod]{scrreprt}

\usepackage{amsmath,amssymb}
\usepackage{iftex}
\ifPDFTeX
  \usepackage[T1]{fontenc}
  \usepackage[utf8]{inputenc}
  \usepackage{textcomp} % provide euro and other symbols
\else % if luatex or xetex
  \usepackage{unicode-math}
  \defaultfontfeatures{Scale=MatchLowercase}
  \defaultfontfeatures[\rmfamily]{Ligatures=TeX,Scale=1}
\fi
\usepackage[]{libertinus}
\ifPDFTeX\else  
    % xetex/luatex font selection
\fi
% Use upquote if available, for straight quotes in verbatim environments
\IfFileExists{upquote.sty}{\usepackage{upquote}}{}
\IfFileExists{microtype.sty}{% use microtype if available
  \usepackage[]{microtype}
  \UseMicrotypeSet[protrusion]{basicmath} % disable protrusion for tt fonts
}{}
\makeatletter
\@ifundefined{KOMAClassName}{% if non-KOMA class
  \IfFileExists{parskip.sty}{%
    \usepackage{parskip}
  }{% else
    \setlength{\parindent}{0pt}
    \setlength{\parskip}{6pt plus 2pt minus 1pt}}
}{% if KOMA class
  \KOMAoptions{parskip=half}}
\makeatother
\usepackage{xcolor}
\setlength{\emergencystretch}{3em} % prevent overfull lines
\setcounter{secnumdepth}{5}
% Make \paragraph and \subparagraph free-standing
\ifx\paragraph\undefined\else
  \let\oldparagraph\paragraph
  \renewcommand{\paragraph}[1]{\oldparagraph{#1}\mbox{}}
\fi
\ifx\subparagraph\undefined\else
  \let\oldsubparagraph\subparagraph
  \renewcommand{\subparagraph}[1]{\oldsubparagraph{#1}\mbox{}}
\fi


\providecommand{\tightlist}{%
  \setlength{\itemsep}{0pt}\setlength{\parskip}{0pt}}\usepackage{longtable,booktabs,array}
\usepackage{calc} % for calculating minipage widths
% Correct order of tables after \paragraph or \subparagraph
\usepackage{etoolbox}
\makeatletter
\patchcmd\longtable{\par}{\if@noskipsec\mbox{}\fi\par}{}{}
\makeatother
% Allow footnotes in longtable head/foot
\IfFileExists{footnotehyper.sty}{\usepackage{footnotehyper}}{\usepackage{footnote}}
\makesavenoteenv{longtable}
\usepackage{graphicx}
\makeatletter
\def\maxwidth{\ifdim\Gin@nat@width>\linewidth\linewidth\else\Gin@nat@width\fi}
\def\maxheight{\ifdim\Gin@nat@height>\textheight\textheight\else\Gin@nat@height\fi}
\makeatother
% Scale images if necessary, so that they will not overflow the page
% margins by default, and it is still possible to overwrite the defaults
% using explicit options in \includegraphics[width, height, ...]{}
\setkeys{Gin}{width=\maxwidth,height=\maxheight,keepaspectratio}
% Set default figure placement to htbp
\makeatletter
\def\fps@figure{htbp}
\makeatother

\KOMAoption{captions}{tableheading}
\makeatletter
\@ifpackageloaded{bookmark}{}{\usepackage{bookmark}}
\makeatother
\makeatletter
\@ifpackageloaded{caption}{}{\usepackage{caption}}
\AtBeginDocument{%
\ifdefined\contentsname
  \renewcommand*\contentsname{Table of contents}
\else
  \newcommand\contentsname{Table of contents}
\fi
\ifdefined\listfigurename
  \renewcommand*\listfigurename{List of Figures}
\else
  \newcommand\listfigurename{List of Figures}
\fi
\ifdefined\listtablename
  \renewcommand*\listtablename{List of Tables}
\else
  \newcommand\listtablename{List of Tables}
\fi
\ifdefined\figurename
  \renewcommand*\figurename{Figure}
\else
  \newcommand\figurename{Figure}
\fi
\ifdefined\tablename
  \renewcommand*\tablename{Table}
\else
  \newcommand\tablename{Table}
\fi
}
\@ifpackageloaded{float}{}{\usepackage{float}}
\floatstyle{ruled}
\@ifundefined{c@chapter}{\newfloat{codelisting}{h}{lop}}{\newfloat{codelisting}{h}{lop}[chapter]}
\floatname{codelisting}{Listing}
\newcommand*\listoflistings{\listof{codelisting}{List of Listings}}
\makeatother
\makeatletter
\makeatother
\makeatletter
\@ifpackageloaded{caption}{}{\usepackage{caption}}
\@ifpackageloaded{subcaption}{}{\usepackage{subcaption}}
\makeatother
\ifLuaTeX
  \usepackage{selnolig}  % disable illegal ligatures
\fi
\usepackage{bookmark}

\IfFileExists{xurl.sty}{\usepackage{xurl}}{} % add URL line breaks if available
\urlstyle{same} % disable monospaced font for URLs
\hypersetup{
  pdftitle={Рады видеть вас в бюро ритуальных услуг «Маятник»},
  pdfauthor={Александра Булгакова},
  colorlinks=true,
  linkcolor={blue},
  filecolor={Maroon},
  citecolor={Blue},
  urlcolor={Blue},
  pdfcreator={LaTeX via pandoc}}

\title{Рады видеть вас в бюро ритуальных услуг «Маятник»}
\author{Александра Булгакова}
\date{2016-01-01}

\begin{document}
\maketitle

\renewcommand*\contentsname{Table of contents}
{
\hypersetup{linkcolor=}
\setcounter{tocdepth}{1}
\tableofcontents
}
\bookmarksetup{startatroot}

\chapter{Рады видеть вас в бюро ритуальных услуг
«Маятник»}\label{ux440ux430ux434ux44b-ux432ux438ux434ux435ux442ux44c-ux432ux430ux441-ux432-ux431ux44eux440ux43e-ux440ux438ux442ux443ux430ux43bux44cux43dux44bux445-ux443ux441ux43bux443ux433-ux43cux430ux44fux442ux43dux438ux43a}

\bookmarksetup{startatroot}

\chapter{Часть I. Социальный маятник
\{-\#chapter-1=``\,``\}}\label{ux447ux430ux441ux442ux44c-i.-ux441ux43eux446ux438ux430ux43bux44cux43dux44bux439-ux43cux430ux44fux442ux43dux438ux43a--chapter-1}

\emph{И все, что я любил -- я любил в одиночестве.}

\emph{--- Эдгар Аллан По}

\section*{1}\label{1}
\addcontentsline{toc}{section}{1}

\markright{1}

Забывшись тревожным сном, -- антибиотики и неудавшаяся ночь давали о
себе знать сильнее чем следовало -- Элеонора невнятно выругалась и
отвернулась к стене, однако мучения не прекращались, и вскоре девушка
осознала, что толчки, которые её подсознание поначалу сочло результатом
дуэта местных дорог и старого поезда, оказались всего лишь попытками её
разбудить. Всё ещё не открывая глаз, девушка зевнула и потянулась
(насколько это позволяли сделать убогие габариты спальной полки). Чьи-то
пальцы упрямо продолжали сжимать её плечо и в итоге Нора сдалась.

Обернувшись, она увидела лупоглазое, испещрённое угрями лицо, что
расположилось в пугающей близости, и инстинктивно подалась назад, от
чего её голова тут же поцеловала вешалку для полотенец. Пожилая
проводница сделала шаг назад и, теперь стояла со сложенными на груди
руками вполне довольная проделанной работой.

-- Як си кунялося? \footnote{Как спалось? (здесь и дальше -- укр.
  диалект)}

Элеоноре, потирающей ушибленный затылок, понадобилось какое-то время,
дабы понять, о чем её спрашивают. В итоге, девушка неуверенно заметила:

-- Поезд всю ночь трясло.

Проводница угрюмо качнула головой.

-- Ниська має бути файна штрика. \footnote{Сегодня дорога должна быть
  хорошей}

Нора медленно кивнула, не имея ни малейшего понятия о том, что бы это
могло значить.

-- Вы не подскажите, через скольк\ldots{}

Где-то в другом конце вагона раздался звук бьющегося стекла, и голова
женщины тут же исчезла в дверном проёме.

-- Гузиця!\footnote{Задница!} -- заявила она с таким видом, словно это
слово многое объясняло, а затем уже более рассержено: -- А шляк бы
трафыв эту фіндюр\ldots{}\footnote{Да чтоб паралич разбил эту бл\ldots{}
  (с польского наречия)}

Конец фразы затерялся, так как проводница уже мельтешила по коридору,
оставляя за собой шлейф неведомых ругательств. Нора, которая до этого
утра думала, что владеет украинским в совершенстве, взглянула на часы.
До момента прибытия оставалось ещё около двух часов.

За это время словарный запас нашей героини успешно дополнили ярчайшие
образчики местного фольклора. Такие как \emph{лайдак}, \emph{шмаркатий},
\emph{чмурик}, \emph{алік}, \emph{харлак}, \emph{йолупега},
\emph{гамазей}, а также особенно полюбившееся Элеоноре словечко, которое
едва ли имело отношение к колоритной брани, но звучало гордо и
неоднозначно -- \emph{нацицюрник}.

-- И на кой чёрт было будить меня в такую рань?

По ту сторону двери раздались шаги и мгновение спустя, похерив правила
приличия, на пороге вновь объявилась проводница, словно почувствовавшая,
что о ней вспоминают не в лучшем ключе.

-- Сьорбать будеш, кобіто? \footnote{Пить будешь, девушка?}

Девушка тут же кивнула головой, потому что и раньше слышала этот вопрос.

-- Чашку чая, пожалуйста.

-- Скільки?

-- Простите?

-- Скільки цукру? \footnote{Сколько сахара?}

-- О нет, я пью чай без сахара.

Услыхав эту фразу, проводница выпучила свои и без того громадные зеньки
\footnote{Глаза}. Выглядело это жутковато, и девушке почудилось, что
глазные яблоки вот-вот вырвутся на свободу. Тем временем женщина окинула
Элеонору ошарашенным взглядом, предназначавшимся по меньшей мере для
представителей внеземной расы. В подобной ситуации Агата без лишних
раздумий послала бы собеседника на хуй, -- плевать, пуцка это, или
пуцька \footnote{Варианты написания вышеупомянутого органа.} -- но до
знакомства с последней оставалось почти девять часов, так что Нора лишь
сдержанно улыбнулась.

Проводница не удостоила этот жест ответом и удалилась.

Нора нехотя вылезла из-под одеяла и принялась собирать постель.

День начался как-то не очень.

\section*{2}\label{2}
\addcontentsline{toc}{section}{2}

\markright{2}

Девушка подвязала волосы и, укутавшись в шарф, вышла за пределы душного
купе. Дело было не в отсутствии воздуха и даже не в абстинентном
синдроме, который медленно, но уверенно подкрадывался к её организму, --
просто Нора решила, что ещё одной встречи с курьером из ада, по каким-то
странным и неведомым причинам выступающим здесь в роли проводницы, она
не переживёт.

Продвигаясь по вагону, Элеонора бросила взгляд на проплывающий за окнами
пейзаж и отметила про себя, что привлекательность картины за стеклом
возрастает прямо пропорционально её отдалению от родного города. С этими
мыслями она миновала последнее купе, туалет, открытая дверь которого
демонстрировала всякому нуждающемуся малоприятное зрелище, и вышла из
вагона.

С зажатой в пальцах сигаретой она стояла в крохотном продуваемом
помещении, соединяющим седьмой и восьмой вагоны, прямиком под значком с
изображением жирной линии, перечеркивающей сигарету и надписью ``Не
палити''\footnote{Не курить!}. Как и в коридоре, пол здесь был
непростительно грязным и от предыдущего его отличало лишь присутствие
дюжины окурков -- свидетелей того, что Элеонора была далеко не
единственной, кто нарушал вышеуказанное правило. Несмотря на обещания
проводницы, которых девушка всё равно не разобрала, поезд продолжало
трясти, так что свободной рукой ей приходилось держаться за пыльные
поручни.

Не лучшее место для аллергика.

Чай уже ждал на столике, когда Элеонора вернулась в своё купе. Слишком
крепкий, но без сахара.

\section*{3}\label{3}
\addcontentsline{toc}{section}{3}

\markright{3}

-- Виходок зачиняється!\footnote{Туалет закрывается!} -- оповестила
проводница.

Поезд прибыл по расписанию, чем немало удивил своих обитателей. В десять
минут девятого Нора захлопнула книгу и внимательно осмотрела своё купе,
не забыв заглянуть под спальную полку, где однажды оставила кошелёк.
Плацкартные вагоны были забиты под завязку, но в вагоне номер восемь
едва ли была занята четверть мест, посему девушка ехала одна, против
чего она нисколько не возражала -- загадочные спутники бывают только в
старых романах; на деле же их роли обычно исполняют храпящие алкоголики
и счастливые обладатели грязных носков, а то и оба варианта сразу.

Проведя в дороге более восемнадцати часов, Элеонора чувствовала себя
выжатой, но вместе с тем понимала, что засиделась. Львов встретил её
тихим утром и мокрым снегом. Ступив на перрон, девушка вдохнула свежий
воздух и ощутила, что усталость начинает испаряться. При себе у неё была
дамская сумочка и кожаный чемодан. Старомодный, он вполне тянул на
ручную кладь.

Здание вокзала оказалось на удивление просторным и уютным. Выполненное в
пастельных тонах, оно удачно вписывалось в представление девушки о новом
месте жительства: воздух был пропитан запахами шоколада и
свежесваренного кофе; с разных сторон доносились отрывки фраз,
произнесённых, казалось, на всех языках мира. Выйдя на улицу, Элеонора
задержалась на какое-то время, чтобы полюбоваться старинными фасадами
здания. Онлайн-путеводитель гласил, что открытое в ноябре тысяча
восемьсот шестьдесят первого года (возведенное в тысяча девятьсот
четвёртом), строение это исторически было первым железнодорожным
вокзалом на территории современной Украины. Она деловито кивнула и
сделала сэлфи.

Покончив с осмотром, девушка постояла ещё немного напротив здания
вокзала, дожидаясь пока из кадра удалятся туристы, после чего
сфотографировала его и направилась к нескольким десяткам такси, что
выстроились в аккуратный ряд у самого бордюра.

До центра города простиралось всего два с половиной километра, но
девушка решила взять такси, тем самым сохранив силы. По словам хозяйки,
комната, насчёт которой она договорилась, будет готова только к пяти
часам вечера, так что сегодня Норе предстояла долгая прогулка.

Она воткнула в уши любимые ракушки и окунулась в атмосферу декабрьского
утра.

\section*{4}\label{4}
\addcontentsline{toc}{section}{4}

\markright{4}

Для начала Элеоноре хотелось подняться на самый верх пресловутой ратуши,
дабы всецело ощутить вкус нового города, ну, и, конечно, ввиду
очаровательной панорамы, которую она не единожды видела на чужих
снимках. Однако, вскоре выяснилось, что ратуша находится в пешеходной
части Львова, куда строго настрого запрещен въезд. Шофер -- пожилой
вуйко\footnote{Мужчина} в бордовом берете и с усами щеточкой -- высадил
девушку на проспекте Свободы и подробно объяснил, как добраться до
нужной ей точки.

Поблагодарив мужчину и выйдя из машины, Элеонора первым делом заметила
хмурое каменное сооружение, возвышающееся в центре небольшого сквера.
Его она также узнала. Сама фигура Тараса Шевченко ничем не отличалась от
прочих памятников, увиденных Норой в других украинских городах:
стандартной высоты, выполненная из бронзы, покоившаяся на гранитном
постаменте, но было здесь ещё кое-что, заставляющее прохожих затаить
дыхание. По левую сторону от поэта находилась двенадцатиметровая стела
-- так называемая «Волна национального возрождения», передняя сторона
которой символически изображала историю страны со времен Киевской Руси
до прошлого века, вместе с героями произведений Кобзаря; обратная же
передавала всё тот же мотив, начиная с двадцатого века. Фигурные рельефы
вызывали коктейль из тоски и благоговения, а разглядывать их можно было
до захода солнца. И хотя это сооружение было далеко не в её вкусе, Нора
провела у памятника с четверть часа, затем подкурила и перешла дорогу,
направляясь в сторону вычурных зданий и бесконечных кофеен.

Ещё одной особенностью этого места были выложенные брусчаткой дороги: в
совокупности с узкими улочками они создавали особую, домашнюю атмосферу.
Свернув направо, девушка шагнула в заснеженный переулок и медленно
зашагала вдоль трамвайных путей, что привели её к Доминиканскому собору.

Гуляя по старому городу, Нора ощущала родство с этим местом и на её
губах вновь поселилась мечтательная улыбка. Приятно было наконец
вырваться из приевшегося индустриального городка, единственной
перспективой жизни в котором была возможность уехать.

Обойти центр не составляло труда -- городок был небольшим, но каждый
квадратный метр последнего являл собой произведение искусства, так что
последующие два часа Элеонора провела за изучением местности, которая к
её приятному удивлению оказалось премного богаче того, что предлагал
путеводитель. Первым действительно полюбившимся ей местом стал
Доминиканский монастырь: выполненная в камне, наружная часть собора,
включающая в себя роскошные статуи, удивительно гармонировала с золотым
интерьером. До этого момента девушке не доводилось видеть столь
сдержанного барокко.

Разглядывая алтарь, она заметила вошедшего в костёл мужчину, на руках у
которого сидела белокурая девочка, бережно замотанная в пуховик. Увидев
переливающиеся фасады, она протянула к ним ручки, после чего вдруг
надула розовые губки и обиженно спросила:

-- Почему я не могу здесь жить, папочка?

Последнее слово прозвучало особенно чётко, выдавая южный диалект, и
Элеонора успела подумать, что она не единственная туристка, гуляющая по
незнакомому городу ранним снежным утром. Однако, ответ последовал на
местном украинском.

-- Бо то э культурна пам'ятка, доце -- лицо мужчины осветила улыбка. --
Не реви, ушитко то добрі, бо дам тобі цуцлик.\footnote{Потому, что это
  культурный памятник, доча. Не реви, все же хорошо, а то дам тебе
  соску.}

-- Я уже большая для цуцлика, -- заявила девочка, но тем не менее
расхохоталась.

Мужчина придал своему лицу наигранно задумчивое выражение.

-- Хм\ldots{} Большая кажеш?\footnote{Говоришь} Тоді мабудь і лизанку
купувати не варто\ldots{}\footnote{Тогда, наверное, и леденец покупать
  не стоит\ldots{}}

Оказавшись истинной женщиной, малышка чмокнула отца в щёку, и по
добродушному взгляду мужчины стало ясно, что он купит ей не один
леденец.

«И почему я не могу здесь жить?» -- подумала Элеонора, глядя на
удаляющееся из собора семейство.

Забавно, но ближе к вечеру она с немалой долей иронии вспомнит эту
мысль.

\section*{5}\label{5}
\addcontentsline{toc}{section}{5}

\markright{5}

Навернув пару кругов по Старому городу, Нора вновь вышла на площадь
Рынок и очень огорчилась тому, что не смогла попасть в ренессансную
Чёрную каменицу, о которой она читала, что та получила свой цвет
совершенно таинственным для прошедших столетий образом. Девушке нравился
чёрный, (и об этом она также вспомнит грядущим вечером) а потому ей
хотелось узнать, как зданице выглядит внутри.

Поражало её и удачное сочетание античности с повседневной жизнью города.
Многое поражало.

Запечатлев ещё несколько строений и статуй, Нора поднялась по ступеням.
Она очутилась около памятника Ивана Фёдорова, который, если верить
путеводителю, был возведен в честь четырёхсотлетия книжной печати на
украинской земле. Но заинтересовало девушку кое-что другое: всю площадь
подле постамента занимал огромный букинистический рынок. Являвшаяся по
своей эстетичной природе тем ещё графоманом, Элеонора с новыми силами
взялась за изучения ассортимента и вскоре утяжелила свой багаж. Отрадно
было и другое: здесь оказалось возможным найти не только редкие, давно
снятие с продажи издания, но и старые пластинки. Девушка любила винил
(это-то их с Агатой и объединило) и не пожалела бы на его покупку денег,
будь у нее при себе проигрыватель. Она покинула площадь, дав себе
обещание обзавестись последним в скором времени. На протяжении всего дня
снег то усиливался, то затихал чтобы начаться вновь.

Следующей остановкой стала Ратуша. Девушка не была уверена в том,
правильно ли она рассчитала свои силы, ведь по мере прогулки чемодан в
её руке становился всё тяжелее, а лишенная сна ночь всплывала в памяти,
но подойдя к интересовавшему её зданию Элеонора уже не сомневалась в
том, что увидит волшебную панораму. Во-первых, в своих желаниях она была
упряма, во-вторых, ратуша выглядела не такой уж и высокой. К тому же,
девушка любила прогулки в одиночестве, особенно если альтернатива как
таковая отсутствовала.

У входа в здание городского совета её традиционно приветствовал каменный
лев. Внутри оказалось тепло и просторно. Быстрыми шагами Нора преодолела
четыре этажа, время от времени выглядывая в окна. Иногда на её пути
встречались работники администрации, что суетливо перескакивали из
одного кабинета в другой. Следуя указателям, Элеонора очутилась в тесной
комнатке -- пропускном пункте, дальняя дверь которого служила выходом на
крышу. Кабинка сторожа пустовала. Девушке не хотелось задерживаться, так
что она просунула в окошко розовую купюру и направилась к двери.

Для начала её вниманием завладели огромные старинные часы, шестеренки
которых, несмотря на свой возраст, продолжали вертеться. Обожающая
стим-панк, Элеонора долгое время не могла отвести взгляд от механизма.
Даже воздух этого места проникся минувшими столетиями, а тишина и
пустынность придавали помещению загадочность. Девушке представилось, что
она только что вошла в ТАРДИС, которая перенесла её в средину
девятнадцатого века -- такое неописуемое чувство единения с
произведением искусства.

Впереди её ожидало чуть более четырёхсот деревянных ступеней и поначалу
эта цифра не казалась Норе значительной, но, не пройдя и половины своего
пути, девушка осознала, что покорение ратуши после насыщенного
прогулками утра -- идея не из лучших. Отдышавшись, она наклонилась через
перила и взглянула вниз: идеальная симметрия раскинувшейся под ногами
картины создавала иллюзию зеркал.

Чем выше Элеонора поднималась, тем уже становился проход, из квадратной
лестница превращалась в спиральную, а перила и вовсе исчезли. Теперь
идти приходилось медленно: земля то и дело уходила из-под ног. Один раз
девушка даже споткнулась, не обошлось и без этого. Перед её глазами тут
же предстала картина красочного падения.

Тем временем подъем стал настолько резким, что Нора уже не видела ничего
кроме недавно побеленной стены по правую сторону от себя. Путь её
занимал не более двадцати минут, но где-то на середине дороги Нора
потеряла счет времени и не поверила своему счастью, когда за очередным
поворотом вдруг наткнулась на приоткрытую дверь, из которой пробивался
свежий ветер.

Элеонора с облегчением вздохнула и вышла на смотровую площадку.

Снегопад усилился. На улице заметно похолодало. Одна снежинка упала
прямиком на нос девушки, но охваченная открывающимся с ратуши видом она
этого не заметила. Панорама оказалась воистину чудесной: в какую сторону
не глянь, на глаза попадались неописуемые львовские красоты, будь то
величественные храмы, или же разноцветные олдфэшеневые\footnote{Old-fashion
  (англ.) -- старомодный} домики. Последние полюбились ей особенно
сильно -- аккуратные и в то же время хаотично разбросанные дворики,
разделенные тонкими улочками; дома с покатыми крышами, схожие, но не
идентичные; яркие верхушки, стены, выполненные в мягких тонах и
бесчисленное количество кирпичных дымоходов.

Местами виднелись трамвайные пути, по которым лениво передвигался
транспорт. Элеонора стояла у самого края и наблюдала за одним из
трамваев до тех пор, пока тот не скрылся из виду, растворившись в зимней
белизне. Она пожалела, что на крыше часовой башни до сих пор не открыли
какую-нибудь кофейню. Конечно, нынешнее состояние ратуши максимально
соответствовало прошлому, но теперешняя погода и очаровательные виды то
и дело наводили девушку на мысли о домашнем уюте и горячих напитках.

Вспомнилось ей и другое -- поднебесный чердак в старом доме, крохотное
помещение, доверху заваленное книгами, сувенирами и флаконами сладких
духов -- место, где девушка провела большую часть своей лишь начинающей
становиться взрослой жизни.

Нора тряхнула головой, словно отгоняя от себя неудавшиеся мысли. Так или
иначе, для ностальгии было ещё слишком рано.

\section*{6}\label{6}
\addcontentsline{toc}{section}{6}

\markright{6}

С приходом холодов темнело в этих краях рано. К четырём часам Элеонора
расправилась с обедом, а в начале пятого попросила принести ей ещё одну
чашку заманчиво пахнущего латте.

Ближе к вечеру погода изменилась в лучшую сторону. Ветер поутих, и сидя
в дальнем углу кофейни, девушка наблюдала за тем, как крупные хлопья
снега плавно кружили в оранжевом свете фонарей, расположившихся вдоль
бульвара. Нора обхватила рукой чашку с дымящимся напитком; второй она
переворачивала неразлинованные страницы ежедневника, просматривая их
содержимое. Прошлой ночью, в ожидании запаздывающего сна, девушка
заполнила своим изящным почерком несколько листов. Поначалу это должны
были быть всего лишь частички водоворота мыслей, кружащего в её голове,
но постепенно стал вырисовываться сюжет.

А начиналось повествование со следующих слов.

\begin{quote}
\emph{Я всегда считаю секунды между молнией и громом, определяя таким
образом расстояние на котором `сверкает'. Этому в детстве научил меня
отец, объяснив разницу между скоростью света и скоростью звука. Я помню,
как мы сидели летней ночью возле реки, он светил фонариком в небо и
говорил, что этот свет долетит до ближайшей звезды через восемь лет. Мне
тогда самой было около восьми, и этот срок казался огромным. Сейчас я
подумала, что если кто-то на этой `ближайшей звезде' получил наше
послание, то вскоре мне стоит ждать ответа.}
\end{quote}

Элеонора вычеркнула парочку не приглянувшихся ей слов и перевернула
страницу.

\begin{quote}
\emph{Я шла по просторному нефу в полном одиночестве, мои шаги
разносились гулким эхом и смешивались со звуком стучащих по стеклу
капель дождя. Замедлив шаг, я подошла к широкому витражному окну, свет
из которого в солнечные дни делил коридор на две равные части.
Рассматривая узоры из цветного стекла, я остановила свой взгляд на
розах, аккуратно составленных из угловатых алых осколков. Дождь тем
временем прекратился и застывшие на лепестках прозрачных цветов капли,
напоминали крупную росу, выступившую ранним утром после холодной ночи.}

\emph{Повернувшись на 180° я облокотилась на подоконник и начала
рассматривать сводчатый потолок, возвышающийся в трёх метрах надо мной.
Утончённые переходы и вычурные розетки не переставали поражать меня
своей плавностью и точёностью, от них веяло стариной, и в моей голове
уже начала разыгрываться история, сродни средневековым легендам, как
вдруг в кармане пальто зазвонил телефон.}
\end{quote}

Эта часть записей напоминала Элеоноре о давно минувшем лете, проведенном
в Оксфорде. Несмотря на то, что дожди не прекращались, дни выдались
удивительно жаркими, а вместе с тем не прекращался и наплыв туристов.
Норе, которая предпочитала в уединении знакомиться с прекрасным,
приходилось просыпаться с зарей: только так она могла гулять по городу,
а не протискиваться сквозь толпу иностранцев, плевавших на историю и
делающих сэлфи на фоне каждого столба.

Какое-то время наша героиня разглядывала «точёность», размышляя, есть ли
такое слово в принципе, и если нет, стоит ли его употреблять? В итоге
решила пока оставить.

Далее в тексте следовал разговор, и если с описаниями, по её скромному
мнению, у Норы проблем не возникало, то в плане диалогов девушка особой
уверенности не чувствовала. Составляя их, она никак не могла отделаться
от чувства наигранности.

\begin{quote}
\emph{\ldots как вдруг в кармане пальто зазвонил телефон.}

\emph{-- Привет, Имя, -- произнесла я в трубку.}

\emph{-- Добрый вечер, леди Имя. Когда вы наконец покинете свою башню и
явитесь отужинать со мной? Убийство дракона возлагаю на вас.}

\emph{-- Я как раз точу свой меч, чтобы с ним расправиться, --
произнесла я в тон речам Имя.}

\emph{-- Ещё не сдала?}

\emph{-- Нет, брожу пока тут. Профессор должен освободиться к шести. }

\emph{-- Но мы же договаривались увидеться в шесть, -- голос Имя
прозвучал расстроено.}

\emph{-- Ну ты же знаешь, как это обычно бывает. Час задержки -- это ещё
не много. Хотя я сама рассчитывала, что сегодня всё пройдет быстро.
Вечером тут обычно никого особо нет, но видимо у МакДоусона нашлись
какие-то срочные дела, в которые моя скромная персона не входит.}

\emph{-- Слишком уж она у тебя скромна. Не можешь настоять на своём.}

\emph{-- Ты ведешь себя, как Имя, - Имя прыснул. - Давай перенесём на
семь? Обещаю, что не задержусь ни на минуту.}

\emph{-- Ладно, договорились. Ровно в семь возле «Нью Эйдж», -- Имя
поставил особое ударение на слово `ровно'.}

\emph{-- Когда приедешь, я уже промокну насквозь, дожидаясь тебя.}

\emph{-- Дождь закончился, так что тебе это в любом случае не удастся.}

\emph{-- Посмотрим.}

\emph{Я отключилась, не дав Имя ответить и, мечтательно улыбаясь
собственным мыслям о предстоящем вечере, снова повернулась к окну. На
этот раз я смотрела сквозь стекло на нависшие над городом тучи, которые
вдруг озарила тонкая сеть молний, и ровно через шесть секунд послышался
приглушенный стенами гром.}
\end{quote}

Элеонора вскинула бровь и нарисовала жирный вопросительный знак вдоль
всего диалога. Фамилию профессора девушка подчеркнула дважды -- во время
написания чернового текста она старалась на фокусироваться на мелочах, а
в выборе фамилии приставка `Мак-' оказалась первым, что пришло ей в
голову.

Она ещё не знала, будет это повесть, или целый роман, но метила на
научную фантастику. Норе не терпелось закончить рукопись -- такое
призрачное ощущение чего-то великого появлялось каждый раз, когда
сознание нашей героини посещала новая идея. В итоге, верхний ящик её
прикроватного столика был переполнен неоконченными рукописями, и хотя
некоторые из них выглядели довольно увесистыми, остановившись однажды,
Нора никогда не возвращалась к старым историям.

Но в этот раз все обещало сложиться иначе.

Погруженная в изучение собственных записей, Нора совершенно забыла о
времени. С неподдельным изумлением она уставилась своими зелеными
глазами на часы, показывающие пять вечера, и попросила как можно скорее
принести счет.

\section*{7}\label{7}
\addcontentsline{toc}{section}{7}

\markright{7}

Объявление о предоставлении жилья Элеонора нашла незадолго до
назначенного срока выезда. Совершенно случайно. Первоначально девушка
собиралась поселиться в каком-нибудь небольшом хостеле и уже по ходу
дела подыскать что-нибудь, удовлетворяющее её потребности, так как
заселиться она собиралась на весьма длительный срок. Однако заметка,
попавшаяся ей на глаза, тут же привлекла внимание Норы.

Её будущая арендодатель писала на украинском. Перевод же объявления
выглядел бы следующим образом.

\begin{quote}
\emph{На продолжительное время сдается этаж старинного здания с видом на
культурные достопримечательности города. В апартаменты входят две
спальни, одна из которых с балконом, ванная комната и просторная
гостиная. Есть беспроводной интернет. Помещение расположено в тихом,
безлюдном месте, десять минут ходьбы до трамвайной остановки.}
\end{quote}

Ниже указывалась цена аренды, невероятно низкая для такой роскоши, и это
окончательно убедило Нору зарезервировать новое жилье.

Элеонора не любила опаздывать, но делала это с завидной регулярностью.

«С видом на культурные достопримечательности,» -- напомнила себе девушка
с толикой облегчения.

Значило ли это, что пункт назначения находится вблизи от центра? Она
полагала, что да. И так как сама Нора находилась в центре, немного
времени на дорогу у неё всё ещё оставалось.

Девушка миновала Оперный театр, на ходу любуясь чарующей скульптурной
композицией, отражающей радости и страдания жизни, что возвышалась над
коринфскими ордерами в убаюкивающем свете тысячи огней: крылатые музы,
олицетворяющие гениев Славы, Музыки и Трагедии. Чуть ниже располагалось
несколько десятков античных фигурок, разбросанных по всему зданию. В
вечернее время, за пеленой падающего снега казалось, что они пляшут на
фасадах здания. Шагая по заснеженной аллее, Нора ощутила, как по её телу
пробежали мурашки, вызванные увиденной картиной.

Вскоре ей на глаза попался крохотный туристический трамвайчик,
разъезжающий вокруг площади и периодически останавливающийся у
выдающихся мест, что располагались буквально на каждом углу. Девушка
подошла к нему и назвала нужный ей адрес, поинтересовавшись, не проходит
ли экскурсия мимо этого места.

-- Перепрошую?\footnote{Прошу прощения?} -- удивленный взгляд.

Элеонора повторила нужный ей адрес.

Водитель и его спутница переглянулись, после чего первый отрицательно
покачал головой и поспешил вернуться к экскурсии, а Нора -- чуть более
чем обескураженная -- продолжила свой путь.

Трамвай подошел быстро и, видимо, ввиду погоды был абсолютно пуст.
Прислонившись к стеклу, Элеонора слушала музыку и наслаждалась
проплывающей за окном картиной. Казалось бы отдохнувшие в кофейне ноги
заныли с новой силой; девушка уже в сотый раз прокляла свою любовь к
каблукам.

Спустя какое-то время снег перестал и теперь лишь изредка мелькал за
окном. Старый город постепенно сменился пейзажами, которые мало чем
отличались от предыдущего места жительства Элеоноры. Трамвай ехал
слишком долго, и тень сомненья постепенно окутывала девушку. В итоге,
она достала телефон и заглянула в электронную почту.

-- Вот оно.

В своем заключительном письме Агата писала:

\begin{quote}
\emph{Вам нужен трамвайный маршрут №7, что проходит через проспект
Свободы, в паре шагов от Оперного. Сойдите на улице Мечникова, (13737)
это третья с конца остановка. Выйдя из трамвая, увидите врата под
остроконечной каменной аркой. Входите. Сверните налево и идите прямо,
пока не встретится памятник Франко.}
\end{quote}

(Ответ также пришел на украинском, и в этом месте было написано
«Франка», а не «Франку», но наша героиня то ли не заметила подмены, то
ли попросту сочла ее случайной оговоркой.)

\begin{quote}
\emph{Потом направо и вдоль сосновой аллеи, в конце которой Вы увидите
высокое здание из черного камня. Оно здесь единственное, так что не
спутаете. Если надумаете взять такси, адрес: Мечникова 33/8, или же
попросите остановить у Лычаковского музея, Вас поймут.}

\emph{P.S. Пускай Вы и обещали быть в пять, напомню, что в ворота Вы
должны зайти не позже шести вечера, иначе придется ждать утра.}

--- А.
\end{quote}

Ещё при первом прочтении письма постскриптум показался норе странным.
Теперь же он выглядел тревожным, особенно с учетом того, что до
указанного времени оставалось всего ничего. Девушка вздохнула и сделала
музыку громче, изо всех сил стараясь взять верх над накатившей
сонливостью.

\section*{8}\label{8}
\addcontentsline{toc}{section}{8}

\markright{8}

Когда Элеонора сошла с трамвая, с шести часами вечера её отделяли всего
три минуты. Батарейка в телефоне предательски сдохла. Будучи выходцем
двадцать первого века, Нора не имела при себе фонарика. Незнакомая,
пугающе пустынная улица была полностью покрыта тьмой и столь разительно
отличалась от той части города, в которой девушка провела последние семь
часов. К счастью, светлые каменные врата, о которых писала Агата,
величественно выделялись во мраке, и Нора со всех ног поспешила к ним.

Открыты.

Но вокруг -- ни души. Какое-то время она простояла у входа, опрометчиво
вглядываясь в темноту, а спустя несколько шагов в неопределенном
направлении, всё-таки, заприметила танцующий лучик белого света, но тот
был слишком далеко, да и к тому же находился по правую руку, тогда как
девушке предстояло идти в противоположную сторону.

И она пошла.

На протяжении первой минуты все было в порядке, но мгновеньем спустя
Элеонора наткнулась на какой-то высокий камень и во весь рост
растянулась на земле.

-- Фак! -- вскрикнула девушка.

Живя с не знающими английского языка родителями, в подобных ситуациях
она привыкла изъясняться именно таким образом.

-- Хренова бичара\footnote{Bitch! (англ.) -- Сука!}!

И еще спустя пару секунд:

-- Ну и дик\footnote{Dick (англ. жарг.) -- хуй.} с ним.

Холода она не чувствовала, а потому решила, что не тронется с места,
пока глаза не привыкнут к темноте.

Чемодан по-прежнему оставался в её руке.

Она не знала, как долго пролежала в сугробе, глядя на верхушки сосен,
сужающиеся там, где по её мнению должно было быть небо, но не думала,
что с момента падения прошло много времени. Наконец, зрение начало
адаптироваться, и Элеонора поняла, что уже может различать тонкие
очертания стволов.

Хороший знак. Девушка понемногу начала подниматься.

Нора машинально взглянула на свой личный камень преткновения. Тут-то всё
и прояснилось. Лычаковский -- девушка вспомнила, где прежде слышала эту
фамилию: всё в том же путеводителе.

Её преградой действительно был камень, надпись на котором гласила:

Наталья В. Гробовец

1809 -- 1849

Она обнаружила, что находится на кладбище, вероятно, открытом для
посещений до шести часов вечера.

\section*{9}\label{9}
\addcontentsline{toc}{section}{9}

\markright{9}

До этого дня Элеонора никогда не замечала за собой
койметрофобии\footnote{Coimetrophobia -- боязнь кладбищ.}, но в
сложившихся обстоятельствах вычурные старинные надгробия,
расположившиеся по обе стороны её пути, выглядели зловещими. Никаких
проблесков света, никакого ветра, или шума автомобилей. Последние
пристанища мёртвых встречали и провожали её в кромешной тишине. Для
полноты картины не хватало лишь густого тумана, плавно вытекающего
из-под какой-нибудь могилы.

Памятником Франка также оказалось надгробие писателя.

«И впрямь культурная достопримечательность,» -- девушка нахмурила брови.

Сворачивая направо, она в полной мере не ожидала увидеть обещанное
строение, выполненное из черного кирпича. Происходящее скорей напоминало
хорошо продуманный розыгрыш. Но помимо всего прочего Нора была самым что
ни на есть филантропом, а потому решила дать этой истории шанс и дойти
до конца. К тому же, возможность вызвать такси отсутствовала, а пройти
весь путь вновь не было сил.

Нора испытала смесь удивления и облегчения, заприметив то самое здание.
Ввиду своего цвета, оно в последний момент выпрыгнуло из темноты.

Возведенное в три этажа, строение служило прекрасным примером раннего
барокко и словно вышло из-под руки самого Пёппельмана: утонченные, но в
то же время сильные пилястры; идеально выдержанные пропорции изящно
украшали статуи скорбящих атлантов, что притаились среди цветочных
арабесок; те же орнаменты непринужденно свешивались вдоль колонн и
полуколонн, венчая богатым узором центральный вход. Вместе с тем, здание
имело каркасную систему строения, что придавало ему особенно мрачный
вид, в несколько раз увеличивая реальный размер.

Вывеска над парадным входом гласила:

МАЯТНИК

И подпись ниже:

Бюро ритуальных услуг

-- И почему меня это не удивляет? -- произнесла Нора и тут же вздрогнула
от звука собственного голоса, предавшего тишине ещё большее могущество.

Девушка собрала последние силы в кулак и быстрыми шагами поднялась по
высоким ступеням. На месте окон почивали высокие витражи, а аналогом
звонка служила медная ручка-стучалка, традиционно выполненная в форме
маятника и оказавшаяся чересчур увесистой. После третьего стука в одном
из витражей зажегся свет. Послышались неспешные шаги и звук
открывающихся замков.

Дверь отворилась, и на крыльцо вышла особа, прекрасно вписавшаяся во
время и место действия: высокая, бледная, темноволосая фигура в
свободном, покрывающем ступни чёрном платье; шляпка в тон одеждам
натянута на затылок; достающие до бедер волосы были прямыми и скрывали
треть лица; на шее -- подвеска в виде маятника, но наиболее странным
элементом оказались старомодные солнцезащитные очки, столь неуместные в
это время суток.

«Должно быть, у нее проблемы с алкоголем,» -- мелькнуло в голове Норы.

Нельзя сказать, что она слишком уж ошиблась.

Определить возраст стоявшей на ступенях личности пока не представлялось
возможным.

-- Значит, снег уже выпал\ldots{} -- произнесла та невероятно низким
голосом, и тут же добавила: -- Вы опоздали на 84 минуты. Это хорошо: не
люблю просыпаться до наступления темноты.

-- Вы Агата?

Хозяйка дома ничего не ответила, а лишь медленно кивнула и протянула
девушке бледную руку. Внимание Элеоноры тут же привлек увесистый
перстень в форме черепа, выглядевший на удивление женственным в
сочетании с длинными ногтями-семечками.

Они обменялись рукопожатием.

-- Меня зовут Элеонора, но достаточно будет простого обращения «Нора».

Уголок рта арендодателя, если её все еще можно было считать таковой,
странно дернулся. Лишь спустя сутки наша героиня узнала, что это
движение означает улыбку. Элеонора не могла с точностью этого
утверждать, но казалось, что Агата внимательно наблюдает за ней.

-- Рады видеть вас в бюро ритуальных услуг «Маятник», Элеонора, --
изрекла Агата.

Она отступила в сторону, предлагая гостье войти.

\section*{10}\label{10}
\addcontentsline{toc}{section}{10}

\markright{10}

Следуя тускло освещенными коридорами за хозяйкой бюро, Элеонора никак не
могла отделаться от мыслей о том, что она добровольно собирается
заселиться в украинскую версию замка Дракулы, (именно таким он
представлялся девушке во время чтения романа, и, посетив в последствии с
десяток «оригинальных поместий», она осталась более чем разочарованной)
а экскурсию по дому ей проводила гранжевая версия Мартиссии Адамс.

«Тудудудум-пам-пам, тудудудум-пампам, тудудудум, тудудудум, тудудудум,
пам-пам, -- услышала она знакомый мотив в собственной голове, а вслед за
ним: -- Дети ночи\ldots{} Какую музыку они заводят!»

Нора прокашлялась, желая скрыть улыбку.

Она опустила взгляд и попыталась сосредоточиться на шумной картинке.
Ноги шагающей впереди Агаты терялись в темноте касающегося пола платья,
так что Норе чудилось, что её спутница не идет, а парит в парочке
сантиметров над землей.

-- У вас очень необычный дом, -- вежливо заметила девушка. -- Я бы
сказала, выдающийся.

Ответа не последовало.

-- Могу я узнать, как давно было построено бюро?

-- Спустя год после снования кладбища, -- ответила Агата, и, заметив на
лице собеседницы замешательство, добавила: -- В тысяча семисот
восемьдесят восьмом.

Ступенька под ногой Норы устало скрипнула, словно подтверждая
вышеназванную дату.

-- Что ж, это\ldots{} впечатляет.

Когда они преодолели первый лестничный пролет, хозяйка дома
остановилась.

-- Весь первый этаж занят под бюро ритуальных услуг, -- она махнула
рукой. -- В правом крыле находится прощальный зал, а левое занято мои
кабинетом и другими помещениями, о предназначении которых вам едва ли
хочется слышать.

Проследив за этим движением, Нора увидела лишь темноту.

Девушка поёжилась.

-- Здесь же находятся ваши апартаменты. Ну, а последний этаж мой и он
мало чем отличается от этого.

Внутри было довольно свежо, но всё же лучше, чем на улице. Рука Агаты на
мгновенье исчезла в темноте, нащупывая выключатель, и вскоре к огням
холла присоединилось с десяток светильников. Стены были увешаны
картинами, взятыми в витиеватые рамки всевозможных форм и размеров;
мебели -- совсем немного, казалось, она втрое старше нашей героини.

Мысленно Элеонора отметила, что внутри здание выглядит куда меньше,
несмотря на музейный интерьер, но ничего не сказала.

Агата подняла ладонь вверх и, медленно согнув тонкие пальцы, двинулась
вдоль коридора, приглашая девушку следовать за ней. Последняя не
заставила себя ждать. Сквозь пустующий дверной проём они проникли в
просторную гостиную.

Первым объектом, завладевшим вниманием девушки, стал резной камин, в
несколько раз превосходящий в размерах тот, что остался в её доме.

«Бывшем доме,» -- напомнила себе Элеонора.

В конце комнаты виднелись двойные высокие двери, под стать потолкам.
Хозяйка кивнула в сторону первых.

-- Располагайтесь, -- произнесла она, и Нора поняла, что ей никогда не
удастся привыкнуть к этому загробному голосу. -- Могу предложить кофе,
хотя ближе к ночи я предпочитаю вино. Это удобно, потому что в дневное
время мне глаз не разомкнуть.

-- Кофе был бы очень кстати.

Агата кивнула и вскоре исчезла за пределами комнаты.

Мебель в первой спальне полностью повторяла обстановку гостиной: резная,
дубовая и очень-очень старая. Единственным отличием стала двуспальная
кровать с кованной спинкой, отбрасывающей причудливые тени на черного (а
вы сомневались?) цвета стены. В гостиной же на её месте покоился
диванчик из алого бархата.

Закрыв за собой дверь, Нора громко выдохнула и поспешила отделаться от
изводивших её весь день туфель. Спустя долю секунды те уже летели в
одному богу известном направлении.

От дубленки девушка избавилась в более мягкой манере, но тоже без особой
любви. Следом отправились свитер и джинсы. Чемодан временно прилег под
кроватью -- сил на раскладывание вещей у девушки уже не было.

Она улыбнулась сложившейся ситуации и растянулась поперек ложа.

\section*{11}\label{11}
\addcontentsline{toc}{section}{11}

\markright{11}

Элеонора прикрыла глаза и направила душ в свою сторону, с облегчением
почувствовав, как поток тёплой воды смывает с её лица остатки утреннего
макияжа. Усилием воли она заставила свою задницу, а вместе с ней и
другие части тела, покинуть пределы кровати. Теперь тёплая, практически
горячая струя ласкала кожу, вгоняя девушку в ещё большую сонливость.

Ванная комната примыкала ко второй спальне -- очередное просторное
помещение. Выполненное в кофейных тонах и обставленное элементами
современной мебели, оно разительно отличалось от предыдущих комнат.
Впервые ощутив этот контраст, девушка вновь подумала о ТАРДИС.

Вымотанная, уставшая до невозможности, она откисала под душем, лениво
наблюдая за тем, как тонкие струйки воды стекали по запотевшему стеклу
кабинки. Время от времени, Нора мысленно корила себя за
неосмотрительность: это же каким надо обладать легкомыслием, чтобы мало
того, что в последний момент взяться за поиски жилья, так ещё и не
удосужиться проверить место его нахождения, и это при том, что в
расположении человечества уже лет десять имеется такая крутая штука как
Гугл-поиск.

-- Расслабься, -- раздраженно бросила девушка. -- Выспись для начала, а
завтра уже подыщешь что-нибудь адекватное.

-- А-д-е-к-в-а-т-н-о-е, -- спустя полминуты повторила Элеонора.

Да уж, это точно не о её новых апартаментах.

Нора широко зевнула и потянулась за полотенцем.

Душ определённо не предавал бодрости.

*

Норе думалось, что в ванной комнате она провела от силы десять минут, но
на деле оказалось, что процесс этот занял втрое больше времени.
Возможно, в какой-то момент она просто отключилась, ведь такое случалось
и прежде. Так или иначе, когда девушка вошла в гостиную, сжимая в руках
сумочку, дожидавшийся её напиток уже успел остыть.

-- Как вам здесь нравится? -- осведомился низкий хрипловатый голос.

Элеонора непроизвольно взглянула на его источник. Агата раскинулась
поперек одного из кожаных кресел, перебросив ноги через подлокотник, и
выглядела слегка поддатой. В левой руке та держала бокал-тюльпан,
наполненный бардовой жидкостью, напоминавшей вино; в правой -- дымилась
сигарета. Шляпа исчезла, (в мягком свете раскачивающейся над их головами
люстры, Нора видела, как пряди волос гоздині\footnote{Хозяйки}
переливались янтарным цветом) но очки по-прежнему были на месте.

Силясь отыскать вежливый ответ, девушка еще раз взглянула на
собеседницу. Голова Агаты была чуть повернута, но зеркальная поверхность
очков не позволяла определить траекторию взгляда последней.

Нора дружелюбно улыбнулась. На всякий случай.

-- Ваш дом напоминает мне Дрезденскую Цитадель Пёппельмана.

Тишина.

-- Вы ведь знаете, о чем я?

Агата медленно кивнула.

Однако, весь её внешний вид кричал о том, что женщине -- или девушке,
Нора всё ещё не могла определить возраст своей новой знакомки, так что
решила пока думать о ней как о даме -- было совершенно наплевать на
происходящее вокруг.

-- Присаживайтесь, -- наконец, отозвалась Агата, и Нора вдруг осознала,
что до сих пор стоит посреди комнаты.

Приземлившись на диванчик, который был не столь мягок, каким казался при
первом взгляде, Элеонора заметила, что на ногах так-себе-собеседницы
надеты ни то криперы, ни то мартина -- в общем, какие-то глянцевые
говнодавы неизменного чёрного цвета.

«Ноги есть, -- подумала Нора. -- Это уже хорошо».

-- Агата?

Бледное лицо, вернее та его часть, что не была скрыта очками,
повернулась в сторону Норы. Плавно и неторопливо. Девушку уже понемногу
начинала раздражать эта медлительность, вызванная то ли алкоголем, то ли
чёрт его знает чем ещё.

Но девушка всё-таки решилась задать кипящий в её голове вопрос.

-- Но почему вы не написали, что дом находится на кладбище? -- мягко
спросила она, и увидела бровь, выглянувшую из-под очков
так-себе-собеседницы. -- То есть\ldots{} Я хочу сказать, вы, конечно,
написали `Лычаковского', но\ldots{}

-- Я ведь упомянула памятник Франка, -- безразлично заметила Агата. --
Имей я ввиду статую, а не надгробие, я бы написала `Франку'.
Элементарная грамматика украинского языка.

Теперь уже Нора молчала.

-- К тому же, много ли вы знаете здешних достопримечательностей, которые
бы могли сравниться с некрополем?

«Тысячу и одно,» -- подумала Элеонора, но, как любой филантроп, говорить
она этого, естественно не стала.

Тем временем Агата осушила свой бокал, выбралась из кресла и направилась
в другой конец гостиной.

Девушка откинулась назад и принялась разглядывать окружающую её
обстановку: стены были увешаны картинами, практически все -- портреты, и
хотя девушка видела их впервые, работы показались ей, как она бы
сказала, весьма выдающимися.

Несмотря на предвкушение грядущих странностей, Нора ощущала дискомфорт,
сидя в тишине, нарушаемой лишь завываниями внезапно разыгравшегося на
улице ветра. Она отхлебнула холодного кофе, который -- о чудо! --
оказался без сахара и предприняла новую попытку завязать беседу.

-- Как давно вы здесь живете?

-- Вы действительно хотите это знать? -- вновь подняла брови Агата.

Она уже вернулась с новым, до краев наполненным бокалом, в придачу к
которому, решив не мелочиться, прихватила и бутылку, этикетки на которой
не было

-- Люди часто задают вопросы, которые их не интересуют, чтобы потом
получить ответы, которые, в свою очередь, интересуют их ещё меньше, --
изрекла Агата, заметив следы замешательства на лице гостьи. -- А
делается этот ритуал ради какой-то мифической вежливости. Конечно,
можете рассказать, чем вы занимаетесь, например, но, бога ради, не
заставляйте меня участвовать в этом утомительном процессе! --
так-себе-собеседница наигранно отмахнулась. -- Но полагаю, приятней
будет просто пропустить бокальчик-другой. Вино и музыка -- вот лучшие
собеседники.

Не дожидаясь ответа, она протянула Норе бокал, а сама приложилась к
бутылке.

-- Берите-берите. Кофе ваш уже, поди, давно остыл.

Всё ещё удивленная внезапным монологом хозяйки дома, Нора взяла бокал.
Она отхлебнула вина (то оказалось сладким, с примесью чайной розы) и,
сама не зная отчего, принялась рассказывать о своей деятельности.

Агата ухмыльнулась, но перебивать не стала.

-- По образованию я менеджер по управлению проектами\ldots{}

Девушка поднялась со своего места и, выудив из сумочки пластиковую
визитку, протянула последнюю Агате, которая уже успела плюхнуться
обратно в кресло. Та молча взяла карточку.

--\ldots{} Пару лет работала в одной крупной компании, расположившейся в
моем родном городе, но южные индустриальные пейзажи мне, мягко говоря,
чужды, так что я дала себе слово покинуть город при первой же
возможности. Прошлось во многом себе отказывать, но, тем не менее, мне
удалось собрать подходящую сумму для того, чтобы пару лет безбедно жить
и заниматься тем, к чему лежит душа.

Агата не спросила, откуда родом её гостья, и Нору это вполне устраивало.

Она взглянула на так-себе-собеседницу. Полулежа, та беззаботно
потягивала вино и без особого интереса вертела меж тонких пальцев
визитную карточку. Норе уже было невмоготу наблюдать эту навевающую
сонливость меланхолию и она поспешно переключила свое внимание на
висящие перед ней портреты.

Одну из картин Элеонора, все-таки, узнала. Эта была «Женщина дождя»:
высокая утонченная фигура, облаченная в черные одежды; поверх мертвенно
бледного лица виднелась широкая бесформенная шляпа, с полей которой
стекали дождевые капли; на поджатых губах играла знакомая ухмылка --
сходство с хозяйкой дома сразу же бросалось в глаза. Картина вызывало
противоречивые чувства тревоги и умиротворения, но первым делом
завораживал отсутствующий взгляд, лучащийся из-под опущенных ресниц
рисованной девушки. Глаза были белыми.

«А что если она носит очки потому, что\ldots?» -- Элеонора нахмурилась и
тут же отогнала от себя эту глупую мысль.

Она сделала жадный глоток вина.

-- А душа у меня лежит к писательству, -- девушка застенчиво улыбнулась,
отчего на щеках проступили очаровательные ямочки. -- Вот только проблема
в том, что за всё время мне не удалось закончить ни одной рукописи.
Потому я и выбрала Львов -- маленький старый город, идеальное место для
писателя.

Ещё глоток.

-- Знаете, в идеале мне хотелось бы открыть своё издательство. Пускай
небольшое, но\ldots{}

Нора вздрогнула и замолчала. На долю секунды девушка увидела, как
«Женщина дождя» подняла веки и взглянула прямиком на неё своими
белоснежными глазами. Элеонора испуганно моргнула, а когда вновь
взглянула на картину, увидела, что изображенная на ней особа по-прежнему
смотрит куда-то вниз.

«Твою-то мать, вроде не горилку пью, а привидится же такое!»

Нора также заметила, что хозяйка дома прекратила вертеть в руках
карточку и теперь с интересом наблюдала за ней.

-- Что с вами?

-- Вино ударило в голову, -- тихо произнесла Элеонора. -- Должно быть, я
переутомилась.

-- Возьмите сигарету.

-- Ох, у меня есть\ldots{}

Девушка принялась рыться в сумочке, ей внезапно захотелось перекурить.

Агата вновь поднялась, но на этот раз не за вином. Она направилась к
стоявшему позади Норы серванту и на протяжении нескольких минут была
занята чем-то вне поля зрения нашей героини. Вскоре в гостиной разлились
сладкие звуки музыки, ставшие мёдом для ушей Элеоноры.

-- Винил! -- восхищенно вскрикнула она и обернулась. -- Шестьдесят
третий год?

-- Шестьдесят первый, -- небрежно ответила Агата, но холода на лице
так-себе-собеседницы заметно поубавилось.

«Возможно, мне стоит задержаться здесь ещё ненадолго», -- расслаблено
подумала Элеонора.

К тому же, в её телефоне собралось более сотни фотографий, которые сами
себя не обработают и в социальные сети на опубликуют.

\section*{12}\label{12}
\addcontentsline{toc}{section}{12}

\markright{12}

Дело близилось к полуночи.

Элеонора беспокойно ерзала в постели. Она ощущала себя уставшей и
изношенной, но окунуться в сон никак не удавалось. Виной всему была
давящая тишина, создаваемая то ли локацией, то ли предназначением
похоронного бюро. Нора понятия не имела, находились ли в здании так
называемые клиенты Агаты, и всеми силами старалась об этом не думать.

Какое-то время до её ушей еще доносились свинговые мотивы, льющиеся из
гостиной, но вскоре они исчезли: вероятно, Агата поднялась в свои
апартаменты. Нора пролежала в тишине больше часа. В итоге она
выскользнула из постели и направилась к высокому окну. Она приподняла
раму и достала из шкафа запасное одеяло. Тишина сменилась завыванием
ветра. Холод и внезапно разыгравшаяся вьюга не добавляли уюта, но, всё
же, были лучше мёртвого затишья.

«Я лежу в огромном здании, которое находится посреди старого кладбища. В
кромешной темноте. Поблизости ни души, лишь покойники. Последние,
возможно, находятся ближе чем я думаю\ldots{} -- Нора вздрогнула и
натянуло одеяло до самых глаз. -- Прекрати! Сейчас же переключи ход
мыслей».

-- Погода в этих краях так непостоянна, -- заплетающимся языком
произнесла девушка и наконец провалилась в тревожный сон.

*

Девушке снилось, что она бродит по кладбищу, не в силах отыскать выход.
В какую сторону не глянь, ей улыбались призрачные очертания надгробий,
сверкающие в лунном свете. Откуда-то издалека лилась музыка -- смесь
виолончели и контрабаса -- прекрасная, но вместе с тем зловещая в
сложившихся обстоятельствах. Снега и в помине не было, зато слышались
шаги. Их слегка приглушенный звук эхом отдавался в темноте и Нора никак
не могла понять, где находится её преследователь.

Вдруг она почувствовала на плече чьи-то цепкие пальцы, обернулась и
\ldots{}

Показанные в кино пробуждения после кошмаров обычно сопровождаются
криками и резким поднятием, однако в случай нашей героини был вовсе не
из этих. Элеонора распахнула глаза и медленно перевернулась на бок, но с
подушки не подорвалась; дыханье её оставалось ровным.

«Склеп, вот что мне напоминает этот чёртов дом».

Она вновь сомкнула веки, но тут же с ужасом прислушалась: шаги не
исчезали. Казалось, кто-то бродит в стенах\ldots{} Шаги смолкли, а затем
вновь раздались прямиком за её спиной, но это было невозможно, ведь
спинка кровати примыкала к наружной стене второго этажа.

Девушка с трудом отогнала эти мысли и вынудила себя прекратить
прислушиваться.

«Ко мне в гости пожаловал сонный паралич, оттуда и тревожность, --
сказала себе Нора. -- Все логично, я же гуляла целый день».

Она шевельнула кистью, убежденная, что ничего не выйдет. Увы, девичьи
пальцы послушно коснулись ладони.

Шаги тем временем раздавались то тут то там, заставляя шуршать половицы.
После очередного такого скрипа Элеонора сдалась. Она нащупала телефон и
включила на всю громкость «El Pintor». На середине третьей песни Нора
почувствовала, что проваливается в сон.

Она не знала, в котором часу проснулась, но решила, что прошло как
минимум 30 минут, потому что музыка в её ушах смолкла. Видимо, альбом
закончился.

Из сна её опять вырвали шаги.

На этот раз Элеонора лежала неподвижно с едва приоткрытыми веками, боясь
пошевельнуться. Нет, кто-то определенно точно только что пробежал в
полуметре от нее\ldots{} Вот только там была стена, смотрящая на
кладбище.

Шаги раздавались уже у окна, а затем исчезли так же внезапно как и
появились. По коже Элеоноры скользнул холод, когда она увидела картину,
заставившую сердце забиться где-то на уровне горла: большую часть
витража заслонил вытянутый силуэт, вырисовавшийся буквально из неоткуда.
Сутулый, со свисающими вдоль тела руками, он был на голову, а то и
больше, выше хозяйки дома.

Силуэт стоял неподвижно, повернутый в сторону кровати. Девушка
чувствовала на себе его взгляд, а когда незваный гость сделал шаг в её
направлении, Нора выпрыгнула из постели и метнулась в сторону выхода --
именно там находился выключатель (вот вам первый минус больших комнат).
Бежала она секунд десять, слыша лишь свое истерически бьющееся сердце,
но успела заметить, что силуэт взмахнул руками.

Дальше классика жанра: когда напуганная до усрачки Нора обернулась,
комната была пуста. На неё смотрели лишь три пары портретных глаз.

В ту ночь девушка спала с включенным светом.

\section*{13}\label{13}
\addcontentsline{toc}{section}{13}

\markright{13}

Элеонора вынырнула из сна за час до полудня. За окном блестело
пушкинское утро: стекла были исписаны ледяными узорами, редкие снежинки
мерно парили за окном в лучах зимнего солнца. Утро впрямь оказалось
мудренее вечера и, как всегда бывает в подобных историях, Нора сочла
произошедшее ночью дурным сном.

Как и все герои этих самых историй, она ошибалась.

Девушка сладко потянулась и поплелась в гостиную. Таким славным утром
дом выглядел куда гостеприимней -- даже компанию «Женщины дождя» Нора
сочла приятной.

На камине её ждала записка.

\begin{quote}
\emph{В погребе уйма вина, но о завтраке придется заботиться самой.
Через дорогу от кладбища есть парочка киосков, но в них вряд ли найдется
что-нибудь пригодное для употребления. Ближайший известный мне
супермаркет находится по адресу Соборная площадь 14. Добраться туда
можно все тем же трамваем №7.}

\emph{P.S. До темноты на третий этаж не ходите. Можете, конечно,
рискнуть, но, говорят, я избиваю людей, помешавших моему сну, а потом
этого даже не помню.}

--- А.
\end{quote}

На этой веселой ноте Элеонора наскоро оделась и отправилась по
указанному адресу. Хотя она и предпочла трамвай прогулке, путь всё равно
был неблизким, и, заняв одиночное место у окна, девушка написала ещё
несколько страниц то ли грядущего романа, то ли повести.

Действие разыгрывалось в нескором будущем. Смесь древней архитектуры и
космических технологий увлекала Нору, так что за описания она бралась с
нескрываемым восторгом. Да и диалоги теперь вызывали меньше подозрения.
В старый город наша героиня вернулась довольная проделанной работой.

Времени до наступления темноты оставалось более чем достаточно. Прежде
чем отправиться за продуктами, Нора задержалась, чтобы позавтракать в
одной из милых кофеен, мостившихся на каждом шагу.

«Світ Кави» находилась около Каплиці Боїмів, и девушка с разочарованием
подумала, что нет в русском языке слова, которое с точностью могло бы
передать суть этого сакрального здания. На ум приходила лишь «часовня»,
но это совсем не туда.

Заведение примыкало к светлому жилому зданию. Вход украшали каменные
вазоны, в которых росла желтая ветреница, и пускай до Нового года
оставался практически месяц, веранда была увешана множеством гирлянд.
Такую обстановку наша героиня обычно величала «ламповой».

«Должно быть, вечером здесь особенно уютно\ldots» -- подумала Нора и
вошла в кофейню.

Несмотря на ранее время, все места были заняты. Официант предложил
девушке плед и, сделав заказ, та уселась в одно из плетеных кресел под
навесом. На улице Элеоноре нравилось гораздо больше: она считала, что в
большинстве своем работа писателя заключается в наблюдении, поиске
различных типажей характера и их изучении, а потому, ожидая заказ,
увлеченно наблюдала за оживленной улицей.

\section*{14}\label{14}
\addcontentsline{toc}{section}{14}

\markright{14}

Над кладбищем сгущались поздние сумерки. Снег эволюционировал в дождь;
ветер гонял над надгробиями капли воды. Вытянувшись во весь рост, Агата
Рахманинова лежала поверх небрежно застеленной кровати с доживающей свой
век сигаретой в зубах; её босые ноги практически доставали до конца
ложа. Граммофон извергал из себя второй и последний альбом Джой Дивижн.
На прикроватном столике, соседствующий с солнцезащитными очками, её
дожидался томик Ремарка, заложенный на середине двадцатигривневой
купюрой.

По правую сторону от себя Агата услышала шаги и тут же потянулась за
очками, но опустила руку, когда увидела представшую посреди комнаты
фигуру.

-- Владислав, -- произнесла она, выдыхая сигаретный дым. -- Тебе нужно
чаще попадаться мне на глаза, иначе я всё время забываю о твоём
присутствии.

Девушка отложила сигарету и села на кровати.

Фигура угрюмо развела руками и опустилась рядом. Это был широкоплечий
альбинос, ростом под два метра, немногим старше самой Агаты.

-- Видел уже нашу гостью?

Владислав коротко кивнул.

-- И что думаешь?

-- Человек, -- изрек Ласло с таким выражением лица, словно говорил о
крайне мерзких вещах. -- Но красивая.

-- Не думаешь, что пора нанести ей визит? -- предложила Агата и
похлопала приятеля по плечу. -- Я имею в виду, нормальный визит.

Девушка сделала яркий акцент на предпоследнем слове.

-- Я постараюсь, -- пообещал Ласло.

Его голос сквозил неуверенностью.

Из приоткрытого окна до них донесся звук захлопнувшейся двери.

-- Вернулась, -- произнесла Агата и бросила взгляд на часы в форме
черепа, украшавшие восточную стену. -- Без двух минут шесть. Снова в
последний момент.

Она ухмыльнулась, а в случае Агаты это означало самую что ни на есть
улыбку.

-- Пойду я.

Владислав поднялся и быстрым темпом зашагал в конец комнаты. Еще
мгновенье, и он растворился в очертаниях стены.

-- Возвращайся к ужину! -- крикнула, Агата, надевая очки. -- Сыграем в
карты.

*

В дверь тихонько постучали. В привычной скупой манере Агата предложила
гостье войти. На пороге тут же возникла Элеонора. Её раскрасневшееся от
холода лицо смотрело на хозяйку похоронного бюро извиняющимся взглядом.

-- Я услышала музыку, и решила, что вы не спите\ldots{}

Агата махнула рукой, приглашая девушку присесть. Она успела натянуть
очки незадолго до прихода последней, и теперь на лице шатенки сияла
прежняя безмятежность.

Нора освободилась от шарфа и присела напротив так-себе-собеседницы.

-- Хотите кофе? -- предложила она. -- Я также купила кое-какие продукты,
но не решилась искать кухню без вас.

Агата вопросительно подняла бровь.

-- Боялась наткнуться на какого-нибудь покойника, -- объяснила Нора и
виновато улыбнулась. -- Кстати, вам самой не жутко здесь находиться?

-- Что вы имеете в виду?

Нора немного замялась, боясь оскорбить чувства собеседницы, которой
царившая здесь атмосфера, казалось, пришлась по душе.

-- Одной, посреди кладбища\ldots{}

-- Я абсолютный мизантроп, -- невозмутимо изрекла та, -- и меж двух зол
предпочитаю компанию мёртвых. К тому же, где вы весь день гуляли,
Элеонора?

Девушку удивил такой вопрос. Она решила, что Агата пытается перевести
тему, и пускай с так-себе-собеседницей Нора была знакома лишь второй
день, это не было на неё похоже.

-- Гуляла по Старому городу.

-- Конкретней?

-- Прошлась по центру, позавтракала в «Світе Кави», том что находится
около Часовни Бомоив\ldots{}

-- Во-о-от, -- протянула Агата и Нора непонимающе поглядела на неё. --
Долго вы находились около часовни?

-- Часа два, если не больше, но почему вам это интересно?

-- Вот вам интересный факт на сегодня: каплиця была возведена в начале
семнадцатого века на территории городского кладбища того времени.
Конечно, сейчас старый город старательно вымощен брусчаткой, но вы ведь
понимаете, что сегодня успешно позавтракали в компании многовековых
захоронений?

-- Вы серьезно?

Агата медленно кивнула и ухмыльнулась.

-- Мертвые приносят гораздо меньше хлопот, как видите. Всё зависит от
визуального восприятия. С ранних лет меня привлекают кладбища. Зовется
этот подарок судьбы тафофилией -- почитайте на досуге, если хотите.

Тем временем потусторонний голос Кёртиса в последний раз пропел «Where
have they been?»\footnote{«Где они были?» -- повторяющаяся строка из
  композиции «Decades» группы Joy Division.}, и проигрыватель умолк.

-- Пойдемте, -- поднимаясь, произнесла Агата. -- Покажу вам кухню.

\section*{15}\label{15}
\addcontentsline{toc}{section}{15}

\markright{15}

На протяжении оставшихся до наступления следующего года недель, Элеонора
уяснила одну вещь о хозяйке «Маятника» -- неизменно оставаясь темной
лошадкой, Агата никогда не распространялась о своём прошлом, но с
неумолимым интересом поддерживала две темы: литература и смерть. Избегая
последнего, можно было скоротать приятный вечер у камина и, хотя в
бытовых вещах её сожительница по-прежнему была так-себе-собеседницей,
интерес Элеоноры постепенно возрастал.

Уже тогда она начинала задумываться, что из Агаты вышел бы неплохой
антагонист для её грядущего романа\ldots{}

Однако, тем вечером Элеонору ожидало ещё одно открытие.

Кухня оказалась расположенной в противоположном крыле третьего этажа.
Как и ванная комната, она была оборудована в стиле двухтысячных и
выглядела неуместной на фоне остального интерьера бюро ритуальных услуг.

После кратковременной дискуссии, во время которой Агата с безразличным
видом опустошала первый за этот вечер бокал, дамы сошлись на том, что
ужином (а кому завтраком) им послужит пицца, и вскоре принялись за
готовку. Вернее, готовила Нора, в то время как её сожительница
разглагольствовала о хемингуэевских героинях и лениво потягивала вино.

-- \ldots выходит, что среди перечисленных барышень нет ни одного
сильного характера, что слегка удручает.

-- А как же Пилар? -- бросила Нора, которая пропустила большую часть
монолога.

Она покончила с укладкой ингредиентов на корж и с облегчением сунула
противень в духовку.

-- Пилар скорее исключение из общего правила. Плюс, говорю же, она и
близко не является возлюбленной главного героя, а посему я делаю вывод,
что Хем концентрируется исключительно на силе духа своих мужских
персонажей, чего не скажешь о Ремарке. Взять хотя бы его Элизабет.

Элеонора устало опустилась на высокий стульчак и плеснула себе воды.
Прикончившая третий бокал Агата взглянула на эту картину крайне
неодобрительно, но ввиду тёмных очков жест этот остался незамеченным.

Нора не знала, в чем скрывался секрет внезапной болтливости её
собеседницы: в вине, или подходящей теме, но решила непременно
воспользоваться ситуацией.

-- Послушайте, -- начала она, -- сегодня вечером, поднявшись на третий
этаж, я слышала не только музыку\ldots{} Мне кажется я слышала голоса.

Агата ничего не ответила.

-- Я бы не стала спрашивать просто так, но прошлой ночью произошло
кое-что странное.

-- Вот как?

-- Долгое время я не могла уснуть, а когда удалось, меня разбудили шаги.
Я понимаю, как это звучит\ldots{} Кто-то буквально бродил в стенах, но!
Это ещё не самое стрёмное: в какой-то момент у моей кровати появилась
высокая фигура, возникшая словно из неоткуда.

Договорив, Нора в ожидании посмотрела на так-себе-собеседницу.

«Сейчас она решит, что я свихнулась, и попросит съехать в ближайшем
будущем, -- пронеслось в голове девушки. -- Возможно, это не худший
вариант».

Последовавший вопрос более чем удивил Нору.

-- В это время вы часом не слушали музыку?

Девушка робко кивнула.

-- Пост-панк? -- спокойно уточнила Агата? -- Что это было: Кьюр, Джой
Дивижн? Или, может быть, что-нибудь поновее, Бон Ивер, скажем?

-- Интерпол, -- тихо ответила Нора, пораженная до глубины души.

-- Тем более!

Губы Агаты изогнулись в подобии улыбки.

-- Да будет вам, -- произнесла она, завидев побелевшее личико Норы.

-- Почему вы спросили о музыке? -- только и нашлась девушка.

Агата осушила свой бокал.

-- Потому что только так можно выманить Владислава из его окопов.

Нора удивленно приоткрыла губы.

-- С ударением на «и», -- между тем заметила Агата.

\section*{16}\label{16}
\addcontentsline{toc}{section}{16}

\markright{16}

Владислав Мирославский был местным таксидермистом\ldots{}

-- Таксидермистом? -- с недоверием переспросила Элеонора.

Агата задумчиво кивнула.

-- Некоторые клиенты бюро прибегают к подобного рода услугам, --
объяснила она, и Нора брезгливо поёжилась. -- Но чаще всего Ласло
выступает в роли визажиста. Знаете, старается приукрасить трупики, чтобы
те не выглядели очень уж плачевно.

Так вот, Владислав Мирославский был местным таксидермистом, а по
совместительству и ярым социофобом, что лишь укрепляло их дружбу с
Агатой: руководствуясь разными причинами, оба всё же не питали любви к
человечеству и старательно избирали круг общения. Однако, если Агата
могла находится в компании незнакомцев, о Владиславе никак нельзя было
сказать того же. С годами боязнь общества, поселившаяся в голове
последнего, лишь укрепилась и, так как работа в бюро («К моему
глубочайшему сожалению,» -- заметила Агата) ритуальных услуг отчасти
подразумевала периодическое присутствие в здании живых людей, друзьям
пришлось искать компромисс.

Как выяснилось, Владислава не пугали люди, которые не могли его увидеть,
так что вскоре Агата нашла выход. После долгих лет, проведенных в старом
здании, ей были известны все тайные ходы, комнаты и лестницы, скрытые от
посторонних глаз, а таковых в доме оказалось не меньше комнат. В
большинстве своем, ходы представляли незамысловатую оптическую иллюзию:
они тянулись промеж толстых каменных стен и оканчивались у наружной
стороны какой-нибудь спальни; стена в том месте переходила в двойную, но
из-за однотонных черных обоев, которыми стены были обклеены с обеих
сторон, этого никак нельзя было заметить невооруженным взглядом.

С помощью таких коридоров Владислав мог свободно перемещаться по дому,
без страха быть замеченным. Жил он, кстати, в одной из тайных комнат,
находившейся на последнем этаже.

Хозяйка дома отвела Нору в дальнюю гостевую спальню и велела ей стать в
то самое место, где минувшей ночью девушка видела испугавший её силуэт.

-- Здесь, -- произнесла Элеонора, остановившись в метре от окна. -- Но я
ничего не вижу.

Агата отрицательно покачала головой.

-- Вплотную к окну.

Нора послушно сделала два шага вперед.

Справа от неё все так же была черная стена. Руководствуясь рассказом
Агаты, девушка вытянула руку и ахнула: кончики пальцев скрылись к
темноте. Она бросила вопросительный взгляд в сторону Агаты.

Та кивнула, и Нора сделала еще несколько шагов в темноту, а когда
наткнулась на стену, интуитивно свернула влево. Вскоре лица коснулось
что-то гладкое и лёгкое. Этим чем-то оказалась чёрная как смоль
занавесь, откинув которую, Элеоноре открылся узкий, но хорошо освещенный
каменный коридор. Она ещё какое-то время заинтересованно разглядывала
это место, после чего вернулась в спальню.

-- Надо же! -- восхищенно выпалила девушка. -- Как тонко продумано!
Выдающееся изобретение!

Агата довольно кивнула.

-- Шторку я сама повесила. Остальное шло в комплекте с домом.

-- А картины? -- вспомнила Элеонора. -- Вчера мне показалось, что
«Женщина дождя» взглянула прямиком на меня.

Агата вновь кивнула.

-- Бюро украшает около ста пятидесяти портретов, но резать эту мне было
сложней всего. «Женщина дождя» -- моя любимая. К счастью, Владислав
проделал всё очень осторожно.

Они уже практически достигли кухни, когда послышался равномерный гудок
-- пицца была готова.

Однако, у Норы оставался ещё один вопрос.

-- Но почему вы не сказали мне раньше?

-- О Владиславе?

-- Да.

-- Понимаете, -- ухмыльнулась Агата, поправляя очки, -- порой я и сама о
нем забываю.

\section*{17}\label{17}
\addcontentsline{toc}{section}{17}

\markright{17}

-- Раз уж вы благополучно пережили встречу с Ласло и не сбежали,
предлагаю перейти на «ты», -- Агата отправила в рот кусок пиццы и
поспешно залила это дело вином.

Она проделывала сей ритуал с каждый куском, и Норе не удалось скрыть
своего удивления.

-- Когда много пьешь, любая еда становится пресной, -- добавила Агата.
-- Так что скажите?

-- Всеми руками «за»! -- весело ответила Элеонора. Её лицо осветила
широкая улыбка. -- В таком случае, позволено ли мне спрашивать, откуда у
тебя столько вина?

Дама устремила бледное лицо в стороны Норы.

-- Я так понимаю, ты не спускалась в винный погреб?

-- Нет, а что там?

Очки помешали миру увидеть очередной саркастичный взгляд Агаты
Рахманиновой.

-- Вино.

-- А я думала мексиканцы, -- таким же безразличным тоном заметила
Элеонора.

Агата прекратила жевать, отставила бокал и\ldots{} расхохоталась.

-- Тарантиновские диалоги, -- отдышавшись, сказала она. -- Мои любимые.

Девушка улыбнулась в ответ, ожидая продолжения.

-- Вино я произвожу сама, за домом есть крохотный виноградник.

-- Правда?

Агата ответила утвердительно и в этот раз наполнила два бокала.

-- Бартендер, художник, арендодатель и владелица бюро ритуальных услуг,
-- перечислила Агата. -- Временами выступаю в роли пост-мортем
фотографа.

-- Так значит похоронное бюро не твой основной род деятельности?

Дама в чёрном пожала плечами.

-- В последние годы дела в бюро идут не то чтобы очень, потому я и
решила сдавать пустующий этаж. Наверное, здесь всё дело в моей харизме.

-- А что с ней?

-- Она отсутствует.

За стеной послышались приглушенные шаги.

«Крадется, паразит!» -- подумала Элеонора.

Обе девушки взглянули в сторону единственной картины, висящей в
отведенной под кухню комнате -- гротескного портрета Гоголя,
выполненного в красно-синих тонах.

Глаза Николая Васильевича оставались на месте и вскоре шаги начали
отдаляться.

-- Передумал, -- разочарованно произнесла Агата.

Она потянулась к бутылке, как вдруг недовольно вскрикнула.

-- Вот чёрт?

-- Что случилось? -- встревожилась Нора.

-- Винишко, -- печально произнесла Агата. -- Оно закончилось.

\section*{18}\label{18}
\addcontentsline{toc}{section}{18}

\markright{18}

Солнцезащитные очки особы по имени Агата Рахманинова имели зеркальное
покрытие и едва ли применялись по назначению. Будучи натурой
компанейской, Элеонора как могла пыталась подстраиваться под
мэнсоновский режим\footnote{Имеется в виду распорядок дня Мэрлина
  Мэнсона: музыкант как-то признался, что предпочитает бодрствовать по
  ночам.}, так что уже на третий день пребывания в ``Маятнике''
собственное отражение являлось Элеоноре чаще чем солнечный свет.

Девушка просыпалась задолго после полудня с твердым намерением не
сомкнуть глаз до двух, а то и трёх часов ночи. Обычно она проводила в
постели ещё какое-то время: читала и бороздила просторы всемирной
паутины. Затем отправлялась на кухню, где с грациозностью шанхайского
рыцаря варганила поздний завтрак, остерегаясь преждевременного
пробуждения хозяйки дома. Светлыми днями, распрощавшись с голодом, Нора
отправлялась гулять по городу, и каждый раз открывала для себя новые,
приятные сердцу места, что удивительно, учитывая масштабы Львова. Если
же погода оставляла желать лучшего, девушка возвращалась на свой этаж,
чтобы почитать новости и укрепить графоманию.

Одним таким дождливым днем, Нора сидела у окна гостиной, наблюдая за
тем, как зимнее солнце плавно опускается за горизонт. Паркет в радиусе
метра от неё был усыпан скомканными страницами -- работа над рукописью
продвигалась ужасно медленно.

Избавившись от очередного листа и крепко выругавшись, девушка, наконец,
задействовала эффект болта и спустилась в винный погреб, решив как
следует набраться по этому поводу.

Когда вино обратилось мёдом, а до встречи с дном бутылки оставалось
всего ничего, Нора услышала приближающиеся шаги. Наивно решив, что
сегодня Агата поднялась раньше, девушка обернулась. Естественно, она
никого не увидела.

«Владислав, чёртов ты засранец!» -- мелькнуло в голове.

Изредка она слышала льющиеся за стенами шаги, так что уже не предавала
этому особого значения.

«А у него хороший вкус», -- улыбнулась девушка: в тот вечер она слушала
Колдплей.

Следующая мысль поразила Нору не своим контекстом, но тем фактом, что
она раньше не задала себе этот вопрос.

«Почему он просто сам не ставит музыку?» -- поразительно, как временами
хмельное сознание рождает трезвые мысли.

Позже Агата объяснила, что Владислав старается не нарушать тишину из-за
страха быть услышанным.

-- Радуйся, что дела обстоят именно так, -- ухмыльнулась гоздиня. -- В
отсутствии гостей он завывает как кретин и сутки напролет может слушать
одну и ту же песню.

-- Ты преувеличиваешь, -- вежливо заметила Элеонора, думая о том, что
предмет их обсуждения, возможно, сидит за ближайшей стеной, внимательно
вслушиваясь в диалог.

-- Вовсе нет. Достаточно просто изучить его профиль на ласт.фм. Там всё
задокументировано.

Так вот, услышав шаги и определив их источник, Нора продолжила играть в
Хемингуэйа и вскоре её унесла сладкая дрёма.

*

-- Ну, надо же, -- раздался знакомый потусторонний голос.

Нора медленно выплыла из глубин сна.

Открыв глаза, она первым делом встретилась со своим залипшим личиком --
ухмыляющаяся Агата склонилась над ней. Высокая, в траурных одеждах, она
походила на предвестника смерти, кем, в общем-то, и являлась. В руках
дама держала пустую бутылку, которую Нора выронила аккурат перед тем,
как провалиться в сон.

-- Мне ещё рано, -- простонала Элеонора.

На лице Агаты проступило недоумение.

Она хмыкнула, уселась напротив плохо соображающей девушки и закурила.

-- Есть вопрос, -- произнесла Нора, когда ход её мыслей вернулся в
прежнее русло.

Агата подождала, пока собеседница откашляется, и протянула ей бокал.

-- Ой, нет! -- запротестовала девушка, заметив в нем вино.

Она хотела спросить о предназначении очков, но в последнюю секунду
передумала, вспомнив вчерашний разговор. Каждый раз, когда Элеонора
пыталась затронуть относительно личные темы, так-себе-собеседница
превращалась в генератор рандомных фактов, так или иначе связанных со
смертью. Так она узнала, что урной для изобретателя пачки Принглс
послужило его же детище -- банка-тубус, а еще -- что на японском и
китайском языках число четыре созвучно слову ``смерть''.

-- Почему ты решила открыть бюро?

-- Однажды я сказала своей матери, что если она и может меня за что-то
судить, так это за оказание ритуальных услуг без лицензии, -- прозвучал
ответ.

Нора облегченно вздохнула. Она и не подозревала, что это была ссылка на
очередную увлекательную историю\footnote{«Единственное, за что Вы меня
  можете осудить, так это за оказание ритуальных услуг без лицензии» --
  слова Джона Уэйна Гейтси, американского серийного убийцы прошлого
  века, также известного как «убийца-клоун» благодаря своим выступлениям
  на детских праздниках.}.

И слава богу.

-- На самом деле, ``Маятник'' был здесь ещё задолго до моего рождения.
Похоронный дом, так он тогда назывался. Видела дома, стоящие по ту
сторону кладбища? -- тонкая рука вынырнула из моря чёрных складок,
указывая в правую сторону.

-- Многоэтажки?

Агата кивнула.

-- В тёплую пору года моё детство проходило далеко отсюда, но всё
остальное время я жила в одном из этих домов, -- чёртово гетто с
прекрасным видом на некрополь. Последний с детства служил мне парком и
игровой площадкой, за неимением других вариантов. Чудом окончила школу,
никого не убив, а спустя несколько лет, когда дом дал трещину и жить в
нём стало совсем невмоготу, стала подыскивать альтернативу. Денег было
чуть менее чем ничего, а потому приличное жилье в районе Старого города
мне едва ли удалось бы арендовать. Я даже задумывалась, а не поселиться
ли на улице Джона Леннона, но название оказалось её единственным плюсом.

Словом, ситуация вырисовывалась плачевная. Уже тогда я начала
предпринимать первые попытки виноделия. Как-то вечером стояла у окна с
полным бокалом новехонького вина. Аромат напиток имел превосходный, но
при более близком знакомстве выяснилось, что употреблять его невозможно.
Быстро и много пить я умела уже тогда и, сделав это по неосторожности,
поняла, что меня вот-вот вывернет наизнанку. На путешествие в ванную
времени не оставалось, и я выблевала всё в окно. Детище моё оказалось
воистину убийственным: трясло минут пятнадцать, и наконец разогнувшись,
я почувствовала себя Энди Дюфрейном, выбравшимся из Шоушенка. Никогда
ещё воздух не казался мне столь сладким, а закат не виделся таким
прекрасным. Глубоко и прерывисто дыша, я задержалась у окна, пытаясь
насладиться видом и заодно придти в себя. Тогда-то взгляд и задержался
на «Маятнике». Я вспомнила, что уже с полжизни не видела, чтобы в нем
проводились церемонии прощания и следующим утром связалась с нужными
людьми.

Выяснилось, что владелец похоронного дома давно покинул этот бренным
мир, не оставив после себя наследников. В итоге здание перешло в руки
внука троюродной тетки покойного, который к тому времени жил в Одессе и
не имел ни малейшего желания заниматься подобного рода бизнесом. Юноша
даже не пытался продать бюро, которое с момента смерти хозяина стояло
нетронутым. Услышав моё предложение, он первым же поездом примчался во
Львов для оформления документов. Вероятно, боялся, что я передумаю.

Но я не передумала. Здание полюбилось мне с первого взгляда, да и район
был подходящий, учитывая моё отношение к социуму. Ласло сказал, что это
судьба.

-- Владислав?

Агата кивнула.

-- Мы и тогда были соседями.

К этому моменту Элеонора уже окончательно оправилась от алкогольного
сна. Она попросила подкурить, после чего спросила:

-- И ты тут же возобновила работу бюро?

Привычная ухмылка.

-- За кого ты меня держишь? Первые полтора месяца я просто жила здесь на
деньги, оставшиеся с суммы компенсации старой квартиры, но в скорости и
они закончились. Тогда я решила возобновить работу «Маятника». А почему,
собственно нет? Всего-то требовалось получить лицензию и наладить
торговые связи. Остальное и так было при мне.

-- И ты жила здесь одна всё то время? -- поинтересовалась Нора.

Агата помрачнела, если это выражение вообще подходит для описания дамы в
стиле Эдгара По. Она нервно провела пальцами по оправе очков. Губы
слегка вздрогнули.

-- Владислав въехал позже, -- уклончиво ответила Агата. -- После
открытия бюро.

Она предпочла превратиться в так-себе-собеседницу и Норе не удалось
вытянуть из нее что-нибудь существенное. Не считая историй о смерти,
разумеется.

Тем вечером Агата впервые удалилась в спальню раньше Элеоноры.

«Возможно, она просто плохо спала,» -- с надеждой подумала девушка.

Однако, глубокой ночью, поднимаясь на кухню за новой порцией бодрящего
кофе, (она не оставляла попыток справиться с рукописью) Нора слышала
печальную музыку, льющуюся из апартаментов хозяйки дома.

\section*{19}\label{19}
\addcontentsline{toc}{section}{19}

\markright{19}

Первая неделя пребывания в «Маятнике» подходила к концу.

Элеонора все так же не виделась с ночным призраком, но время от времени
слышала как Владислав бродит по своим каменным коридорам.

После каждых ночных посиделок таинственная владелица похоронного бюро
нравилась ей все больше: вопреки своей отстраненности (наследственной,
или приобретенной -- Нора не знала) Агата обладала неким шармом.
Загадочность, вот что влекло девушку. Она уже знала, что Агате известно
о смерти и теперь хотела выяснить, что же та знает о жизни.

При взгляде, брошенном со стороны, создавалось впечатление, что Агата
прошла через несколько столетий, и плевать хотела на каждое из них. Но
так ли это было на самом деле? Возможно, она и вовсе не покидала
пределов родного города.

-- О жизни, о смерти\ldots{} -- скользнуло в голове, и Элеонора мигом
просияла: она вдруг поняла, как завоевать внимание Агаты грядущим
вечером.

Относительно новой приятельницы её интересовали как минимум две вещи.
Во-первых, Элеонора никак не могла определиться с возрастом Агаты.
Вблизи та выглядела молодой, если не сказать юной: лицо и руки были
матовыми, нетронутую морщинами кожу нарушали лишь едва заметная морщина
на лбу и складка у левого уголка губ. Результат любимых жестов Агаты,
решила девушка, ухмылки и вздергивания бровей.

Однако, голос Агаты звучал довольно взросло, а иронично-саркастичные
нотки, граничащие с безразличием, только добавляли лет.

Тогда Элеонора решила записать всё то, что узнала об Агате за прошедшую
неделю. К тому же, с зажатой между пальцев ручкой девушке всегда
думалось гораздо проще.

\begin{quote}
\emph{Агата как-то упомянула, что она возобновила работу «Маятника»
почти шесть лет назад\ldots{} Или четыре? Я всегда путала эти числа.}
\end{quote}

Нора закусила губу. Ей также не составляло труда принять оранжевый цвет
за зеленый и наоборот, но к счастью это не имело отношения к делу.

Перечислив все, что она смогла вспомнить о хозяйке дома, Нора в
недоумении уставилась на свою записную книгу: подобные подвиги должны
занять лет эдак сорок.

«Окей, -- решила девушка. -- Оставим пока цифры».

В конце концов о возрасте Агаты она могла узнать и от Владислава, и хотя
на сегодняшний день контакт с последним казался немыслимым, Элеонора с
робкой надеждой верила, что однажды ей удастся провести вечер в компании
своего призрака, когда оба будут по одну и ту же сторону стены.

Куда больше Элеонору интересовало предназначение солнцезащитных очков на
лице подруги.

«Подруги?»

Она на мгновенье задумалась, вспомнив относительно недавний диалог с
хозяйкой дома.

В тот вечер девушка намеревалась позвонить домой, желая справиться о
делах, и в ходе разговора рассказать о пребывании в новом городе,
естественно, опустив все траурные подробности.

-- Маме приятно будет услышать, что я завела друга в свою первую неделю
во Львове, -- поделилась Нора.

Агата нахмурилась.

-- Я не завожу друзей, -- заметила та.

Девушка с недоверием взглянула на свою собеседницу.

-- А что насчет Владислава?

Привычная ухмылка.

-- Ласло был рядом задолго до того, как я узнала о предназначении дырки
между ног.

«По всей видимости, здесь не поспоришь», -- с грустью подумала Элеонора,
но вопреки этому начала все чаще воспринимать Агату как друга.

Она вернулась к записям.

\begin{quote}
\emph{Сначала я решила, что причина очков кроется в алкоголизме, но уже
следующим утром заметила, что Агата носит их и до того, как взяться за
выпивку. К тому же, у нее, похоже, не бывает похмелья.}

\emph{Слепота сразу отпадает: временами мы читали в гостиной по
очереди.}

\emph{Возможно, это астигматические очки, затемненные лишь с одной
стороны? В выборе цвета Агата не отличается особой оригинальностью, так
что я не удивилась бы, окажись моё предположение верным.}

\emph{Светобоязнь -- самая логичная причина, но ведь (даже если забить
на то, что она вообще не бодрствует в светлое время суток, а кстати,
почему?) я опять-таки видела Агату в очках чуть ли не в кромешной тьме.
Допустим, светобоязнь может быть симптомом какой-то другой болезни, а
предмет моих размышлений уже настолько сросся с очками, что снимает их
только перед сном?}

\emph{А, может, они наоборот -- с подсветкой?}
\end{quote}

Нора улыбнулась, довольная собой. Последний вариант уж точно ни к селу
ни к городу, но тот, что был до него действительно виделся девушке
наиболее реалистичным.

\begin{quote}
\emph{В таком случае, нет ничего дурного в том, чтобы поинтересоваться
предназначением очков.}
\end{quote}

На этой одушевляющей нотке девушка захлопнула блокнот и отправилась в
гостиную, дабы проверить, не появилась ли Агата. Нора провела там всего
несколько минут -- задержалась, чтобы поставить музыку и немного
повозилась с проигрывателем.

Агата ещё не проснулась, но по возвращении в спальню Элеонору ожидал
сюрприз. Записная книга, о которой девушка точно помнила, что
захлопнула, лежала там же, где её оставили, только вот открытая на
последней записи, к которой чужая рука бережно добавила лишь одно
предложение низким, закругленным почерком.

\begin{quote}
\emph{Не стоит это делать.}
\end{quote}

*

-- Не стоит это делать, -- медленно произнесла Элеонора, словно хотела
попробовать фразу на вкус.

Никаких подписей, объяснений, или иных инструкций; даже восклицательный
знак отсутствовал. Только «не стоит это делать». Под стать владелице
дома, Владислав умел быть таинственным.

«Тогда расскажи сам», -- чуть ниже написала Нора.

Девушка покинула пределы спальни, оставив блокнот открытым.

Она проголодалась и устала дожидаться Агату.

\section*{20}\label{20}
\addcontentsline{toc}{section}{20}

\markright{20}

-- Gnaghi! -- сказал Франческо Делламоре.

-- Шикарно! -- сказала Агата.

-- Есть такое, -- согласилась Элеонора.

-- Определенно мой любимый фильм, -- продолжала дама. -- Расставляет все
по полочкам, создавая при этом ещё больший бардак в голове.

-- Вообще-то, я надеялась, что ты его не видела.

Нора смущенно улыбнулась.

Ответом послужила широкая бровь, традиционно выглядывающая из-под очков.

-- Считаешь, вы похожи?

Агата не ответила.

-- Я имею ввиду, тебе действительно нравится жить на окраине кладбища?
-- добавила Нора. -- Не многим пришлось бы это по душе.

-- Тебе действительно нравиться пить чай без сахара? -- прозвучал
бесстрастный ответ.

Элеонора задумалась, после чего улыбнулась.

-- Я тебя поняла.

Агата кивнула.

*

Ожидания оказались напрасными: Владислав больше ни слова не сказал. Даже
на бумаге.

-- Как продвигается работа над книгой? -- внезапно спросила Агата.

Расположившись в своем кресле, она несколько часов просидела с книгой в
руках не проронив и слова, а потому успевшая забыть о её присутствии
Нора непроизвольно вздрогнула, услышав низкий потусторонний голос.

За это время девушке удалось выдавить из себя не более пяти страниц, и
хотя по сравнению со вчерашними результатами этот явно был достижением,
Нора осталась недовольна.

Ближе к ночи холода вновь решили напомнить о себе. Ветер скользил за
окном, лаская ветви деревьев, и те время от времени касались витражей.
Элеонора сидела в постели, бесцельно перебирая волосы: несмотря на то,
что часы уже оповестили о наступлении трёх ночи, спать не хотелось.
Параллельно девушка штудировала интернет, подыскивая достойный внимания
фильм. Она в который раз отметила про себя, как странно техника
сочетается с убранством её апартаментов.

Спустя минут десять таких поисков Элеонора услышала топот. Кто-то вновь
прогуливался по стенам. Шаги тем временем приближались, на сей раз звуча
гораздо четче: казалось их обладатель не пытался скрывать своего
присутствия. Остановились они у самого окна.

По ту сторону стены раздался стук.

-- Войдите? -- нерешительно произнесла Элеонора.

Как и в прошлый раз, силуэт возник из ниоткуда. Сначала показалась
бледная рука, затем плечо и, наконец, спрятавшееся за очками лицо: на
этот раз Нору навестила гостья, а не гость.

Перед ней выросла Агата. Поверх траурных одежд было накинуто пальто
соответствующего цвета.

-- Я вижу, ты не спишь, -- произнесла дама и указала на стену. --
Слышимость отличная, если находишься внутри.

-- Сна ни в одном глазу, -- только и нашлась Нора.

С её лица не сходило удивление, заметив которое, Агата объяснила:

-- Так быстрее.

-- Ты откуда, или куда? -- поинтересовалась девушка.

Агата сдвинула шляпку на затылок и застегнула пальто.

-- Собираюсь прогуляться, хочешь присоединиться?

Нора отложила телефон.

\section*{21}\label{21}
\addcontentsline{toc}{section}{21}

\markright{21}

К тому времени как дамы вышли на улицу ветер разгулялся не на шутку, и
вскоре Норе открылось, что её загадочная спутница по совместительству
является ругательницей высшего класса: шляпку Агаты несколько раз
уносило в ночь.

Элеонора же, будучи уроженкой городка, который она не стала бы называть
и при самых критических обстоятельствах, к шторму привыкла, а потому
шляпок не носила. В паре метров от входа в бюро находилась парочка
изредка работающих фонарей. На удивление, той ночью они были исправны.
Нора стояла на гранитной тропинке, дожидаясь пока Агата возьмет верх над
ветром в битве за любимый головной убор. Девушка чувствовала как по коже
гуляют ледяные порывы дыхания ночи; в глазах отражался свет уличных
фонарей, превращая их из зеленых в медовые -- поворачивать назад совсем
не хотелось.

Погода навеивала воспоминания о доме. Перед глазами всплывали ветреные
вечера: пустынные грязные улицы, объятые лишь серой чередой бесконечных
многоэтажек.

Нет, поворачивать назад совсем не хотелось.

-- Графиня в метаньях, -- сквозь пелену воспоминаний прозвучал голос
Агаты.

Судя по звуку, дама уже отвоевала шляпку и теперь находилась в паре
метров от Элеоноры, но вне площади озаренной желтым светом фонарей.

-- Что?

-- Спрашиваю, идёшь ли ты? Я здесь уже пару минут стою.

Нора старательно всмотрелась в темноту и, наконец, увидела собеседницу.
Последнюю выдавали лишь огоньки света, плавно отражающиеся в стёклах
очков.

-- Прости, -- ответила Нора. Она сделала пару шагов в направлении Агаты.
-- Ничерта не вижу. Ты гармонируешь с ночью.

Девушка не могла утверждать наверняка, но она буквально почувствовала,
как подруга ухмыльнулась.

-- И темнота вокруг неё шумела, словно прибой сумрачного моря, --
изрекла Агата.

-- Ремарк?

-- Ремарк.

Нора улыбнулась: высказывание её понравилось.

-- Вообще-то он говорил о дереве, -- добавила Агата, -- но кому какое
дело, когда речь заходит о поэтике.

-- Это вполне в твоём стиле.

Агата остановилась и взглянула на Нору, которая, наконец, начала
различать в темноте очертания предметов.

-- У меня есть стиль? -- озадаченно поинтересовалась Агата.

Нора бросила в её сторону удивленный взгляд.

-- Шутишь?

Дама покачала головой.

Элеоноре понадобилось секунд десять на размышления.

-- Мрачный винтаж, -- с уверенностью произнесла она.

-- Хм\ldots{} А мне нравится.

Агата расхохоталась.

-- Ещё бы, -- вторила присоединившаяся к ней девушка.

*

-- Не то чтобы я была сильно удивлена, -- спустя дюжину метров заметила
Элеонора, -- но где-то в глубине моей души все же таилась робкая надежда
на то, что мы погуляем вне кладбища.

Ответа не последовало.

-- Просто я по-прежнему едва ли вижу в темноте, так что, если у тебя в
очках не спрятаны фонари дальнего виденья, в чем я немножко сомневаюсь,
близок час моего падения в\ldots{}

-- Погоди, -- спокойно сказала Агата.

-- Да?

-- Просто погоди.

Нора непроизвольно закатила глаза.

-- Ладно. А куда идти-то?

Агата приоткрыла рот и тут же клацнула языком. Так, словно собиралась
указать девушке нужное направление, но в последний момент передумала.
Нора, в свою очередь, без особого труда догадалась, какой пункт
назначения ей хотели предложить, а посему решила повременить с
расспросами.

Удалось ей это так себе.

-- На днях я размышляла\ldots{}

-- Это полезно. Продолжай в том же духе.

Филантроп-Элеонора с лёгкостью проигнорировала этот ответ, как поступала
и со всеми последующими колкими фразочками, слетевшими с уст владелицы
похоронного бюро.

-- \ldots я размышляла и заметила странную штуку\ldots{}

«Далеко не одну, -- подумала Нора, -- но, говорят, надо начинать с
мелочей».

-- Только одну? -- словно прочитав её мысли, переспросила Агата, и Нора
сказала собеседнице то же, что говорила себе минуту назад.

-- Я слушаю.

-- Как тебе удается заниматься прощальными церемониями, если те
традиционно проходят в дополуденное время, когда ты видишь десятый сон?
-- на одном дыхании спросила девушка.

-- С большим трудом и нечеловеческими усилиями.

Нора с недоверием взглянула на Агату.

-- Ты ведь не оставляешь родственников усопшего предоставленными самим
себе?

-- Было пару раз, -- последовал ответ. -- На самом деле, я уже месяцев
пять не занималась работой. Потому и подала объявление о сдаче этажа.

-- Ясно.

Шли они с четверть часа.

Ветер продолжал играть с одеждой и волосами. Вчерашний снег обратился
тонким льдом, создавая для девушки особенно травматические условия.
Ориентироваться в пространстве было непросто. Приложив усилия, Нора
заключила, что они сделали крюк вокруг бюро и теперь продвигались вглубь
кладбища.

-- Ещё один поворот, -- Агата нарушила тишину неожиданно жизнерадостным
голосом.

Девушка с радостью преодолела этот самый поворот.

Открывшийся вид заставил Элеонору остановиться и медленно выдохнуть
ночной воздух. Перед ней простиралась аллея склепов, усеянная множеством
белых огоньков. Первой мыслью, пришедшей в голову девушки были
светлячки. Однако, откуда им взяться в разгар декабря, да еще и на
Западе Украины?

-- Солнечные лампы, -- догадалась Нора.

С их помощью кладбище выглядело очаровательным и совсем не страшным.
Каменные, увенчанные соснами аллеи скорее походили на задний дворик
какого-нибудь загородного дома, нежели на обитель мёртвых. Для получения
полной картины, добавьте к этому всеобъемлющую тишину, нарушаемую лишь
мерным шепотом веток.

-- Выглядит -- не побоюсь этого слова -- уютно, -- зачарованным голосом
произнесла девушка. -- Это ты сделала?

Агата согласно кивнула головой.

-- Что же, должна признать на этот раз ты меня действительно поразила!

-- Я сделала это и для себя, но в первую очередь для Ласло. В канун
прошлого нового года этот любитель безлюдных мест подвернул ногу вон у
того склепа и еще часа четыре провел в сугробе.

Агата извлекла из внутреннего кармана пальто портсигар, (в белом свете
фонариков Нора увидела надпись: «Восемь из десяти курильщиков умирают.
Двое оставшихся, видимо, живут вечно») и подкурила. Нора проделала то же
самое.

Со свойственной ей меланхолией, Агата склонилась над одним из склепов и
принялась поджигать толстую свечу, стоящую на постаменте.

-- Зачем ты это делаешь? -- удивилась Нора.

-- Родственники усопших регулярно просят меня зажигать свечи на могилах
их близких. Считают, что таким образом духи мёртвых находят покой.

-- И ты с ними согласна?

-- Мне за это платят, -- просто ответила Агата. -- К тому же, процесс не
занимает много времени.

-- А у меня он вот едва ли вызывает доверие.

Агата откинула с лица прядь волос и направилась к следующему склепу.

-- Полноте, -- сказала она. -- Индонезийцы вот вообще любители
подвешивать на скалах гробы со свеженькими покойниками, а затем дружно
отмечать это д ело.

-- Это еще для чего?

-- Ну, они верят, что горы являются проводниками между земным и небесным
мирами, или что-то вроде того. А вот младенцев они хоронят в дуплах, но
только тех, у кого еще не прорезались зубы.

Лицо Норы выражало сомнения, смешанные с удивлением.

-- Только не говори, что дерево здесь выступает проводником душ этих
несчастных\ldots{}

-- Именно.

Девушка хмыкнула. Она подошла к соседнему склепу и вскоре зажгла
тамошнюю свечу.

-- А ты сама что об этом думаешь? -- поинтересовалась Элеонора.

-- Слишком долгий выйдет рассказ, -- произнесла Агата, не отрываясь от
дела. -- Но детей я в дуплах не прячу, можешь быть спокойна на этот
счет.

-- У меня есть время.

Какое-то мгновенье Агата смотрела на свою спутницу, вероятно, раздумывая
над её вопросом. Затем дама согласно кивнула.

-- Мне хочется думать, что умирая мы не исчезаем бесследно, но начинаем
новую жизнь, всё ещё овеянные привкусом предыдущей. Сами того не
подозревая, круг за кругом как Роланд\footnote{Роланд Дискейн --
  основной персонаж «Темной башни», цикла романов Стивена Кинга.} идём к
своей Башне, преследуя схожие цели. Пускай каждый раз новый, он, всё же,
зависит от предыдущего, что может проявляться в мельчайших деталях.

К примеру, мне представляется как в конце века в мои руки совершенно
случайно попадет бутылка вина, сделанного некой Агатой Рахманиновой в
далёком двух тысячи пятнадцатом. Наслаждаясь совершенным вкусом напитка,
я буду удивляться тому, что человек, живший и усопший столетия назад,
обладал алкогольными предпочтениями, идентичными моим. Быть может, с
первым глотком меня настигнет дежавю, но чувство это будет столь
мимолётным, что ощутить я его смогу лишь глубоко в подсознании.

Возможно, распитие такого вина натолкнет меня на верную мысль, или
заставит совершить поступок, который в будущем приблизит меня к цели, а,
возможно, тем вечером мне просто удастся славно надраться -- оба
варианта более чем приемлемы.

Или, допустим, ты. Вообрази, как однажды будешь держать книгу, над
которой сейчас работаешь, и осознавать неописуемое родство с автором. Со
мной такое часто случается.

Ещё мне хочется верить, что за пределами одной жизни мы снова и снова
встречаем своих людей. Понимаешь, тех самых, кто ускользнул в этом
варианте бытия, но это уже совсем банальные мысли.

Нора слушала с широко распахнутыми глазами.

-- Это хорошие мысли, -- подытожила она, когда монолог Агаты подошел к
концу.

Агата улыбнулась, но как-то невесело.

Норе тут же вспомнилась строка из стихотворения любимого поэта её
сожительницы, которую последняя озвучивала по поводу и без: «Луна
улыбалась, но мне показалась улыбка её неживой»\footnote{Цитата из
  стихотворения Э.А. По под названием «Вечерняя звезда».}.

-- Так хочет думать моё сердце. Однако, -- Агата коснулась виска
указательным пальцем, -- всё находится здесь. Это досталось мне вместе с
интеллектом.

-- Что всё?

Агата вновь улыбнулась. На этот раз гораздо мягче.

-- Атеизм.

*

До рассвета было не менее трёх часов. Погода по-прежнему оставляла
желать лучшего, когда наши героини достигли конца аллеи склепов.

Агата зажгла последнюю свечу и плотнее запахнула полы пальто. Она
взглянула на свою спутницу.

-- Пожалуй, пора возвращаться, -- с готовностью произнесла Нора, стуча
зубами.

-- Пожалуй, пора, потому что я продрогла настолько, что сейчас с
радостью бы поменялась местами с Ярославом Олеговичем, -- заметила дама.

-- С кем?

Агата молча указала на склеп, расположенный за спиной её собеседницы, и,
обернувшись, Нора прочла надпись:

Сплавинский

Ярослав Олегович

1836 -- 1917

\section*{22}\label{22}
\addcontentsline{toc}{section}{22}

\markright{22}

Агата оказалась не единственным художником в доме -- рисование являлось
ещё одном талантом Элеоноры. В своих портретах ей удавалось удивительно
сочетать реализм с абстракционизмом, и, как и каждый художник, девушка
уделяла особое внимание работе над изображением глаз.

Нора запаслась кофейно-ромовым коктейлем, к которому её приучила Агата и
теперь сидела на крыльце, периодически кутаясь в плед. На дворе стояло
позднее утро вторника. С неба просеивался мелкий дождик. Хотя над
кладбищем царил хмурый штиль, Нора чувствовала, что грозы этой ночью не
миновать, и была совсем не против -- осенне-зимняя свежесть ей
нравилась. Но не так сильно, как горячий коктейль, конечно.

Нора сделала глоток бодрящего напитка и слегка улыбнулась -- возможность
напиваться с утреца была ещё очередной славной особенностью львовской
жизни: планировка города явно не располагала к вождению автомобиля, от
чего девушка, проведшая за рулем последние три года, (очень трезвые три
года) получала особое наслаждение.

На коленях покоилась новая записная книга, что разнилась от предыдущей
лишь увеличенным форматом: правая рука Норы уже около часа бесцельно
вращала карандаш. Девушка призвала на помощь всё, что только смогла
отыскать в своей светлой головушке: воображение, интуицию и прекрасную
память на лица.

Безрезультатно. Лежащий перед ней портрет оставался незавершенным. Не
хватало лишь глаз, в остальном сходство с оригиналом было безупречным:
шелковые локоны, которым в перспективе предстояло принять цвет молочного
шоколада, ниспадали с невысокого лба и струились далеко за пределы
рисунка; темные брови правильной формы, (левая, как всегда, более
изогнута, над правой виднелась родинка) под вздернутым носом
располагался аккуратный рот. Сложенные бантиком губы терялись в
сравнении с кончиком носа, посему первые их обладательница подводила
темной помадой. Схожим образом Агата старалась выделить практически
отсутствующие скулы. Всё это Норе удалось передать с завидной точностью.

Но глаза\ldots{}

Их девушка не смогла увидеть даже в своем воображении.

Когда время близилось к трём после полудня, а попытки изобразить свою
таинственную сожительницу Элеонора уже давно променяла на занятия
графоманией, которая тем днем продвигалась как нельзя кстати, на
противоположном конце аллеи, ведущей к дому, показалась фигура. Ею
оказался плохо одетый мальчишка двенадцати, или тринадцати лет от роду.
Тощий, облаченный в потускневшую фуфайку и затёртые до дых брюки,
(которые, к тому же, выглядели размеров на пять больше нужного) он
приветливо помахал рукой. Поверх копны грязных волос не хватало лишь
соломенной шляпы -- ну точно персонаж, сошедший со страниц романов Марка
Твена.

Элеонора не могла похвастаться опытом общения с детьми, но при
вступлении в диалог с братьями нашими меньшими придерживалась всего двух
верных правил: всегда улыбаться и никогда не сюсюкаться.

-- Цієї ночі трафилось дещо важливе\footnote{Этой ночью произошло
  кое-что важное.}, -- заявил мальчик, опуская приветствия.

Он достиг крыльца и держался так, словно знал девушку не первый год.

-- Хорошее, или напротив? -- поинтересовалась Нора.

Малец усмехнулся и клацнул языком, напоминая хозяйку «Маятника».

-- Та хіба ж буваюсь в світі речі, вийнятково гарні чи злі?\footnote{Да
  разве бывают в мире вещи, исключительно хорошие или плохие?} -- с
задумчивостью древнегреческого философа произнес он.

Нора полагала, что бывают, но спорить не стала: самоуверенность ребенка
забавно разбавляла хмурый день.

-- У кия два кінці,\footnote{Палка о двух концах.} -- мальчишеская
улыбка продемонстрировала зубы заядлого курильщика. -- Зрештою, як не
живеш, а труни не минеш, та й годі!\footnote{Как не живешь, гроба не
  избежишь, и ладно!}

Девушка помолчала, не зная, чем ответствовать этому потоку народной
мудрости.

-- А жіночка в чорному ще почиває?\footnote{А женщина в чёрном еще
  отдыхает?} -- наконец, осведомился юный философ.

Элеонора мельком глянула на часы.

-- Можешь быть уверен.

Малец в свою очередь внимательно оглядел сидящую перед ним девушку.

-- Кола під очима, розтягнутий светр, байдужість до макіяжу та ще й
зошит на колінах\ldots{} Ти, напевно, письменниця!\footnote{Круги под
  глазами, растянутый свитер, безразличие к макияжу, да еще и тетрадь на
  коленках\ldots{} Ты, должно быть, писатель?}

Элеонора смущенно улыбнулась.

-- Але з тих, хто тільки починає. Та не дивуйся ти так! Сором'язливість
тебе видає: сама ще не знаєш, чи гідна цим займатися, -- лицо мальчика
растаяло в довольной улыбке. -- Я файно знаюся в людях.\footnote{Но из
  тех, кто только начинает. Да не удивляйся ты так! Застенчивость тебя
  выдает: сама еще не уверена, достойна ли этим заниматься. Я хорошо
  разбираюсь в людях.}

-- Багато часу проводиш у суспільстві?\footnote{Много времени проводишь
  в обществе?} -- Нора и сама не заметила, как переключилась на
украинский.

-- Бач чого захотіла! -- ребенок оказался хохотуном в десятом поколении.
-- Господь милував -- багацько читаю.\footnote{Ишь чего захотела! Бог
  миловал -- много читаю.}

-- Так я до чого хилю, -- продолжил тот по окончании театральной паузы.
-- Може даси мені якогось свого зуба?\footnote{Так я к чему веду. Может,
  дашь мне какой-нибудь из своих зубов?}

-- Даруй?\footnote{Прости?}

-- Ну, зуба, чи щось таке.

-- Нащо воно тобі?\footnote{Зачем тебе это?}

-- На випадок успіху твоїх творів, звісно! Якось одна мізковита жіночка
вилучила двадцять дві тищи за різець Джона Леннона. І це у євро! Так-то
справи й робляться!\footnote{На случай успеха твоих произведений,
  конечно! Как-то одна умная женщинка получила двадцать две тысячи за
  резец Джона Леннона. И это в евро! Так-то дела и делаются!}

Нора залилась смехом.

-- Ну то як, ми домовилися?\footnote{Ну так что, мы договорились?} --
спросил малец с лукавой улыбкой, что ни на минуту не сходила с его лица.

-- Я подумаю, -- уклончиво ответила Нора.

Этому её научила Агата, сама того не ведая.

Малец пожал плечами.

-- Піймав не піймав, а поганятися можна.\footnote{Попытка не пытка.}

Он расстегнул фуфайку и принялся исследовать содержимое карманов, в
следствии чего девушка увидела леденец, надкушенный коржик, горсть
монет, коробок спичек, моток ржавой проволоки и странного рода предмет,
походивший на железную рогатку о четырёх концах.

«Он же не собирается торговаться за мой зуб?» -- с сомненьем подумала
Элеонора.

Тем временем юный предприниматель нашел что искал, и протянул девушке
квадратный конверт бордового цвета.

-- Передаси це жінці в чорному.\footnote{Передашь это женщине в чёрном.}

Нора осмотрела предмет: на лицевой стороне конверта красовалось знакомое
изображение маятника, на обратной лермонтовским почерком вывели:

\emph{Агате Р.}

\emph{Отправитель: Валерий С.}

Завершивший свою миссию приверженец английских традиций уже удалялся из
поля зрения.

-- Постой! -- крикнула ему вслед Нора. -- Что мне сказать Агате?

Малец ответил не оборачиваясь.

-- Скажи, що заскочив Гек. Вона зрозуміє.\footnote{Скажи, Гек забегал.
  Она поймет.}

«Гек, -- улыбнулась Нора. -- И как я сама не догадалась?»

*

Вечер не задался с первого акта.

Часы пробили начало девятого. Вошедшая в гостиную комнату Агата
сообщила, что не выспалась. Нора же, в свою очередь, впервые увидела
приятельницу в домашней одежде -- помимо очков на Агате виднелся лишь
халат, само собой черный; волосы растрепались до невозможности и
непослушными прядями вились вокруг лица, аля Дженис Джоплин.

Дама прикрыла рукой зевок и плюхнулась в кресло.

-- Ты проспала больше четырнадцати часов, -- в недоумении произнесла
Элеонора.

-- У меня нет проблем с простейшими вычислениями, -- сухо заметила Агата
и, помедлив, добавила: -- Чего нельзя сказать о сне.

-- Это уж точно.

-- Не сплю сутками, чтобы затем спать сутками, а, проснувшись, вновь не
спать сутками\ldots{} Винишка ради, спрячь этот жалостливый взгляд!

Нора повиновалась.

-- И так было всегда? -- спросила она.

-- Нет. Но другой жизни я не помню.

У девушки сложилось впечатление, что ещё немного и Агата вновь обратится
так- себе-собеседницей, так что она поспешно подкурила и с превеликим
интересом взялась изучать собственные ногти.

Агата потягивала излюбленный напиток и наблюдала за этой картиной со
свойственным ей безразличием.

Выждав несколько минут, Нора решила сменить тему.

-- До Нового года менее двух недель, -- начала она. -- Может, пора
заняться украшением дома? Я могла бы хоть завтра съездить за ёлкой, если
ты не против.

-- Я против.

Элеонора нахмурилась.

-- Ну чего ты? -- произнесла девушка как можно мягче.

-- Не люблю праздники. Их приближение слишком переоценено и зачастую
предзнаменует одни грядущие неприятности, так что никаких ёлок, но в
подсобке найдётся парочка сосновых гробов. Разрешаю тебе нарядить их.

Нора бросила недоверчивый взгляд в сторону говорившей, но та то ли
проигнорировала, то ли и вовсе не заметила этот жест.

-- Шутишь?

-- Вовсе нет. Подсобка слева от лестницы первого этажа, -- спокойно
ответила Агата.

Нора вздохнула, всеми силами стараясь не изображать Роберта Дауни
младшего.\footnote{Т.е. не закатывать глаз.}

-- Я говорю о твоем отношении к праздникам, -- произнесла она, про себя
добавив: «саркастичная ты бестолочь», а затем: -- Тебе не кажется, что
здесь слишком много предрассудков?

Агата покачала головой.

-- В общем-то, нет.

-- Послушай, -- не сдавалась Нора. -- Позволь мне обустроить все? Хотя
бы в гостиной и холе. Обещаю, проблем не будет!

Агата не ответила. Её внимание привлёк конверт, лежащий на кофейном
столике.

-- Ах да, Гек заходил, -- вспомнила Нора, проследив за взглядом подруги.
-- Странный малый\ldots{}

Агата приподнялась и нехотя потянулась за посланием. Лицо её помрачнело
еще до того как конверт был вскрыт.

-- Понеслась, -- невесело бросила Агата.

\section*{23}\label{23}
\addcontentsline{toc}{section}{23}

\markright{23}

Элеонора распахнула глаза задолго до официального наступления утра.
Пространство вокруг поглотила привычная тьма, под воздействием которой в
комнатах девушки, несмотря на их богатое убранство, царило лишь два
цвета: черный и серый. От этой мысли сразу делалось как-то одиноко.

Она лежала на правом боку и разглядывала свои ноги, гадая почему часы
перед рассветом -- самые холодные в сутках. Говорят, именно это время
является для гробовщиков наиболее плодотворным потому, что большинство
потенциальных покойников умирает незадолго до восхода солнца. «Отходят»
-- так принято говорить в больницах, постояльцы которых прекрасно
осведомлены о вырисовывающейся статистике, отчего не скрывают своего
страха перед предрассветными часами. Многие не хотят оставаться в
одиночестве и башляют последние, предназначавшиеся для поездки в
солнечную Испанию или обустройства загородного дома сбережения ночным
сиделкам.

Часы в гостиной пробили четыре утра. Эхо разнеслось по всему этажу, если
не зданию. Девушка подумала об умирающих. Сотни тысяч одиноких душ,
разбросанных по всему миру, которые бессонно лежали в своих постелях,
готовясь расстаться с жизнью.

Нора выдохнула и медленно подняла голову, словно только пробудилась ото
сна. Она лениво направилась в душ и, прежде чем включить воду, замерла,
прислушиваясь к стенам, надеясь услышать шаги, или звуки музыки. Однако,
немногие постояльцы «Маятника» еще спали в своих постелях. Тишину не
нарушало ничего, кроме разыгрывающейся на улице вьюги.

Душ не занял много времени. По крайней мере, Норе хотелось на это
надеяться.

Нужно было выдвигаться. День обещал быть тяжелым и, судя по всему, очень
долгим.

*

Дорога, по которой она шла, казалась девушке чьим-то неудачным
наброском. Солнце по-прежнему таилось за горизонтом, посему и голые
кроны деревьев, объятые инеем, и сугробы с выглядывающими из них
надгробиями, и даже бесконечное количество снежинок, что кружили вокруг
-- все было серым. Элеоноре на мгновенье подумалось, что она попала в
старый черно-белый снимок. Один из тех, которые за гривну можно
приобрести в любой букинистической лавке. Только никто бы не стал
покупать фотографию кладбищенской дороги, кроме, разве что, Агаты. Ради
нее-то Нора и поднялась в такую рань.

На половине пути Элеонора обернулась. Похоронное бюро чёрнело в конце
тропы, сливаясь с темнотой, но отнюдь не это привлекло внимание нашей
героини. Девушка ожидала увидеть собственные следы, оставленные на
снежном покрове. Их не было. Снежный поток бесцеремонно скрыл улики еще
до того, как Нора успела достигнуть нужного места. Теперь все выглядело
так, словно она никогда не была в «Маятнике», не входила в него минувшим
вечером, не сидела на крыльце со своим альбомом в руках и не выходила
этим утром. Да и самого здания, судя по открывающемуся виду, там никогда
не было. Конечно, Нора оставалась в своем уме и прекрасно знала, что
«Маятник» на месте, дожидается наступления рассвета, дабы выплыть из
темноты и возвыситься над нарядившемся в белое некрополем, как Башня
Роланда возвышается над полем роз.

Тем не менее, девушку не покидали странные мысли. Нора продолжила свой
путь и теперь складывалось ощущение, что она не идет, а парит над
землей, едва ли касаясь снежного покрова. В голове пронеслись старые
строки, повествующие одновременно о печальной и комической ситуации,
вырванные из «Морфия».

\begin{quote}
\emph{Какая пустыня. Ни звука ни шороха\ldots{} И вот я вижу, от речки
по склону летит ко мне быстро, и ножками не перебирает под своей пёстрой
юбкой колоколом, старушонка с желтыми волосами\ldots{} В первую минуту я
её не понял и даже не испугался. Старушонка как старушонка. Странно --
почему на холоде старушонка простоволосая и в одной кофточке?.. А потом:
откуда старушонка? Какая?.. на десять верст кругом -- никого. Туманцы,
болотца, леса! А потом вдруг пот холодный потёк у меня по спине --
понял! Старушонка не бежит, а именно летит, не касаясь земли. Хорошо? Но
не это вырвало у меня крик, а то, что в руках у старушонка -- вилы.}
\end{quote}

Нора не то чтобы призирала наркоманов, но искренне надеялась, что сцена
эта является вымышленной, хотя в глубине души знала, что это не так:
творчество Михаила Афанасьевича всегда оставалось более чем
автобиографичным.

Она осознала, что пытается разглядеть снежинку, застывшую на кончике
носа, и ускорила шаг.

Временами пропитанный недостатком сна разум выдает крайне странные
мысли. Поразительный долбоебизм.

*

Мальчишка, назвавшийся Геком, поджидал девушку сразу за поворотом. На
нем была всё та же фуфайка, на этот раз застегнутая. Примостившись на
могильной плите известного украинского поэта, малец вальяжно курил
самокрутку. Исходя из его внешнего вида, парнишка едва ли мог позволить
себе покупку табака. Нора бросила взгляд на самодельную сигарету, гадая,
что же скрывается за газетным обрывком.

Она вспомнила, как лет эдак тринадцать назад соседские дети мастерили
нечто подобное. Желая казаться взрослыми, они пытались курить сено,
чайные и лавровые листы\ldots{} Словом, все, что смогли разыскать дома,
а один парнишка, чьё имя не стоит упоминать во избежание конфликта,
клялся, что курил лобковые волосы своего деда, (у самого-то они ещё не
выросли) которые срезал поздним вечером, когда последний задремал у
телевизора. Дурбецало этот божился, что на вкус его самокрутка
неотличима от табака. Спорить никто не стал, но и пробовать тоже.

Элеонора ещё раз пристально оглядела самокрутку Гека и решила не
развивать эту мысль.

-- Здоровенькі були\footnote{Приветствие.}, -- промямлил Гек. Из-за
зажатой в зубах самокрутки это прозвучало как «Сдовенькы буы».

-- И тебе не хворать, -- машинально ответила Нора.

Рядом с легко одетым мальчиком она -- закутанная в чёрную шубку и как
минимум три шарфа -- чувствовала себя гребанной матрешкой. Ещё обиднее
оказался тот факт, что несмотря на все старания Элеонора по-прежнему
ощущала холод.

Для поездки на трамвае час выдался слишком ранним. Гек, явившийся в
некрополь пешком, заявил, что на дорогу им потребуется около двух часов.
То же время уйдет на ожидание трамвая, который вряд ли появится раньше
рассвета. Нора взглянула на мальчика: от длительного пребывания на
холоде на его щеках расцвели розы, а само лицо, казалось, вот-вот
покроется инеем.

-- Почему бы нам просто не взять такси?

Гек просиял искренней детской улыбкой, открывавшей миру умилительные
ямочки на щеках, и признался, что никогда не ездил на такси.

-- Точніше, було крапаль, та я не певен, чи то можна взагалі рахувати,
адже я не мав грошей, а тому вилетів в автівки на останньому світлофорі
та згинув у невідомому напрямку.\footnote{Точнее, немного было, но я не
  уверен, можно ли это вообще учитывать, ведь у меня не было денег, а
  потому я вылетел из автомобиля на последнем светофоре, чтобы исчезнуть
  в неизвестном направлении.}

Это предложение малец выпалил на одном дыхании. Он тут же закашлялся.
Норе понадобилось приложить усилия, дабы подавить желание оповестить
своего спутника о вреде курения. Во-первых, в этом не было нужды, так
как Гек (как и любой здравомыслящий человек) был осведомлен о
последствиях курения табака -- если это вообще был табак -- чуть более,
чем полностью. Во-вторых, Элеонора, начавшая курить в пятнадцать, и сама
успела пригубить парочку сигарет с момента пробуждения. В конце концов,
какая разница, во сколько начинать курить: годом раньше, двумя годами
позже -- раковые палочки всегда остаются раковыми палочками, и раз уж
речь зашла об этом, то лучше и вовсе не курить. Но люди любят то, что их
убивает.

-- Стівен Кінг писав, що табак може бути корисним для сформувавшихся
легенів\footnote{Стивен Кинг писал, что табак может быть полезен для
  сформировавшихся легких.}, -- произнес мальчик, заметив как Элеонора
поглядывает на его самокрутку.

Девушка подняла бровь и с интересом перевела взгляд на собеседника.

-- Ты читал «Тёмную Башню»?

Лицо Гека сделалось таким серьезным, словно он собирался заявить, что не
просто прочитал, но и собственноручно написал этот цикл. Вместо этого
малец лишь многозначительно кивнул.

-- Я багацько читав. У Валерія.\footnote{Я много читал. У Валерия.}

*

Валерий был крёстным отцом Агаты. Неофициально, разумеется, ибо
последнюю никогда не крестили. Он держал магазинчик антикварных вещичек
под названием «Ретроспектива», войдя в который, Элеонора мигом
сообразила, откуда в похоронном бюро все эти пластинки, патефоны и
многое другое добро, включающее в себя также серебряные готические
подсвечники и первое полное русскоязычное собрание работ Хемингуэя.

Внутри лавки яблоку негде было упасть. Она представляла собой три до
блеска вычищенные комнаты, доверху заставленные товарами на любой вкус и
цвет, начав перечисление которых, рискуешь сыграть в «Гаргантюа и
Пантагрюэль».\footnote{Сатирический роман в пяти книгах французского
  писателя Франсуа Рабле, в котором поразительно много лишних деталей, а
  перечисление таких вещей как содержимое банкетного стола может
  занимать далеко не одну страницу.} Пахло здесь так, как пахнет старый
чердак уютным августовским вечером.

Протискиваясь вслед за своим юным проводником промеж стеклянных витрин и
книжных стендов, Нора то и дело останавливалась, завидев очередную
заворожившую её вещь.

-- Хутчіш!\footnote{Поторапливайся!} -- бросил через плечо Гек с
нетерпением, свойственным всем детям. -- Матимеш ще гору часу на
екскурсію.\footnote{У тебя еще будет море времени на экскурсию}

Нора в последний раз взглянула на удивительную печатную машинку средины
прошлого века, стараясь запомнить, где та лежит, и шагнула вслед за
мальчишкой в дальнюю комнату.

Из-за прикрытой дубовой двери доносилась бодрая на первый взгляд
мелодия, которую девушка незамедлительно узнала. Это был Дел Шеннон, чей
лирический герой гулял под дождем и думал всякие думы. Преимущественно
невесёлые.

Преимущественно о любви.

Отворив дверь, Элеонора немало удивилась, обнаружив, что в комнате нет
никого кроме неё, Гека и вот этого орущего паренька с пластинки.

Перехватив её взгляд, малец загадочно улыбнулся. Так, словно они с Норой
хранили какой-то до ужаса важный секрет.

-- Та це я кімнати обжив і поставив платівку незадовго до відправлення
на зустріч із тобою. Не хотів, аби крамниця зустріла тебе похмурим
обличчям, бо тобі, мабуть, і «Маятника» вистачає, -- Элеонора с
облегчением отметила, что сегодня её новый друг не сыпал афоризмами. --
До того ж, мені і самому якось моторошно лишаться тут
наодинці.\footnote{Так это я комнаты обжил и пластинку поставил
  незадолго до отправления на встречу с тобой. Не хотел, чтобы лавка
  встретила тебя хмурым лицом, ведь тебе, наверное, и «Маятника»
  хватает. К тому же мне и самому как-то жутковато оставаться тут в
  одиночестве.}

-- Розумію\footnote{Понимаю.}, -- кивнула она и продолжила уже на
русском: -- Ты здесь живешь?

-- Раніше іноді перебивався, а тепер так, певно що живу. Взагалі-то, це
дядін кабінет\footnote{Раньше иногда перебивался, а теперь да, наверное,
  живу. Вообще-то, это кабинет дяди.}, -- последнее утверждение Гек
произнес с едва заметным смущением.

Движением руки он предложил Элеоноре присесть, а сам устремился в
дальний конец комнаты, обещая вернуться с домашним горячим шоколадом,
равного которому не сыщешь на всем белом свете. Ожидая такси, девушка
изрядно подмёрзла и горло уже начинало побаливать.

Однако, лицо таксиста, завидевшего молодых людей, карабкающихся по
забору закрытого кладбища в половине пятого утра и грациозно летящих в
направлении сугроба\ldots{} -- оно того стоило.

Конечно, не только это привело нашу героиню в «Ретроспективу».

Нора ощутила резкий перепад температур и поспешила избавиться от верхней
одежды.

Дальняя комната служила Валерию квартирой, которая хоть и являла
просторное помещение, совмещающее спальню, кабинет и кухню, выглядела
практически пустой, по сравнению с предыдущими залами. Высокие окна с
деревянной обделкой открывали вид на отрывок старого города.

Элеонора умостилась за широким письменным столом. Тот был вплотную
приставлен к северной стене, вдоль которой имелась полка, доверху
заставленная фотографиями в рамках, соседствующими с различной
дребеденью. Снимки виднелись и под полкой, булавками приколотые к чёрным
обоям. Девушка обернулась: Гек возился с чайниками у комода напротив
окна, которое само по себе походило на тускнеющий снимок.

Нора откинула с лица пепельно-черные волосы и с нескрываемым интересом
принялась разглядывать фотографии.

*

На многих снимках была Агата. Без очков и улыбающаяся если не во все
зубы, то хотя бы вполовину -- Нора не сразу её узнала. Что-то в этой
улыбке показалось девушке знакмым, но ощущение это промелькнуло так
быстро и неясно, как чувство дежавю во сне.

«В конце концов, все подвыпившие улыбаются одинаково», -- подумала Нора
и напрочь забыла об этом.

Даже на снимках, запечатлевших Агату в более поздний период (Нора тут же
заметила шляпу с широкими полями и траурные одежды, столь характерные
для ее сожительницы) дамочка не носила очков и Нора смогла разглядеть
тёмно карие, практически черные глаза, края которых были по-кошачьи
изогнуты. Наша героиня сочла этот взгляд слишком очаровательным, дабы
скрывать его за зеркальной перегородкой, но решила повременить с
расспросами.

Если и имелось время для поднятия таковой темы, это определённо было не
оно.

Размышления прервал голос мальчишки.

-- Вуаля! -- радостно воскликнул Гек, протягивая дымящуюся кружку.

Нора тут же приложилась к напитку. Шоколад оказался слишком редким и
скорей походил на неумело приготовленное какао. Мальчик, заподозрив
неладное, с надеждой взглянул на новообращённого дегустатора.

-- Очень вкусно, -- улыбнулась Нора.

Гек ответил незамедлительно, словно ожидал появления этой фразы.
Элеонора подумала, что именно так дела и обстояли.

-- Дядя мене навчив. Агата каже, що то лайно собаче, проте дядя запевнив
колись, що я можу й краще. Тож я практикуюсь. Іноді.\footnote{Дядя меня
  научил. Агата говорит, что это дерьмо собачье, но дядя как-то убедил,
  что я могу и лучше. Так что я практикуюсь. Иногда.}

Гек указал на один из снимков, которые девушка и без того продолжала
рассматривать, сделанных, судя по всему, в каком-то баре. В центре
композиции (на фоне барной стойки, за которой Агата что-то усердно
смешивала в французском шейкере) улыбался мужчина средних лет. Как
несправедливо бывает в отношении сильного пола, возраст лишь придавал
ему привлекательности: высокие скулы, копна непослушных волос и
невероятно яркие синие глаза первым делом приковывали взгляд. Морщинок
-- всего ничего. Парочка тёмных локонов падала на высокий лоб. Немалого
роста и крепкого телосложения, без малейших признаков обвисшего брюшка.

«Прямо-таки герой советского кино», -- мелькнула мысль в голове девушки.

-- Дяді тут п'ятьдесят шість,\footnote{Дяде тут пятьдесят шесть.} --
сообщил Гек.

Элеонора круто вскинула брови и дважды моргнула, недоверчиво, но в то же
время заинтересованно, поглядывая на собеседника.

-- Сколько?!

Малец, вероятно, давно уже привыкший к подобной реакции, спокойно
закончил мысль.

-- Саме тоді ми й познайомились. Ну, не в барі, звісно.\footnote{Как раз
  тогда мы и познакомились. Ну, не в баре, конечно.}

\section*{24}\label{24}
\addcontentsline{toc}{section}{24}

\markright{24}

Гек, чье настоящее имя, как выяснилось, было Дармедонт (Нора дважды
сдавлено хихикнула и одарила мальчика извиняющимся взглядом) оказался
большим любителем вести повествование. Делал он это совершенно
непринужденно и, можно даже сказать, беспристрастно. Так, словно
пересказывал сюжет хорошо знакомого фильма, а не отрывок собственной
жизни. Элеоноре это даже нравилось: по своей натуре она была
сентиментальной девушкой и даже реклама растворимого кофе порой могла
вызвать слезы на прекрасном аристократическом личике. В эти минуты Норе
думалось о том, как же хорошо, что сожительница её оказалась куда менее
эмоциональной дамочкой, ибо двум таким нюням пришлось бы несладко в
одной гостиной.

Уже позже ей открылось, что Агата, которая при первом, а также парочке
последующих взглядов напоминает закоренелого дудаиста, по сути куда
более эмоциональна. Нет, она не ревела над сопливыми романтическими
комедиями и никогда не умилялась детишкам, но поразительная
чувствительность владелицы «Маятника» проявлялась в куда более странных
вещах. Бывало, в момент разговора, или же просто занимаясь каким-нибудь
привычным делом, вроде чтения, Агата вдруг застывала и, не шевелясь,
смотрела куда-то перед собой. В такие моменты выражение лица дамы было
олицетворением тоски, а наличие тонированных очков делало такую картину
ещё более непонятной. Если в момент подобного происшествия в руке Агаты
была зажата сигарета она могла спокойно догореть, обжигая при этом
пальцы владелицы, да так, что та ничего и не замечала.

Случались и другие ситуации, когда эмоциональная палитра Агаты состояла
лишь из злости и всех оттенков раздражения в сторону вещей, которые, по
мнению нашей героини, вовсе не стоили таких нервов. Агата была чертовски
свободолюбива и даже такая ерунда как разлинованные блокноты была
способна тут же вызвать ярость на лице, казалось бы, безразличной ко
всему особы, особенно смущало Элеонору: привычные плавные движения
контрастировали с резкими нотками в голосе, что угрожали обратиться
неконтролируемыми приступами ярости, вызванными чёрт его знает чем.
Тогда Агата казалась девушке воображаемой героиней какой-нибудь
экспрессивной пьесы и Нора даже ловила себя на мысли, что буквально
может предсказать следующее действие своей подруги.

Однако, все это имело место быть позже.

Сейчас же Элеонора сидела за столом дальней комнатки «Ретроспективы» и,
подложив ладони под голову, с неподдельным интересом слушала рассказ
Гека.

*

Дело было не в том, что у Гека не было семьи.

Хотя, ладно, дело было именно в этом, но не в том смысле, в котором
общество привыкло воспринимать эту фразу. Малец ни разу не был в приюте
и, до поры до времени, не скитался холодными улицами в поисках ночлежки.
Фактически он был частью полноценной семьи среднего достатка -- по
крайней мере так значилось в документах. Отец зарабатывал на жизнь тем,
что по шестнадцать часов в день торчал в местном филиале одной
невхеренной канадской it-компании, (которая отказывалась облагать
налогом своих сотрудников, создавая последним немало хлопот) где
довольствовался почетной должностью офисного планктона и безуспешно
пытался избавиться от красноречивого акцента во время мастерски
отрепетированных шаблонных ответов на вопросы обескураженных клиентов.
Любимыми выражениями папаши Гека были «Please clarify your
question»\footnote{Пожалуйста, уточните свой запрос.(англ.)} и «Have you
tried to turn off and turn on?»\footnote{Вы пробовали включить и
  выключить? (англ.)}, хотя последнее всё ещё давалось ему с трудом.

Дело свое мужчина любил, пускай оно и состояло из беспрекословного
выполнения обезьяньей работы, обсуждения политики с коллегами и, конечно
же, того, чтобы вставлять please\footnote{Пожалуйста (англ.)} и
kindly\footnote{Будьте добры (англ.)} в каждую сводобную дырку диалога с
арабами, которые ждали, что за десять баксов в месяц теподдержка вдоль и
поперек будет вылизывать им задницы.

На самом деле рабочий день был девятичасовым, но за
оувертаймы\footnote{Overtime (англ.) -- переработка.} приплачивали в
полтора, а по праздникам (вы, блядь, не поверите!) -- в два раза больше.
Лаврентий не выходил на работу лишь по воскресеньям и в день Рождества
Христова. Причиной последнего послужил тот случай, когда в разгар этого
самого праздника мужчина попытался отчаянно впустить Иисуса в сердце не
менее радикально настроенного клиента-мусульманина.

Итак, всё бы ничего, если бы оставшееся от лишенной смысла работы время
Лаврентий проводил не с божьим сыном, а со своим собственным, но Геку
удавалось поболтать с отцом от силы пару раз в неделю. К тому же,
подобные разговоры всегда сводились к одной и той же фразе: «А как бы на
твоем месте поступил Иисус?»

В четыре года Гек понял, что Иисуса отец любит премного больше, чем его
самого. В шесть пообещал себе, что непременно разобьет нос этому
бородатому всеми любимому мужику в платье, -- было это задолго до
явления миру Кончитты Вурст, вы не подумайте -- ну а в восемь\ldots{} На
девятом году жизни Гек осознал, что бороться за отцовскую любовь в
принципе бессмысленно, а уж тем более с мифическим зомби, так что ранним
весенним утром он, как всегда, ушел в школу чтобы уже никогда не
вернуться в отчий дом.

Мать Гека была той еще штучкой. Днем она скиталась по улицам, уговаривая
прохожих протестировать на себе пробники убогой косметической марки,
представителем которой она являлась, а по вечерам продолжала
распространять эту идею в интернете.

У мальчика также был старший брат, о котором достаточно сказать, что
лучше бы его не было.

*

В календаре наступило третье марта, но погода и не думала сворачивать на
весну. Дождь лил без устали уже вторые сутки, делая улицы не только
грязными, но и пустынными. Сумерки в такие дни наступали рано. Мальчик
знал это, а потому заставил шагать быстрее свои непривыкшие к подобным
нагрузкам ноги. В пути он провёл уже около тридцати двух часов, из
которых спал всего шесть в одном заброшенном сарае, удачно встретившемся
посреди голого поля. Теперь же нехватка сна и здоровой еды с лихвой
сказывалась на растущем организме, и Гек чувствовал не просто усталость:
шагая, он ощущал каждую ноющую мышцу. Болели не только ноги, руки
почему-то сделались слишком тяжелыми и, свисая вдоль тела, заставляли
мальчика упасть на холодную землю. Время от времени тот закладывал их за
голову, но и это не могло помочь надолго. Спина также давала знать об
изнуренности и буквально кричала мольбами о помощи, которые наш герой
так умело игнорировал. Он был готов к чему угодно, -- сколиозу, бронхиту
и даже мифическому гепатиту, которым так любили пугать в школе, -- но
только не возвратиться домой.

К началу второго дня пути Гек решил обратить всё в игру. Ему хотелось
знать, сколько может пройти его тело без каких-либо остановок. Так
мальчик приноровился выполнять на ходу многие вещи: есть, пить,
просматривать карты и, конечно, справлять малую нужду. Последнее он
находил особенно забавным, если не забывал повернуться к ветру спиной.

Малец также делал на ходу различные упражнения -- все, что только мог
припомнить с занятий физической культурой и это помогало, но,
опять-таки, лишь на какое-то время. Однако, хуже всего дела обстояли с
шеей. Казалось, что её несколько раз обернули вокруг оси.

Несмотря на физическую изнеможенность, морально Гек был на высоте.
Происходящее казалось первой совершенной им вещью, пускай и сложной, но,
наконец-то, имевшей смысл. Когда серое небо принялось тускнеть ещё
сильнее, извещая о приближении темноты, мальчик понял, что едва ли
успеет отыскать для ночлега новое заброшенное сооружение. За прошедшие
часы пейзаж перед глазами едва ли изменился: пустынная дорога полна
грязи и выбоин, в которых уже успело накопиться немало дождевой воды;
вдоль обеих сторон трассы тянутся бесконечные ряды тополей. Утратившие
листву деревья выглядели мертвыми. Ветер то и дело заставлял их ветки
тянуться вниз, создавая тем самым весьма удручающую картину. Далее были
поля. Холодные голые поля и ни одного, черт бы его побрал, сооружения.

Тогда Гек представлял, как однажды напишет рассказ, а может и целую
повесть, о парнишке, который день за днем вынужден идти только вперед,
не имея возможности остановиться, -- это было до того, как Гек всерьез
увлекся литературой и с удивлением обнаружил, что такая книга существует
уже не одно десятилетие.

Темнота накрыла мальчика в считанные минуты, но, дождавшись, пока глаза
к ней привыкнут, Гек заставил себя прошагать ещё полтора часа.
Периодически ему попадались мелкие тропинки, уходящие в глубь поля.
Возможно, на конце одной из них мальчика и ждала деревушка с парочкой
заброшенных хат, но с трассы края тропинок видно не было, к тому же, наш
герой уже твердо решил не сходить с главной дороги.

Взошла луна и дождь вдруг резко прекратился. В то же время ветер задул
сильнее, заиграл тощими ветвями и загудел среди пустошей. Пришлось
признать, что сегодня ночевать доведется под открытым небом. Гек
вздохнул, сошёл с трассы и, сделав пару неуверенных шагов в темноту,
вдруг провалился в кювет. Он здорово ушиб локоть и разревелся. Не
столько от боли, сколько от усталости и беспощадного осознания того, что
он один в целом мире. Конечно, малец понимал это и прежде, но узнать
разницу между понятиями «одинок» и «один» можно лишь прочувствовав
последнее.

Ручейки слез продолжали испещрять его лицо, когда, шмыгнув носом, Гек
осознал, что в кювете ветер чувствуется не так сильно. Он вытер рукавом
нос, подложил под голову рюкзак и тут же забылся тревожным сном.

\section*{25}\label{25}
\addcontentsline{toc}{section}{25}

\markright{25}

Очнулся Гек от собственного кашля. До рассвета было далеко, так что он
позволил себе ещё какое-то время полежать, разглядывая ветви деревьев,
возвышающихся на фоне ночного неба. Потянувшись, Гек сморщился от резкой
боли, пронявшей спину и нехотя сел. Во рту раскинулась настоящая Сахара,
но жажда уступала место голоду. Он вывалил содержимое пищевого
контейнера на колени и с жадностью принялся поглощать бутерброды. Дышать
было как-то тяжело, и наш герой с горечью осознал, что нос его заложен
со вчерашнего вечера.

Проблема восьмилетних мальчиков, устраивающих побеги из дому, по большей
части заключается в том, что они понятия не имеют, как устроен быт, так
что их познания в данном вопросе более чем отличаются от реального
положения вещей. Гек был смышлен и, ввиду своего вынужденного
одиночества, гораздо самостоятельнее сверстников, однако, он по-прежнему
оставался ребёнком и многое не мог предусмотреть. Время, проведённое в
компании телевизора, сказывалось на нём, в общем-то, положительно:
мальчик знал о насилии больше, чем стоило, а потому, собираясь в путь,
не забыл и о средствах самообороны. Оружия в их доме отродясь не
водилось, как и газовых баллончиков, но, пораскинув мозгами, Гек пришел
к выводу, что последнее с лёгкостью можно заменить. Теперь в боковом
кармане его рюкзака хранился новенький флакон лака для волос. Мать
разозлится, заметив пропажу, -- тут Гек не сомневался, как не сомневался
и в том, что исчезновение лака мать заметит гораздо раньше, чем
отсутствие его самого -- но оно того стоило.

Так вот, Гек был смышленым малым. Он прихватил с собой компас, карты,
фонарик со сменной батареей, кое-какую одежду и прочие полезные мелочи,
которые по его (и, в принципе, правильному) мнению могли понадобиться в
пути, но мальчик не позаботился о главном. Он совершенно забыл про
дождевой плащ, но, что ещё хуже -- не вспомнил о лекарствах. Оно и
понятно: стоило Геку приболеть, таблетки, сиропы и порошки от кашля
заведомо ждали в аптечке над раковиной. Мальчику не приходилось покупать
их самому, так что в последний момент он попросту не вспомнил о
лекарствах.

Наличные, что Гек имел при себе, были его собственными сбережениями за
последние несколько лет, и представляли собой суму, лишь немного
превышающую среднюю заработную плату. Этого хватило бы на оплату пары
дней в отеле и покупку билета на поезд в оба конца, но Гек понимал, что
денег у него всего ничего, а потому надо их беречь. К тому же, без
документов билет бы ему не продали, а рисковать мальчик не хотел.

К обеду пейзаж по обе стороны дороги начал меняться. Поля смешались с
лесопосадками, поначалу редкими, а затем более широкими. Кое-где вдали
начинали проступать домики. Сверившись с картами и своими записями,
мальчик решил, что, если не будет останавливаться, достигнет областного
центра с наступлением следующего рассвета. В городе он купит еды, отыщет
автовокзал и обзаведётся билетом на утренний автобус в любой конец
страны. Конечно, Гек мог провернуть это в своем собственном городишке и
в таком случае уже был бы в тепле и далеко от грязной трассы, но кое-что
здесь, всё же, останавливало. Он понимал, что спустя пару дней полиция
начнет его искать, то есть опрашивать горожан, и был уверен, что
окажется единственным восьмилетним мальчиком, покупающим билет на первый
автобус. Его описание узнают и вскоре отыщут даже в новом городе.

А больше всего на свете Гек не хотел быть пойманным. Сперва он решил
незаметно покинуть район, в котором жил, и удалиться как можно дальше от
города. Даже в этой глуши по мокрой от дождя дороге иногда проезжали
машины. Завидев их приближение, мальчик тут же прятался в овраге или
среди деревьев. Дом он покинул в школьной форме и надетой поверх неё
куртке, которые вскоре спрятал в одной из полевых ям для компоста, и
переоделся в широкую толстовку с капюшоном и заношенную фуфайку. Эти
вещи он нашёл в гараже. Отец не носил их уже много лет, так что вряд ли
станет искать, отчего шансы быть неузнанным у мальчика увеличивались.

Около семи вечера Гек сошел с трассы: вдали показался свет от чьих-то
фар. Мальчик зашёлся ревущим кашлем, пытаясь углубиться в посадку и
параллельно надеясь, что в автомобиле работает радио. Наш герой
справился с кашлем и теперь тяжело дышал, пытаясь взять себя в руки. Его
трясло уже второй час. Голова кружилась, а ноги путались на ходу,
периодически увлекая лицо своего владельца прямиком в лужи. Гек взялся
за ствол какого-то широкого дерева и попытался вернуться на дорогу. Его
усилия были напрасными. С трассой его разделяло каких-то двадцать
метров, но преодолевал их Гек бесконечно долго, чтобы в итоге
обнаружить, что так и не сдвинулся с места. Определенно, за этот день
дела его сделались гораздо хуже. Котелок совсем не варил, но страшнее
всего было другое -- толстяк в костюме клоуна.

Клоун преследовал его с наступления темноты. Точнее сказать было сложно:
мальчик совершенно потерялся во времени. Минуты казались ему часами, а
часы вдруг проскальзывали как пара секунд. Иногда мальчик оглядывался по
сторонам и не мог вспомнить, как попал сюда. Некоторые отрезки времени
просто исчезали. Но клоун вот не спешил исчезнуть, он держался на
небольшом расстоянии от мальчика, словно выжидая, пока тот упадет без
сил.

Впервые Гек заметил силуэт среди деревьев, когда остановился, пытаясь
прочистить нос, и вдруг обнаружил, что весь мокрый. Холода больше не
чувствовалось, только жара -- невыносимая жара и жажда теплились в
голове мальчика. Он поднял тяжелые веки и обратил взгляд в сторону
деревьев, пытаясь понять, есть ли ветер. Деревья медленно покачивались,
оставляя за собой расплывающиеся следы. Воздух перед глазами сгустился и
поплыл, дышать было невозможно. В голове мальчика пронеслась отдалённая
мысль о том, что он под водой. И все это под водой. Холод, темнота и
бесконечная влага только подтверждали эту мысль.

Мальчику сделалось страшно. Он обхватил себя руками, но сразу опустил
их, изнывая от жары. По щекам потекли слезы и тут же смешались с
ручейками пота, заполнившими всё лицо.

ПЛЫВИ.

Голос в голове. Чужой голос говорил ему, что дышать вовсе и не нужно.
\emph{Под водой никто не дышит, глупенький.}

ПЛЫВИ.

МЫ ВСЕ ЗДЕСЬ ПЛАВАЕМ. СКОРО И ТЫ ПОПЛЫВЕШЬ.

Гек замер. Последняя фраза окончательно вогнала его в ступор. Поплывешь?
Он сказал «поплывешь»? Или он сказал\ldots{} \emph{Полетишь?}

Шея болела пуще прежнего. Мальчик осознал, что всё ещё смотрит на
верхушки деревьев и медленно опустил голову. В нескольких шагах от него
стояла высокая фигура, в руках которой были\ldots{}

«Не шарики, пожалуйста, только не\ldots»

Да, это были воздушные шарики. Шарики, которые совсем не раскачивались
на ветру. Фигура ослабила хватку и парочку шаров устремились в небо.

Гек вскрикнул.

Вернее, попытался, но его измученное горло издало лишь хриплый стон.

Фигура сделала шаг по направлению к мальчику. Ещё несколько шаров
вылетело из рук существа.

МЫ ВСЕ ЗДЕСЬ ЛЕТАЕМ.

Гек никогда не слышал этого голоса, но узнал бы его где угодно. Он знал
эти шарики. Он знал эту фразу лучше чего бы то ни было. В темноте сложно
было разглядеть детали внешности этого существа, но Геку не особо-то и
хотелось. Он знал, кто это, знал его имя. Пеннивайз\footnote{Танцующий
  клоун Пеннивайз -- главный антагонист в романе Стивена Кинга «Оно»,
  зло, являющееся нашему миру в различных образах, основным из которых
  является вид ужасного клоуна с охапкой ярких шаров в руках.} --
единственный детский страх, так никогда и не покидавший Гека.

По всей видимости, не напрасно.

Клоун вытянул вперед руку, отпуская в небо оставшиеся шары и сделал шаг
в сторону мальчика. Тот широко раскрыл глаза и, издав новый скрипучий
стон, бросился прочь. Он выскочил в центр дороги и бездумно несся
вперед.

Но не долго.

Ватные ноги запутались, и Гек нырнул лицом в одну из глубоких луж,
раскинувшихся посреди трассы. Боли от падения он совсем не почувствовал,
зато было кое-что другое. От ледяной воды его мысли мгновенно
прояснились. Жар слегка сошел и мальчик растерянно обернулся.

Клоуна нигде не было. Ещё бы, его и не могло здесь быть: персонажи
Стивена Кинга не преследуют маленьких мальчиков на пустынных трассах. По
крайней мере, вне штата Мэн.\footnote{Именно там разворачивается
  действие большинства историй автора.}

Гек на удивление быстро забыл о странном происшествии, как это всегда
случается с дурными снами, и продолжил свой путь, мечтая о чашке -- нет,
целом чайнике -- горячего чая.

Увы, просветление длилось недолго.

Пеннивайз преследовал его, вновь и вновь появляясь в разных местах. Вот
он бредёт позади мальчика, а вот вдруг бледная рука в клоунском
облачении появляется из темноты всего в паре метров от нашего героя. Гек
не понимал, как существу удаётся перемещаться с такой скоростью, но
времени на подобные размышления у него всё равно не было. На улице вновь
сделалось холодно, да так холодно, что никакой чай бы не помог. Геку
было страшно и невыносимо плохо. Он не помнил, когда в последний раз его
так лихорадило.

И вот теперь, спустя неизвестное ему количество времени, Гек стоял во
тьме, держась за ствол мокрого дерева и силясь вернуться на трассу. Он
пробыл здесь так долго, что вдалеке замелькали фары следующей машины.
Гек тяжело вздохнул и беспомощно опустился наземь.

Секундой позже мальчик удивленно вскинул брови и улыбнулся. Из зарослей
по другую сторону появилась собака. Мохнатая и крохотная, невероятно
походившая на плюшевую игрушку, она была щенком лайки. Тем самым, о
котором всегда так мечтал Гек. В свете луны белоснежная шерстка отливала
синевой, придавая щенку ещё большее сходство с игрушкой. Полные доверия
глазки-пуговки смотрели на Гека с восторгом. Мальчик не мог поверить
своему счастью. Если следом не покажутся хозяева лайки, он сможет
оставить щенка себе и тогда они будут вместе путешествовать по стране.

Гек знал, что сможет следить за собакой. Он мечтал о ней с пяти лет и
успел за это время прочесть уйму сопутствующей информации. Удивительно,
что именно сейчас удача повернулась к нему лицом, ведь деньги, которые
мальчику пришлось взять с собой, изначально предназначались именно для
приобретения пушистого друга.

Его размышления прервал глухой звук удара. Улыбка застыла на лице Гека,
медленно искажаясь и обращаясь гримасой боли. Он закричал. На этот раз
по-настоящему. К тому моменту щенок уже успел добраться до середины
дороги, когда пролетевший мимо автомобиль размазал по асфальту его
крохотное тельце.

Мальчик закрыл лицо руками и зарыдал. Хуже быть просто не могло.

Но было хуже.

Пеннивайз вернулся. Он вышел из темноты и опустился на колени рядом с
тем местом, где ещё мгновение назад, виляя хвостом, стоял щенок. Сейчас
же там виднелись только пятна крови и разорванные внутренности, что
казались черными под покровом ночи. Теперь Гек имел возможность
рассмотреть своего преследователя. Стоило мальчику подумать, что улыбка
на лице существа выглядит как-то иначе, как клоун тут же её снял.
Вернее, снял то, что оказалось небольшой маской, скрывающей часть лица,
и мальчик понял, что у клоуна нет нижней челюсти. На ёе месте виднелись
едва ли зажившие кровавые ошмётки.

Тем временем существо старательно соскребло с земли остатки бедного
животного и с неумолимой легкостью принялось поглощать добычу. В страхе
и отвращении, наш герой попытался отползти назад, но тело его не
слушалось. Оставалось только следить за ужасной картиной.

Клоун покончил с трапезой. Он поднял голову и резко повернул её в ту
сторону, где прятался мальчик. Затем сделал то, что парой секунд позже
Гек интерпретировал как самую страшную улыбку на всем белом свете.
Рот\ldots{} Вернее сказать, то, что от него осталось, не шевелился, но
Гек услышал, как из его собственной головы звучит голос существа.

МЫ ВСЕ ЗДЕСЬ ЛЕТАЕМ.\footnote{«Мы все здесь летаем», -- известная фраза
  вышеупомянутого персонажа.}

ОН ПОЛЕТЕЛ. Я ЧУВСТВУЮ, КАК ОН ЛЕТИТ.

ТЫ ТОЖЕ ПОЛЕТИШЬ.

В глазах у Гека потемнело. Тело мальчика упало на прелую листву. Он
потерял сознание.

\section*{26}\label{26}
\addcontentsline{toc}{section}{26}

\markright{26}

После этих слов последовала драматическая пауза. Приличия ради Элеонора
выждала пару секунд, после чего как бы невзначай взглянула на часы.
Особой надобности в этом не было. Девушка и так знала, что опаздывает.

-- Впечатляющая история\ldots{} -- заметила она.

-- Це ще навіть й не четвертушка!\footnote{И это ещё даже не четвертая
  часть!} -- тут же парировал Гек.

-- \ldots{} но я не думала, что она настолько длинная. Я с удовольствием
послушаю, или прочту продолжение, но, боюсь, Агата уже ждёт меня.

Мальчик слегка нахмурился. Казалось, рассказ слишком увлёк его, позволяя
на время забыть о реальности.

-- Ты ведь уже записал это? -- тут же спросила Нора, стараясь напомнить
мальцу о его писательских порывах.

Тот покачал головой.

Гек убрал со стола пустые чашки и выдвинул верхний ящик. Его содержимое
оказалось потрепанной картонной папкой. Мальчик взглянул на неё лишь на
мгновенье, но этого хватило, дабы Нора заметила бесконечную скорбь на
лице Гека, которую тот так умело пытался скрыть.

-- Тут я зібрав усі документи та долучив декілька карток на випадок,
якщо Агата загубила свої.\footnote{Тут я собрал все документы и добавил
  несколько фотографий на случай, если Агата потеряла свои.}

Элеонора кивнула.

-- Я приєднаюсь до вас разом з усіма, а потім розповім тобі ще частину
своєї історії, якщо захочеш.\footnote{Я присоединюсь к вам вместе со
  всеми, а потом расскажу тебе ещё часть своей истории, если захочешь.}

-- Конечно.

Девушка отложила папку и принялась надевать верхнюю одежду.

-- Скажи только, -- произнесла она, шагая в сторону выхода, -- у тебя в
ту ночь прыгала температура?

Малец ухмыльнулся, тут же сделавшись прежним Геком.

-- Ще б пак! -- гордо сказал он. -- Падала до тридцяти п'яти, а потім
стрибала до сорока и знову те саме. Бачила колись таке?\footnote{Еще бы!
  Падала до тридцати пяти, а потом прыгала до срока и снова то же самое.
  Видала когда-нибудь такое?}

\section*{27}\label{27}
\addcontentsline{toc}{section}{27}

\markright{27}

Дабы подготовиться к церемонии, Агата поднялась на рассвете, как и
обещала. Возможно, что и раньше: вернувшись домой, Элеонора видела её из
окна своей спальни, плывущую средь могильных плит. Фигуру Агаты окутывал
туман, скрывая края платья -- столь густым слоем он стелился над землёй.

Наступившим утром хозяйка «Маятника» выглядела совсем скверно:
ссутулившаяся и обхватившая себя руками, она держалась неуверенно, хотя
была в одиночестве. Глядя на эту походку, Нора задалась вопросом: спала
ли сегодня её подруга?

Девушка принялась укладывать волосы, изредка поглядывая на кладбище, где
Агата бесцельно бродила взад-вперед, периодически исчезая в тумане. Нора
поняла, что все ещё не согрелась и отправилась принять горячий душ, а
когда вернулась в спальню траурная дамочка скрылась за пределами
открывающегося из неё вида. Девушка простояла у окна ещё пару минут,
надеясь, что та покажется, после чего пожала плечами.

-- Наверное, ушла вглубь кладбища, -- произнесла Нора, обращаясь к самой
себе.

Она взяла с подоконника чашки, оставшиеся после минувшего вечера. Пора
заняться завтраком. Обернувшись, Нора вздрогнула; посуда чудом осталась
в руках. У изголовья её кровати неподвижно сидела Агата. Закинув ногу на
ногу, она спокойно смотрела на девушку.

-- Давно здесь сидишь? -- только и нашлась Элеонора.

-- Не то чтобы очень.

Агата не шелохнулась. Сотканная из бледной кожи, в тусклом свете торшера
она напоминала восковую фигуру.

-- А ты давно за мной наблюдаешь?

-- Не то чтобы очень.

Завтрак получился американским -- яичница с беконом, большую часть
которой Агата отправила в мусорное ведро. Как девушка и подозревала,
прошлой ночью её компаньонка не сомкнула глаз и теперь один за другим
опустошала бокалы с таким рвением, словно пьянство могло компенсировать
отсутствие сна.

Расправившись с завтраком, Нора бегло осмотрела комнату и, не обнаружив
никаких следов присутствия Владислава, нахмурилась: находится в тишине
больше не было сил. Она буквально ощущала облако скорби, тяготившее над
Агатой.

В итоге девушка остановила внимание на портрете, украшавшем
противоположную стену. Нора указала на личико Николая Васильевича.

-- Не твоих ли рук дело?

Агата кивнула не глядя.

-- А что насчет Гинзберга?

Еще один кивок.

-- Булгаков, Есенин, Бродский, Гоголь, Гинзберг, Берроуз, Томпсон, Уэлш,
Буковски, Кинг, Хилл, Стокер, Коллинз, По, Лавкрафт, Хемингуэй, Ремарк
-- всего девятнадцать, среди которых три Эдгара, -- произнесла Агата
измученным голосом.

-- Я знала, что они одного авторства! -- с запалом вскрикнула Нора,
которая всегда хотела разбираться в искусстве. -- Эти мрачные контрасты,
гротескная подача с вкраплениями сюрреализма\ldots{} Много времени ушло
на эту серию?

-- Лет семь.

Элеонора не оставляла попытки разговорить Агату, тускнеющую на глазах.

-- Ты изображаешь только любимых авторов?

-- Ты наблюдательна.

Никакой ухмылки.

-- И кто на очереди?

-- Кристи. Пишу её уже четвертый месяц, но выходит слишком много деталей
и это раздражает.

Следующая фраза Элеонора стала внезапной даже для неё самой.

-- Хотелось бы мне оказаться в этом списке.

-- Заверши рукопись, -- серьёзно напомнила Агата.

-- Я о каждой так говорю, но эту непременно закончу. Ещё немного и
выйдет две третьих.

Агата одобрительно кивнула.

Или просто кивнула. Кто её разберет.

*

Стрелка часов медленно ползла в сторону восьмерки. Этой ночью владельцы
похоронного бюро постарались на славу: внизу всё было готово ещё до
восхода солнца. Элеонора не знала, каким образом её соседям удалось
уговорить представителей морга доставить тело глухой ночью, но
подозревала, что инициатива исходила от Агаты.

Девушке представилось ночное кладбище, каким, благодаря Агата, она
прекрасно помнила это место; теряющиеся в тумане силуэты, несущие гроб
по направлению к «Маятнику»\ldots{} Или в гроб покойного клали уже
здесь? В любом случае, картинка вырисовывалась не из самых
воодушевляющих.

За неполный месяц, проведенный в «Маятнике», Нора ни разу не осматривала
первый этаж, (не считая крестовых походов в винный погреб) видимо, не
располагая подходящим траурным настроением. Этим утром ей представилась
такая возможность.

Со слов Агаты девушке было известно, что правое крыло занимали кабинеты
Рахманиновой и Мирославского; левое же было отведено под зал прощаний.
Туда Нора и отправилась первым делом.

Зал прощаний представлял собой огромное помещение, по сравнению с
которым терялся простор гостиных комнат. Убранное в красно-золотые тона
оно больше походило на холл какого-нибудь театра: потолки подпирали
мраморные горгульи, со всех сторон виднелись выступающие образы львов.
Помимо настенной иллюминации, здесь имелись старинные колыхающиеся
люстры со свечами вместо ламп -- они-то и придавали комнате сакральное
очарование, создавая атмосферу уединения. Высокие витиеватые подсвечники
возвышались и вдоль всего зала.

Две третьих помещения занимали бархатные стулья, (какие прежде Элеонора
видела лишь во дворцах-музеях) насчитывалось их более двух сотен. У
дальней стены стоял гроб с усопшим. Тени от свечей мягко падали на лицо
Валерия, придавая его чертам умиротворение. За свою жизнь девушка видела
покойников раз пять, но впервые подошла так близко, не чувствуя при этом
и толики страха -- с атмосферой Агата не прогадала.

-- Словно он прилег отдохнуть, -- глухо произнесла Агата. Она вновь
бесшумно выплыла из-за спины девушки. -- Хотя так, пожалуй, говорят обо
всех покойниках.

-- Он будто улыбается\ldots{}

-- Видишь две эти тусклые линии? Вдоль скулы и у края губ. Их сложно
заметить в желтом свете, но всё дело в тенях.

-- Твоих рук дело?

-- Освещение -- да, однако для взаимодействия с тенями холст должен быть
идеален, а это уже заботы Ласло.

-- Вот как\ldots{}

-- Знаешь, они никогда не встречались, -- туманно произнесла Агата, - по
крайней мере, в нормальном понимании этого слова, но в своём письме
Крёстный просил заняться\ldots{} привести его в порядок именно в
«Маятнике».

Нора внимательно оглядела свою собеседницу, раздумывая, стоит ли задать
следующий вопрос.

-- Раз уж речь зашла об этом, могу я задать ещё пару вопросов? --
решилась она.

-- Почему нет?

-- С твоих слов и не скажешь, что Валерий был при смерти. Он умер от
патологии?

-- Нет. Тромб оборвался. Это было внезапно для всех, включая его сердце,
так что, к счастью, Крёстный ничего не заметил.

-- А письмо?

Агата улыбнулась.

-- Он написал его, сидя в моем кабинете лет эдак пять назад. Тем днём мы
начинали подготовку к открытию бюро. Утром привезли фирменные конверты,
одним из которых Валерий и воспользовался. Скорее шутки ради, но письмо
он, всё же, сохранил. Вместе с этим, -- она указала на безымянный палец
левой руки, на котором переливался чёрно-синий агат. -- Вообще, в
отношении смерти Крёстный владел уникальной философией --
вудиалленовской -- предпочитал не иметь ложных надежд и воспринимал уход
как нечто более приятное. Возможность сократить свои расходы, например.

Нора хохотнула и озадачено взглянула на Агату, испугавшись собственного
невежества, но та лишь одобрительно кивнула.

-- Всё так, -- произнесла дама в чёрном. -- Он предлагал смеяться над
всем, и смерть не стала исключением. В последней строке своего
прощального письма крёстный спрашивает, а не называет ли мой друг,
Владислав, трупиков пёсиками?

Девушка в недоумении подняла брови.

-- Потому что у них холодные носики, -- объяснила Агата.

Нора вновь рассмеялась.

-- А он называет? -- отдышавшись, уточнила девушка.

Уголок рта хозяйки «Маятника» дёрнулся.

-- Не исключено.

\section*{28}\label{28}
\addcontentsline{toc}{section}{28}

\markright{28}

Очередной проигрыватель расположился у входа в церемониальную комнату. В
отличие от гостиных, пластинок здесь было куда меньше.

Пришедшие проститься с Валерием горожане начинали понемногу
подтягиваться, и пускай подавляющее большинство последних облачилось в
чёрное, ни один из них не мог тягаться с Агатой в количестве излучаемого
траура.

Наши героини стояли по обе стороны дверной арки, помогая гостям
рассаживаться. Воздух переполняли цветочные запахи, излучаемые букетами,
что были уложены вокруг гроба. С каждой минутой их становилось всё
больше. Вскоре цветы уже заполнили треть помещения, и Нора -- едва ли
счастливая обладательница аллергии, набирающей обороты с приходом тепла,
-- возблагодарила Валерия за решение покинуть этот бренный мир туманным
декабрьским утром.

Тогда девушка и подумать не могла, чем обернётся сегодняшняя церемония.

Гости тем временем всё прибывали. «Утром, в полдень» зал впитал свыше
четырёх сотен душ, что вдвое превышало количество стульев. Некоторые
выражали хозяйке «Маятника» свои соболезнования, но большинство
явившихся попросту не замечало Агату, (что её более чем устраивало) они
сбивались в кучки и шумно продвигались в конец зала, не забывая при этом
жалостливо всхлипывать.

Мальчишка по прозвищу Гек тоже был здесь. Нора видела как тот
прикладывал к лицу платок, насквозь промокший от слёз. Заприметив
девушку, Гек учтиво кивнул -- от бывалой смешливости не осталось и
следа.

(Элеонору это не удивило: боль утраты зачастую чувствуется не сразу,
уступая дорогу шоку, к тому же, несмотря на свою самостоятельность, Гек
по-прежнему оставался ребёнком.)

Девушка ответила тем же.

Элеонора слыла натурой впечатлительной, а потому обычно избегала
присутствия на похоронах, или свадьбах, находя эти события одинаково
печальными. Вот и сейчас в её горле поселился мерзкий ком, а на глазах
проступили слёзы.

-- Полагаю, время начинать, -- бесстрастно произнесла Агата. Её взгляд
упал на Нору. -- Даже не думай. От слёз твой тонкий аристократический
носик превратится в деревенское рыло.

Что есть, то есть.

Нора попыталась выдавить из себя улыбку. Кое-как удалось.

-- Как это у вас происходит? -- тихо спросила она.

-- Владислав, как и копатели, не церемонии не присутствует. Он тепло
относился к Валерию, но Ласло как-никак таксидермист-визажист, о чем
гостям хорошо известно. Людям больно видеть очередное напоминание о
смерти их близкого, а Владу это только на руку, потому церемонию я
провожу одна, но Ласло\ldots{}

-- Он всегда рядом, -- Нора провела пальцами по двойной стене.

-- Именно.

-- И с чего начинается церемония?

Агата ответила не сразу. Она уже сосредоточенно рылась в пластинках.
Тонкие пальцы, наконец, выхватили нужную вещь, и дама вернулась к
беседе.

-- Процесс до неприличия банален: я ставлю тоскливую мелодию -- никаких
«Аве Марий» в моём доме! -- красивую лирическую мелодию, приглушаю свет
и на протяжении последующего часа эти окаянные, -- Агата махнула рукой в
сторону зала, -- ревут, содрогаясь в обоюдных конвульсиях, периодически
сдабривая это дело фразочками вроде: «На кого ж ти мене
залиииишииииив?!»\footnote{Кому же ты меня оставил?!}

Не считая чарующего убранства зала, картина в точности повторяла
похоронные церемонии, на которых Норе всё же довелось присутствовать в
своём родном городе. Однако, от Агаты она ожидала большего.

-- А дальше?

-- А дальше покойника выносят во двор и придают земле, тем самым
сопутствуя процессам разложения.

Лицо Элеоноры в удивлении подалось назад, от чего на шее мелькнула тень
от несуществующего второго подбородка.

-- Не может быть, -- озадаченно произнесла Нора и, предвидя намечающийся
рассказ Агаты о четырёх стадиях естественного разложения трупа,
поспешила добавить: -- Но кто-нибудь же произносит речи?

Агата вздохнула.

-- Ты набралась этого из зарубежных фильмов, -- заявила она и поднесла к
виску указательный палец. -- Вот где прощаются с близкими и произносят
речи. Сюда же приходят исключительно с целью выплакаться.

-- Ты так спокойно это воспринимаешь\ldots{}

-- О нет, -- она резко выставила вперед правую руку, -- всхлипывания
меня раздражают. Обычно я стою у самого проигрывателя, пытаясь заглушить
их.

Агата поставила пластинку, и зал окунулся в отчаянно прекрасную мелодию,
к счастью оказавшуюся не первородным пост-панком, от которого девушка
уже изрядно подустала.

Она узнала мелодию.

-- Думаешь, это подходящая песня? -- с сомнением протянула Нора.

Агата вновь не спешила с ответом: она принялась задергивать шторы.

-- Композиция, конечно, замечательная\ldots{}

-- Естественно.

-- \ldots но, мне кажется, её лучше слушать\ldots{} когда всё живы.

Адам Хёрт тем временем пророчил неизбежное:

From the moment you start up this miserable life

You should know that it's only a matter of time

'Till you die

'Cause we're all gonna die

Start saying goodbye!\footnote{Текст композиции Hurt под названием
  Flowers: (англ.) с того момента, как ты начал эту жалкую жизнь, ты
  должен понимать, что это только дело времени -- когда ты умрёшь.
  Потому что мы все умрём. Начинай говорить «до свидания».}

Покончив с окном, Агата повернулась к залу и застыла со сложенными на
груди руками. Улучив момент, Нора окинула её взглядом: она всё ждала,
что подруга вот-вот заплачет, слишком уж подозрительным казалось её
спокойствие. Но дама стояла ровно, с высоко поднятой головой и, кажется,
даже подпевала. Утренняя рассеянность вконец развеялась.

(Лишь позже Норе открылось, что слабой Агата бывает лишь наедине с
собой.)

And sarcastic song writers

Guys with tattoos, drummers and bassists and engineers too

And every last person that sang in this room

They will die

'Cause we're all gonna die\ldots{}\footnote{(англ.) И авторы
  саркастических песен, ребята в татуировках, барабанщики, басисты и
  инженеры тоже, и все до последнего, кто подпевает в этой комнате --
  они умрут, потому что мы все умрём\ldots{}}

Звуки музыки близились к своему лирическому завершению.

-- Однако\ldots{} -- начала Агата.

Нора с интересом взглянула на неё.

-- Сегодняшний случай не стоит приписывать к повседневным. Это и моя
боль тоже. У них было четыре минуты и тридцать восемь секунд для слёз,
так что пора сворачивать эту драму. Элеонора?

-- Да?

-- Будь добра\ldots{}

-- Чем смогу?

Агата вновь запустила пальцы в виниловую стопку.

-- Поставь её через тридцать секунд, -- произнесла Агата и тут же
исчезла в толпе.

Нора взглянула на полученную пластинку и удивленно вскинула брови.

«Ну, женщина в чёрном, надеюсь, ты ничего не напутала\ldots»

\section*{29}\label{29}
\addcontentsline{toc}{section}{29}

\markright{29}

Знаете, говорят, что жизнь не похожа на кино. Да, она в корне отличается
от сопливых романтических комедий, кишащих неправдоподобными
хэппи-эндами, или, скажем, идеальной рекламы кофе, но это уж точно не
относится к душевным фильмам. Большинство в кино на такие не ходит,
предпочитая им шумные блокбастеры, где все куда-то бегут, однако
Элеонора как раз таки относилась к тому самому меньшинству. Тем днём,
стоя в отведенном под церемонии прощанья зале, девушка зачарованно
впитывала разыгравшуюся перед ней трагикомедию, в которой каждый шаг,
каждое движение главной героини напоминали любимые киноленты.

Вопреки всему вышесказанному, происходящее вокруг Нора находила вполне
реальным и пропитанным жизнью, с какой бы иронией это не звучало в
сложившихся обстоятельствах. Именно тогда Агата перестала быть для неё
лишь красивым картонным образом.

Пластинка транслировала одушевляющие нотки, кричащие об идеальной жизни.
Уверенным шагом Агата поднялась на возвышение и с безмятежностью
прислонилась к дальней стене. Музыка Моби вызвала в толпе неодобрение,
но героине нашей было на это более чем наплевать.

Вот она потянула железную цепочку и за гробом с усопшим появился
огромный белоснежный экран. Невидимый проектор, которым, вероятнее
всего, руководит Владислав, начинает транслировать мерно всплывающие
фотографии.

Нора не была поклонницей слайд-шоу, считая их признаком дурного вкуса,
но это было весёлым исключением. Вместо привычных фотографий вроде «наш
парень в школе», «наш парень окончил университет» и «наш парень
женился», гостям показывались совершенно иные воспоминания, идеально
гармонирующие с музыкальным сопровождением.

Подобно персонажу Лиама Нисона, Агата мягко улыбнулась залу и начала
показ. Нора видела как дрожат губы её подруги. Вскоре из-под очков
показались слезинки, но Агата плавно смела их, продолжая улыбаться. На
экране один за другим всплывали снимки.

\begin{quote}
\emph{Вот Валерий сидит в плетёном кресле посреди гостиной второго
этажа. Он курит трубку с гротескной серьезностью, а за его спиной
виднеются пятки убегающего Владислава.}

\emph{Вот он стоит перед парадным входом «Маятника» с лопатой в руках;
на мужчине рабочая одежда и огромная соломенная шляпа, в зубах -- всё та
же трубка. Здание выглядит заброшенным, так что Нора думает, что это
первые дни возобновления бюро. Рядом, обхватив Крёстного за шею, Агата
демонстрирует ровные зубы. На вид девушке около двадцати, волосы едва ли
касаются плеч; на Агате надет пуловер со звездами и старомодный рваный
комбинезон -- никаких траурных одежд.}

\emph{Вот Валерий закинул Агату на плечо и дикаприевской походкой шагает
вдоль кладбища, по-прежнему держа трубку в свободной руке.}

\emph{Вот он обнимает целый галлон вина, обмотанный подарочной лентой.
На лице проступает невиданное блаженство.}

\emph{Вот мужчина обнимается с унитазом, показывая в камеру средним
палец: вероятно, в те дни Агата ещё не до конца обучилась виноделию.}

\emph{Вот Валерий самоотверженно пытается выбраться из свеженькой
могильной ямы, в которую угодил по собственной неосмотрительности. Он
делает вид, что злится, однако глаза -- синее синего -- выдают улыбку.}

\emph{На следующем кадре погода изменилась. С широкой улыбкой на глазах
Валерий стоит на четвереньках посреди центрального катка; на ногах
виднеются коньки, он боится стоять на льду.}

\emph{Вот Валерий вновь курит, но на этот раз не трубку. В его ладонях
зажат новенький бульбулятор. Комната заполнена плотным дымом. Поблизости
валяется Владислав, которому, кажется, наплевать на собственную
социофобию.}

\emph{Вот Валерий испуганно смотрит на пустую стену, обхватив руками
горло.}

\emph{Вот одурманенный он мчится по коридору, тогда как не менее
обкуренная Агата пытается остановить Крёстного.}

\emph{Вот Валерий выпал из окна и довольно растянулся на крыльце. К
счастью, это был первый этаж.}
\end{quote}

Тем временем Агата склонилась над гробом, всматриваясь в лицо усопшего и
напоминая Элеоноре персонажа Блума.

-- Она думает, в действительности ли он улыбается, -- вслух произнесла
девушка, чем приковала к себе парочку и без того негодующих взглядов.

-- КУРВА СКАЖЕНА!\footnote{Сука бешенная!} -- крикнул кто-то по
соседству.

-- Я почти уверена, что за очками скрывается безмятежность, -- ответила
Нора.

\begin{quote}
\emph{Вот Валерий в одних плавках стоит на траве. Рядом уже практически
принявшая свой привычный вид Агата, а около неё тот самый андерграундный
молодой человек. Валерий смеется и держит на тыльной стороне ладони
горящую свечу.}

\emph{Вот Агата горит, а молодой человек снимает с себя блейзер, умирая
от смеха.}

\emph{Вот Валерий в своем магазинчике\ldots{}}
\end{quote}

Снимки продолжали исчезать и появляться. Агата зажала в зубах сигарету и
нервно шарила по карманам в поисках огонька. Вдруг из ниоткуда появилась
мужская рука, держащая включенную зажигалку. До ушей доносилось:

Oh,

We close our eyes\ldots{}

The perfect life, life

Is all we need.\footnote{Текст песни Moby под названием Perfect life:
  (англ.) ох, мы закроем свои глаза\ldots{} Идеальная жизнь -- вот все,
  что нам нужно.}

Агата подкурила, но рука Владислава не исчезла. Огонёк вдруг начал
колыхаться в такт музыке. Агата улыбнулась и принялась подпевать.
Завидев эту картину, Нора разразилась смехом.

Она услышала недовольный шум толпы и тут же сделала музыку громче.

Тем временем Агата избавилась от сигареты не совсем корректным способом,
швырнув её в вазу с цветами. Дама склонилась над покойником, крепко
поцеловав его в обе щеки, и закрыла крышку гроба, отчего публика начала
негодовать сильнее. Самая отважная старушенция, которая твердо
вознамерилась скорбеть не менее часа, в гневе бросилась на Агату. Роста
в последней -- полтора метра с кепкой, так что дама без труда её
остановила. Однако, неудержимая бабуля решила нанести ещё один удар.

Музыка продолжала литься. Где-то в паре метров от гроба, виляя бёдрами,
танцевал Ласло, периодически натыкаясь на стены.

-- Не смей меня, блядь, трогать! -- крикнула Агата.

Вот непослушная бабуля уже летела в направлении вышеупомянутой вазы.

«Хоть бы она никого не сожгла\ldots» -- пронеслось в голове Элеоноры.

Девушка не сразу осознала, откуда такие мысли, но следующее действие всё
прояснило.

Подставка для гроба оказалась ничем иным, как стальным прозекторским
столом, скрытым под бархатными тканями. Агата нашла ручку и принялась
катить гроб к выходу. Толпа охала и, естественно, ахала. Никто не
аплодировал и не двигался в такт музыке, как это бывает в глупых
фильмах, но никто также не осмелился остановить хозяйку «Маятника».

На средине пути к Агате присоединился заплаканный Гек. Лицо мальчика
отражало благодарность и глубочайшую скорбь, тогда как выражение
физиономии самой дамы было сложно охарактеризовать: она была серьёзна,
но в то же время безмятежна и абсолютно точно уверена в том, что делает.

«Это же Хешер!» -- ликующе подумала Элеонора.

Вот девушка уже обнаружила себя расталкивающей буйствующих гостей.
Увидела, как прорывается к гробу и становится частью истории. Сама того
не ведая, Нора плакала: очень уж она впечатлительна.

Втроем им удалось преодолеть зал как раз к концу мелодии. Шагали наши
персонажи спокойно и неспешно, так, словно везли Валерия прогуляться.
Мелькающие со всех сторон лица -- разгневанные и озадаченные -- казались
им лишь размытым фоном.

The perfect life. The perfect life. The perfect life.

Лица соприкоснулись со свежим ветром. Они вышли во двор.

-- Дамы и господа, -- торжественно произнесла Агата. -- Объявляю первую
за всю историю существования «Маятника» десятиминутную церемонию
прощанья оконченной.

*

Да, жизнь это кино. Независимое авторское кино, разумеется.

\bookmarksetup{startatroot}

\chapter{Часть II. Гром и Молния
\{-\#chapter-2=``\,``\}}\label{ux447ux430ux441ux442ux44c-ii.-ux433ux440ux43eux43c-ux438-ux43cux43eux43bux43dux438ux44f--chapter-2}

\emph{(Девушка, указывая на забор кладбища:)}

\emph{-- А зачем вам колючая проволока? По ночам сюда залезают?}

\emph{(Гробовщик безразлично:)}

\emph{-- По ночам отсюда вылезают.}

\emph{к/ф Влюбленный гробовщик}

\section*{30}\label{30}
\addcontentsline{toc}{section}{30}

\markright{30}

Широкая полоса леса раскинулась под небесным сводом. Морозный воздух
покрывал щёки девушки колючими поцелуями. Погода обещала быть бодрящей:
термометр показывал шесть градусов ниже нуля.

В это время года рассвет наступал не ранее семи утра. Агата подкурила
очередную сигарету. Сегодня солнце за её спиной поднялось лишь без
четверти восемь. Столетние сосны, туи и ели тянулись густой стеной,
конца которой было не видать. Верхушки деревьев ещё сохранили на себе
остатки вечернего снега и теперь вовсю переливались под лучами холодного
солнца.

Ветра не было и перед глазами девушки застыла неподвижная картина,
единственным изменением в которой была игра света. Ни одна ветка не
шелохнулась пока солнце медленно поднималось над лесом, так что
открывшийся вид скорее напоминал фотоснимок, которому постепенно
предавали теплые тона. Последнее навело Агату на мысль о социальных
сетях, а те, в свою очередь, напомнили о Норе.

Лесной пейзаж всегда действовал на владелицу «Маятника» умиротворяющее,
и пускай она не была здесь более десяти лет, мысли девушки частенько
возвращались к тихим полянам и одиноким тропинкам, усеянным сухими
иголками и пожелтевшей листвой. Стоило Агате почувствовать, что руки
опускаются, а измученное тоской сердце колотится так, словно вот-вот
выскочит из груди, предвещая нервный срыв, она тут же закрывала глаза и
переносилось в это чарующее место. Порой девушке приходилось облачиться
в наушники, забраться в постель и некоторое время провести так,
погрузившись в воспоминания о днях, когда в её мире не было места
печали, а жизнь была полны покоя, изучения литературы и прогулок по
волшебным тропинкам, ведущим к излюбленным местам.

Сейчас, стоя на одной из тех самых дорог и купаясь в холодных рассветных
лучах, Агата Рахманинова с едва уловимым удивлением осознала, сколь
многое она забыла. Естественно, стирать детские вспоминания -- привычное
дело для человеческого сознания, особенно когда речь идет о мелочах,
которым в мире взрослых не принято придавать большого значения. Но
значение у них всё-таки было. Прошлое казалось ей размытой плёнкой,
годами пылившейся в дальнем углу воспоминаний, но стоило Агате вернуться
в родные места, как кто-то в её голове взялся за старый ящик и принялся
старательно проявлять снимки, постепенно придавая им чёткости.

Тогда она поняла, что уже давно не помнила, как старательно вырисовывала
контур леса, сидя на этом самом месте; не помнила, как кузен бродил во
сне и однажды чуть не потерялся в чаще; не помнила, как в канун Ивана
Купалы сама гуляла по ночному лесу в поисках цветущего папоротника,
дарующего бесконечную силу (включающую ясновиденье и власть над нечистым
духом); не помнила, как лежала на печи, всматриваясь в крохотное окошко,
открывающее лоскут звездного неба; не помнила, как неделями мастерила
плот для покорения кристально чистых озер. Она осознала, что прожила
последние годы, не зная, что однажды принимала у собаки роды; не знала,
что когда-то забиралась на самую верхушку сосны в поисках мобильной
связи; не знала, как прекрасен лес во время грозы; не знала, что именно
здесь услышала свою первую андерграундную пластинку; больше не знала,
каким был у него почерк, не знала его отчества, не знала, почему Адам
оставил её.

Агата закрыла глаза. Это были не те мысли. Она глубоко затянулась, а
затем выдохнула, создавая перед собой облако из пара и сигаретного дыма.
Подсознание вновь сыграло с ней злую шутку, и она знала, что всё дело в
ассоциациях. Девушка распахнула успевшие покраснеть глаза. Поблизости не
было ни одной живой души, так что очки ждали её в домике, что остался за
спиной. Может, оно и к лучшему.

Нужно было двигаться, иначе эмоциональная волна могла повториться,
нахлынув с новой силой. Агата избавилась от погасшей сигареты.

Она ступила на заснеженную тропу и вскоре исчезла под покровом леса.

\section*{31}\label{31}
\addcontentsline{toc}{section}{31}

\markright{31}

Над долиной царил на удивление холодный вечер. Ветер заставлял деревья
перешептываться, лишь начиная снимать с них листву. В печи приветливо
шуршали обугленные брёвна. Сегодня небо было как никогда звёздным --
девочка наблюдала за ним, свесив с лежанки не знавшие загара ноги и
глядя в окошко, что размещалось под самым потолком. По своим размерам
последнее не превосходило тетрадного листа.

Стояла середина сентября тысяча девятьсот девяносто седьмого года.

Девочка -- Агата, как вы уже, должно быть, догадались -- держала в руках
кружку с практически остывшим чаем. Не отводя взгляда на протяжении вот
уже четверти часа, она старательно рассматривала ту часть неба, что была
видна с печи. Вглядывалась в мерцающие светила, перебирала в голове все
известные ей созвездия, историю их названий и пыталась вспомнить
расположение зодиаков на небесном атласе. Затем вновь возвращалась
мыслями к картине перед глазами, стараясь распознать виднеющиеся
созвездия.

В шестилетнем возрасте Агата уже начинала страдать нарушением сна.
Особых мучений по этому поводу девочка не испытывала. Дискомфорт
сказывался лишь в тех случаях, когда её пытались переучить. Только лет
восемь спустя стало ясно, что дело вовсе не в бессоннице: биоритмы Агаты
действительно отличались от общепринятого режима сна, но по сути никаких
нарушений не было. По никому не известным причинам, подобные опции
обосновались в организме девочки еще до её рождения, соседствуя с
мизантропией и пороком сердца.

Наша героиня ещё несколько минут сверлила взглядом звездное небо, после
чего сдалась. Она вспомнила о чае и принялась жадно поглощать его. Ввиду
специфики своего организма в людях Агата больше всего ценила -- и с
возрастом это совсем не изменилось -- умение мириться с её привычками.
Дедушка был ранней пташкой, даже слишком. Он просыпался в четыре утра,
завтракал в шесть и к семи был уже у калитки, готовый совершить первую
за день прогулку. Не смыкавшая глаз всю ночь Агата с радостью к нему
присоединялась и вместе они неспешно бродили по лесным просторам в
поисках грибов и новых, не виданных ранее, озер и полян, что так умело
прятались в глубинках леса. Иногда они заходили так далеко, что попадали
в берёзовую рощу, а бывало, наоборот, держались ближе к дому, собирая
дрова и хворост.

Так или иначе, спать Агате хотелось лишь к обеду и именно то, что
дедушка не только не мешал её сну с полудня до наступления вечера, но и
принимал эту привычку, так радовало нашу героиню. Было кое-что ещё: за
всю свою жизнь Агата ни одного «утра» не провела без того, чтобы отпить
чёрного чая, к которому позже добавилась сигарета. Дедушка же начинал и
заканчивал свой день чашкой очень крепкого кофе, но к моменту
пробуждения на ступенях лежанки Агату всегда ждала огромная кружка чая.
Обычно девочка вытягивала руку, опустошала половину чашки и лишь потом
выползала из-под одеяла.

С годами, кстати, мало что изменилось.

Вернёмся же к звездам. Этот вечер с самого начала отличился от всех
предыдущих, потому что, распахнув глаза, девочка и думать забыла о чае.
Она резко подскочила и поспешно уставилась в окошко. Отсутствовало и
привычное желание позвать Булю и ещё с часок поваляться на теплых
простынях, обнимая собаку. Агата слышала, что есть хороший способ быстро
избавиться от дурного сна, а в тот день ей привиделся именно такой.
Нужно было как можно скорее подойти к окну и стараться сконцентрировать
своё внимание на открывающемся из него виде до тех пор, пока память не
сотрет неприятные сновидения. Стоит сказать, что прежде данный способ
действительно работал, но только не в этом случае.

Прошло минут двадцать, и Агате всё же пришлось это признать. Она
опустошила кружку и начала спускаться с печи за новой. Девочка пыталась
отвлечься от кошмара, погрузившись в книги, но мысли путались, так что
ей приходилось по несколько раз перечитывать один и тот же абзац, что не
могло не вызывать раздражения. Затем Агата опрометчиво силилась
отвлечься с помощью ещё одного любимого занятия -- рисования, но!
Результат оставлял желать лучшего. Не потому, что картинка получилась
так себе: Агата была совсем рёбенком и как все дети мнила себя великим
художником. Результат оказался не очень как раз из-за того, что девочка
изобразила именно то, что так сильно её напугало. Агата резко выдохнула
и отшвырнула рисунок в корзину с макулатурой, надеясь, что следующим
утром тот пойдет на растопку печи.

Она вдруг поняла, что вся дрожит и даже немного разозлилась на себя за
это. Девочка не могла понять, почему увиденное во сне её так напугало.
Она никогда не испытывала страха перед ведьмами, вампирами и прочей
нечистью. Более того, Агата была поклонницей Гоголя, так что список её
детских интересов составляли исключительно потусторонние и
паранормальные темы. Это ещё сильнее пугало девочку: именно тот факт,
что она испытывает странный, необъяснимый страх перед, казалось бы,
безобидной детской фантазией, окончательно вогнал Агату в ступор. Она
была настолько напугана, что ещё около часа не могла заставить себя
покинуть пределы спальни. Чувство голода и тоски по любимому питомцу,
всё же, возымело верх, и в итоге девочке пришлось отправиться на кухню,
но, даже обнимая собаку, она не могла забыть странной картины,
пригрезившейся ей накануне.

Сделать это не удалось и во время завтрака, который был для дедушки,
скорее, ужином.

\section*{32}\label{32}
\addcontentsline{toc}{section}{32}

\markright{32}

Спустя тринадцать зим Агата с немалой грустью вновь обрела свое
ненадолго утраченное одиночество. Алкоголь никогда не был в силах
заглушить душевную боль девушки, а иногда и вовсе служил топливом для
очага тоски в её сердце, так что владелица «Маятника» изо всех сил
хваталась за спасательный круг, которым для неё служила литература.
Правда крылась в том, что даже чтение не было способно заставить
разбитое сердце забыть о пережитых горестях, но кое-что книги всё же
меняли: они помогали заморозить мысли, отложить произошедшее, пускай и
не в самый далёкий уголок сознания, и попросту тянуть время с надеждой
на то, что когда девушке всё же придется вернуться к случившемуся, буря
в её душе поутихнет.

Конечно, тогда Агата и подумать не могла, что ждёт её дальше, а потому и
не предполагала, что подобное состояние навсегда останется для неё
единственным возможным.

В любом случае, девушка читала с неумолимой скоростью. Страницу за
страницей и книгу за книгой, временами забывая о еде и отвлекаясь разве
что на работу. Приносившее ей деньги дело отнюдь нельзя было назвать
предсказуемым. Случались загруженные дни, но были и те, в которые никто
почему-то не спешил умирать. Их Агата проводила не выбираясь из постели,
а если девушка всё-таки решала покинуть её пределы, то лишь для того,
чтобы обосноваться в старом кожаном кресле с привычной книгой в руках.
Временами на неё накатывало внезапное тягостное вдохновение, и Агата
бралась за кисти. Рисовала она чуть ли не в один присест: не могла
уснуть, пока картина оставалась незавершенной, так что за мольбертом
девушка проводила по несколько суток, но чаще всего её вечера занимала
литература. Именно в один из таких мрачных и ничем не примечательных
вечеров она взяла в руки бутылку Шато Шасс Сплин и «Призраков двадцатого
века».\footnote{«Призраки двадцатого века» -- сборник рассказов
  американского писателя Джо Хилла.}

Любимый напиток Байрона был приятным на вкус и практически оправдывал
потраченную на него зарплату, но книга оказалась ещё лучше. К
наступлению утра было прочитано двенадцать из четырнадцати рассказов.
Пускай ни один из них не ввел Агату в состояние, хотя бы отдаленно
напоминающее страх, местами она находила повествование не просто
приятным, но и трогательным, вызывающим то самое трепещущее ощущение под
ребрами и какой-то там ложечкой. Однако, ход событий резко переменился,
когда пришло время для тринадцатого рассказа. История настораживала
девушку с первых страниц, а вернее, со второго абзаца.

\begin{quote}
\emph{«За тобой гонятся карточные люди, -- объяснила она, -- дамы и
короли. Они такие плоские, что могут проскальзывать под закрытыми
дверями».}
\end{quote}

Эти строки вызывали в душе девушки необъяснимую тревогу, что лишь
усиливалась по мере развития сюжета. Агата не знала тому причин и, не
прерывая чтения, всячески гнала от себя странные мысли.

Безуспешно.

Повествование не было страшным. Оно было жутким. Под конец истории
девушку всецело охватило чувство дежавю, смешавшееся с тревожным
волнением. Она отложила книгу и какое-то время просидела, разглядывая то
противоположную стену, то собственные руки, и пытаясь понять, чем вызван
столь внезапный калейдоскоп эмоций. Естественно, чем больше Агата
погружалась в мысли, тем быстрее от нее отдалялся ответ. В итоге, наша
героиня уже не могла вспомнить, о чём думала минуту назад. Так всегда
бывает с\ldots{}

«С дурными снами», -- произнесла про себя Агата.

Невнятная мысль промелькнула в её голове яркой вспышкой, чтобы исчезнуть
прежде, чем девушка смогла что-либо понять.

Внезапно захотелось рисовать.

Она растерянно взглянула на книгу и, не зная, что ещё делать, вернулась
к чтению. Девушку не пугали события, что настигли главного персонажа в
лесу, хотя они и включали в себя образ маленького призрачного мальчика,
в одной пижаме разъезжающего на винтажного вида велосипеде.
Настораживала скорее атмосфера и странное поведение героев рассказа.
Конечно, было тут и кое-что ещё. Такое бледное и непритязательное, что
Агата с трудом улавливала производившую этот шум волну, но вместе с тем
понимала: именно та являлась истинным источником её тревоги.

Покончив с книгой, девушка предприняла последнюю попытку найти причину
ощущений, столь резко её охвативших. Агата чувствовала, что когда-то
определенно знала ответ, но память стёрла эту информацию за
ненадобностью. Понимала, что это что-то в действительности совершенно
неважное, деталь, ни в чем не играющая роли, но, тем не менее, чувство
страха, пугающее скорее своей внезапностью, не покидало девушку всё
утро.

К обеду она отдалась сну, а, очнувшись, начала новую книгу.

С наступлением следующего дня Агата и думать забыла о случившемся. Она
не вспоминала карточных людей на протяжении последующих пяти лет. До
того самого момента, пока сердце и приближающийся нервный срыв не
привели её обратно в лес.

\section*{33}\label{33}
\addcontentsline{toc}{section}{33}

\markright{33}

Утро сменилось днем. Уютный слой перламутрового снега укрыл собой лес и
долину, напоминая глазурь и превращая небольшие строения, --
преобладающая часть которых пустовала уже долгие годы -- разбросанные то
тут, то там, в пряничные домики. За прошедшие часы температура воздуха
поднялась до нуля. Ветра по-прежнему не было, и снежные хлопья сонно
кружили в пространстве, играя солнечным светом.

Ближайший домик располагался всего в сорока двух шагах от леса. Он был
самым обычным деревянным жилищем с низкими потолками, но на фоне
всеобъемлющей природы казался неким сказочным местом. Несмотря на то,
что новогодние празднества остались позади, здесь царило волшебство.

И так было всегда.

Вскоре на конце тропинки показалась фигура, облаченная в траурные
одежды. Она вышла из чащи и плавно двигалась по скрипучему снегу в
сторону единственной виднеющейся тут калитки.

Агата замерла в двух шагах от входа во двор и внимательно осмотрела дом.
Он казался ещё меньше, чем запомнился ей из детства. Однако, вопреки
звенящей тишине и тому факту, что при контакте с дверными проёмами
дамочке теперь приходилось сгибаться в три погибели, (чувствуя себя
Гендальфом, посетившим Шир) Агата ощущала, как при взгляде на знакомые
ступени и заснеженные окошки, её тело наполняло поразительное
спокойствие.

В открывающемся пейзаже, казалось, чего-то не хватало. Всего на
несколько секунд Агата прикрыла глаза, позволяя детским воспоминаниям
выплыть наружу, а распахнув их, сразу же поняла, в чём дело.
Отсутствовала главная деталь, превращающая домик в колыбель уюта -- печь
давно не топили, и, конечно же, из кирпичной трубы на крыше не валил
пар.

Агата слегка улыбнулась воспоминаниям и поспешила внутрь, намереваясь
разогреть печь и искренне надеясь, что не забыла, как это делается.

*

Оказалось, забыла.

Дамочка залилась глубоким кашлем. Она бросилась к окнам и принялась
открывать их одно за другим, сшибая на своем пути всякого рода хлам.
Последнее мало заботило нашу героиню, хотя она в полной мере
почувствовала упавший на её ногу тяжелый табурет.

Говорят, первый блин комом, но это ещё ничего. Чуть не лишить себя жизни
в столь неподходящий для этого момент -- куда неприятнее. Спальня и
соседствующий с ней предбанник были заполнены чёрным дымом, который,
вопреки стараниям, без остановки валил из печи.

Как выяснилось позже, Агата поместила в печь слишком много поленьев,
напрасно полила их керосином и недостаточно широко отодвинула створку
дымохода, после чего благополучно погрузилась в сон, к счастью, не
слишком глубокий, дабы пропустить нехватку кислорода.

Когда с устранением этого недоразумения было покончено и в комнате вновь
сделалось возможным дышать, Агата умыла лицо и руки, покрытые слоем
сажи, и вернулась в постель, с недоверием поглядывая на печь.

Спать совершенно расхотелось.

\section*{34}\label{34}
\addcontentsline{toc}{section}{34}

\markright{34}

Темнота накрыла лес ещё до того, как часы успели пробить четыре. Слишком
рано, даже для этих мест. Истосковавшаяся по сну Агата сидела у печи,
зачарованно глядя на пламя и бесцельно перебирая старые стопки бумаги.
Вскоре её взгляд застыл на одной точке, а лицо не выражало ничего, кроме
еле уловимого отчаянья. Мыслями Агата была так далеко, что не сразу
заметила, как её накрыло прежнее волнение, от которого она столь
самонадеянно бежала сюда.

Хозяйка «Маятника» вздрогнула и, заметив, что пламя вот-вот погаснет,
принялась бросать в него пожелтевшую от времени бумагу. Все ещё не в
силах позабыть об угнетающих мыслях, то и дело всплывавших в её голове,
Агата постаралась переключить внимание на что-нибудь другое. Вскоре она
с немалой толикой безразличия разглядывала стопку бумаги, лежавшую у её
ног. В основном там были древние журналы и газеты времен Советского
Союза, но временами попадались старые дедушкины стихотворения. Девушка
бережно отряхивала их от пыли и складывала на кофейном столике.

В какой-то момент Агата осознала, что жажда сна, наконец, взяла верх над
её мрачными мыслями. Она облегченно вздохнула и, швырнув в огонь
очередной выпуск «Вязания», уже предвкушала грядущие объятья Морфея,
когда её рука замерла над стопкой, в глазах отразилась тревога, а из
прикрытого рта вырвался резкий выдох, повторяя события многолетней
давности.

На коленях девушки лежал детский рисунок. Пыльный, побледневший и
местами порванный, при этом он был, всё же, узнаваем и запечатлел
высокие тонкие силуэты, водившие хороводы внутри зеркала. Фигуры были
изображены с помощью красной и черной красок, но Агата узнала бы их,
даже будь они бесцветными.

Карточные люди: короли, дамы и валеты из её сна! Карточные люди, которые
отображались лишь по ту сторону зеркала. Тут-то Агате и вспомнилась
«Маска моего отца»\footnote{Рассказ из вышеупомянутого сборника Джо
  Хилла.}, которая однажды навеяла на неё малопонятный ужас. Там были
строки, поразительно напоминавшие детский кошмар, похороненный в недрах
памяти.

\begin{quote}
\emph{«Я отвел взгляд -- не мог не отвести -- и случайно посмотрел в
зеркало на комоде. Покрывала слегка сдвинулись в сторону, и в отражении
я увидел, что моего отца ласкает карточная женщина -- пиковая дама. Её
чернильные глаза устремлены вдаль, на теле -- нарисованные
одежды\ldots»}
\end{quote}

Конечно, в рассказе никто не водил хороводов. Странностей и без того
хватало.

Агата же, в свою очередь, пришла к выводу, что в её сне происходила
много больше событий, но сцена с жутким танцем оказалась единственным
следом, отпечатавшимся в сознании. Большего ей знать и не хотелось.
Подумать только, она каким-то образом смогла не только вспомнить сон,
привидевшийся ей почти два десятилетия назад, но и испугаться таких
воспоминаний!

Девушка вопросительно подняла брови, сомневаясь в собственной
вменяемости. Сложившуюся ситуацию она одновременно находила смешной и
грустной. Агата загнала себя в глухой лес, желая успокоить нервы, но
сейчас всё выглядело так, словно она в действительности страдает
душевным расстройством. Меньше всего ей хотелось оказаться сумасшедшей,
особенно\ldots{}

Нет, совсем не хотелось.

Агата отправила в огонь злополучный рисунок. Вслед за ним оправилась
парочка новых поленьев.

Так закончился первый день в лесу.

\section*{35}\label{35}
\addcontentsline{toc}{section}{35}

\markright{35}

Гек не забыл о данном обещании и продолжил свой рассказ. Часть истории
он поведал на следующий после похорон день. Это помогло разбавить
гнетущую атмосферу, царившую в «Маятнике» после смерти Валерия. Мальчик
говорил о том, как, мучаясь от безумных скачков температуры, провел
несколько дней в состоянии трупа. Продвижение к цели сделалось для него
мучительно медленным, и в один едва ли прекрасный момент Гек потерял
сознание прямиком посреди трассы.

Он пришёл в себя лишь когда над дорогой взошла полная луна,
потревоженный вспышкой света, внезапно озарившей его лицо. Мальчику
пригрезилось чудище с огромными жёлтыми глазами. Осознав, что перед ним
автомобиль, Гек попытался отползти на обочину, но был слишком слаб для
подобного рода действий.

Запоздалое везение всё же не подвело нашего героя. Спустя полчаса он
очнулся на заднем сиденье машины со смутными воспоминаниями о
предшествующих этому событиях. Водитель был мужчиной приятной
наружности, немногим моложе Иисуса, и чем-то даже походил на последнего.
Гек при всём желании не мог запомнить его имени. Спаситель уложил
несчастного ребенка в автомобиль и без лишних раздумий помчался в город.
Придя в себя, малец несколько раз поблагодарил спасителя, на что тот
лишь встревожено улыбнулся и попросил ни о чем не беспокоиться. Затем
сказал что-то о больнице, куда Геку ни за что нельзя было попадать. С
таким же успехом он бы мог явиться прямиком в полицейский участок.
Мальчик старался не терять контроля над телом, но периодически
проваливался в сон. Проснувшись в очередной раз, он увидел, как сквозь
запотевшие стекла переливались огни города. Геку удалось сбежать на
ближайшем светофоре, оставив позади недоумевающего водителя.

Мальчик также поведал, как с трудом добрался до аптеки и, обзаведясь
антибиотиками, отправился прямиком на вокзал, где и провел ту ночь.
Утром выяснилось, что таблетки и какой-никакой, но, всё же, сон сделали
своё дело и состояние больного улучшилось.

История, казалось бы, небольшая, но благодаря красноречию Гека и его
любви к деталям, она заняла весь вечер. Мальчик начал рассказ во время
ужина и закончил его, стоя рядом с Норой, наблюдавшей за тем, как
хозяйка «Маятника» зажигала могильные огоньки.

-- Найліпше мені ввижається та мить, коли я, ледве розплющивши беньки,
поплентався до привокзального сортиру. Вперше за увесь час свого шляху я
побачив власне відображення в люстерці і сказати, що побачене мене
лишень вразило -- не сказати нічого. Звичайно, в мене не з'явилась
борода, чи ще щось навзнак мужності. Важко втелепати, що саме змінилось,
але то був не я. З туалетного дзеркала у мене вдивлявся хтось
інший.\footnote{Лучше всего мне видится тот момент, когда я, едва
  разлепив глаза, поплелся в привокзальный сортир. Впервые за всё время
  своего пути я увидел собственное отображение в зеркальце и сказать,
  что увиденное меня просто поразило -- не сказать ничего. Конечно,
  бороды у меня не появилось, или ещё какого-нибудь признака мужества.
  Сложно смекнуть, что именно изменилось, но это был не я. Из туалетного
  зеркала на меня таращился кто-то другой.}

Этими словами Гек закончил очередную порцию рассказа, и слегка
поклонился в ответ на шутливый реверанс Элеоноры. Наблюдавшая эту
картину Агата, вероятно, не единожды слышала вышеописанную историю.

-- Ты забыл свою любимую фразу, -- без особых эмоций заметила она,
зажигая новую свечу.

Гек на мгновенье задумался, после чего просиял.

-- Трагічна жінка має рацію! -- улыбаясь во все тридцать два, сказал
малец. -- Саме тоді я й став Геком.\footnote{Трагическая женщина права.
  Именно тогда я и стал Геком!}

\section*{36}\label{36}
\addcontentsline{toc}{section}{36}

\markright{36}

В «Маятнике» наступление Нового года не отмечали, в традиционном
понимании этого слова, как не отмечали Рождество, день всех влюбленных и
любые другие праздники, связанные с убранством комнат. С большим трудом
Элеоноре удалось получить разрешение развесить в гостиной парочку
неоновых гирлянд.

-- Бутылка хорошего вина послужит лучшим товарищем во время любого
празднества, -- произнесла Агата во время очередной попытки Норы убедить
её хоть раз установить в доме ёлку.

-- Но на твоем столе и так каждый вечер имеется бутылка, -- с обидой
пробормотала девушка.

-- Значит, будет две, -- отрезала Агата.

Она опустилась в кресло и раскрыла книгу на заложенном месте, давая
понять, что обсуждение окончено.

Ещё какое-то время Элеонора посидела с надутыми губками, но вскоре
поняла, что это едва ли принесет плоды, и поплелась в свою спальню.
Девушка собиралась отправиться домой тридцатого числа. Как бы она ни
презирала город, в котором волей судьбы ей пришлось расти, Нора находила
крайне бессмысленным празднование нового года вне семейного круга. Дома
её ждал вкусный обед, теплые объятья и, в конце концов, чертова ель.

Девушка убедилась, что дверь за ней закрылась, после чего отворила
дверцу бездонного шкафа. Нора внимательно осмотрела находящийся перед
ней предмет -- новогодний подарок Агате.

Это был масштабный портрет и, возможно, лучшая вещь, из всех ею
когда-либо нарисованных. Выполненная в холодных тонах картина с
невообразимой точностью отображала внешние данные владелицы похоронного
бюро. Единственное исключение составили глаза. На их месте Элеонора
изобразила графики радиоимпульсов какого-то там пульсара, носившего
невыносимо трудное название, словом, эмблему дебютного альбома Джой
Дивижн. Надломленные линии горизонтально текли вдоль всего холста,
предавая изображению некую абстрактность, создающую удачный контраст с
реализмом общей картины. Этим Нора гордилась больше всего.

Была у неё привычка, немало раздражавшая саму девушку: когда
какая-нибудь из её картин вплотную подходила к логическому завершению и
казалась Норе чудесной, в голове нашей героини вдруг зарождалось
яростное желание продолжить рисунок. Как вы уже могли догадаться, в
итоге, она перебарщивала с деталями, лишая собственное творение его
первоначального шарма.

Вот и сейчас руки девушки предательски зачесались, умоляя добавить пару
темных линий в-о-о-о-н в тот уголок, и Нора поспешно захлопнула дверцу
шкафа.

Из гостиной доносились отголоски незнакомой мелодии, вызывая в душе
противоречивые чувства. Прихватив свою записную книжку, Элеонора уже
отворила дверь спальни, намереваясь вернуться в отчужденную компанию, но
так и застыла на месте. Среди музыки -- пелось что-то о фотографиях и
подходящих словах -- девушка различала ещё один голос: тихий и лишенный
эмоций он полностью соответствовал представлениям о своём владельце,
таившихся в голове нашей героини.

Владислав продолжал говорить, но девушке так и не удалось разобрать ни
слова. Она надеялась, что успеет понять что-нибудь в перерыве между
песнями, но вот мелодия заиграла сначала, а таинственный голос и вовсе
исчез. Нора вздохнула и вошла в гостиную.

-- Что играет? -- спросила она, желая избежать тягостного молчания
соседки.

-- The Cure, -- не поднимая головы, произнесла Агата. -- Музыка для тех,
кто глубоко несчастен вместе с тем рад этому.

Элеонора с недоверием взглянула на хозяйку «Маятника», пытаясь понять,
не шутит ли она. Временами Агата говорила весьма странные и пафосные
вещи, но делала это столь непринужденно, что сказанное начинало казаться
забавным.

-- Это как?

-- Ну, для тех, кто ни жив, ни мёртв.

Девушка ещё с мгновенье постояла в дверях, дожидаясь более внятного
объяснения, но его не последовало и Норе ничего не оставалось, кроме как
опуститься на диван.

С чувством гнетущей подавленности, которое всегда сопровождало
отсутствие вдохновения во время её литературных порывов, Элеонора
принялась перечитывать рукопись. Написано было всего ничего, но девушке
нравилось каждое слово и каждое событие её будущей книги. Своих
персонажей Нора находила идеальными и прекрасно знала, как должен
развиваться сюжет. У неё никогда не возникало проблем с облачением
мыслей в слова.

Именно это теперь и угнетало.

С каменным, как ей казалось, лицом Нора уже раз шесть перечитала ранее
написанное, но так и не смогла выжать из себя и двух абзацев. Что бы ни
приходило в голову, оно казалось девушке отталкивающе неуместным.

-- Ну й чого ти журишся?\footnote{Ну, и почему ты унываешь?} -- вдруг
спросила Агата.

Нора вздрогнула. За время напряжённой работы мозга она успела позабыть о
наличии в комнате кого и чего бы то ни было, кроме злополучной рукописи.
Девушка в недоумении уставилась на подругу. Нора не могла с точностью
сказать, что удивило её больше: внезапная смена Агатой языка, или тот
факт, что та вообще заметила изменения в её настроении.

-- По-моему, мы все здесь сошлись во мнении, что это моя прерогатива --
печалиться, -- добавила Агата.

На миг её лицо осветила улыбка, в которой проскальзывала едва уловимая
печаль.

-- Так что тебя омрачило?

Нора осознала, что за всё время не промолвила ни слова. Подобное
поведение казалось добродушной девушке невежливым, и та поспешила
ответить.

-- Книга. Она не пишется.

-- Давно? -- только и спросила Агата, хотя тон её был едва ли
удивленным, превращая сказанное в утверждение.

Элеонора кивнула.

-- Давно.

-- Это всё персонажи, -- опять не вопрос. -- Они недостаточно живы.

Глаза нашей героини удивленно расширились. Агата говорила так уверенно,
словно вдруг обрела способность к чтению мыслей.

-- Я ведь\ldots{} Не давала тебе прочесть рукопись? -- медленно
протянула Нора.

-- Нет.

-- И не рассказывала, о чём она?

-- Нет.

-- Тогда откуда..?

Агата вновь улыбнулась..

-- Это и так ясно по тому, как ты выглядишь, когда пишешь.

-- Что ты имеешь в виду?

-- Я имею в виду что, что бы там не лежало в основе твоего сюжета, ты
сама с натяжкой веришь в то, о чём пытаешься писать, -- спокойно
объяснила Агата. -- Так всегда бывает с красивыми историями, авторы
которых не испытывали держатся описываемых ощущений. Они понятия не
имеют, о чём говорят, а потому подсознательно на расстоянии от
персонажей.

Девушка помедлила, удивленная обширностью ответа, а после выпалила всё,
что таилось в её мыслях.

-- Я работаю над романом меньше месяца, а в душе такое чувство, словно
это тянется целую вечность. Искусство должно приносить радость, почему
же тогда все мои амбиции обращаются мучениями? Фак, да мне просто
хочется создать что-то прекрасное, но всё, за что я бы я ни взялась,
оказывается пустым и в итоге у меня просто опускаются руки. Это
невыносимо!

Агата отложила книгу и поднялась с места, дабы приглушить музыку.

-- Твоя проблема, -- начала она, -- как раз кроется в убежденности в
том, что искусство должно приносить радость. Да, порою это так, но! Оно
приносит радость потребителю, но никак не автору, и уж тем более не во
время самого процесса создания. Изливать душу чертовски сложно, а именно
этим ты и должна заниматься и единственной радостью, которую ты
получишь, станет облегчение -- момент, когда понимаешь, что, наконец,
сказал пером всё то, что хотел сказать.

Владелица похоронного бюро помедлила, наблюдая за реакцией своей
собеседницы, после чего с небывалой мягкостью в голосе задала вопрос.

-- Ты вообще знаешь, что хочешь сказать своей книгой?

На этот раз Элеонора ответила без промедлений.

-- Я хочу создать новый мир, прекрасный на первый взгляд\ldots{}

-- Нет, -- прервала её Агата. -- Я не спрашиваю тебя о сюжете. Идея --
вот что должно теплиться в твоей голове на протяжении всей работы над
романом. Ты можешь писать о будущем и прошлом, в котором никогда не
жила, можешь вообразить новые миры и галактики, о которых ничего не
знаешь и, изучив, или, опять-таки, придумав сопутствующую информацию
клепать сюжет, облагораживая его красивыми деталями. При этом ты должна
понимать, что всё вышесказанное -- лишь декорации, окружающие твою идею.

Представь, что повествование -- это тесто, а детали, такие как место и
время, придают ему форму, делают более привлекательным в чужих глазах,
но никак не влияют на вкус. Ты не обязана отправляться на Луну, если
хочешь сделать её местом действия своей истории, но ты должна пережить
то, о чем пишешь, а вернее писать только о тех чувствах и эмоциях,
которые ты в действительности пережила. В противном же случае твоё
творение, (даже если ты заставишь себя его закончить) выльется одной
лишь красивой картинкой, которая не будет цеплять никого, включая тебя
саму.

Я не прошу тебя сейчас выбросить свой опус, хотя история о новом, на
первый взгляд прекрасном мире, кажется мне более чем банальной в наши
дни. Я прошу тебя прекратить вечерами напролет сверлить взглядом
рукопись и попробовать разобраться в себе, понять, что ты уже успела
пережить. Найди эти эмоции и извлеки на поверхность, вне зависимости от
того, хорошие они или плохие.

Скажи мне, что ты чувствуешь?

-- Растерянность, -- угрюмо произнесла девушка.

Агата недовольно хмыкнула.

-- Ой, да ну это и так ясно. А глубже?

Элеонора развела руками. Выглядела она ещё более опечаленной, чем в
начале разговора. Раздражение от неудачных творческих попыток исчезло,
оставив лишь горечь и жалость к самой себе.

-- Боюсь, я ничего не чувствую, -- с сожжением призналась Нора. -- Моя
жизнь не хорошая и не плохая, она совершенно обычная. Тут ты абсолютно
права: я не знаю, что пытаюсь сказать своей книгой. Никогда об этом
толком не думала. По-моему, во мне просто нет ничего такого глобального,
и это заставляет меня чувствовать себя заурядной. И огорчает.

-- Что ж, если это так, тебе стоит написать о человеке, который боится,
что ему нечего сказать миру, -- заметила Агата.

Следующая фраза, поразила девушку сильнее, чем сам разговор.

-- Уверена, в тебе есть гораздо больше, чем ты способна себе
представить.

Сейчас больше всего на свете Норе хотелось, чтобы её собеседница
оказалась права, но, размышляя о двадцати прожитых годах, она не могла
представить ничего, способного хоть немного затронуть чью-нибудь душу.

-- Почему ты так думаешь? -- наконец спросила Элеонора.

-- Копай глубже, -- просто ответила Агата.

Она опустила голову и вновь углубилась в чтение. Разговор окончился так
же внезапно как и начался.

\section*{37}\label{37}
\addcontentsline{toc}{section}{37}

\markright{37}

Снег понемногу начинал таять. Ещё недавно обрамлявший деревья своим
сверкающим кружевом иней постепенно исчезал, обнажая голые ветви.
Вопреки стремительному приближению нового года, над Львовом повис
ноябрь.

Элеонора гуляла по кладбищу, пределы которого ей вовсе не хотелось
покидать в такую погоду, как бы странно это не звучало. Она стала лучше
понимать хозяйку «Маятника» и теперь воспринимала окрестности скорее как
задний дворик, а не место скопления траура, или чего бы там ни было ещё.
На улице было мерзко и холодно, периодически валил мокрый снег, что тут
же исчезал при контакте с землей, так что отходить далеко от тепла Нора
не желала, а кладбище -- с его многочисленными холмами и аллеями --
служило прекрасным аналогом парка, когда в девушке просыпалось желание
размять ноги.

Размышляя о странном, если не сказать душевном, разговоре с Агатой и
своей жизни в целом, Элеонора немного взбодрилась, когда пришла к выводу
о том, что то самое, переменчивое и захватывающее, о чём стоило бы
рассказать читателю, происходит с ней прямо сейчас.

Луж в округе прибавилось и, спустившись с холма, наша героиня увидела
небольшую табличку, что выглядывала из сугроба, успевшего значительно
сдать в размерах за последние несколько дней. Надпись гласила:

\emph{«Червей не копать!»}

Девушка иронично улыбнулась этой находке и продолжила прогулку.

Ветер скользил меж надгробий, напевая понятный лишь ему одному мотив,
когда, преодолев немалое расстояние, Элеонора решила вернуться в
«Маятник». Сквозь небесную оболочку теперь просеивался мелкий дождик.
Девушка уже миновала аллею склепов и, дрожа от холода, шагала под
навесом массивных сосен Эллиота. Большие чёрные вороны сделались
регулярными посетителями некрополя. Их крики разносились по всей
территории и, сливаясь с атональным воем ветра, создавали атмосферу
обречённости. Птицы виднелись на склепах, крестах и памятниках, голых
ветвях и витиеватых калитках из тёмного метала, так что, уловив боковым
зрением какое-то движение справа от себя, Нора сперва приняла его за
компанию очередного ворона. Однако, поддавшись интуиции, она
обернулась\ldots{}

\section*{38}\label{38}
\addcontentsline{toc}{section}{38}

\markright{38}

\ldots{} чтобы увидеть высокого молодого человека, облачённого в чёрное
пальто строгого покроя. Тот стоял подле одного из излюбленных надгробий
Агаты, являвшего собой широкую каменную кровать, украшенную орлами с
гордо расправленными крыльями; её изголовье служило ещё одним ложем, на
котором, поместив головку на руку, вечным сном забылась каменная
женщина. Черты лица и складки её одежд были столь искусно изготовлены,
что при длительном взгляде на фигуру казалось, словно платья мерно
вздымаются от вздохов дремлющей.

Именно это сейчас и делал незнакомец: не отводя глаз, рассматривал
памятник, в то время как Нора разглядывала его. Несмотря на время и
место, молодой человек выглядел так, будто ожидал старого друга.

Заметив чьё-то присутствие, незнакомец обернулся, и Нора поразилась
тому, как он был красив. Тем временем молодой человек слегка улыбнулся и
вскинул руку в приветственном жесте. Элеонора ответила тем же, и
незнакомец вернулся к безмолвному изучению памятника.

Слегка растерянная, наша героиня продолжила идти в сторону похоронного
бюро.

*

Её поезд отходил в два часа и двадцать четыре минуты. За последние годы
поезда стали на удивление пунктуальны. Они прибывали и отправлялись по
расписанию в точности до секунды, так что Норе пришлось несколько раз
свериться с билетом, дабы запомнить, что поезд отходит в четырнадцать
двадцать четыре, а не в четырнадцать двадцать шесть -- она не была
рассеянной, но с детства путала цифры, составляющие в сумме число
десять.

Каменное обрамленное колоннами крыльцо «Маятника» было усеяно воронами,
как и сам дворик похоронного бюро. Птицы разгуливали по оккупированной
территории размеренным шагом, перекрикиваясь скрипучими голосами и
оставляя на талом снегу дорожки хрупких следов. Присутствие девушки их
ничуть не смутило. Пернатые чувствовали себя полноправными хозяевами
кладбища. Один ворон разместился в бесстрашной близости от входной
двери, так что Норе пришлось отодвинуть упрямую птицу ногой, прежде чем
войти внутрь. Та недовольно каркнула и отпрыгнула в сторону, чтобы
мгновеньем позже вернуться на прежнее место.

Затворив за собой массивную дверь, Элеонора попала в знакомую атмосферу
уюта. Внутри было блаженно тепло. Воздух пропитался сосной с тонкой
ноткой кислинки, которой веяло из приоткрытой двери чулана, где
содержимое медных чанов как раз проходило стадию мацерации. Откуда-то
сверху доносились усыпляющие звуки музыки.

Норе вдруг подумалось о том, как же приятно вернуться в «Маятник» с его
высокими потолками, ветхой мебелью и вечной неизменностью. Девушка
улыбнулась, дивуясь этим мыслям: она провела здесь только месяц, но
теперь бюро отзывалось в её душе приятными нотками ностальгии.

Утро близилось к девяти часам. Нора ожидала найти Агату на привычном
месте -- дремлющей в кожаном кресле, но когда девушка принялась
подниматься по широким ступеням, та выплыла из своего кабинета. По всей
видимости, владелица бюро направлялась в сторону винного погреба. Вид у
неё был довольно странный: скулы подведены, а губы пылают алым, очки на
прежнем месте; на затылок сдвинута шляпа федора, из под которой
старательно уложенные волосы волнами ниспадали на грудь, а ниже\ldots{}
Наспех накинутый халат и махровые тапочки, а между зубами зажата
сигарета. Создавалось чувство, словно в спешке Агата успела привести
себя в порядок лишь наполовину, или наоборот -- ещё не до конца
подготовилась к сну, сложно было сказать наверняка, учитывая режим
дамочки.

Та остановилась, заметив присутствие Элеоноры, после чего медленно
кивнула. Вопреки комичному внешнему виду, выражение лица Агаты выдавало
в ней мастера похоронного дела.

-- Кто-то умер? -- поинтересовалась Нора и лишь затем поняла, как
двусмысленно прозвучал вопрос.

-- Моё светлое будущее.

Агата произнесла эту фразу тоном, лишенным каких бы то ни было эмоций, а
затем вдруг разразилась странным, чуть ли ни нервным смехом.

Не знавшая, что сказать, девушка изумленно застыла на лестнице. Агата
поймала её взволнованный взгляд, и выражение лица хозяйки «Маятника»
вновь сделалось непроницаемым.

-- Не переживай, -- произнесла она в привычной холодной манере. -- Всего
лишь ещё один заказ.

\section*{39}\label{39}
\addcontentsline{toc}{section}{39}

\markright{39}

Под конец минувшего года произошло ещё кое-что занятное. Прежде чем
покинуть бюро ритуальных услуг, чтобы сесть в лишенный комфортабельности
поезд, Нора намеревалась вручить Агате её новогодний подарок. Ввиду
распорядка дня последней, девушке пришлось сделать это поздним вечером,
предшествующим дню отъезда.

Очередная церемония прощанья закончилась много часов назад, но, войдя в
гостиную, Элеонора отметила, что лицо владелицы «Маятника» по-прежнему
сохраняло траурный вид.

-- Мой поезд отправляется в четырнадцать двадцать шесть\ldots{} --
начала девушка.

-- Четырнадцать двадцать четыре, -- поправила её Агата.

Нора смущенно улыбнулась. Агата слегка склонила голову таким образом,
что очки мешали понять, смотрит ли она в книгу, или на свою собеседницу.
В любом случае, девушка продолжила.

-- Думаю, к тому времени ты уже будешь видеть десятый сон, так
что\ldots{} -- она помедлила, пытаясь понять, куда же всё-таки смотрит
Агата.

Если та и заметила наличие свертка, который девушка держала в руках, то
виду не подала.

-- Я хотела отдать тебе это прежде, чем отправиться в постель.

Нора сделала пару шагов к креслу, в котором полулежала Агата, и
протянула ей квадратный свёрток. Бледная рука скользнула в сторону
наглухо запечатанного портрета. Её обладательница положила предмет на
журнальный столик и после длительной паузы монотонно произнесла:

-- Благодарю.

-- Не хочешь узнать, что в нём? -- спросила Нора.

Она продолжала улыбаться, не зная, что ещё делать.

-- Позже.

Слегка разочарованная, Нора всё же сказала то, что собиралась.

-- Знаешь, никогда бы ни подумала, что найду себе друга за столь
короткое время. Откровенно говоря, я очень в этом сомневалась. Думала
днями буду сидеть одна в чужом городе, не зная, куда себя деть, но
теперь у меня такое чувство, словно мы знакомы вечность.

-- Давай без реверансов, -- ответила Агата.

-- Ой, да не хмурься ты всё время! -- воскликнула девушка.

Её лицо осветила широкая улыбка, а затем, поддавшись минутному порыву,
Нора наклонилась и обняла подругу. Она указала на принесенный свёрток.

-- Тебе понравится.

Тень улыбки скользнула на лице дамочки в чёрном.

-- Не сомневаюсь.

Элеонора выпустила её объятий и зашагала к выходу. Она остановилась у
самой двери.

-- Мы увидимся через неделю, -- пообещала девушка.

И нарушила обещание.

\section*{40}\label{40}
\addcontentsline{toc}{section}{40}

\markright{40}

Следующим утром Элеонора вновь проснулась раньше, чем рассчитывала. До
выхода оставалось больше двух часов, а то немногое, что она собиралась
взять в дорогу, уже давно было сложено, так что девушка сидела на кухне
с планшетом в руках, гуляя по социальным сетям. Вообще-то она собиралась
выпить чашку кофе и провести первую половину дня скитаясь по городу, но,
умостившись за столом, поняла, что в ближайшее время вряд ли захочет
сдвинуться с места.

День обещал быть пасмурным. Ветер барабанил дождем в окна, а из старого
радиоприемника раздавалась мелодия «Летнего места» Макса Штайнера, столь
поразительно контрастирующая с погодой. Звуки музыки обволакивали
девушку сладостными грёзами и заставляли чувствовать себя счастливой.
Она не могла припомнить, когда в последний раз слышала эту мелодию, но
вскоре добавила её в свой плейлист и, включив повтор, с головой
окунулась в графоманию.

Неожиданно для себя самой, Элеонора преодолела барьер в семь сотен слов.
Воодушевленная музыкой, она продолжала писать, прерываясь лишь для того,
чтобы подлить себе кофе и приближаясь к тысяче -- заветному количеству
исторгнутых на бумагу слов для всех начинающих авторов.

Наша героиня как раз собиралась отхлебнуть бодрящего напитка, когда
атмосферу её душевного покоя внезапно нарушила Агата. Растрепанная, но с
очками на привычном месте, она влетела на кухню в наспех запахнутом
халате, из-под которого виднелись оборки белья. Конечно же, черного.

Определенный эффект произвело и то, что Агата буквально вышла из стены.

«Ещё один тайный ход», -- подумала девушка.

Элеонора так и застыла с чашкой в одной руке и карандашом -- в другой,
вопросительно глядя на подругу. Та замедлила шаг и, подойдя к плите,
повернула ручку. Только теперь Нора заметила, что духовка была включена.

-- ДОВОЛЕН ТЕПЕРЬ? -- крикнула Агата, обращаясь к стенам. Она
повернулась к ошарашенной Норе и, пытаясь убавить злость в голосе,
произнесла: -- Это Ласло, кулинар хренов! Поставил в духовку свои
блядские артишоки и съебался к чёртовой матери, завидев твоё
приближение.

Норе понадобились силы, чтобы не расхохотаться. Она постаралась
вспомнить что-нибудь печальное. В голову пришли кадры из
«Землян»\footnote{«Земляне» -- документальный фильм Шона Монсона о
  проблеме жестокой эксплуатации животных людьми для различных целей.} и
это помогло. Девушка открыла рот, собираясь как-то успокоить подругу, но
та тут же выставила вперед руку, демонстрирующую фарфоровую кожу.

-- Нет! Не заговаривай со мной, даже не думай! Я безумно зла, если меня
будят раньше времени, я же говорила тебе, когда мы\ldots{}

Концовка фразы исчезла за пределами кухни вместе с Агатой прежде, чем
Нора успела опомниться.

Девушка пожала плечами. Мысль о том, что боязливый Владислав к тому же
знатный кулинар не могла не вызвать у неё смех.

«Такие дела!» -- написала она, тем самым завершая отрезок текста
размером в тысячу слов.

\section*{41}\label{41}
\addcontentsline{toc}{section}{41}

\markright{41}

Узнав о нелюбви подруги к интернет-общению и социальным сетям в целом,
Нора сочла её странной и старомодной, но ни капельки не удивилась. Это
привело к тому, что девушка проводила вечера за чтением и написанием
электронных писем. Такое занятие пришлось ей по душе и Нора даже
удивилась, почему никогда прежде не развивала свои писательские навыки
подобным образом.

Как и разговоры с Агатой, её послания по-прежнему не касались личных тем
и пестрили различными фактами, но Элеонора с облегчением заметила, что
теперь среди них встречались и те, что не имели отношения к тафофилии.

\begin{quote}
\textbf{\emph{Вчерашняя церемония и вовсе лишила меня последних сил.
Хоронили какого-то цирюльника. Проститься явилось человек сорок, не
более. Выполнив привычный ритуал «посмотрел-положил цветочки-поплакал»,
народец начал расходиться, и всё бы ничего, но осталась престарелая
дамочка с пугающе высоким начёсом, которая никак не желала уходить. Ты
ведь знаешь, как сложно мне бодрствовать после полудня! Я могу снести
ещё часок-другой, а именно столько должна длиться церемония прощанья, но
эта особа опустилась на колени у изголовья гроба и продолжала реветь
даже когда поздние сумерки сменились кромешной темнотой. В подобной
ситуации на Ласло полагаться не стоит, сама понимаешь, а я ведь не
бармен, чтобы торчать здесь до последнего клиента! Наёмные ребята (я
имею в виду, могильщиков) тоже замучались ждать и ушли домой ещё до
заката, предложив этой плаксе самой нести гроб. }}

\textbf{\emph{Короче говоря, я устала ждать пока ситуация прояснится.
Поднялась в свои комнаты, но поняла, что уже не смогу сомкнуть глаз.
Знаешь, как это бывает, когда бодрствуешь слишком долго -- второе
дыхание, или вроде того. Мертвецу, видимо, придётся провести под крышей
«Маятника» ещё одну ночь, дожидаясь возвращения копателей. Они наотрез
отказываются выносить гроб под покровом ночи. Говорят, это дурная
примета\ldots{} Странно, что прежде я слышала подобную ересь только в
отношении мусора.}}

\textbf{\emph{Сейчас время близится к полуночи. Я сижу на кухне,
наблюдая как Ласло возится с ужином и наполняю желудок кофе, пытаясь
хоть как-то побороть усталость. Сквозь стены до моих ушей продолжает
доноситься плачь этой полоумной. Кажется, никакой пост-панк не способен
его заглушить.}}

*

\textbf{\emph{Эта женщина наконец ушла. Её стоны сменились шумом ветра.
Всё ведет к тому, что над городом вот-вот разразится гроза. В голове у
меня прояснилось и появилось резкое желание рисовать. Чувствую, что ещё
не скоро смогу уснуть, но руки уже тянутся к кистям, так что вскоре я
вынуждена буду прервать своё письмо. Только выпью ещё чашечку
чего-нибудь горячего.}}

\textbf{\emph{Кстати, ты знала, что первая чашка кофе была предложена
европейцам в 1626-м году в Риме? Другие источники сообщают, что кофе был
привезён в Европу в 1615-м году, а чай в 1610. Так или иначе, эта
информация не приближает нас к ответу на мой вопрос: как люди вообще
жили до этого, а главное -- зачем?»}} \textgreater{}\\

\textbf{\emph{А.}}
\end{quote}

На этой философской ноте и заканчивалось письмо Агаты. Нора представила
её, рисующую свои вычурные полотна под грозовым покровом, ведущую
соседство с покойником. Девушка рассмеялась подобному Арт Хаусу и
взялась писать ответ.

\section*{42}\label{42}
\addcontentsline{toc}{section}{42}

\markright{42}

Минула неполная неделя с момента отъезда Элеоноры, так что в «Маятнике»
её ждали со дня на день. Жизнь текла своим трагическим чередом, не
привнося сюрпризов в будничность обитателей похоронного бюро, -- как
мёртвых, так и живых, -- чем последние оставались, в общем-то, довольны.

По крайней мере, внешне.

После недавнего возобновления работы «Маятник» стал местом проведения
шестнадцати прощальных церемоний, что было прекрасным результатом,
особенно во время зимних праздников.

-- Взгляни на них, -- произнесла Агата, стоя у окна своей спальни.

На улице было сыро и ветрено. Шёл снег с дождем. На крыльцо бюро
ритуальных услуг высыпалась группка людей. Вздрагивая и скорбно
сгорбившись, они следовали за шестью мужчинами, что несли гроб. Агата
наблюдала за этой картиной, пока похоронная процессия не обогнула дом и
скрылась в недоступной взору части кладбища.

-- Лица словно с обложки Sopor Aeternus\footnote{Sopor Aeternus \& The
  Ensemble of Shadows (лат. Вечный сон и англ. Ансамбль теней) --
  немецкий проект, единственным участником которого является Анна-Варни
  Кантодеа, создающий композиции в таких жанрах как дарквэйв, неофолк и
  готик-рок с элементами средневековой музыки.}. Разыгрывают трагедию,
которой сами же верят. Я слышала, как одна из женщин за глаза порицала
свою товарку за то, что её одежды недостаточно чёрные, а скорее тёмно
синие\ldots{} Смешно! Как будто это имеет значение.

Владислав многозначительно кивнул.

Он оторвался от ноутбука и внимательно изучил выражение лица подруги. В
последние дни Агата была слишком эмоциональной. Его это беспокоило.
Ласло вёл общение с немногими, -- да что там, их можно было пересчитать
на пальцах одной руки -- но! Он всегда разбирался в поведении своих
близких. Чаще всего знал, что что-то гложет людей ещё до того, как они
сами это осознавали.

Агата сделалась нервной. Два дня назад ей вдруг не угодила лампа,
которая годами стояла в кабинете, а вчера он нашел владелицу «Маятника»
убирающей снег с крыльца, чего за той прежде никогда не наблюдалась.
Агате не было особого дела до общества, но теперь она говорила о нём
подозрительно часто.

-- Все эти сегодняшние тренды! Патриархальная система питается за счёт
продаж косметики и бритвенных станков, заставляя всех вокруг считать
волосы на ногах чем-то постыдным. Выбрасываем время и деньги, а затем
слегка калечим естественное состояние своего тела -- вот что мы делаем,
дабы соответствовать нынешним представлениям общества.

Очки Агата сняла как только вошла в спальню и сейчас Ласло мог видеть её
глаза. Тщательно скрываемое ненавистью, в них всё же сквозило отчаянье.

-- А смерть, -- продолжала Агата. -- Ты видел, что они сделали со
смертью?!

Владислав отрицательно качнул головой, хотя этого, в сущности, и не
требовалось: Агата тут же продолжила свой монолог.

-- Человечество только начало подавать признаки здравого мышления,
отходя от Библии, и адом нас не запугать, но нет! Теперь каждому
встречному-поперечному напоминают о смерти! По телевиденью реклама
станков для бритья чередуется с роликом, в котором зрителя просят
позвонить своим старикам, ведь кто знает, когда пробьёт их час\ldots{} А
затем на экране, как бы невзначай, тут же появляется их новое средство
для профилактики сосудочно-сердечных заболеваний. Чёрт, да смертью
теперь торгуют!

Она замолчала и потянулась за сигаретой.

-- Пойду приготовлю что-нибудь на ужин, -- сказала Агата и скрылась в
дверном проёме.

Владислав давненько не слышал, чтобы в одном разговоре поднималось такое
количество глобальных вопросов, но последняя фраза и вовсе казалась
тревожным звоночком.

Он подхватил ноутбук и последовал за подругой.

\section*{43}\label{43}
\addcontentsline{toc}{section}{43}

\markright{43}

\begin{quote}
\emph{Дорогая Агата,}

\emph{Погода разыгралась не на шутку. Снег валит не прекращая уже
который день, из-за чего света нет во всей округе. По вечерам я включаю
генератор, чтобы зарядить технику и ненадолго почувствовать себя белым
человеком, но тепла он практически не даёт. Дом теперь напоминает
морозильную камеру и мне приходится напяливать по три свитера, чтобы не
превратиться в айсберг.}

\emph{Я уже упоминала, что мой коттедж находится в некотором отдалении
от города. Так вот, дорога, что ведёт к трассе, превратилась в снежный
тоннель, преодолеть который не в силах даже мой внедорожник. Если бы во
мне и нашлось мужество пробраться сквозь ледяные заросли и подняться к
трассе, шансы достигнуть городской черты едва ли утешительные -- ввиду
снегопада транспорт не ходит уже которые сутки, а садиться в кабину к
дальнобойщику я пока не готова.}

\emph{В первые дни заточения меня тешила мысль о том, что подобное
уединение -- у камина и при свечах -- лучшая атмосфера для творчества.
Увы! Я по-прежнему не написала ни одной достойной строчки, не считая
этих писем. По большей части я разбираю старые вещи, валяюсь с собаками
и читаю книги, а по вечерам смотрю что-нибудь. Ты скажешь, что это
прокрастинация, но я действительно не могу ничего написать. То есть, я
пробую всё время, но каждый раз натыкаюсь на глухую стену, чей рост
прямо пропорционален моим попыткам. Кажется, чем сильнее я стараюсь, тем
менее живыми становятся мои собственные персонажи. Я больше не вижу их
так ясно, как видела тем вечером в поезде, когда впервые взялась за
книгу. Эта проблема преследует мои писательские амбиции на протяжении
всех тех лет, которые я пытаюсь творить.}

\emph{Скажи, что ты об этом думаешь?}

\emph{С любовью,}

\emph{Нора}
\end{quote}

Элеонора отправила письмо. Она обвела печальным взглядом свою комнатку с
низким скошенным потолком: весь пол был устлан скомканной бумагой,
яростно отброшенной в стену -- её неудавшимися попытками.

В голове мелькнула мысль: а не поджечь ли эти рукописи прямо здесь, не
подбирая их с ковра? Девушка тут же прикинула, как долго до её дома
будет добираться пожарная машина, в такую-то погоду, подумала о своих
собаках и поднялась, чтобы разогреть ужин. Она удивилась, услышав звук,
сопровождающий прибитые нового письма.

Ещё больше она удивилась, развернув входящее сообщение: приветствия были
опущены, как и подпись, а само послание содержало лишь одну фразу:

\begin{quote}
\textbf{\emph{«Твои персонажи ругаются матом?»}}
\end{quote}

-- Чё? -- произнесла Нора, вскинув брови.

От кого, от кого, но от Агаты она ожидала большего красноречия. Ну, или
хотя бы очередной истории о каком-нибудь покойнике. Девушка вскинула
брови, с недоумением глядя на эту фразу.

Тем не менее, она взялась писать ответ.

\begin{quote}
\emph{Нет, мои персонажи не ругаются матом. По крайней мере на страницах
романа, но почему ты спрашиваешь? Откровенно говоря, я ожидала от тебя
большего красноречия, ведь это ты у нас любительница старомодных
посланий. 😊}
\end{quote}

Ответ не заставил себя долго ждать.

\begin{quote}
\textbf{\emph{Пошути так ещё раз и получишь современное послание по всем
известному адресу. Если тебе от этого станет легче, можешь представить,
что я крайне озабочена твоими писательскими проблемами, а потому весь
вечер сижу на почте и телеграфирую тебе незамедлительно.}}

\textbf{\emph{Теперь перейдем к сути. Говоришь, твои персонажи не
ругаются матом? Тогда, боюсь, у них никогда не было прыщей и волос на
ногах, а походы в туалет этим несчастным вообще строго-настрого
запрещены. Я веду к тому, что ты пытаешься идеализировать своих героев и
выбрала для этого самый идиотский способ: ты идеализируешь их в
собственных глазах. Пускай страницы твоей книги и не будут усыпаны
первосортными ругательствами, -- не каждому из нас дано стать Буковски и
ты правильно делаешь, что боишься опошлить этим свою историю -- но в
мыслях-то ты должна понимать, что пишешь о людях, а не о сказочных
героях. О людях, которые ссорятся, ругаются матом, мастурбируют,
выдергивают волоски над верхней губой и опорожняют по утрам свои
желудки.}}

\textbf{\emph{Применяй это где и сколько посчитаешь нужным, но не
игнорируй.}}
\end{quote}

Это послание Нора перечитала несколько раз и, каким бы смешным в своей
серьёзности не казался текст, девушка пришла к выводу, что в нём и
впрямь есть смысл. Сама мысль о том, что Ричард (к тому времени она уже
определилась с именем главного персонажа) подтирает задницу прежде чем
отправиться завтракать, смущала её и никогда прежде не приходила в
голову.

\begin{quote}
\emph{Думаю, здесь ты права. Я воспользуюсь твоим советом, чтобы придать
своим персонажам более\ldots{} живой вид, что ли, но я так и не знаю,
как взяться за перо с прежним энтузиазмом. Я имею в виду, как вообще
отыскать нужные мысли и построить сюжет, когда в моей голове вдруг
простираются бесплодные земли?}
\end{quote}

\begin{quote}
\textbf{\emph{Ты спрашиваешь у меня совета потому, что я много читаю?
Это не совсем логично, дорогуша. В литературе я всего лишь наблюдатель.
Потребитель, если угодно. Я многое беру, но ничего не возвращаю обратно.
Мои советы -- лишь взгляд со стороны, подкрепленный мнениями различных
авторов, чьи эссе я читала прежде. Мне и сочинение в школе никогда не
удавалось написать таким, каким его хотели видеть. Не думаю, что сама я
смогла бы написать хотя бы треть книги. Стиль, сюжет, тема, предыстория,
завязка, кульминация, развязка, цель, вывод -- слишком много рамок, в
которые приходится вгонять свои мысли. Мне это не по душе.}}

\textbf{\emph{Я могу сказать, что не так в твоей книге, выступая в роли
читателя. Однако, если тебя интересует сам процесс создания -- это не ко
мне. Ознакомься с эссе По или Уайльда, в них ты найдешь много полезного,
но для начала прочти первое, что должно бросаться в глаза другим авторам
при виде библиографии Стивена Кинга, а именно `Как писать книги'.}}
\end{quote}

\section*{44}\label{44}
\addcontentsline{toc}{section}{44}

\markright{44}

Возможно ли, чтобы человек сгорал от душевных терзаний, но вместе с тем
был смиренней покойника? Владислав старался не сводить глаз с подруги и
чем пристальнее он наблюдай за Агатой, тем сильнее убеждался в том, что
буря не за горами.

Бюро ритуальных услуг пришлось возобновить работу в день смерти Валерия,
вне зависимости от того, хотели ли этого его обитатели: вопрос и не
поднимался. Однако, с виду Агата была не против. По утрам она беседовала
с клиентами, сидя у изголовья массивного стола красного дуба, а в
полдень обычно открывала церемонию прощанья. Её движения были плавными,
голос практически загробным в своем спокойствии, а лицо не выражало
ровным счётом ничего.

И всё же, такие мелочи как внезапное упоминание в разговоре давно
забытых тем, или расчистка снега на крыльце, зарождали в голове
Владислава подозрения, удерживая его в напряжении.

Агата никогда не пыталась стать душей компании, постоянно витала в свих
собственных готических облаках и даже не пыталась скрыть этого, но в
начале второй недели января набиравшая силу рассеянность стала пугать.
Стоило хозяйке «Маятника» взять в руки кисти, и она не могла сомкнуть
глаз, пока картина не была завершена. Исключение составлял разве что
портрет Кристи, с которым Агата не могла справиться уже который год. Ей
удалось не думать о картине и вернуть себе сон, но теперь дамочка
изображала иные сюжеты: радостные моменты в мрачных тонах -- отрывки из
прошлой жизни, которые Ласло с трудом мог вспомнить.

В отличие от Агаты.

Лишившаяся сна и пребывающая в океане тревог, понятных лишь ей одной,
Агата писала ночами и работала днем. Подобное состояние омрачало её и
без того безрадостный вид и вскоре хозяйка похоронного бюро начала
напоминать собственных клиентов. Несколько раз Владислав находил её с
чашкой чая в руках, бесцельно смотрящую в окно, или стену. Последнее не
играло роли, ибо взгляд девушки был более чем отсутствующий. Она
забывала затушить сигарету и не замечала этого, даже когда окурок
обжигал хрупкие пальцы. Одним пасмурным утром Агата позабыла надеть
очки, спускаясь вниз для проведения очередной церемонии, и поняла это
лишь вечером, по привычке собравшись их снять.

В сущности, Владислава можно было назвать невероятно глупым и
одновременно невероятно мудрым человеком. Он долгое время изучал
поведение подруги, пытаясь понять, что здесь предпринять.

Продолжая размышлять над этим вопросом, Ласло сидел в своём стенном
укрытии и наблюдал за тем, как Агата рисует чьи-то глаза на лиловом
фоне. Так прошло несколько часов, когда девушка вдруг отложила кисти и
сделала несколько шагов назад, желая осмотреть полученный результат.

-- Так, что же мне делать? -- спокойно произнесла Агата.

Она и головы не повернула, но вопрос прозвучал так, словно все это время
друг находился подле неё и они болтали без умолку. Ласло помедлил с
ответом, предположив, что Агата беседует сама с собой, что было бы не
впервой, но, та позвала его по имени. Причем использовала полную версию
-- очередной тревожный звоночек.

-- Ничего, -- ответил Ласло, и вышел из укрытия, слегка пристыженный
внезапным разоблачением.

-- Мы должны прекратить это, Ласло. Иначе я просто сойду с ума. Так не
может продолжаться, неужели ты не понимаешь, к чему всё идет?

Агата повернулась к собеседнику, и тот мог видеть, что её шея и лицо
перепачканы краской. Она глубоко вздохнула и добавила:

-- Снова.

-- Ничего не предпринимай, -- повторил Владислав. -- Иначе всё
разрушишь\ldots{}

-- Но я больше не могу жить этим днём сурка!

-- \ldots как той весной.

Агата вздрогнула.

Упоминание «той весны» мигом охладило её пыл. Девушка обхватила себя
руками и с тоской посмотрела на Владислава.

-- Ты прав, -- тихо сказала она.

Ласло положил руку на плечо подруги, и та накрыла её своей ладонью. Так
-- подавленные и отчаянные -- они простояли несколько минут, не проронив
ни слова.

-- Пойду, -- сказала Агата.

Она открыла рот, собираясь ещё что-то добавить, но, видимо, передумала.

Владислав кивнул.

-- Ты привыкнешь, -- пообещал он. -- Я же привык. В конце концов, это
тянется не первый год.

Агата медленно покачала головой.

-- Это неправильно, -- вновь заметила она, обращаясь, скорее, к себе
самой, после чего вышла из мастерской.

Владислав продолжил стоять в средине комнаты, разглядывая пятна краски,
отпечатавшиеся на полу и стенах, эскизы и художественные инструменты --
ему нравилось находиться в здешнем творческом хаосе. Ласло понимал,
почему подруга стала проводить свои вечера за мольбертом: царившая в
мастерской атмосфера действовала на него умиротворяющее.

Взгляд Ласло остановился на картине, над которой Агата только что
трудилась. Местами краска не успела высохнуть, отчего изображение
казалось слишком ярким, но Владиславу оно всё равно нравилось. С
картины, полные доброты и терпения, на него смотрели изумрудные глаза.

Это были глаза Элеоноры.

\section*{45}\label{45}
\addcontentsline{toc}{section}{45}

\markright{45}

Двумя днями позже уставшая и сонная Агата плюхнулась в любимое кресло,
намереваясь провести вечер в компании музыки и бутылочки красного
полусладкого. Ладно, двух бутылочек и ещё одной полной лишь на треть,
оставшейся со вчерашнего вечера.

В любом случае, наша героиня наскоро расправилась с вчерашней бутылкой и
тут же переключилась на новый запас. Она нехотя взяла в руки ноутбук и
принялась набирать письмо, параллельно потягивая винишко и сокрушаясь о
временах славной совиной почты.

\begin{quote}
\textbf{\emph{Элеонора,}}

\textbf{\emph{Сегодня мне, наконец, удалось поспать. Проснувшись, я
вдруг поняла, что от тебя не было вестей уже несколько дней, которые
пролетели для меня как одни сутки\ldots{} Так или иначе, надеюсь, твои
пальцы не отмерзли и не были съедены в этом снежном заточении. Не то,
чтобы я осуждала подобный ход событий, просто тебе явно будет удобней
писать ответ, имея как минимум по три пальца на каждой руке.}}

\textbf{\emph{Ты когда-нибудь слышала о 4D стимуляторе смерти?}}

*\textbf{Сомневаюсь.}

\textbf{\emph{Дело в том, что я внезапно осознала, что мне наскучило
будничное однообразие. В своё время я перепробовала не одну профессию и
лишь в «Маятнике» почувствовала, что нахожусь на своём месте, но даже
занятие любимым делом не гарантирует увлекательного существования. Я
полагаю, для поддержания духа человеку всё время нужно развиваться.
Именно поэтому я решила что-нибудь изменить, взяться за новый проект.
Учитывая специфику моего дела, это едва ли кажется возможным, но, поверь
мне, в смерти порою кроется гораздо больше, чем в самой жизни. Вернее
сказать, последнюю зачастую познают через первую, что открывает передо
мной множество дверей.}}

\textbf{\emph{Вернемся же к предмету разговора. В Китае уже не первый
год успешно работает 4D стимулятор смерти. Формально, это проект,
созданный для смертельно больных и всех тех, кто хочет испытать на себе
воздействие летального исхода, вернув тем самым вкус к жизни. Эта штука
позиционируется как игра, позволяющая почувствовать каково это -- быть
кремированным и вновь родившемся в матке. Если вкратце, участников
помещают в безопасный гроб, а затем в печь, где благодаря сочетанию
света, горячего воздуха и умению разработчиков создаётся подлинное
ощущение горения. Затем клиентам открывается реалистичная проекция
матки, сопровождаемая биением сердца. Им является яркий свет,
представляющий собой тоннель, по которому они должны ползти, осуществляя
перерождение.}}

\textbf{\emph{Нет, я не собираюсь открывать нечто подобное, хотя занятие
увлекательное, ты не находишь?}}

\textbf{\emph{Я всё же намерена начать обдумывать проект на смертельную
тематику, пускай и не требующий подобных технологий. Ты как никто другой
знаешь, как много мне известно о смерти, ведь я не только знакома со
всеми её атрибутами, но и с психологической стороной вопроса. Посему я
хотела бы вести семинары. Прямо здесь, в «Маятнике», я располагаю
идеальной атмосферой для того, чтобы беседовать о смерти, раскрывая не
только поэтические аспекты упокоения.}}

\textbf{\emph{Естественно, мои ученики будут делать надписи на
собственных надгробиях, писать прощальные письма, выбирать для себя
погребальные одежды, воссоздавать пост-мортем фото и подбирать дизайн
гроба, в котором потом должны будут пролежать несколько часов. Однако,
вместе с тем я расскажу о вскрытии, разложении и прочих радостях
посмертных приключений тела.}}

\textbf{\emph{Вот как я это вижу.}}

\textbf{\emph{Даже если откинуть первоначальную идею о том, что оное
занятие призвано напомнить о радостях жизни\ldots{} Глупо нести в себе
столько информации и не делиться ею с другими. Хотя\ldots{}}}

\textbf{\emph{С другой стороны, может мне просто расширить границы
своего виноделия?}}

\textbf{\emph{Возвращайся скорее.}}

\textbf{\emph{А.}}
\end{quote}

\section*{46}\label{46}
\addcontentsline{toc}{section}{46}

\markright{46}

Спустя две чашки чая и бутылку вина Агата захлопнула книгу. Она заметно
взбодрилась, обнаружив, что её ждет новое письмо.

\begin{quote}
\emph{Дорогая Агата,}

\emph{Не знаю, возможно ли это, но, кажется, снегопад стал ещё сильнее.
Боюсь, в ближайшие дни я не смогу тебе писать. Генератор вот-вот
опустеет, а использовать оставшийся в машине бензин я пока не готова,
надеясь, что вскоре снежная буря утихнет и я смогу выбраться из своего
ледяного дворца.}

\emph{Без света здесь жутко. Особенно по ночам. Я подхватила простуду,
так что всё время лежу под тремя одеялами, слушаю завывание ветра,
которое не в силах заглушить даже накрепко закрытые металлопластиковые
окна. Высокая температура заставляет меня дремать практически весь день,
но сон этот едва ли можно назвать здоровым: я просыпаюсь каждый час.
Если не от кашля, то от холода.}

\emph{Ночами я лежу без сна. Слушаю вьюгу за окном и чувствую себя
Боггартом как в этом сериале, помнишь? Буквально слышу у себя в голове
голос Рэдклиффа с его шикарным британским акцентом, который все время
повторяет всего два слова: CAN'T SLEEP}\footnote{Нора говорит о сцене из
  мини-сериала «Дневник юного доктора», основанного на одноименном
  романе Михаила Булгакова.}.

\emph{Хорошо, еды в доме хватит на месяц вперед.}

\emph{Касательно твоих внезапных лекторских порывов, предлагаю обсудить
это лично, но пока лучше сосредоточься на вине.}

\emph{Ты знаешь, я прочла ту книгу Кинга. В общем-то мне понравилось, но
и в противном случае я бы не рискнула тебе говорить обратное -- мало мне
проблем что ли?\\
😊 }

\emph{Так вот, я не совсем понимаю, как именно это должно мне помочь?
Первая половина книги состоит из его воспоминаний, местами
довольно\ldots своеобразных; вторая -- это сборник лингвистических
советов, едва ли применимых к русскому языку в принципе.}

\emph{Объяснишь?}

\emph{С наилучшими пожеланиями,}

\emph{Нора}
\end{quote}

На этот раз ответ Агаты вновь явился незамедлительно.

\begin{quote}
\textbf{\emph{Согласна относительно вина. Сейчас как раз занимаюсь
тестированием.}}

\textbf{\emph{Боггарт -- это привидение из «Гарри Поттера», хотя в свете
последних событий я бы не удивилась, узнав, что ты чувствуешь себя
именно им, а того доктора звали Бомгард.}}

\textbf{\emph{Что касается книги, обрати внимание на первую половину.
Воспоминания Кинга -- прекрасный образец осознания череды событий,
которая сделала его самим собой и привела к занятию литературой в
принципе. Я уже говорила тебе это: перебери в голове свою жизнь,
углубись в воспоминания, разложи их по полочкам и пойми, зачем вообще ты
хочешь писать. И ситуация прояснится.}}

\textbf{\emph{Признаюсь, меня слегка беспокоит твоя ситуация.
Выздоравливай, а затем убирайся к чёрту из этой глуши. Как можно
скорее.}}

\textbf{\emph{А.}}
\end{quote}

Агата несколько раз удостоверилась, что сообщение доставлено, но ответа
не последовало.

-- Должно быть спит, -- предположил Владислав, с тревогой наблюдавший за
тем, как подруга ежеминутно проверяет почту.

Агата одарила его красноречивым взглядом, выражавшим посыл по тому
самому хорошо известному адресу и Ласло, который буквально избегал
конфликтных ситуаций, поспешно ретировался на свой этаж.

К рассвету хозяйка «Маятника» опустошила последнюю бутылку.

Ответа от Норы по-прежнему не было.

*

Не было его и следующим утром.

И днем позже.

\section*{47}\label{47}
\addcontentsline{toc}{section}{47}

\markright{47}

Моросил мелкий дождь. Небеса объединяли в себе палитру сине-серых
оттенков, переходящих в насыщено лиловый, и теперь походили на огромный
синяк, охватывающий город.

Дело шло к закату, когда мальчик вышел из здания вокзала. Он вновь
проспал весь день и теперь с интересом разглядывал новое место. Это уже
превратилось в своего рода традицию: сойдя с автобуса или электрички,
Гек отсыпался в зале ожидания, а затем проводил целые сутки, гуляя по
новому городу. Подобные прогулки приносили нашему герою духовный подъём,
но вместе с тем и подавленность -- неотъемлемую спутницу одиночества.
Гек понимал, что ведёт себя крайне расточительно, но ничего не мог с
собой поделать. Куда бы он ни ступил -- мальчик не чувствовал, что
находится на своём месте, а потому продолжал колесить по стране.

Одиннадцать городов спустя Гек обнаружил, что от его начальных запасов
осталось чуть больше десятой части, и начал ездить зайцем. Автобусы для
такого не годились, а вот электрички -- вполне. К тому же это оказалось
ещё веселей, не надо было думать, какой город окажется следующим:
садишься в первом вагоне и просто покидаешь электричку на той станции,
где входит контролер. Однажды Гек не успел выйти до начала движения, и
ему даже пришлось прятаться от проверяющего на крыше электрички.

Такое вот веселье.

-- Серьезно, -- сказала Агата. -- Хватит.

Гек прервал свой рассказ и вопросительно взглянул на молчаливую
собеседницу.

-- Я миллион раз слышала эту историю. И каждый раз всплывают новые
детали, в достоверности которых я очень сомневаюсь. Дай мне нормально
выпить.

-- Щас девять утра.

-- Вот именно. Скоро придут клиенты.

Мальчик театрально закатил глаза.

-- Ну што ты начинаешь. Хотя бы дослушай до того момента, где я говорю,
какого цвета были сумерки в мой первый день во Львове\ldots{}

-- Пепельно бледные.

Жест с глазами повторился. Затем Гек печально вздохнул.

-- Тебе просто нужна новая история, -- мягко произнесла Агата.

-- Та мабудь.\footnote{Да, наверное.}

-- Ну, так иди и твори её. И дай мне, наконец, спокойно выпить.

-- Думаешь, мне есть про што рассказывать? -- спросил мальчик, пропуская
мимо ушел последнюю фразу Агаты. -- Нора всегда слушает мои истории. Но
только старые. Ей они правда нравяца.

-- Ей понравятся и новые. Кстати, а почему ты разговариваешь с Элеонорой
исключительно на украинском?

-- Ну, во-первых, это красиво, -- с улыбкой начал Гек.

-- А во-вторых?

-- А во-вторых мне нравица её реакция. Кажеца, она не всё понимает, но
очень стараеца этого не показывать.

Агата ухмыльнулась винному бокалу.

Мальчик не унимался.

-- Но\ldots{} Думаешь, мне ещё найдеца про што рассказать?

Теперь пришла очередь Агаты вздыхать. Она поправила очки и
воспользовалась этой возможностью.

-- Тебе только двенадцать.

Мальчик ничего не ответил. Он вытащил из кармана старый портсигар
Валерия и принялся сосредоточенно лепить самокрутки.

Минут десять спустя, когда Агата уже решила, что может отдохнуть от
бесконечной болтовни своего юного друга, тот вновь поднял голову и
нарушил долгожданную тишину.

-- Двенадцать лет -- это так скучно, -- сказал Гек и затянулся.

\section*{48}\label{48}
\addcontentsline{toc}{section}{48}

\markright{48}

Владелица «Маятника» стояла у входа, наблюдая за тем, как люди (один в
гробу, остальные своим ходом) покидают похоронное бюро. С гостиной
второго этажа доносилась спокойная музыка, и Агата пыталась понять, что
же всё-таки удивляет её больше: внезапное наличие джаза в этих стенах,
или то, что Владислав вдруг обнаружил своё присутствие, не дождавшись
пока дом опустеет.

Она решила, что склоняется к последнему варианту и в смятении поднялась
по ступеням.

Гостиная была окутана теплом, излучаемым камином, который растопили
впервые в этом году. Журнальный столик оказался уставленным узкими
винными бутылками: по две каждого сорта. Рядом виднелась стопка
шоколадных плиток.

Агата ещё раз обвела взглядом открывшуюся картину.

-- Тааак? -- протянула она.

Владислав стоял спиной к входу и, склонившись над проигрывателем,
тщательно перебирал пластинки. Он обернулся, и Агата увидела, что друг
держит в руках аптечку. Не ту, которую можно было найти в ванной комнате
-- её собственную, обычно спрятанную под кроватью.

Это насторожило девушку больше всего.

Агата мало что понимала в происходящем. Она смотрела на друга,
разрываясь между недоумением и желанием рассмеяться. Ласло, видимо,
испытывал нечто похожее, потому что с минуту они стояли молча.

Первой заговорила Агата.

-- В чём тут дело? -- полуулыбаясь спросила она.

Владислав не отвечал. Склонив голову, он блуждал взглядом по комнате.
Так прошла ещё минута.

-- Присядь, -- наконец, выдавил из себя Владислав.

Он вновь притих. Агата знала, как временами бывает сложно заставить
Владислава говорить. Как-то он молчал почти две недели, а одним утром
просто вошёл в гостиную и сказал, что ему нравится игравшая там песня.
Со стороны Агаты прозвучал вполне логичный вопрос «Почему ты молчал все
это время?», на что Ласло ответил: «Потому что только в тишине вселенная
даёт ответ!»

В общем, Агата не стала спорить и послушно опустилась в кресло.
Владислав вручил ей аптечку, сел рядом и открыл ноутбук.

-- Тебе это не понравится, -- предупредил он.

Агата вопросительно вскинула брови.

-- Тут пришло письмо\ldots{}

-- И?

Ласло ничего не ответил. Он лишь повернул компьютер так, чтобы подруге
был виден текст письма.

Письмо оказалось от Норы. Короткое, написанное впопыхах, оно заставило
Агату содрогнуться даже при повторном прочтении.

\begin{quote}
\emph{так темно и холодно. а хуже всего пустынно}

\emph{я ничего не вижу вчера я слышала шаги но здесь никого нет и не
было}

\emph{кроме как за дверью}

\emph{точнее, в дверном проёме}

\emph{давно}

\emph{голоса вокруг везде они а потом снова белая тишина}

\emph{не думаю что здесь когда нибудь вообще был дом слишком сыро}

\emph{холодн я видела так много}

\emph{сколько тебе лет7 снова.каждый раз}

\emph{неужели я никогда не закончу свою книгу?}
\end{quote}

\section*{49}\label{49}
\addcontentsline{toc}{section}{49}

\markright{49}

Над «Маятником» повисла напряженная атмосфера. Казалось, даже солнце
спешило скрыться за горизонтом. Пламя в камине погасло, а свет в
гостиной так никто и не включил, так что теперь комната медленно
погружалась в полумрак, наполняясь серыми оттенками.

«Рапсодия в чёрно-голубых тонах» началась сначала. Комната виделась
девушке отрывком из чёрно-белой ленты, и звучание мелодии лишь усиливало
это впечатление. Агата съежилась и тряхнула головой, пытаясь отделаться
от гнетущих мыслей. Она подумала о прошлом и о тускнеющих со временем
фотографиях.

Но прошлое не было выцветшим. Набирая яркости, оно в этот самый момент
разливалось вокруг неё красками, заставляя сердце учащенно биться.

Оправившись от потрясения, Агата написала ответ, состоящий всего из двух
слов:

\begin{quote}
\textbf{\emph{Где ты?}}
\end{quote}

Она не раз пыталась звонить, но, по всей видимости, телефон девушки
давно разрядился.

Голова раскалывалась до такой степени, что находиться в собственном теле
было некомфортно. Иных вариантов, к сожалению, не предлагалось. Агата
коснулась лба рукой, посидела так какое-то время, затем сняла очки и
зажгла читальную лампу.

Владислав увидел её пропитанное болью лицо и чуть было не отшатнулся.
Такая обстановка угнетала его, вызывая желание выбежать из гостиной и
укрыться в другом, менее скорбном месте.

Агата не обратила на это никакого внимания. Она машинально потянулась к
бутылке. Погруженный в молчание, Ласло тем не менее подоспел вовремя,
чтобы откупорить сосуд. Он извлек из кармана нож сомелье и вскоре
протянул девушке наполненный вином бокал. Та взяла его, но, к удивлению
своего друга, не сделала ни глотка.

Плечи Агаты поникли. Прикрыв глаза, она сплела изящные пальцы вокруг
тонкой ножки бокала и сидела в застывшем состоянии, пребывая где-то вне
комнаты.

*

Решив, что подруга уснула, Владислав осторожно взял из её руки бокал.
Ему и самому давно хотелось спать, так что он убедился, что вино и
прочие успокоительные меры находятся вблизи от Агаты, и вскоре поднялся
на свой этаж\ldots{}

\section*{50}\label{50}
\addcontentsline{toc}{section}{50}

\markright{50}

\ldots а когда проснулся, Агаты уже не было.

Вместе с ней пропал чемодан и несколько бутылок. В холе Владислава ждала
записка, оставленная тонким почерком, понять который были в состоянии
лишь избранные сего мира.

\begin{quote}
\textbf{\emph{Ласло!}}

\textbf{\emph{Проснувшись, я обнаружила, что сижу в одном из кресел
церемониального зала. Не помню, как и для чего спустилась вниз. Однажды
я проснулась на краю платформы от света, что исходил от приближающегося
поезда. В ту ночь мои руки тряслись так же, как и сейчас.}}

\textbf{\emph{Думаю, мне нужно как можно скорее убраться из этих стен,
иначе я просто утону в пучине безумия, да ещё и тебя прихвачу с собой.
Говорят, собаки уходят из дома в преддверье смерти, а мне всегда
нравились пёсики, ты знаешь. Однако, я просто собираюсь привести себя в
порядок.}}

\textbf{\emph{Не переживай, я отменила ближайшие заказы, так что тебе
никто не должен докучать. Умрут как-нибудь в другой раз.}}

\textbf{\emph{Я не намерена отсутствовать долго, но ты уж присмотри там
за вином. И Элеонорой, если она вдруг даст о себе знать до моего
возвращения. }}

\textbf{\emph{Я должна держать себя в руках, если хочу ей помочь.}}

\textbf{\emph{А.}}
\end{quote}

Владислав развел руками и отправился в погреб.

Плакал его кулинарный конкурс.

\section*{51}\label{51}
\addcontentsline{toc}{section}{51}

\markright{51}

Все это возвращает нас к тому моменту, когда печальная и обеспокоенная
Агата нашла себя стоящей перед полосой бесконечного леса, купающегося в
первых лучах рассвета.

*

Последнее утро в лесу выдалось настолько пасмурным, что, открыв глаза,
Агата не сразу поняла, в каком времени суток она очутилась. Дни здесь
протекали плавно. Преисполненные покоя и заснеженных тропинок, они
погружали нашу героиню в долгожданный покой, уводя от реальности, как
это бывает подвластно вину, или хорошей книге.

Девушка приподнялась на локтях и выскользнула из постели. Будущее
по-прежнему виделось ей туманным и сумрачным, но одно Агата знала
наверняка: она была готова вернуться в «Маятник», и если ситуация не
прояснилась, собиралась всецело погрузиться в работу. К счастью, хоть с
этим проблем не возникнет: всегда кто-нибудь да умирал.

-- В виденьях темноты ночной, -- Агата услышала собственный голос, --
мне снились радости, что были; но грезы жизни, сон денной мне сжали
сердце -- и разбили.\footnote{Цитата из стихотворения Эдгара Аллана По
  под названием «Сон» в переводе Валерия Брюсова.}

Агата выглянула в окно, продолжая повторять четверостишье в своей
голове. Отрывки когда-то прочитанных фраз периодически бессознательно
выплывают из памяти, когда литература является неизменным спутником
твоей жизни -- это была одна из многочисленных странностей Агаты,
привычек, которые раздражали только её саму

-- В виденьях темноты ночной\ldots{} -- прошептала девушка и скривилась,
осознав, что говорит вслух.

Из окна на неё смотрела зима. В уголках стекло взялось морозом и
походило на новогоднюю раму, обрамлявшую старую открытку: снег
по-прежнему покрывал лес и долину, но за ночь над ними сгустились тучи,
делая пейзаж неестественным.

Впечатление было такое, словно над домиком застыл вечер, хотя солнце
встало не более часа назад. Все вокруг предвещало ливень, если не грозу.
Наблюдая за погодой, Агата лишь на мгновение усомнилась в своих
намереньях прогуляться по лесу. Она любила лес во время грозы, пускай и
не вспоминала об этом на протяжении долгих лет.

В сознании яркими вспышками пронеслись воспоминания и в голове вроде как
прояснилось.

-- В виденьях темноты\ldots{} Да ну вашу ж мать!

Агата наспех натянула на себя одежду и покинула дом, опустив привычную
чайную церемонию.

*

О последнем она пожалела сразу же, стоило лишь ступить под холодное
январское небо.

Быстрым шагом Агата шла по нетронутому снегу. Тонкие прутики ветвей
переплетали друг друга, образуя над ней изящную арку. Вблизи гулял
ветер. Девушка ощущала его присутствие, но не чувствовала особого
холода, словно замерзшие деревья могли укрыть её от мороза.

Под ногами хрустел снег, и это вызывало в душе нашей героини
настойчивое, почти детское желание обойти как можно больше пустующих
полян. Ей хотелось весь день только и заниматься тем, что оставлять
следы своих ботинок на блестящем белом покрытии. Чувство это возникло
спонтанно и, по сути, было таким же бессмысленным, как лопанье пузырьков
на оберточной бумаге, когда не совсем понимаешь, зачем это делаешь, но
просто не можешь остановиться.

Так прошло около часа, а может и двух: Агата не особо задумывалась о
времени. Мысли девушки занимало другое -- воспоминания, которые, подобно
костям домино, обрушивались на неё, приводя в движение остальные
частички чётко выстроенной линии. Но это были хорошие воспоминания, так
что девушка не возражала и позволила себе полностью погрузиться в мысли.

Однако, зачастую проблема воспоминаний как раз и состоит в том, что это
счастливые моменты, о которых говорят только в прошедшем времени.

\section*{52}\label{52}
\addcontentsline{toc}{section}{52}

\markright{52}

Как уже упоминалось, за минувшие годы наша траурная героиня успела
позабыть, как сильно любила лес во время грозы. В детстве она неизменно
заполняла плетеную корзину рисовальными принадлежностями, покрывалом,
термосом (а иногда и двумя: с чаем и кофе) и целым самодельным пирогом.
Последний был всего лишь покупными вафельными коржами, щедро
перемазанными вареной сгущенкой, но это отнюдь не делало его менее
вкусным. Весь этот походный набор Агата брала с собой, отправляясь на
длительную прогулку сквозь леса, луга и рощи.

Она обожала дни напролет бродить окрестностями: природа сделалась для
неё не только источником неисчерпаемого вдохновения, но и тем, что время
от времени даровало девочке уйму новых приятных мест.

Находить никем не тронутые кусочки леса нравилось Агате ничуть не меньше
рисования. В такие моменты, обнаружив очередной чарующий дикий уголок,
девочка чувствовала себя так же прекрасно, как и по завершению картины
-- ощущала превосходство. Не над другими, но над самой собой.

Так вот, однажды Агата набрела на тихое лесное озеро, расположенное на
поляне, что была в самой чаще. Едва ли это место и впрямь можно было
назвать нетронутым: на берегу виднелась деревянная беседка, сколоченная
чьими-то заботливыми руками, хотя действие это, судя по виду сооружения,
производилось несколько десятков лет назад.

В тот день она вернулась из лесу гораздо позже обычного, чем немало
напугала дедушку. Сидела под навесом и смотрела, как первые капли дождя
соприкасаются с гладкой поверхностью кристально чистого озера, а затем
лежала, наблюдая сквозь просветы в крыше беседки за тем, как над лесом
разыгрывается настоящая гроза. Подобное зрелище внушало девочке
благоговейный трепет и, вместе с тем, создавало некое чувство уюта,
понятного лишь ей одной. Перед могуществом природы, а также ввиду
уединения, Агата чувствовала себя совсем крошечной и вовсе не была
против этого. Открывшееся ей зрелище словно несло в себе первозданную
тайну, более никому не доступную.

Пожалуй, на самом деле всё обстояло не совсем так. Повзрослевшая Агата
не полностью позабыла вещи, которые когда-то очень любила. Она просто
вспоминала о них лишь в счастливые моменты своей жизни, о которых нельзя
сказать, что их было много. Мысли о лесе возродились в душе девушки в
тот самый момент, когда она засыпала в сладких объятьях Адама, и Агате
безумно захотелось, чтобы они оба очутились в её детском воспоминании --
старой беседке, спрятанной у лесного озера.

И вот, в последний день её пребывания в лесу, заснеженная тропинка
привела Агату к тому самому месту. Не без сожаления она увидела, что
озеро давно высохло и то, что когда-то было тихим водоёмом, сделалось
теперь всего лишь одинокой ямой, доверху засыпанной ветками и мусором.

Озера не было, Адама тоже, да и сама она больше не была той мечтательной
девочкой и думала о себе исключительно как о владелице бюро ритуальных
услуг, которая время от времени не против пригубить стаканчик-другой.

«Как метафорично», -- мысленно произнесла Агата, спустившись на дно ямы.

Она с грустью взглянула на останки бывшего озера, думая о том, что
некоторые воспоминания лучше не пытаться воскресить, и подняла голову,
разглядывая небеса, которые к тому моменту уже начинали светлеть. Слегка
раздосадованная увиденным, наша героиня посетила ещё несколько уголков,
где, как она помнила, должны были серебриться другие озера. Однако, и
здесь ничего хорошего её не ожидало. Вероятно, что-то произошло с
почвой, потому как на привычных глазу местах теперь простирались лишь
высохшие ямы.

В конце концов, тоскующая по ушедшим временам Агата повернула в обратную
сторону. Спустя два с лишним часа она прошла по собственным следам в том
месте, где сквозь заснеженные ветви елей проглядывалась черепичная крыша
небольшого дедушкиного домика, но Агата не замедлила шаг.

Было в этих краях местечко, которое, в противовес пропавшим озерам, уж
точно оставалось неизменным. В детстве и года отрочества наша героиня не
единожды гуляла вдоль этого отрезка земли -- крохотного кладбища,
раскинувшегося на окраине леса, что за последние годы заметно
расширилось.

Благодаря расположению кладбища, да ещё и потому, что с каждым годом
численность жилых домов в деревушке всё сокращалось, (а это не могло не
нагонять ужас на оставшихся в ней обитателей) никто здесь не прибегал к
употреблению хорошо известного глагола «умирать». О покойниках в этих
краях говаривали, что их унесли под сосны.

«Хорошо, хоть что-то остаётся на прежнем месте», -- подумалось Агате,
когда она остановилась у резного железного креста -- по всей видимости,
первой могилы, положившей начало лесному кладбищу.

Пора было возвращаться.

Гроза тем днём так и не показалась.

\section*{53}\label{53}
\addcontentsline{toc}{section}{53}

\markright{53}

Дорога, примыкающая к станции, оказалась насквозь истоптана и
представляла собой болотистое месиво, некогда бывшее снегом --
поразительный контраст с лесными тропинками, так прекрасно
характеризующий итог всего, к чему прикасаются люди.

Полоса облысевших тополей, что тянулась вдоль коричневой дороги,
показалась Агате чуть ли не прекраснейшим из городских пейзажей:
утонченные пики деревьев устремляли свои конечности к небу; последнее
было ослепительно белым, отчего сотни ветвей, испещряющих его края,
казались следами карандаша, которые кто-то оставил на листе бумаги во
внезапном порыве задумчивости, как бывает при затянувшихся телефонных
разговорах.

-- И одиночеством всегдашним полно всё в сердце и природе\footnote{Цитата
  из стихотворения Бориса Пастернака под названием «Осень».}, -- не
отдавая себе в этом особого отчета, произнесла Агата.

Девушка свернула за угол и её ботинок оказался в непосредственной
близости с бездомным мужчиной, спящим под стенами вокзального здания.
Будучи привыкшей к подобным подаркам судьбы, Агата вовремя увернулась и
успела удержаться на ногах.

Она сверилась с часами -- пятнадцать тридцать восемь. В ту же минуту
громкоговоритель откашлялся и объявил, что с первого пути отбывает её
поезд. Очень вовремя -- Агата как раз ставила ногу на его выдвижную
лесенку. Она не выносила ожидание, а потому никогда не опаздывала, но
всегда приходила исключительно в последний момент, секунда в секунду.

Девушка нырнула вглубь вагона и вскоре оказалась в своем купе. Внутри
было холоднее, чем на улице, но её это ничуть не удивило. Здесь всегда
было предлагалось только два варианта: вагоны либо отапливались как
последний круг ада, либо не отапливались вообще -- привычное дело для
украинских поездов.

Пальто Агата снимать не стала. Она повесила шляпу на сомнительного вида
крючок и, поплотнее укутавшись в шарф, надела наушники.

Поезд тронулся.

Агата бросила прощальный взгляд на проплывающие мимо деревья. Над ними,
кружась, парили снежинки.

-- И одиночеством всегдашним полно всё в сердце и в природе.

-- И шо вы хочтэ цим сказать?

Агата перевела взгляд и увидела пожилую проводницу, вероятно, явившуюся
для проверки билета. Женщина была натуральной блондинкой и, хотя
лоснящиеся фиолетовые пятна на переносице и в уголках глаз отлично
подчеркивали бледно фиалковые глаза, она по всем показателям попадала в
группу риска людей, наиболее подверженных заболеванию раком кожи --
такой была первая мысль Агаты.

Вместе с тем девушка уже в который раз отметила, как же жители глубинок
любят коверкать два прекрасных языка, образуя из них раздражающее нечто,
причем в каждом регионе делают это по-своему.

«Пожалуй, проведя с месяцок в поезде, можно с легкостью написать книгу
под названием «100 и 1 способ извратить язык вашего города», подумала
Агата, протягивая проводнице билет, но вслух сказала:

-- Несите чай.

*

По мере её приближения к дому падавшие с небес хлопья снега становились
всё реже, пока окончательно не превратились в дождь. Прислонившись к
окну, Агата лениво наблюдала пейзажи, проносившиеся по ту сторону
стекла. У них был цвет сухой травы и опавших листьев.

Потому как преимущественно это и была сухая трава да опавшие листья.

Открывающаяся глазу картина напоминала те самые шотландские равнины, что
на весь мир славятся своей почвой для соложеного виски.

Агата отложила книгу и прибавила звук. Она слушала Боба Дилана -- в
дороге все слушают Дилана, и это всегда казалось ей самой естественной
вещью на свете.

Дождь обратился ливнем.

Часом позже Агата сошла с поезда.

\section*{54}\label{54}
\addcontentsline{toc}{section}{54}

\markright{54}

Что ж, наша героиня получила свою грозу, правда с опозданием на сутки.

Погода разыгралась над Львовом с той же беспощадностью, с которой
таксист, что вёз Агату в домой, устанавливал цены на свои услуги.
Однако, особого выбора здесь не предоставлялось.

Она без устали ёрзала на заднем сиденье автомобиля, слишком тесном для
ног человека, имевшего рост выше среднего. За залитым дождевой водой
окном с трудом можно было различить родные улочки, что, ввиду грозы,
были неестественно пустынными как для послеобеденного времени. Местами
попадались одинокие трамваи, брошенные мокнуть посреди своего пути;
никто не выходил из бесконечных бутиков и кофеен, а мерцающие огни
витрин и вовсе исчезли -- казалось, мир вокруг забылся коллективным
дождливым сном.

Агата без особого интереса отметила, что за всю дорогу ей не встретился
ни один пешеход. Светофоры тоже были отключены.

Единственным светом оказалась молния. Её серебристые полосы то и дело
пронзали сгустившиеся тучи, и на несколько ослепительных мгновений
возвращали городу его прежний вид.

Водитель сунул в рот сигарету и рискнул приоткрыть окно, но спустя пару
секунд окатил себя содержимым ближайшей лужи. Он чертыхнулся и избавился
от теперь уже мокрой никотиновой палочки. Приподняв бровь, Агата
наблюдала эту занимательную картину из-под своих очков. Последние,
кстати, никак не давали водителю покоя: он то и дело нервно поглядывал в
зеркало дальнего вида.

Девушку едва ли это волновало. Малознакомые люди вообще частенько
удостаивали её любопытными, а то и подозрительными взглядами и причиной
тому не всегда были очки.

Мужчина вновь поднял голову, так что Агата увидела пару заинтересованных
глаз, неотрывно следивших за ней на протяжении нескольких секунд, что,
учитывая погоду и неработающие светофоры, было весьма рискованным
действием.

Подумав об этом, Агата ухмыльнулась: она знала статистику дорожных
смертей и, в отличие от своего спутника, воспользовалась ремнём
безопасности. Как это чаще всего и бывало, кривая улыбка сделала своё
дело. Водитель моргнул и вернул внимание трассе.

Желая избежать внезапно возникшей неловкости, он вспомнил о музыке и
включил радио. Вернее, попытался это сделать, но ни одна из радиостанций
не хотела толком работать. Проигрыватель болезненно извергал из себя
отрывки каких-то песен, более напоминающих эффект зажеванной пленки,
которые тут же поглощались белым шумом.

Последние звуки вышли особенно угнетающими. Такие обычно сопровождают в
фильмах различные паранормальные явления.

«А может и не только в фильмах», -- подумалось Агате.

Мысль показалась ей настолько уместной и нелепой одновременно, что
девушка расхохоталась. Затем поняла, что не считая одноразовой ухмылки,
на протяжении всей дороги молчала с каменным лицом.

Эта мысль, подкрепленная ошалелым взглядом водителя, вызвала у неё новый
приступ смеха. Агата догадывалась, что тут, наверное, кроется что-то
нервное, но решила, что смеяться всё же лучше, чем пускать слёзы, а
потому даже не пыталась себя сдерживать.

Несчастный мужчина (а был он таким ещё с того самого момента, когда
услышал о пункте назначения) ничего не сказал. Он сжал руль с такой
силой, что на пальцах побледнели фаланги.

Водитель не сразу вспомнил о том, что нужно выключить радио и какую-то
часть пути хохот дамочки в чёрном сопровождался звенящим белым шумом,
изредка перекрываемым раскатами грома.

*

Облегченно вздохнув, таксист дал газу и вскоре скрылся за пеленой дождя.

Агату окатило холодной водой, и она тут же перестала смеяться.

«Точно нервное», -- решила девушка и неспешно зашагала ко входу на
кладбище. Спешить было некуда: она и так насквозь промокла за считанные
секунды. Сухой оставалась лишь та часть головы, что пряталась под шляпой
-- и то, надолго ли?

Шагая сквозь ливень, наша героиня думала о том, как же всё-таки приятно
возвращаться домой. Стило лишь увидеть знакомые очертания памятников, и
на душе тут же стало легче. Бюро ритуальных услуг всегда было её
крепостью. Откуда бы Агата не возвращалась и какое бы дерьмо не
переживала, в «Маятнике» всё было на своих местах: пластинка в
граммофоне, вино в погребе, Ласло на чердаке, ну и какой-нибудь покойник
в зале прощаний.

Пропитанная этими мыслями, девушка чуть ли не с любовью наблюдала за
тем, как дождь поглаживал многолетние надгробия. Вскоре из темноты
выросли очертания похоронного дома. За спиной полыхнула молния и Агате
показалось, что она видит чье-то бледное лицо в окне собственной
спальни, но, моргнув, поняла, что за стеклом темно и пусто.

Молния ударила с новой силой. Последующих восьми секунд девушке хватило,
дабы подняться по непростительно скользким ступеням. Она отворила дверь,
и в тот самый момент над кладбищем разнесся оглушительный раскат грома.

Такое вот эффектное появление.

\section*{55}\label{55}
\addcontentsline{toc}{section}{55}

\markright{55}

Ещё одна вспышка небесного света мелькнула прежде, чем девушка закрыла
за собой дверь, и она успела увидеть пустующий холл. Светильники не
работали. Вероятно, где-то замкнуло электричество, благодаря чему
освещения лишился не один квартал.

Как выяснилось позже, в «Маятнике» её ждали все, включая Нору,
вернувшуюся накануне, и парочку новых клиентов, ныне покойных. С
последними Агата решила разобраться утром, полагая, что им уже всё
равно.

Звук закрывающейся двери отдался эхом. Прислушавшись, Агата уловила
резкое движение: кто-то (кто бы это мог быть?) пронесся сначала в одну,
затем в другую сторону. Внутри оказалось слишком темно, дабы хорошо
различать силуэты вокруг, но девушка не сомневалась в источнике шума --
такой глухой звук издавали лишь половицы стенных тоннелей.

Значит, Владислав был дома.

Первым делом она поднялась на третий этаж. Агата облегченно вздохнула
при виде дюжины свечей, мягким светом озаряющих её собственную спальню.
Сейчас ей меньше всего хотелось блуждать в потёмках.

Стена колыхнулась, и в комнату вошел Ласло. Он одарил подругу изучающим
взглядом.

-- Ну и видок у тебя, -- заключил таксидермист, наблюдая за тем, как
дождевая вода стремительно стекает с краев фетровой шляпки. -- Прямиком
из Трансильвании?

Агата не удостоила этот вопрос ответом. Вместо того девушка поспешно
скинула с себя верхнюю одежду и упала на кровать.

Следующие реплики друзья произнесли одновременно.

-- Света нет с самого утра\ldots{}

-- Вода хоть есть?

Они переглянулись.

-- Да, -- ответил Владислав, и, заметив, что девушка поглядывает в
сторону ванной комнаты, вовремя добавил: -- Холодная.

-- Чёрт бы их побрал!

-- Почему? Тебе разве не хватило?

Ласло ткнул пальцем в направлении окна.

-- Попридержал бы ты эти шуточки для своих кулинарных конкурсов, --
посоветовала Агата. -- Если планируешь добраться туда живым.

Она стянула с лица очки и бережно поместила их на прикроватный столик.

-- Я слишком устала.

-- Очень жаль, -- с каменным лицом заметил Ласло.

Владелица похоронного бюро подозрительно взглянула на друга, однако
выражение его лица оставалось непроницаемым.

-- Это ещё почему? -- не выдержала Агата.

Её собеседник улыбнулся.

-- Нора вернулась.

*

Сонливость как рукой сняло.

Наша героиня подскочила. Она тут же возвратила очки на привычное место,
и сделала несколько шагов к выходу, но вдруг остановилась и взглянула на
Владислава, который в свою очередь заинтересованно наблюдал за ней.

-- Твою мать, -- выдохнула Агата. -- И как она выглядит?

Ласло -- социофоб, гурман, таксидермист, косметолог для мертвецов и
вообще веселый парень -- сделал драматическую паузу, прежде чем
ответить.

-- Среднего роста, брюнетка, полноватая, на вид слегка за
двадцать\ldots{}

Агата несколько устала, к тому же, переволновалась, так что не сразу
поняла, что происходит.

-- \ldots{} глаза зелёные\ldots{}

Когда до неё наконец дошёл смысл слов Владислава, девушка вяло подняла
ладонь в останавливающем жесте.

-- Знаешь, меня вот раньше напрягало твоё молчание, но я как-то
передумала.

-- Ой, да ну расслабься ты, -- улыбнулся Ласло. -- Она в порядке. Разве
что немного простывшая.

-- Как и я.

\section*{56}\label{56}
\addcontentsline{toc}{section}{56}

\markright{56}

В то время как Агата наблюдала за небом-матрицей сквозь разрушенную
временем крышу лесной беседки, другая девушка теснилась в узком проходе
поезда, готовясь сойти на следующей станции.

Если верить расписанию, (а оно в последнее время было подозрительно
правдивым) висевшему на ободранной двери туалета, куда сквозь тонны
грязи -- ей хотелось думать, что это только грязь, но обоняние рисовало
чуть менее утешительную картину -- и изодранные клочья сероватой бумаги,
уже был проложен путь самыми отважными пассажирами\ldots{}

Так вот, если верить расписанию, до Львова оставалось неполных десять
минут. Элеонора провела на этом самом месте последний час. Нос девушки
вёл себя странно: он уже который день был заложен, но всё же
предательски различал самые отвратительные запахи. В купе с Норой ехал
поношенного вида мужик, чей образ можно было описать всего одной фразой
-- дед в трусах, а также две тучные женщины с ребёнком, и если первый
пациент всего лишь излучал всевозможные запахи немытого тела, (особого
внимания, конечно, заслуживали ступни) то бабоньки оказались более
оригинальными в выборе средств массового истребления.

Во-первых, они всё время что-то ели. Жадно накинулись на еду ещё до
того, как поезд впервые тронулся с места. Запасы провизии представляли
собой разнообразие всех возможных салатов, содержащих немалую долю лука,
а также целую связку домашней колбасы, и, казалось, были неиссякаемыми.
Они мгновенно заполнили своими тошнотворными запахами и без того затхлое
пространство, вмещающее к тому же пятерых человек.

Во-вторых, проблема была в ребенке. Будучи ещё дошкольником, он
постоянно орал. Орал и ревел. Зачастую из-за того, что хотел есть, но,
как и Нора, не выносил лука, так что основная часть пути могла бы стать
отличным пособием для взращивания будущих чайлдфри,\footnote{Чайлдфри (с
  англ. child -- ребёнок, free -- свободный) -- общее название для
  людей, осознанно не желающих иметь детей.} но! К сожалению, это было
лишь полбеды. Главный сюрприз поездки открылся девушке чуть позже, когда
ребёнок захотел в туалет.

Мамаши, слишком заботливые, дабы позволить ему воспользоваться
общественным толчком и, вероятно, слишком тупые, чтобы подумать об
окружающих, имели при себе пластиковый горшок, больше походивший на
больничную утку. Так или иначе, за те восемнадцать часов, что наша
героиня провела в поезде, горшок наполнялся не один раз, (и не только
жидкостью) но каждый раз, вместе с содержимым, убирался под стол,
припасенный, видимо, до лучших времен.

Смешение вышеупомянутых запахов сводило Элеонору с ума. Она, вроде как,
не была знакома с трупным зловонием, но всегда представляла себе его
именно так. Под конец поездки девушку буквально начало мутить, и виной
тому вряд ли были одни лишь красноречивые ассоциации.

Никогда ещё Нора не жалела о своей вежливости так сильно, как тем
злополучным днем. Она скучала по своему внедорожнику, что остался в
холодном, заснеженном гараже её загородного домика.

Короче говоря, к утру находиться в купе вместе с тем, чтобы никого не
убить, девушке просто не представлялось возможным. Единственным
более-менее продуваемым местом оказался хвост вагона. Туда наша героиня
и устремилась, прихватит чемоданы.

За стеклом с завидной регулярностью мелькали придорожные столбы. Они
упрямо не желали сменяться городским пейзажем. Элеонора смотрела в окно
слишком долго, к тому же, не ела почти сутки, так что вскоре
почувствовала опасное головокружение. Перспектива потерять сознание
рядом с кучкой дерьма казалась ей не самой удачной, и Нора перевела
взгляд.

Ей на глаза попалась перевернутая мусорная корзина, из середины которой
на девушку грустно взирала одинокая прокладка.

«Представь, если бы мужчины испытывали к насилию такое же отвращение,
какое испытывают к месячным\ldots» -- подумалось Норе.

Она не могла вспомнить, где слышала эту фразу. Возможно, в
«Кэрри»?\footnote{Первая опубликованная книга Стивена Кинга, в начале
  которой главная героиня с ужасом переживает появление первых
  критических дней.} Хотя, читала она её слишком давно, чтобы сыпать
цитатами, да и роман, кажется, освещал совсем другие проблемы.

Поезд тряхнуло.

От резкого толчка девушку повело влево, но она успела устоять на ногах,
схватившись за ручку туалета, чего бы никогда не сделала без крайней на
то необходимости.

За её спиной поочередно начали распахиваться дверцы купе. В коридор
высыпали люди.

-- Ну, слава тебе, господи! -- произнесла Элеонора.

Она первой сошла с поезда.

*

Первые несколько минут, шагая прочь от здания вокзала, девушка была
настолько счастлива очутиться на свежем воздухе, что даже не замечала,
как с неба просеивался легкий дождик.

Однако, вскоре капли принялись всё более настойчиво опускаться на её
плечи, а несколько мокрых прядей волос тут же прилипло ко лбу. Дождь
усилился, и игнорировать его сделалось не так-то просто.

Пестрящие огни витрин игриво отражались на довольном, пускай и
подмерзшем, лице девушки, когда та шагала в направлении трамвайной
остановки. Единички ходили каждые пять минут.

Ещё немного, и она будет дома.

\section*{57}\label{57}
\addcontentsline{toc}{section}{57}

\markright{57}

Впервые за долгое время удача повернулась к Норе лицом, а не тем самым
не требующим уточнения местом. Измученная, но счастливая, девушка вошла
в похоронное бюро, двери которого оказались открытыми для всех желающих.
Она успела подняться на свой этаж, выпила растворимый кофе и только,
выйдя из горячего получасового душа, заметила разыгравшуюся за окнами
грозу.

Свет предупреждающе мигнул.

Нора усвоила предыдущий урок. Она зарядила телефон, планшет и три
съемных аккумулятора до того, как в «Маятнике» исчезло электричество,
после чего зарылась в сладко пахнущие одеяла, намереваясь немного
почитать, но тут же отключилась и проспала целых четырнадцать часов.

*

Проснувшись, Нора почувствовала себя лучше и хуже одновременно. Такое
частенько бывает с людьми, ещё не полностью оправившимися от болезни.

За окнами царила темнота, как и в самом «Маятнике». Девушка зажгла свечи
и прислушалась, стараясь уловить признаки чьего-либо присутствия.

Их не было.

Войдя в похоронное бюро, наша героиня ничуть не удивилась, обнаружив,
что дом объят тишиной. Агата в это время спала, а Владислав -- если тот
и был на месте -- ни за что бы не обнаружил своего присутствия.

Теперь же над кладбищем стояла ночь и Элеонора ожидала найти подругу
бодрствующей и с бокальчиком в руке. Той не оказалось ни в одной из
гостиных комнат. Не было Агаты и в собственной спальне, и в кабинете, и
даже в винном погребе. Войдя в кабинет Мирославского, Нора обнаружила
какого-то обнаженного мужчину, спавшего посреди\ldots{} стола?

Ввиду плохого освещения, наша героиня не сразу поняла, что именно видит
перед собой, а, разобравшись с положением вещей, решила, что на сегодня
с поисками достаточно.

Она вернулась на свой этаж и позвонила Агате. Телефон той был вне зоне
досягаемости.

Скорее разочарованная, чем напуганная, Нора вновь забралась в постель.
Она включила очередную серию «Декстера», -- благо 3G работал безотказно
-- и, лишь увидев заставку, поняла, как голодна.

*

Ближе к утру Нора осознала, что не одна в здании -- кухня излучала
умопомрачительные ароматы готовки -- её это успокоило. Путем несложных
умозаключений, девушка решила, что у плиты возится Владислав: она ни
разу не заставала Агата за приготовлением ужина, да и подобная картина
виделась ей нереалистической и чуть ли не гротескной, противоречащей
всем законам физики и устройству современного мира в целом.

Темнота похоронного бюро девушку не пугала, -- его владелица частенько
приглушала свет -- чего нельзя было сказать о тишине. Ниоткуда не
доносилась музыка, и застывшее в воздухе молчание удручало даже больше,
чем покойник, прикорнувший этажом ниже.

Наша героиня вытянулась на диване, (эту ночь она провела в своей
гостиной) пытая понять, чего ей хочется больше: спать, или же есть. Она
провела языком по обветренным губам и решила, что, во-первых, пить.
Девушка берегла оставшееся в технике электричество, так что не стала
пользоваться телефонным фонарем. Накануне она нашла в своем шкафу
парочку газовых ламп, чему ничуть не удивилась, учитывая место действия,
и теперь пробиралась к выходу из комнаты.

По мере её продвижения по коридору, мягкий свет лампы плавно выхватывал
из темноты картины: всё те же портреты писателей, чередующиеся с
кладбищенскими натюрмортами.

Тьма поглотила всё вокруг. В ней виднелась лишь горящая старинная лампа
и очертания руки, что её держала. Это походило на интерфейс какой-нибудь
мрачной бродилки.

Коридор перешёл в крохотный холл, укрывающий лестницу. Нора помедлила.
Она знала, что в другом конце находятся ещё спальни. Те были заперты и,
по всей видимости, пустовали. Девушка уже не впервые столкнулась с
желанием проникнуть в них и осмотреться как следует. Прежде она отметала
эту мысль, зная, что Агата непременно рассердится, застав её за этим
делом. Она не понимала, откуда в ней такая уверенность, ведь это всего
лишь комнаты\ldots{} Просто знала.

Но сейчас Агаты не было в «Маятнике», а Владислав был полностью поглощен
готовкой. Стоя у основания лестницы, девушка слышала слабые отголоски
его веселого пения.

Нора глубоко вдохнула и сделала несколько неуверенных шагов. Вскоре
перед нашей героиней оказалась массивная дверь черного дуба, мало чем
отличавшаяся от той, что вела в её собственные спальни. Дверь была
покрыта толстым слоем пыли, завидев который Нора инстинктивно зажала
рот, опасаясь чихнуть и тем самым привлечь внимание Владислава. Затем
она вспомнила, что за окном совсем не май месяц и с облегчением опустила
руку.

Единственным чистым местом оказалась дверная ручка.

«Кто-то открывал её не так уж давно», -- подумала девушка и, поддавшись
порыву любопытства, повернула дверную ручку.

Та скрипнула, но не поддалась.

«Заперто! А чего ещё можно было ожидать?»

Девушка собиралась предпринять новую попытку, но её рука замерла в паре
сантиметров от ручки. Нора инстинктивно попятилась назад, не совсем
понимая, что именно её остановило. Лишь достигнув лестницы, девушка
осознала это.

Доносившееся из кухни пение смолкло.

\section*{58}\label{58}
\addcontentsline{toc}{section}{58}

\markright{58}

За свои неполных двадцать семь лет Владиславу пришлось от кое-чего
отказаться. В основном из-за социофобии. Юноша ненавидел свою болезнь,
но вместе с тем лелеял её как собственного ребенка. Он лишился вещей,
которые веками радовали человечество, и перспектив, что были способны
воодушевить кого угодно, но только не его самого. Для них Ласло оставлял
в душе крохотный просвет. Он допускал вероятность того, что общение, а,
быть может, даже влюбленность обладали способностью изменить его жизнь в
лучшую сторону, -- временами парень сокрушался о своем одиночестве и
яростно желал от него избавиться -- но зачастую старался не предавать
этому особого внимания.

Коль на свете и были вещи, способные принести ему истинное, а главное
мирное счастье, думал Владислав, так это самые банальные: музыка,
хорошие драмы и, конечно же, кулинария.

Впервые Владислав окунулся во вкусовые эксперименты более двадцати лет
назад. Дело было поздней осенью. Тогда его семья жила в одном из этих
жутких обшарпанных домов по ту сторону кладбища, которые, к тому же, ещё
и плохо отапливались. Ближе к вечеру Влад сидел на кухне. На нем был
спортивный костюм и две пары вязаных носков, но мальчик то и дело жался
к батарее.

-- Довго ще будеш знущатися над цією зупкую?\footnote{Долго ещё будешь
  издеваться над этим супом?} -- с улыбкой спросила девочка-подросток,
которая уже допивала свой чай.

Это была его старшая сестра. Красивая и спокойная, сколько Владик её
помнил.

Мальчик искоса посмотрел на стол, где его ждала тарелка начавшего
остывать супа. Он вроде как был голоден, но не выносил гороховый суп.
Его вкус казался Владу каким-то\ldots{} простоватым.

Наш герой прошелся по поверхности стола взглядом, пытаясь отыскать
что-нибудь съестное взамен противному супу. На глаза попалась лишь
вазочка с красной смородиной. Её Влад тоже не то чтобы жаловал.

Только прошлым утром мальчик слышал по телевизору о том, что минус на
минус даёт плюс. Тогда это показалось ему логичным. Даже слишком
логичным, чтобы рассказывать об этом из ящика. Естественно, что две
прямые чёрточки, если их правильно совместить, образуют знак + с той же
лёгкостью, с какой могут превратиться в знак равенства.

Влад перевёл взгляд на гороховый суп. Затем посмотрел на смородину. И
снова на суп. Теперь мысль о минусах вдруг открылась ему в новом ключе.

-- Сестра, -- позвал он невероятно серьёзным голосом, а потом задумчиво
потер подбородок, словно желая укрепить эту свою серьёзность.

Девочка с интересом взглянула на брата. Большую часть времени тот
говорил на русском. Её это совсем не удивляло. Влад не общался со
сверстниками, да и вообще мало с кем общался без особой необходимости.
Зато он частенько смотрел телевизор, откуда и перенимал манеру общения:
дело было в девяностых и большинство фильмов по-прежнему транслировались
на русском языке.

-- Сестра, что будет, если бросить в суп ежевику?

-- Спробуй!\footnote{Попробуй!} -- весело ответила девочка, стараясь
подавить смешок.

Влад этого не заметил. С напыщенно серьёзным видом, что так плохо
сочетался с фиолетовыми носочками, он взял у сестры чайную ложку и
принялся кропотливо отсыпать ягоды в тарелку с супом. Мальчик оценивающе
посмотрел на содержимое тарелки, решил, что ежевики уже достаточно, и
попробовал ложечку. К его величайшему разочарованию, вкус супа совсем не
изменился.

Поразмыслив над проблемой, Влад решил, что целиком брошенные в суп ягоды
никак не повлияют на вкус -- нужен их сок. Обрадовавшись новой идее, он
принялся беспощадно давить ягоды, выбрал из тарелки кожуру и только
тогда отважился вновь попробовать суп.

Увы, вышло не очень.

\section*{59}\label{59}
\addcontentsline{toc}{section}{59}

\markright{59}

Во время отсутствия Агаты, Владиславу пришлось несладко. Первые дни
ничего не предвещало беды, но под конец недели в бюро доставили не
одного, а целых двух покойников. Вернее сказать, целым был лишь один из
них, второй же оказался обезглавленной жертвой ДТП.

Так или иначе, над обеими ребятками нужно было потрудиться и, будучи
слишком нелюдимым, дабы отказать клиентам, Владислав принялся за работу.
Он был свидетелем сотен церемоний прощанья, но с ужасом представлял, как
будет проводить одну из них. Ласло не хотел тревожить подругу, (в конце
концов, он уже не маленький) так что, работая над трупами, морально
готовил себя к грядущему контакту с людьми, в тайне уповая на то, что
сама Агата, всё же, вернется к назначенному сроку.

Он услышал, как Нора вошла в «Маятник» ещё до того, как девушка успела
подняться в свои комнаты. У неё, конечно, был ключ, но, как бы ни был
осторожен входящий, громоздкая дверь всегда давала о себе знать.

Элеонора застала Владислава как раз в разгар работы, и бедолаге не
оставалось ничего, кроме как ретироваться на чердак в ожидании лучших
времен. Девушка весь вечер кашляла. Услышав эти хриплые звуки, наш герой
для начала надел медицинскую маску, (в отношении болезни, да и не только
её, он был слегка параноиком) а уж потом вспомнил о просьбе Агаты
присмотреть за Норой. Дождавшись, пока та уснет, Владислав решил
приготовить ей куриный бульон, но, как это частенько с ним бывало,
полностью погрузился в любимое дело и осознал это слишком поздно, чтобы
останавливаться -- он уже поливал пряным соусом золотистую телячью
вырезку.

Ласло редко ужинал за кухонным столом. Это место представлялось ему лишь
рабочей территорией, как и его прозекторский столик двумя этажами ниже.
В лучших кулинарных традициях юноша сервировал свой посеребренный
поднос, помещал туда новые изысканные блюда, (их никогда не бывало
меньше трёх) не забывая при этом о стаканчиках с дижестивом и
аперитивом. Завершив работу, Ласло накрывал поднос и нёс эту красоту на
чердак.

Находиться там ему нравилось больше всего. На чердаке было уютно. Лучше,
чем на кухне и в любой их гостиных комнат. Владислав не мог этого знать,
но во многом его обитель походила на старую спальню Элеоноры.

У основания скошенной стены висел список фильмов и сериалов -- по сути,
его планы на вечер. В похоронном бюро Нора была далеко не единственной,
кто умел экономить электричество. Владислав опустил поднос подле
ноутбука и с любовью посмотрел на список. Сегодня он планировал взяться
за «Красоту по-американски».

-- Скоро вернусь, -- пообещал он комнате.

*

Несмотря на то, что готовка уже с четверть часа как завершилась, кухня
продолжала излучать пленительные ароматы, что бесстыдно рассеивались по
всему дому, и Ласло бы не удивился, приведи они Нору наверх: насколько
он мог судить, девушка не ела с самого приезда.

Сервируя стол для Элеоноры, Ласло напевал старую хипповую песенку,
названия которой никак не мог вспомнить. Что-то про разнобой\ldots{} или
разные барабаны\footnote{Имеется в виду песня «Different Drum» в
  исполнении The Stone Poneys.}, чёрт его знает. Тем не менее, наш герой
всё время прислушивался и был готов в любую минуту покинуть зону риска.

Владислав налил себе чаю, когда, наконец, услышал какие-то скрипы.
Однако, никто не спешил подниматься на кухню.

-- Если кто-то хочет есть, -- мелодично пропел Ласло, -- то может
подняться на кухню через тридцать секунд.

В «Маятнике» вновь повисла тишина.

Желая удостовериться, что его услышали, Владислав повторил сказанное ещё
разок. На лестнице раздались шаги.

Довольный собой, наш герой задул свечи и мгновенно исчез в стенах,
прихватив с собой чай.

Куда же без него.

\section*{60}\label{60}
\addcontentsline{toc}{section}{60}

\markright{60}

Раскаты грома начинали понемногу стихать. Вскоре они и вовсе исчезнут,
но лишь для того, чтобы, собравшись с новыми силами, разразиться
наступившим днём. Насладившаяся ну очень ранним завтраком Элеонора
развалилась за столом, лениво потягивая одну из медовых настоек Агаты.
Она написала шеф-повару благодарственную записку, в конце которой
спрашивала, куда же, ч`рт возьми, подевалась хозяйка бюро ритуальных
услуг.

Норе хотелось ещё немного побыть в тепле кухни. Здешние запахи нагоняли
на девушку сонливость, против чего она ничуть не выступала. Нора сама не
заметила, как уронила голову, прикрыла глаза\ldots{}

(Сменяясь одна за другой, грезы проносились вихрем в её уставшем разуме.
Они пестрили яркими клочками воспоминаний, большинство из которых наша
героиня уже не помнила в момент пробуждения. Многие из них во сне
казались совершенно реальными, остальные же играли роль того, что могло
бы быть, и это совсем не казалось ей странным.

Нора видела Агату, покупавшую медовуху в крохотной лавке, чьи размеры
едва ли превосходили масштаб ванной комнаты. За всё время подруги ни
разу не выбирались в город вдвоем, но сцена с настойкой повторилась,
только теперь бутылка в руках хозяйки «Маятника» сменилась на вишневку.
Это не было странным: Нора давно собиралась предложить Агате прогуляться
по Львову.

В другой раз они шли по пустынному спуску. Воздух веял зимой, но снега и
в помине не было. Плитку под ногами сплошь и рядом укрывал толстый слой
льда. Запрокинув голову, Агата звонко смеялась и беззаботно скользила
вниз с такой беспечностью, словно вместо ботинок на ней была пара
коньков. Добравшись до самого низа, дамочка в чёрном схватилась рукой за
фонарь и, повернувшись вокруг него, остановилась в ожидании подруги.
Нора зачем-то надела каблуки, а потому продвигалась довольно медленно.

-- Кто бросил здесь этот чертов окурок?! -- кричала Агата, но на самом
деле она заливалась смехом.

-- Это всё Павел Владимирович! -- ответил приглушенный голос, звучащий
где-то в стенах.

-- На нем следы помады, -- уже более спокойно произнесла Агата.

-- Я не\ldots{} -- начала Нора.

Агата тут же отмахнулась от её объяснений, давая понять, что и не
подозревает подругу.

Нора захихикала, не в силах больше скрывать улыбку.

Агата вынула остатки сигареты их гроба с покойным -- с точностью Нора
сказать не могла, но, кажется, Павел Владимирович при жизни был
противником курения -- и окинула взглядом церемониальную комнату.

-- Начинаем через десять минут и чтобы больше никаких окурков!

Она слегка коснулась костяшками внутренней стены.

-- Ты слышал, Ласло?

\emph{Был здесь и незнакомец, которого Элеонора встретила на кладбище
незадолго до своего отъезда. Только во сне мужчина не был незнакомцем.
Он откинулся на спинку кровати, перебирая гитарные струны, и\ldots)}

\ldots а распахнув их, увидела, что лампа уже погасла, а часть кухни,
как это ни странно, заливал бледный серый свет, исходивший откуда-то из
холла. Девушка осторожно поднялась и на не до конца очнувшихся ото сна
ногах вышла на лестничную площадку. Тут призрачного света оказалось
гораздо больше.

Элеонора повернула голову и выглянула в окно, чтобы лицом к лицу
встретиться с виновником торжества: обрамленное венком из множества
сухих ветвей над кладбищем поднималось холодное зимнее солнце.

\section*{61}\label{61}
\addcontentsline{toc}{section}{61}

\markright{61}

Во время грозы темнело бессовестно рано, даже можно сказать: не светлело
и вовсе. К пяти часам за окнами похоронного бюро вовсю простиралась
послеобеденная ночь.

Норе что-то вновь нездоровилось. Сунув градусник подмышку, она
пододвинулась к газовой лампе и, согнувшись в три погибели (а иначе она
мало что могла увидеть) принялась приводить в порядок ногти.

Боковым зрением девушка вдруг уловила яркую вспышку света. Она повернула
голову и удивленно вглядывалась в темноту, когда из неё послышался
подозрительный треск. Затем вспышка повторилась и всего на пару
мгновений Нора увидела, что свет исходит от винтажного торшера --
вероятно, единственной лампы, которую забыли выключить.

Дверь в гостиную отворилась без какого-либо предупреждения. Элеонора не
успела толком предположить, кто за ней скрывается: обитатели похоронного
бюро имели общую, временами пугающую привычку -- передвигаться, не
создавая при этом никакого шума.

У входа появилась Агата. Хозяйка «Маятника» держала в руках горящую
свечу, и, раздвоившись, крохотные огоньки плясали, отражаясь в её очках.
Агата переступила порог, направляясь к подруге, и та успела заметить,
что лицо вошедшей -- вернее, та его часть, которую не скрывали очки --
было перепачкано кровавыми подтеками.

Торшер вновь издал этот странный треск и зажегся в полную силу.
Электричество вернулось и Нора, чьи глаза уже отвыкли от яркого
искусственного освещения, почувствовала резкий укол головной боли. Она
зажмурилась.

Это возымело действие и боль в висках начала утихать. Спустя секунд
шесть девушка осторожно открыла сначала один глаз, потом второй. Она
удивленно выдохнула: Агаты не было на месте. Нора вспомнила о
термометре. Тот показывал тридцать девять и два.

-- Неужели померещилось? -- прошептала девушка.

-- Температуришь?

Нора хрипло крикнула и обернулась. Агата оказалась на своем излюбленном
месте. Одному богу известно, как девушка так быстро пересекла комнату,
но теперь она восседала в кожаном кресле так, словно никогда и не
вставала.

-- Ты меня напугала, -- призналась Нора.

Агата хотела извиниться, спросить, как у подруги дела и сказать, что
безумно рада её возвращению, но вместо этого она произнесла всего три
слова.

-- Ты меня тоже.

Элеонора недоверчиво взглянула на собеседницу. При должном освещении
полосы на её лице оказались всего лишь потревоженной дождем косметикой.

Правда была в том, что Агата не могла думать ни о чём другом кроме
как\ldots{}

-- Письмо, -- выпалила она.

Прозвучало это немного резче, чем того хотелось Агате, но теперь это
едва ли её волновало.

-- Письмо? -- удивленно повторила Нора.

Агата кивнула.

*

Ей понадобилось какое-то время и собственный ноутбук, дабы прояснить
ситуацию: в телефоне Элеоноры, как и в её планшете и, вероятно, любом
другом устройстве, которым девушка могла бы воспользоваться для входа в
почту, не было и намёка на злосчастное послание.

-- Наверное, чья-то шутка, -- сказала Нора. Ей не нравилось выражение
лица Агаты: никакие очки не могли скрыть тревоги. -- Ящики частенько
взламывают, особенно если ты пользуешься бесплатным доменом. А я
пользуюсь. Обычно электронки ломают для банальной рассылки спама. Гек
говорил, что его отец работал в отделе по борьбе с подобными взломами.
Правда его быстро попросили -- чувак отказывался блокировать религиозные
рассылки.

Она вновь просмотрела текст письма.

\begin{quote}
\emph{так темно и холодно. а хуже всего пустынно}

\emph{я ничего не вижу вчера я слышала шаги но здесь никого нет и не
было}

\emph{кроме как за дверью}

\emph{точнее, в дверном проёме}

\emph{давно}

\emph{голоса вокруг везде они а потом снова белая тишина}

\emph{не думаю что здесь когда нибудь вообще был дом слишком сыро}

\emph{холодн я видела так много}

\emph{сколько тебе лет7 снова.каждый раз}

\emph{неужели я никогда не закончу свою книгу?}
\end{quote}

-- Звучит стрёмненько, -- прокомментировала Нора, -- хотя, вместе с тем
смешно. Как будто читаешь дневник героинового наркомана, или смотришь
одно из этих псевдопсиходелических видео.

За всё время Агата не проронила ни слова.

-- Я понимаю твою тревогу, но это всего лишь чья-то глупая шутка, --
заверила её Нора. -- Я, конечно, приболела, но не бредила\ldots{} Вот
прямо чтоб настолько. И потом, у меня быстро закончились запасы
электроэнергии. Не волнуйся, \emph{бро}.

Последнее слово вырвалось у неё непроизвольно. Нора рассмеялась, но
вовремя замолчала, увидев выражение лица собеседницы. Девушка в третий
раз перечитала письмо.

-- По всей видимости, они не поленились прочесть нашу переписку, потому
что здесь есть о темноте и\ldots{}

-- Твоей книге, я вижу.

Нора улыбнулась, но губы владелицы похоронного бюро по-прежнему были
плотно сжаты.

-- Хотела бы я знать, кто это написал, -- произнесла Агата.

-- Почему тебя это настолько беспокоит?

Дама в чёрном вздохнула и осмотрелась в поисках винишка. Бутылки нигде
не было.

-- Не нравится мне это, -- только и нашлась Агата, не будучи уверена,
идет ли речь о вине, или же том странном письме.

В итоге, она решила, что сгодятся оба варианта.

\section*{62}\label{62}
\addcontentsline{toc}{section}{62}

\markright{62}

Существует более ста способов того, как можно преподнести смерть. Вот
самые популярные из них:

Перерождение, которое подвластно человеку лишь при условии, что на
протяжении предыдущих жизней он накопил достаточно хороших эмоций,
которые позволяют душе осуществлять любые внезапно возникшие желания,
однако, в этом мире свершено отсутствует разнообразие, дающее
возможность для развития, и когда вышеупомянутые положительные эмоции
иссекают, душа возвращается в низшие миры, дабы вновь переродиться;

Отделение души от тела, с дальнейшей навигацией на небеса, что отражают
теологическую и метафизическую концепцию христианского царства, в
котором бесполые ангелы имеют чины и в компании святых наслаждаются
присутствием Бога, созерцая его деяния;

Остатки античности говорят нам о месте на краю земли, с прекрасным
климатом и избытком сладких фруктов, где чистая душа, освобожденная от
материи, а заодно и земных оков, находит долгожданный отдых;

Ацтеки так вообще придумали три рая, -- ну, чтоб наверняка -- в них
уровень полученного комфорта, грубо говоря, соответствовал уровню
просветления, до которого та, или иная личность успела прокачаться
прежде, чем умерла;

Исходя из нордических традиций, жизнь после смерти не сильно-то и
менялась: воины сражались дни напролет, а ночами только и делали, что
пировали, так что не исключено, что многие из обитателей Валгаллы вообще
не были в курсе того, что на днях умерли;

Если верить индейским племенам, окочурившись, ты будешь жить там, где
заходит солнце и происходит удачная охота, что бы это ни значило;

Эскимосы вот временами видят образы своих усопших в лучах северного
сияния, играющих с головой кита;

Есть ещё всевозможные вариации перехода в новый улучшенный мир, для
которого наш является лишь прообразом.

Как ни странно, трактовок Ада ещё больше, но мы вернемся к реалистичной
точке зрения -- банальному увяданию, где смерть начинается в тот самый
момент, когда мозг прекращает получать кислород, отчего жизненно важные
органы просто не могут выполнять свои функции, проходит аутолиз,
достигает анаэробного бактериального гидролиза и всё заканчивается
трапезой братьев наших меньших.

Такого мнения и придерживалась Агата Рахманинова, составляя план своих
лекций. Нора как раз предложила подруге уделять опарышам поменьше
внимания, когда в кабинет просочился Гек.

Он окинул взглядом владелицу «Маятника».

-- Щось ти занадто бліда, жінко, та й наче схудла\footnote{Что-то ты
  слишком бледная, женщина, и вроде бы схуднула.}, -- заключил мальчик.

Переведя взгляд на Нору, он приподнял воображаемую шляпу, после чего
быстренько пересек комнату и, (хотя в его распоряжении было ещё как
минимум два обитых бархатом стула) умостился прямиком на стол.

Дама в чёрном ничего не ответила, а вот Элеонора мысленно согласилась с
утверждением незваного гостя. Последние двадцать минут девушка искоса
поглядывала на Агату, пытаясь отделаться от мысли, что та сейчас вот-вот
утонет в своем громоздком кресле.

Мальчик принялся энергично мотать ногами, то ли выражая нетерпение, то
ли отдавая дань Йену Кёртису\footnote{Йен Кёртис -- вокалист и
  основатель пост-панк группы Джой Дивижн. Страдал эпилепсией и во время
  выступлений нередко исполнял свой фирменный танец, изображавший эту
  болезнь.}. Агате, кажется, совершенно не мешало происходящее. Левую
руку она изредка отнимала от ножки бокала, чтобы задумчиво провести
кончиками пальцев по столешнице красного дуба; правой же Агата
непрерывно водила над записной книгой, периодически что-то вычёркивая и
дописывая.

-- Выкладывай, -- произнесла она, переворачивая страницу.

Гек, который уже успел приуныть от нехватки внимания, вмиг оживился.

-- Привезли! -- поспешил доложить он, и Нора вздрогнула, подумав об
очередном покойнике, но, вопреки ожиданиям, следующая фраза мальчика
вызвала у неё одно лишь недоумение. -- На этот раз в глазных
яблоках\ldots{}

Во-первых, Гек вдруг заговорил на русском, и теперь речь его звучала
странно и чуть ли не противоестественно. Во-вторых\ldots{} \emph{Что?}

Агата едва ли была поражена услышанным. Она вяло махнула рукой, так что
на пальцах стали видны посиневшие вмятинки от длительного писания.

-- Неинтересно.

-- Ти ж неначе як жартуєш?\footnote{Ты не иначе как шутишь?}

Агата вздохнула.

-- Я абсолютно серьезно, Гек, -- она повернулась к другой девушке. --
Так что думаешь, на втором этапе гниения труп лучше называть
раздувшимся, или зрелым?

Нору застали врасплох: она как раз задумчиво переводила взгляд с
мальчика на Агату, а затем обратно, и уж точно не предавалась
размышлениям о стадиях разложения трупов, как и о любым другим,
связанным с покойниками, думам, пускай временами это было и непросто,
находясь в бюро ритуальных услуг.

Девушка подвисла.

Агата подняла ручку, словно собираясь что-то записать, но затем
повторила вопрос, вновь взглянув на собеседницу.

-- Так, что думаешь?

Та постаралась сосредоточиться и призвать на помощь всю свою вежливость.

-- Эм-м-м\ldots Да-а-а, -- протянула Нора.

Гек предательски хохотнул.

-- Что да? -- Агата вопросительно подняла бровь. -- Зрелый или
раздувшийся?

-- А как патологи называют? -- наконец, нашлась Нора.

-- И так, и так говорят.

-- Я\ldots Я думаю, тебе виднее.

Нора неуверенно взглянула на хозяйку «Маятника». Та плавно водила
головой по сторонам, словно раздумывая, а не согласиться ли с
собственными мыслями. Ответ, вероятно, оказался положительным, потому
что Агата удовлетворенно хмыкнула и принялась увлеченно что-то
вычёркивать.

Мальчик озадаченно следил за этим действом.

-- Ну я піду?\footnote{Ну, я пойду?} -- произнес он, словно, ожидая, что
Агата передумает.

Однако, она лишь безразлично кивнула.

Близилось время обеда, что в «Маятнике» обычно имел место быть ближе к
полуночи. Элеонора вспомнила об обещанной пицце и почувствовала, что кит
в её животе вот-вот примется исполнять ночные серенады. Воспользовавшись
ситуацией, девушка поднялась со своего места.

-- Я тоже, -- сказала она. -- Засиделась что-то.

Еще один неопределенный кивок.

Вскоре дубовая дверь скрипнула и владелица «Маятника» вновь осталась
одна в кабинете, едва ли это замечая.

*

Несмотря на напускную обиду, которой так и сквозили глаза мальчика,
(превратившие нашего героя обратно в ребёнка, каким его представляла
Элеонора, слушая историю одного путешествия) и легкую растерянность
ввиду отказа Агаты от чего бы там ни было, Гек не спешил покидать
«Маятник», стоило ему лишь услышать пленительные запахи выпечки, коими,
казалось, пропитался каждый камешек мрачного здания: как бы там ни было,
малец никогда не был против вкусного обеда. Особенно халявного.

-- Тебе не кажется, что наша барышня чёт приуныла? -- задала вопрос
Нора, стоило им остаться наедине. -- Ну, как в той странной песне,
помнишь? Которая вовсе-то и не песня на самом деле.

Мальчик, который слишком много времени провел в компании Агаты и
Валерия, был равнодушен к интернету, так что понятия не имел, о какой
песне, или не песне идет речь, но мысль Элеоноры он понял, а потому
многозначительно кивнул.

-- Я за цим й зайшов.\footnote{Я поэтому и зашел.}

Гек заметил, что не помнит, дабы Агата хоть раз покидала «Маятник» с
момента своего возвращения из лесу. Она, конечно, и до этого не то,
чтобы часто выходила из так званой зоны комфорта, но прежде мальчик
никогда не видел, чтобы она вообще всё время проводила в кабинете.

-- Чи мені здеється?\footnote{Или мне кажется?} -- малец с надеждой
взглянул на собеседницу.

Та попыталась припомнить какой-нибудь недавний относящийся к делу
случай, но, увы, Нора вдруг осознала, что теперь её подруга даже не
вспоминала о том, чтобы зажечь могильные фонари.

По всей видимости, лицо Элеоноры ответило на вопрос Гека раньше, чем это
удалось сделать самой девушке. Малец в задумчивости клацнул языком и
насупил брови, приняв в равной мере комичный и трогательный вид.

-- Так я і думав\ldots{}\footnote{Так я и думал\ldots{}}

Нора и раньше догадывалась, что что-то не так, но в полной мере осознала
это, лишь обретя единомышленника. Она беспомощно взглянула на
последнего. Мальчик продолжал клацать языком, сосредоточенно глядя в
темноту, за которой скрывался зал для проведения прощальных церемоний.

Внезапно Гек смолк. Его взгляд возвратился к Норе.

-- Не маєш бажання гульнути містом?\footnote{Нет желания прогуляться по
  городу?} -- на лице мелькнула лукавая улыбка.

Элеонора рефлекторно сверилась с часами: без трёх минут полночь -- самое
время размять ноги в минусовую температуру.

-- Почему бы и нет, -- ответила девушка. Она указала на лестницу. --
Только сперва давай разбёремся с пиццей.

\section*{63}\label{63}
\addcontentsline{toc}{section}{63}

\markright{63}

Вопреки настырному холоду, февральская ночь оказалась на удивление
ясной: деревья то и дело пускались в пляс, сопровождаемый порывами
ветра, а над ними золотился тонкий серп полумесяца, окруженный компанией
ярко сияющих звёзд; воздух был свежим, на улицах буквально веяло зимой
-- всё это зачаровывало Элеонору, создавая запоздалую иллюзию Рождества.

Привыкшая путешествовать в одиночестве, наша героиня уже давно
исследовала и досконально знала Старе Місто, а потому, усевшись в такси,
тут же воодушевилась, услышав названный Геком адрес: улица Джона Леннона
не входила в территорию, на которой обычно проходили её прогулки, и Нора
была рада выбраться подальше от центра, который, несмотря на своё
разнообразие, уже успел слегка наскучить.

Ехали они долго.

Однако, достигнув пункта назначения, девушка выглядела разочарованной.
На мгновенье Норе подумалось, что она вдруг зачем-то вернулась в свой
родной город, напрочь состоявший из такого вот пост-советского гетто,
коим, несмотря на своё мелодичное название, оказалась вышеупомянутая
улица.

Такси поспешило убраться вон, и наши персонажи остались стоять в
окружении серых бетонных блоков, по ошибке именуемых домами. И вновь
выражение лица Норы с лёгкостью выдало её мысли. Гек расхохотался.

-- Стане тобі журитися!\footnote{Будет тебе печалиться!} -- весело
заметил он.

Первым делом, говорил мальчик, когда они шагали сквозь ночь, в
направлении одной из этих нагоняющих тоску мнгоэтажек, Норе следовало
постараться расслабиться и спрятать этот виноватый взгляд -- он и по
жизни-то не сулит ничего хорошего, что уж говорить о грядущей встрече с
дилером. Во-вторых, продолжал Гек, ни ей за что -- ни при каких
обстоятельствах -- не стоило упоминать слова «пушка»! Таких парней это
же-е-е-есть как бесит, ну, будто красной тряпкой перед быком
помахать\ldots{}

Всё это время Элеонора молча следовала за своим спутником. Мысленно она
уже вовсю сокрушалась о том, что в принципе согласилась отправиться в
это чёртово гетто. Нет, она любила Агату, а та, как уверял Гек, любила
травку, но чем меньше расстояния оставалось до подъезда, тем сильнее
наша героиня убеждалась в том, что игра не стоит свеч.

Слушая мальчика, она уже приоткрыла рот, собираясь сказать, что
передумала, однако, была слишком вежливой, чтобы перебивать.

Ну, и последнее, спокойно излагал Гек, говорить будет он. Нора же, если
кто вдруг спросит, является его сучкой, или как тут принято
говорить\ldots{}

Тут малец не выдержал и расхохотался что есть мочи. Его смех эхом
отдавался в бетонных стенах. Какой-то дед высунулся из окна соседнего
дома и пропитым голосом предложил шумевшему веселиться в другом, более
подходящем для этого, месте. Например, в заднице.

-- Ты чего это? -- прошептала девушка.

Её спутнику понадобилось время, чтобы успокоиться.

-- Бачила б ти своє обличча\ldots{}\footnote{Видела бы ты свое
  лицо\ldots{}} -- не унимался Гек. -- Ну просто
кобітка-прихуїтка!\footnote{Кобіта -- девушка, тогда как второе слово не
  требует перевода.}

Элеонора остановилась. Она откинула с лица волосы и серьёзно взглянула
на мальчика, чьё тело продолжало сотрясаться от смеха, что так и рвался
наружу.

-- Ой, не ображайся ти, -- махнул рукой Гек. -- Я ж лишень жартував, бо
ти видавалась такою напруженою\ldots{} Той Блискавка взагалі-то файний
хлоп, тобі сподобається!\footnote{Ой, не обижайся ты. Я же всего лишь
  пошутил, а то ты казалась такой напряженной\ldots{} Тот Молния
  вообще-то милый парень, тебе понравится!}

Нора глубоко вздохнула.

Что ж, могло быть и хуже.

\section*{64}\label{64}
\addcontentsline{toc}{section}{64}

\markright{64}

Выкурив парочку сигарет, девушка, казалось бы, успокоилась, но поняла,
что это не так, когда разглядывала своё искаженное отражение в разбитом
зеркале. Вскоре лифт за её спиной захлопнулся. Нора не была уверена в
том, что собирается сказать: она не то хотела предложить своему спутнику
уйти, не то просто собиралась попросить Гека ещё немного
подождать\ldots{}

Как бы там ни было, взглянув на мальчика, наша героиня увидела, что тот
уже приподнялся на носочках и коснулся кнопки звонка. Действие это
отдалось чуть слышным щебетанием птиц, которые, скорее всего, уже давно
умерли.

Нора не была поклонницей ни гонзо, ни пост-модернизма, но все же читала
кое-что как из Томпсона\footnote{Хантер Томпсон -- американский писатель
  и журналист, основатель гонзо-журналистики, автор автобиографических
  книг «Страх и отвращение в Лас-Вегасе» и «Ромовый дневник», персонажи
  которых неизменно пребывают под влиянием наркотических средств.}, так
и из Берроуза\footnote{Уильям Берроуз -- американский писатель и
  эссеист, считается важнейшим представителем бит-поколения. Автор таких
  работ как «Джанки» и «Голый завтрак», в которых, основываясь на
  собственном опыте, детально знакомит читателя со всеми оттенками
  наркотической палитры.}. Короче говоря, она ожидала увидеть перед
собой типичный образец угашенного в хлам наркомана. Однако, открывший
дверь человек вовсе не имел безумный вид: на удивление, он оказался
полностью одет, волосы -- длинные, собраны в хвост, за исключением двух
прядей, аристократично свисавших вдоль лица, которое вовсе не выражало
негативных эмоций. Вместо ожидаемого шприца, зараженного сотнями
вирусов, в руках дилера виднелась потрепанная книга, чью обложку не
позволяло идентифицировать плохое освещение.

Словом, мужчина смотрелся адекватно и, кажется, и впрямь был в своем
уме. Его глаза выражали одно лишь спокойствие. Заметив девушку, Молния
улыбнулся, и Норе понадобилась всего доля секунды, дабы узнать в
стоявшем перед ней мужчине того самого незнакомца с кладбища.

Мужчина ничего не сказал. Он посторонился, жестом приглашая гостей в
квартиру. Позже выяснилось, что обшарпанная двухкомнатка была жилищем
одного из товарищей Молнии, -- того самого, что организовал сделку по
перевозке голландских семян в протезах глазных яблок -- и именно здесь
занимались взращиванием товара.

Гек подмигнул своей спутнице, прошел по крохотному коридорчику и скрылся
за дверью кухни. Было слышно, как он весело с кем-то болтает. Тем
временем Молния помог девушке снять шубку и провел её в дальнюю комнату,
представлявшую собой что-то вроде плохо обставленной гостиной.

-- Вот уж не думал так встретиться, -- наконец, заговорил Молния.

У него был тот самый вампирский акцент.

Мужчина опустился на диван. Это оказалось единственной мебелью, на
которой можно было сидеть, не считая лишившегося двух ножек стула, что
был прислонен к противоположной стене, так что Норе пришлось опуститься
рядом.

-- Я бы предложил кофе, -- произнес Молния, кладя книгу по другую
сторону, так что девушка вновь не успела узнать название, -- или зеленый
чай, но из напитков тут только Метакса. Там где должна быть кухня у
нас\ldots{} Ну, ты понимаешь.

Нора кивнула.

-- Всё в порядке, -- сухо ответила она, хотя едва ли была согласна с
собственным утверждением.

Они находились наедине от силы минуты три, но Нора то и дело ловила на
себе взгляд Молнии. Теплый и, на удивление, уютный, учитывая окружающую
обстановку. Девушка чувствовала, что краснеет.

-- Вы продадите травку Геку? -- неуверенно произнесла она, зная, что
раскраснеется ещё больше, если продолжит молчать.

Молния ухмыльнулся, и девушка вновь поразилась его красоте.

-- Естественно, я-то знаю, для кого этот корабль.

-- Вы с Агатой знакомы?

-- Мы \emph{были} знакомы, -- ответил Молния.

Это прозвучало странно. Нора решила повременить с расспросами. Повисла
неловкая пауза. Первым её нарушил мужчина.

-- А что ты вообще здесь делаешь? -- спросил он, чётко выговаривая
каждое слово. -- В такое-то время\ldots{}

-- Присматриваю за мальчишкой, -- соврала Нора. Она и сама толком не
знала, что здесь делает. -- Не хотела отпускать его одного.

Вскинув брови, её собеседник искренне расхохотался.

-- Присматриваешь? За Геком-то? -- сквозь смех произнес Молния. -- Да в
вопросах выживания он каждому из нас фору даст!

Он вновь рассмеялся, а затем сказал что-то ещё, чего наша героиня не
расслышала. Да это и не было важно. Девушка на мгновенье забыла о
стеснительности и теперь зачарованно разглядывала улыбку сидящего рядом
мужчины, стараясь запомнить этот момент как можно чётче.

-- Пойду всё же справлюсь насчет чая, -- произнес мужчина.

Он поднялся со своего места и Элеонора, наконец, смогла разглядеть
коричневый томик, оставшийся на диване. Им оказался «Вокзал потерянных
снов»\footnote{Второй опубликованный роман Чайна Мьевиля, написанный в
  жанре стимпанк.}.

Так наша героиня влюбилась.

\section*{65}\label{65}
\addcontentsline{toc}{section}{65}

\markright{65}

Последний месяц зимы выдался в действительности странным. Не потому, что
снега не было вплоть до наступления марта -- этим как раз никого не
удивишь. Дело было в том, что данный промежуток времени внес множество
изменений в жизни обитателей «Маятника», из которых далеко не все были
положительными, но большинство событий уж точно оказались
неожиданностью.

Ближе к утру Норе удалось добраться до собственной постели. Никогда
прежде подъем по ступеням не занимал у нее так много времени и, только
зарывшись в одеяла, девушка осознала, что на самом деле не поднималась,
а спускалась по лестнице, ведущей из апартаментов Агаты.

До рассвета было ещё далеко, чему наша героиня несказанно радовалась.
Одному Берроузу известно, что могло бы случиться с её головой при
контакте с дневным светом: черепушка и без того обещала расколоться при
первом удобном случае. Нора постаралась убедить себя, что всё утрирует и
нет у неё никаких мигреней. Вышло как-то не ахти. Она попыталась
отвлечься и восстановить в памяти события последних часов\ldots{}

Ну, насколько это в принципе казалось возможным.

\section*{66}\label{66}
\addcontentsline{toc}{section}{66}

\markright{66}

-- Я не наркоманка, -- безразлично произнесла Агата, когда парой часов
ранее Элеонора вернулась в её кабинет со спичечным коробком, полным
голландской травки. -- Курю только по четвергам\ldots{}

В западной стене раздался глухой стук.

-- \ldots{} и каждую вторую субботу, -- нехотя добавила Агата, отчего
грохот тут же прекратился. -- Временами соседство Ласло становится
невыносимым. Словно живешь с детектором лжи.

-- Сегодня как раз четверг, -- заметила Нора.

Владелица похоронного бюро ничего не ответила. Она высыпала содержимое
коробка на стол и принялась своими длинными ногтями водить по его
поверхности. Элеонора минут пять наблюдала за подругой. Подобная картина
ей быстро наскучила.

-- Я, конечно, не спец, -- начала Нора, указывая на траву, -- но,
по-моему, её не так принимают.

Агата медленно подняла голову. Очки, как обычно, не позволяли сказать
наверняка, однако Нора могла поклясться, что подруга закатывает глаза.
Иного ответа не последовало и на протяжении последующих минут.

-- Почему ты отказалась, если дуешь так часто? -- нарушила тишину Нора.

-- Я бы не сказала, что это часто.

-- Ну так?

-- Мне не хотелось отвлекаться. Думала разобраться с делами, --
размеренно произнесла Агата.

Несколько травинок случайно попали под ноготь, так что она пыталась
исправить ситуацию.

-- Но ты уже три недели не вылезаешь из кабинета! -- не выдержала Нора.
-- Сидишь тут, рассказываешь о гниении тел, -- а мне, знаешь ли, это не
особо нужно, я вообще, если хочешь знать, предпочитаю кремацию --
составляешь могильные надписи, а из проигрывателя вечно доносятся эти
загробные голоса\ldots{} -- она перевела дыхание. -- Да и вид у тебя
чересчур трагический.

Агата пожала плечами.

-- Работа у меня такая.

Нора обошла стол.

-- А на прошлой неделе я нашла это, -- она резко выдвинула верхний ящик.

Тот оказался пуст.

-- Секунду, -- попросила девушка, и выдвинула второй ящик.

Результат был тот же, но наша героиня не хотела сдаваться.

-- Это, -- она попробовала третий ящик.

Также пуст. Их, кстати, было двенадцать.

Агата, которая прежде спокойно наблюдала за разыгравшимся действом,
откинулась в кресле.

-- Попробуй нижний, -- посоветовала она.

За неимением иных вариантов, Элеоноре пришлось прислушаться. К
глубочайшему облегчению девушки, последний ящик не был пуст. Его
заполняли блокноты и бумажные папки, (чёрные, разумеется) поверх которых
красовалась бумажка размером А4 -- печатный бланк, большинство пробелов
в котором заполнял знакомый Норе почерк.

Выглядел бланк следующим образом.

ПРЕДСМЕРТНАЯ ЗАПИСКА

Я, Агата Рахманинова, далее именуемая как Составитель, \_\_\_ числа,
\_\_\_\_\_\_\_\_\_\_\_\_ месяца \_\_\_\_ года совершившая самоубийство
путем асфиксии по причинам слишком банальным, чтобы их называть,
подтверждаю, что упомянутое выше решение было принято мною
самостоятельно, не под влиянием алкогольного, или наркотического
опьянения и без постороннего вмешательства.

Ввиду вышесказанного, заблаговременно

\textbf{\emph{Прошу правоохранительные органы}}

-- избежать заведения уголовного дела об умышленном убийстве Составителя
по ст. 115 УК Украины;

-- опустить проведение уголовного следствия о доведении до самоубийства
по ст. 120 УК Украины, а также исключить любые ссылки на этот и другие
законы, которые могут быть использованы в целях привлечения к уголовной
ответственности третьей стороны;

-- в ходе опроса свидетелей и осмотра места происшествия настоятельно
рекомендую избежать разговора со следующими лицами:

Владиславом Мирославским -- по причинам слабого психического здоровья;

\begin{center}\rule{0.5\linewidth}{0.5pt}\end{center}

\begin{center}\rule{0.5\linewidth}{0.5pt}\end{center}

\begin{center}\rule{0.5\linewidth}{0.5pt}\end{center}

-- упразднить проведение вскрытия исходя из того, что смерть Составителя
наступила ненасильственным путем, посредством асфиксии и по причинам,
приведенным выше.

Для облегчения работы правоохранительных органов к данному документу
также

\textbf{\emph{Прилагаются:}}

-- полная информация о Составителе;

-- пакет документов Составителя;

-- последние записи, сделанные Составителем, кои могут потребоваться для
проведения почерковедческой экспертизы.

\textbf{\emph{В дополнение}}

-- прошу утилизировать тело составителя путем (подчеркнуть нужный
вариант)

погребения (по желанию указать кладбище)
\_\_\_\_\_\_\_\_\_\_\_\_\_\_\_\_\_\_\_\_\_\_\_\_\_\_\_\_\_\_

кремации (по желанию указать крематорий и дальнейшую судьбу праха
Составителя)
\_\_\_\_\_\_\_\_\_\_\_\_\_\_\_\_\_\_\_\_\_\_\_\_\_\_\_\_\_\_\_\_\_\_\_\_\_\_\_\_\_\_\_\_\_\_\_\_\_\_\_\_\_\_\_\_\_\_\_\_\_\_\_\_\_\_
\_\_\_\_\_\_\_\_\_\_\_\_\_\_\_\_\_\_\_\_\_\_\_\_\_\_\_\_\_\_\_\_\_\_\_\_\_\_\_\_\_\_\_\_\_\_\_\_\_\_\_\_\_\_\_\_\_\_\_\_\_\_\_\_\_\_\_\_\_\_\_\_

свой вариант
\_\_\_\_\_\_\_\_\_\_\_\_\_\_\_\_\_\_\_\_\_\_\_\_\_\_\_\_\_\_\_\_\_\_\_\_\_\_\_\_\_\_\_\_\_\_\_\_\_\_\_\_\_\_\_\_

-- *прошу учесть, что по желанию Составителя в момент утилизации на
нём/ней должно быть надето (здесь перечислить свои предпочтения)
что-нибудь чёрное, загляните в мой гардероб -- проблем не будет;

-- *в целях погребения/кремации Составителя, а также проведения
церемонии прощанья, (не обязательно) прошу воспользоваться услугами
(вписать название желаемого бюро ритуальных услуг) вот это будет задачка
не из легких;

-- *прошу учесть пожелания Составителя относительно (подчеркнуть и
вписать нужное)

материала и дизайна гроба
\_\_\_\_\_\_\_\_\_\_\_\_\_\_\_\_\_\_\_\_\_\_\_\_\_\_\_\_\_\_\_\_\_\_\_\_\_\_\_\_\_\_\_\_\_\_\_\_

материала и дизайна урны
\_\_\_\_\_\_\_\_\_\_\_\_\_\_\_\_\_\_\_\_\_\_\_\_\_\_\_\_\_\_\_\_\_\_\_\_\_\_\_\_\_\_\_\_\_\_\_\_

свой вариант
\_\_\_\_\_\_\_\_\_\_\_\_\_\_\_\_\_\_\_\_\_\_\_\_\_\_\_\_\_\_\_\_\_\_\_\_\_\_\_\_\_\_\_\_\_\_\_\_\_\_\_\_\_\_\_\_\_\_\_
\_\_\_\_\_\_\_\_\_\_\_\_\_\_\_\_\_\_\_\_\_\_\_\_\_\_\_\_\_\_\_\_\_\_\_\_\_\_\_\_\_\_\_\_\_\_\_\_\_\_\_\_\_\_\_\_\_\_\_\_\_\_\_\_\_\_\_\_\_\_\_

Больше информации можно узнать, ознакомившись с завещанием Составителя,
оформленном в
\_\_\_\_\_\_\_\_\_\_\_\_\_\_\_\_\_\_\_\_\_\_\_\_\_\_\_\_\_\_\_\_\_\_\_\_\_\_\_\_\_\_\_\_\_\_\_\_\_\_\_\_\_\_\_\_\_\_\_\_\_\_\_\_
и заверенном
\_\_\_\_\_\_\_\_\_\_\_\_\_\_\_\_\_\_\_\_\_\_\_\_\_\_\_\_\_\_\_\_\_\_\_\_\_\_\_\_\_\_\_\_\_\_\_\_\_\_\_\_\_\_\_\_\_\_\_\_\_\_\_\_\_\_

\_\_\_ числа \_\_\_\_\_\_\_\_\_\_\_\_\_\_\_\_\_\_ месяца \_\_\_\_\_
года.

Составитель: Агата Рахманинова ~~~~~~~~Подпись: \_\_\_\_\_\_\_\_\_\_\_г.

* обозначенные сносками пункты не являются обязательными для заполнения

Обратная сторона бланка оказалась разлинованной. «Для заметок» -- так
она была озаглавлена. Когда девушка впервые обнаружила странный
документ, та его часть пустовала, но теперь Агата оставила на ней
послание, несущее вот такую сомнительную информацию:

\begin{quote}
\emph{Хорошо быть кремированным. Больно надо, чтоб тобой черви питались.
К тому же, всегда остаётся шанс превратиться в зомби, что тоже так себе
вариант, я считаю. А тут, развеялся и летишь себе над океаном весь такой
радостный, а потом в дыхательные пути всяким мудакам попадаешь и такой:
«Вот вам, сучки!» и дальше летишь\ldots{}}
\end{quote}

Смеяться Нора, вроде как, не собиралась, но, прочитав последние строки,
просто не смогла удержаться. Девушка опустила бланк на стол и, надеясь,
что её лицо все же приняло серьёзно настроенный вид, взглянула на
подругу.

-- Как ты это прокомментируешь?

Физиономия Агаты оставалась беспристрастной.

-- Ты рылась в моем столе? -- произнесла она, скорее удивленно. --
Раньше ты никогда не рылась в моих вещах\ldots{}

Владелица «Маятника» задумчиво выдохнула и замолчала на половине фразы.

-- У меня закончились сигареты, -- выпалила Нора и тут же поняла, как
неправдоподобно это звучит. -- За окном стояла ночь. Мне не хотелось
выходить из дома. Да и холод такой, сама знаешь.

Агата знала.

-- Странно\ldots{} -- протянула она.

-- Что странно?

-- Странно, что я тебя не заметила, потому как по ночам я всегда здесь.

Нора растерянно шмыгнула носом: она никогда не умела врать.

-- Ну хорошо, это было днём, -- призналась девушка. -- Я волновалась за
тебя! Ты ведёшь себя странно, еще более странно, чем обычно. Спишь от
силы часа три в сутки и постоянно -- всё время -- проводишь в этой
долбанной комнате!

Агата указала на предъявленный ей бланк.

-- Это для моих курсов, -- беглая ухмылка. -- Всего лишь шаблон
посмертной записки. -- Мне нужно было разобраться с кучей дел, я же
говорила.

-- И как, разобралась?

Драматическая пауза.

-- Почти.

\section*{67}\label{67}
\addcontentsline{toc}{section}{67}

\markright{67}

Часом спустя обе девушки полулежа развалились с кожаных креслах, которые
располагались в гостиной Агаты и были точными копиями излюбленного
седалища хозяйки «Маятника», что осталось этажом ниже. За всю свою жизнь
Элеонора ни разу не баловалась с дурью и начинать не собиралась, а уж
тем более не хотела спать в прокуренной травкой комнате, так что наши
героини сочли апартаменты Агаты местом, более подходящим для текущей
сцены.

Разглядывая драп, Агата пришла к выводу, что это «не какая-то там
солома», и ей было бы жалко пускать такую травку на косяки. Не так уж
давно Владислав разбил стеклянный бульбулятор, который был втрое старше
его самого, -- и Агата подозревала, что сделал он это вовсе не случайно
-- так что дамочка обзавелась бутылкой с фольгой и наскоро соорудила
самодельный кальян, он же Дудка, он же Ферик, он же Утка, Бонг, Батл и
многое другое.

Однако, кое-что наши персонажи всё-таки не учли. Драп оказался чрезмерно
хорошим, да таким, что Нора, которая лишь хотела понаблюдать за
процессом, внезапно оказалась пассивным курильщиком, но поняла это
слишком поздно, дабы иметь хоть малейшую возможность что-либо изменить.
Прихода девушка, конечно, не поймала, но густые испарения быстро
заполнили комнату и настроили нашу героиню на всем известную волну смеха
и юмора.

Большую часть времени Нора, чей голос стал заметно выше, хохотала,
останавливаясь только чтобы послушать, что ещё скажет её подруга, а
потом продолжала надрывать живот.

Голос второй девушки вообще познал поразительные метаморфозы: его
переполняли эмоции угара и, вместе с тем глубокого трагизма, так что
речь Агаты теперь скорее походила на страсти актеров шекспировских
времен. Элеонора не была уверена, было бы ей настолько весело без
периодических повествований подруги, но полагала, что да.

За все время Агата лишь раз поднялась со своего места, но при этом всё
время выполняла какое-то действие. Вернее сказать, чувствовала его у
себя в голове, а потом пыталась донести увиденное до собеседницы. Наша
героиня (ну, та, которая была более укуренная) также частенько не могла
понять, смеется она, или плачет, о чём не прекращала спрашивать, но это
уже другая история.

Кстати, она также постоянно повторяла три слова, каждое из которых
почему-то начиналось на «Т»: трагически, тревожно и трогательно.
Последнее так вообще вечно прилетало не к месту.

-- Я\ldots{} знаешь что\ldots{} -- мечтательно начала Агата. Говорила
она медленно, чуть ли не по слогам, периодически срываясь на жуткий
смех. -- Сижу как в соломенном кресле\ldots{}

-- Круто тебе.

-- Не-не-не\ldots{}

-- А я сижу как в дерьме. У меня нога затекла.

Нора попыталась сменить позу. Сие действие отдалось проникновением
тысячи невидимых иголок в вышеупомянутую ногу, и девушка завопила.

-- Это не до конца, -- заявила Агата. -- Я сижу, как в соломенном кресле
сидел бы мужик\ldots{} -- она почувствовала, что мысль вот-вот
ускользнет, и попыталась ускориться, в итоге запуталась, отчего речь её
стала ещё медленнее. -- На улице\ldots{} Сидел бы чувак\ldots{}
Такой\ldots{} Высокий\ldots{} -- истерический смех. -- Сидел бы
такой\ldots{} И ел грибы, а к нему подошел мальчик\ldots{}

Тут Нора не выдержала и разразилась очередным приступом атонального
смеха.

-- Какие грибы, ты чё? Невесомые?

-- Да-да-да! -- подтвердила Агата. Сделала она это с такой яростной
готовностью, словно всю жизнь ждала, пока кто-нибудь да спросит у неё о
невесомых грибах. -- Так вот\ldots{} Мальчик подошел к нему\ldots{} И
такой\ldots{} Я сплю с твоей сестрой\ldots{} Случайно сказал\ldots{} А
тот должен был по сценарию разозлиться\ldots{} На него накричать\ldots{}
А он такой: ммм, ммммммммм\ldots{} та ты садись\ldots{} Для начала
присядь\ldots{} И сиди\ldots{} -- Агата вдруг заплакала, что исказило её
последующие слова. -- Под грибами он был\ldots{} Я себя как-то так и вот
так\ldots{} -- тут её голос сорвался, вновь пропитавшись небывалыми
эмоциями, достойными лучших артистов «Глобуса». -- Трагически! Господи,
как же трагически!

Нора в растерянности уставилась на подругу, не прекращая при этом
смеяться.

-- Чё у тебя все трагически да трагически? -- наконец, заговорила она.

Издав странный возглас, который её собеседнице так и не удалось
идентифицировать, Агата прекратила плакать так же внезапно, как и
начала.

-- Ой-ой-ой-ой-ой! -- заверещала она. -- Не делай такой голос! Как у
дедушки\ldots{} Который повесился\ldots{} В фильме\ldots{}
Советском\ldots Но не умер\ldots{}

\section*{68}\label{68}
\addcontentsline{toc}{section}{68}

\markright{68}

-- Я знаешь что\ldots{}

-- Что?

-- На крыше сижу\ldots{} -- сказала Агата, продолжая совершать свои
таинственные перемещения. -- С сигаретой в руках\ldots{} Смотрю на
закат\ldots{} И голос у меня как у Джонни Кэша\ldots{}

Мечтательность её тона резко сменилась всё тем же трагизмом.

-- Но\ldots Зачем сидеть на крыше, когда ты на самом деле сидишь
взаперти? -- печально спросила Агата. -- Ох, волосы длиннее стали! Какие
кто?

-- Кто? -- Нора непонимающе моргнула.

-- Эти существа на «С». Ты их уже раз пять за ночь упоминала.

Элеонора принялась размахивать руками, словно пытаясь прогнать рвущийся
наружу смех.

-- Была речь, была, -- просмеялась она. -- Хотя нет\ldots{} Таких не
упоминала.

-- Они вертикальные? -- вдруг спросила Агата.

-- Чё? Я такого не говорила.

-- Нет, я сама увидела\ldots{} Ты мне показала\ldots{}

Ещё одна пауза.

-- А они все идут и идут, -- с тревогой сказала Агата.

-- Кто?

-- Ну, люди, которые слышат.

Нора понимала, что это всего лишь действие травы, но тревога подруги
необъяснимым образом настигла её саму. Девушка перестала смеяться. Она
приподнялась на локтях и взволновано взглянула на собеседницу.

-- Кто идет, Агата?

Та помедлила, вероятно, обдумывая ответ.

-- Мне кажется, девять, -- безразлично пояснила Агата.

*

-- Стало понятней, когда в музыке молния ударила, -- оповестила Агата
после десятиминутной тишины.

Лицо её приняло восторженное выражение.

-- Я тебе говорила? -- крикнула Агата. -- Я с самого начала хотела
сказать! Я сейчас так, знаешь, всё как будто бы быстро\ldots{}

-- Да чё ты свой нос так держишь?

Громкий смех.

-- Нет, не отпадет! -- сквозь хохот продолжала Нора.

-- А-а-а, он уйдет, -- взвыла Агата. Она осознала, что и впрямь держит
нос, но до этого момента мысль о том, что тот может сбежать в её голову
как-то не приходила. А вот теперь пришла, и наша трагическая джанки
вцепилась в него всеми пальцами левой руки. -- Уйдет\ldots{} Пускай
будет!

-- Чё ты несешь? -- членораздельно произнесла Нора, уповая на то, что от
этого суть сказанного лучше усвоится собеседницей, хотя и сама она никак
не могла остановить поток истерического смеха. -- Что пускай будет --
нос? Как будто у тебя есть выбор\ldots{}

-- Слушай! -- возбужденно прервала её другая девушка. -- Не
обращай\ldots{} Не обращай внимания, на то, что я нос держу\ldots{}

Но, у Норы уже разыгралась фантазия.

-- Сейчас он улетит! -- крикнула она и рассмеялась. -- Н-о-о-о-с, иди
сюда! Носик мой, носик, останься!

Агата еще крепче сжала свой бедный нос. Дышать теперь приходилось ртом.
Однако, укуренная владелица «Маятника» чувствовала, что просто обязана
поведать то, что собиралась.

-- Постой! НЕ ОБРАЩАЙ, ПОЖАЛУЙСТА, ВНИМАНИЯ НА ТО, ЧТО Я НОС ДЕРЖУ!

-- Но-о-о-с, -- пропела Элеонора.

-- Не-не-не, я должна сказать! -- заявила она, а затем повторила эту же
фразу для пущей информативности. -- Я сейчас как будто бы стою на
балконе\ldots{} Ты только дослушай\ldots{} Я сейчас стою на балконе
таком\ldots Античном\ldots{} В греческом дворце\ldots Может быть\ldots{}
Или нет\ldots Стою на балконе.. Волосы у меня так красиво
подобраны\ldots{} Я держусь за перила\ldots{} Белые\ldots И со стороны
вижу, как я стою и смотрю на то\ldots{} Как я с тобой тут сижу\ldots{}

-- Да?

-- Да! И поэтому всё такое\ldots{} Тарахтит быстро так, быстро! А на
самом деле\ldots{} Всё очень медленно\ldots{} Очень\ldots{}
Трагически\ldots{} Мы так медленно думаем\ldots{} -- она устало
вздохнула.

Нора тоже что-то призадумалась.

-- Ну ничё себе, -- ответила она.

Наступила длительная пауза. Когда Норе уже показалось, что подруга
уснула, та вдруг подняла голову, чтобы неуверенно спросить:

-- В каком дворце?

-- В греческом, -- тут же ответила девушка.

-- Да-да-да-да-да\ldots{} Или греческом\ldots{}

Затем Агата пробубнила ещё что-то нечленораздельное и вновь смолкла.
Спустя четверть часа она вяло махнула рукой в сторону проигрывателя.

-- Включи мне Джой Дивижн, -- поразительно трезвым голосом попросила
Агата.

И тут же отключилась.

\section*{69}\label{69}
\addcontentsline{toc}{section}{69}

\markright{69}

Солнце лишь начинало вставать, украдкой бросая на кладбище первые лучи
своего света. Элеонора распахнула глаза и тут же пожалела о быстроте
этого действия: затылок пронзила сильная головная боль, которая,
казалось, и не утихала с прошлой ночи, а только дремала вместе с
девушкой.

Нора болезненно застонала и осмотрела спальню, надеясь обнаружить здесь
какую-нибудь жидкость. Пить хотелось жутко.

Её разбудил гулкий отзвук шагов, звучавших то ли из стен, то ли из её
собственного воображения, причем, на этот раз наша героиня более
склонялась к последнему варианту. Проспав всего-то пару часов, Нора
чувствовала необъяснимую бодрость, так что вскоре ей пришлось оставить
попытки вновь погрузиться в сон. Девушка с трудом добралась до душа: она
ощущала мерзкую ломку во всём теле, какая бывает, когда встаешь после
долгой болезни.

«Маятник» ещё спал.

Владислав всю ночь провел за просмотром первого сезона «Фарго», и теперь
с чувством выполненного долго сопел на чердаке. Агата так и осталась
дрыхнуть в своей гостиной и, вероятно, собиралась не открывать глаз до
тех пор, пока не компенсирует все бессонные ночи последнего месяца.
Станислав Эдуардович, дожидаясь своей очереди преобразиться к церемонии,
отдыхал в морозильнике мастерской Ласло и, вроде как, не собирался
просыпаться.

Какое-то время Нора бесцельно бродила пустыми комнатами похоронного
бюро, совершенно не зная, чем себя занять. Её обуяла глубокая
беспричинная тоска, -- девушка понимала, что это, видимо, и есть
отходняк, но ничего не могла с собой поделать -- которая позже вылилась
в пиццу и двойную порцию шоколадного латте.

Начала она с уже откупоренной кем-то (ну кто бы это мог быть?) бутылочки
красного бургундского.

В своих мыслях Элеонора вновь вернулась к пустующим апартаментам, --
единственным запертым дверям во всём доме, -- понимая, что сейчас никто
не сможет помешать проведению её скромной экспедиции, но тут же отогнала
эту затею: ничем необоснованное любопытство можно отложить и до лучших,
лишенных мигрени времен. Но, было ещё кое-что, нарушающее её и без того
шаткий покой.

Однажды, в стремлении повысить свои писательские навыки, девушка
штудировала веб-пространство, просматривая статьи для начинающих
авторов. \emph{«У каждой написанной фразы}, -- так начиналась одна из
них, -- \emph{если книга достойная, у каждого свершенного действия
должен быть смысл. Он не всегда лежит на поверхности и ввиду своей
утонченности легко может затеряться среди похожих, искусственно
выдуманных высказываний, или же событий, служащих всего лишь декорациями
в среде обитания этого самого смысла. Читателю предоставляется
возможность взглянуть на верхушку айсберга, дабы в дальнейшем иметь шанс
расставить все точки над «і», но, откровенно говоря, к этому прибегает
крохотный процент реципиентов. Хитрость умелого пера заключается в
красноречивом намеке, сделать который нужно своевременно и без особого
фарса, так, чтобы спустя n-ное количество страниц читатель мог вернуться
к этому моменту и осознать суть написанного\ldots»}\footnote{Цитата из
  моей же старой статьи.}

Несмотря на свою простоту, совет был дельный, и Нора собиралась
непременно использовать его при создании своего романа, как только ей
удастся откопать в собственной голове тот самый смысл, идейный посыл,
который бы можно было передать читателю.

Однако, сейчас речь шла вовсе не об этом.

Всё ещё затуманенное сознание нашей героини в принципе понимало, что
Агата Рахманинова -- вовсе не книга, а чертовски странная дамочка, к
тому же накануне обкурившаяся крепче, нежели собиралась, но, как вы уже
поняли, было тут ещё что-то\ldots{}

«У каждой написанной фразы, -- размышляла Элеонора, -- должен быть
смысл, и укуренный бред Агаты -- не исключение».

Девушка давала себе отчет в том, что её нынешние думы является всего
лишь попытками скрасить унылое утро капелькой приключений, но, вместе с
тем, чувствовала, что наркотический монолог подруги, или какая-то его
часть, в дальнейшем ну просто должна преобразиться в более осмысленные
слова. Причем, скорее всего та, что так встревожила, если не сказать
напугала, нашу героиню.

Прикрыв глаза, Нора без устали перебирала в голове отрывки сказанных
Агатой фразочек, пока не остановилась на\ldots{}

«Они идут и идут. Люди, которые слышат\ldots{} Слышат что -- запах? --
девушка затаила дыхание, словно процесс этот мог помешать ей думать. --
Стоп. Ну, при чём здесь запахи?»

Но Элеонора чувствовала, что мыслит в правильном направлении. Хотя,
возможно, это лишь вино давало о себе знать. Ни для кого не секрет: наша
героиня всегда быстро хмелела.

«Хо-ро-шо, -- глубокий вдох. -- Они идут и идут. Люди, которые слышат
запах. Они идут туда, откуда доносится запах, или уходят, потому что
услышали запах? Нет, это всё не то\ldots»

-- Они идут и слышат запах! -- воскликнула Нора, удивившись неожиданной
громкости своего голоса. -- Люди продолжают идти, несмотря на то, что
слышат запах! Какой-то неприятный запах.

Воодушевившись на некий миг, девушка вновь приуныла. Эти слова
показались слишком нелепыми после того, как она произнесла их вслух.

-- Ой, все, -- устало процедила Нора. -- Мне надо пройтись.

\section*{70}\label{70}
\addcontentsline{toc}{section}{70}

\markright{70}

Элеонора покинула пределы кладбища без лишних раздумий. Пребывание на
холодном воздухе действовало на неё благосклонно: в скором времени в
голове прояснилось.

Ступая по голой, местами покрытой изморозью, земле, Нора ещё не знала,
куда держит путь, пока не увидела приближающийся к остановке трамвай.
Она ускорила шаг и вскоре впрыгнула в пустую кабину. Город просыпался и
спустя несколько остановок трамвай постепенно начал наполняться людьми,
спешащими на работу, или учебу. Нора же никуда не спешила. Уткнувшись в
электронную книгу, она одиноко сидела в самом конце салона, изредка
бросая взгляд на проплывающие за окном городские пейзажи.

Спустя минут двадцать, или около того, занятие это ей наскучило, так что
девушка сошла на первой попавшейся остановке, успев вовремя опомниться и
отскочить в сторону. Помедли Нора ещё секунду, и толпа граждан, что
спешили захватить немногочисленные сидячие места, непременно внесла бы
её обратно в трамвай.

Девушка осмотрелась. Всё вокруг выглядело так, как обычно обстоят дела
во Львове под конец зимы: уютно, но сыро и пасмурно. Её силуэт мелькал в
окнах кофеен и витринах уличных лавок, всё еще мокрых после недавнего
дождя, пока девушка бесцельно шагала сквозь город. Утренняя ходьба
успокаивала, предавала некий смысл начавшемуся дню.

Со временем огни вывесок сменились более спокойными видами. Наша героиня
не знала, куда идет. Ноги сами вели её вдоль кирпичных стен, прокладывая
путь среди пастельных домиков с крохотными балконами. Улочки становились
всё уже, оставляя позади суету и шум машин, когда девушка нырнула под
вычурную арку и поняла, что наткнулась на тупик.

Это был небольшой квадратный дворик, окруженный чередой разноцветных
пятиэтажек. Внутри было тихо -- никакого ветра, а среди постояльцев были
только кошки, сгрудившиеся на скамейке.

Девушка подустала. Прогулка длилась третий час и она чувствовала, что
уже практически обрела былую бодрость духа. Нора плюхнулась на одну из
лавочек, расположенных под сенью широкого дуба и тут же почувствовала,
как на самом деле вымоталась: аукнулись ночные посиделки, подкрепленные
до неприличия коротким сном. Смартфон -- одно из чудес современной
техники, благодаря которому в кармане можно было иметь всё, что душе
угодно: музыку, фильмы и целую библиотеку. Не доставало, разве что, еды,
но и это лишь вопрос времени.

Элеонора собиралась почитать, но вместо этого откинулась назад и
принялась изучать серое небо, проглядывающее сквозь обнаженные ветви,
гадая, успеет ли добраться домой до того, как хлынет дождь. В том, что
ливню суждено быть девушка даже не сомневалась.

Веки тяжелели. Держать глаза открытыми с каждым мигом становилась всё
труднее. Нора бы так и уснула посреди пустующего дворика, если бы не
мелодичный голос, внезапно вырвавший её из небытия. Поначалу девушка
сочла, что он ей пригрезился: слишком уж окрыляющими были слова.

\begin{quote}
\emph{Ах, я вспоминаю ясно, был тогда декабрь ненастный,} \emph{И он
каждой вспышки красной тень скользила на ковер.} \emph{Ждал я дня из
мрачной дали, тщетно ждал, чтоб книги дали} \emph{Облегченье от печали
по утраченной Линор,} \emph{По Святой, что там, в Эдеме, ангелы зовут
Линор, --} \emph{Безыменной здесь с тех пор.}\footnote{Строки из
  стихотворения Эдгара Алана По «Ворон».}
\end{quote}

Нора сладко улыбнулась и, -- она ни с чем не могла перепутать этот
голос, пускай и слышала его всего во второй раз -- приоткрыв глаза,
взволновано огляделась по сторонам. Голос тем временем продолжал.

\begin{quote}
\emph{Обезумевший, уставший, ждал я утра, дня, иль ночи } \emph{-- в
поисках бродил что мочи,} \emph{Содрогаясь от волненья, по утраченной
Линор\ldots{}}\footnote{Попытки говорившего продолжить вышеупомянутое
  стихотворения. Вполне успешные, как по мне.}
\end{quote}

-- А теперь вот, кажется, нашёл! -- весело добавил голос, пронизанный
пленительным вампирским акцентом.

\section*{71}\label{71}
\addcontentsline{toc}{section}{71}

\markright{71}

С рассветом того же дня Молния в задумчивости мерил шагами пространство
собственной кухни, периодически заглядывая в спальню, чтобы бросить
встревоженный взгляд на стоявшую у постели фотографию.

Он жаждал беседы с Норой. Понял это ещё тем декабрьским утром, увидев
бредущую промеж могил девушку. Случайную встречу на кладбище едва ли
принято считать удачным поводом для свидания, но мужчина в первое же
мгновенье осознал, всей душой почувствовал, что должен был с ней
заговорить.

Но он не заговорил. Таким уж был Молния: какие бы чувства и переживания
не таились в его душе, мужчина всегда погребал их под тоннами других,
куда менее волнующих дел.

Сделал Кровавую Мери, но так к ней и не притронулся. От одной мысли об
алкоголе начинало мутить. Молния вернулся в спальню и, обхватив голову
руками, присел на край кровати.

Страх и надежда бурлили в нём, не позволяя сосредоточиться. Мужчина
перевёл взгляд на снимок -- фотографию смеющихся влюбленных,
запечатлевшую его последние любовные отношения, которые уже года два как
прекратили своё существование. За это время он отрастил волосы, побывал
в одиннадцати странах, написал семь песен и трижды представал перед
судом за подозрение в хранении марихуаны, но так и не понял, почему
однажды остался без неё, в безмолвном одиночестве.

Боль утраты никуда не делась, а если и затихала на время, то лишь для
того, чтобы ударить с новой силой. Его сердце не было просто разбито,
оно разбивалось каждый раз, когда Молния вспоминал свою единственную
истинную любовь. Мысли о счастье, которое однажды обрушилось на него с
такой неожиданностью, теперь делались крайне болезненными, как, в
общем-то, и все мысли касательно того, что было. Худшую боль приносили
разве что думы о том, что могло бы быть, но так никогда и не произойдет.

Боль, печаль, тоска, уныние -- существует много слов, относящихся к
субъекту повествования, но ни одно из них не способно в точности описать
агонию разбитого сердца, особенно если это состояние годами не сдает
своих позиций. Молния оказался однолюбом и в полной мере прочувствовал
своё одиночество, осознав, что так и не встретит человека, для которого
смог бы приоткрыть ржавые створки собственной души. Было в этом и нечто
положительное -- никакой новой боли.

А затем он встретил Нору. Увидел её, отворив дверь затхлого
наркопритона, и новая волна чувств охватила мужчину в самый неподходящий
момент: всего минутой ранее он собирался ширнуться. Впервые за свои
неполные тридцать.

Страх предстоящего разочарования (а он был убежден, что именно такое
завершение обретут любые попытки стать счастливым) смешался с
предательским душевным трепетом, когда Молния вновь увидел Элеонору.
Последний, кажется, победил, потому как мужчина, который и раньше в
равной степени верил и в судьбу, и в глубину собственных чувств, просто
не мог поверить, что подобные совпадения случайны.

«Вот бы кто из знакомых умер\ldots{} -- с печальной ухмылкой подумал
Молния. -- Тогда я смог бы наведаться к Агате и встретить \emph{её}, раз
уж судьба вдруг решила быть ко мне благосклонна\ldots{} Встреча
непременно бы состоялась.»

Он рассмеялся то ли глупости, то ли наивности собственных суждений.
Вновь взглянул на фотографию и поспешно убрал её в дальний ящик комода.

*

Время протекало мучительно медленно. Где-то за хмурыми облаками
пряталось полуденное солнце. Накинув поверх махрового халата пальто,
Молния таки взял в руки свой коктейль.

Квартира, в которой он жил, находилась на последнем этаже и располагала
мансардными окнами, занимавшими большую часть потолка спальни, которая,
кстати, была единственной, но весьма просторной комнатой. В моменты
обострения одиночества, мужчине казалось, что в доме даже слишком много
места для него одного.

Молния относился к тому типа людей, которые курят исключительно на
балконе. Зимой эта привычка была особенно неудобной: балкон в его
квартирке не был застеклен. Мужчина подкурил и облокотился о перила,
чувствуя как первые капли дождя касаются привыкшей к теплу кожи.

Нет, он определенно искал встречи с Элеонорой.

Однако, процесс этот не был долгим. Сама того не ведая, девушка нашла
его первой.

*

Нашей героине никогда не нравилось её имя. Полная форма, считала она,
ещё ничего, но вот сокращения звучали ужасно в своей простоте, но
девушка всегда была слишком вежливой, чтобы требовать от окружающих
обращения к себе по полному имени, опасаясь, что такой шаг будет
расценен как высокомерие. Говоря об этом, общение с хозяйкой похоронного
бюро приносило Норе облегчение: та вообще редко обращалась к ней по
имени, но, если и делала это, то называла подругу исключительно
Элеонорой, хотя девушка не помнила, чтобы хоть раз упоминала о
раздражении, которое испытывала, слыша краткую форму своего имени.

-- \ldots{} содрогаясь от волненья по утраченной Линор\ldots{} А теперь
вот, кажется, нашёл!

Нора с трудом, но определила источник звука и, обернувшись, вновь
запрокинула голову. На последнем этаже соседнего дома, пуская дымовые
кольца, ей улыбался Молния.

Она знала лишь одну Линор и то не была знакома с ней лично.

-- Это разве не о мёртвой девушке? -- спросила Нора, чувствуя что, сама
не зная от чего, вот-вот зальется краской.

-- О ней самой. Но лишь первое четверостишье.

Вдали громыхнуло. Секундой позже девушка вздрогнула, ощутив на лице
несколько ледяных капель. Одна из них повисла на кончике носа и никак не
хотела убираться прочь. Нора поспешно смахнула её рукавом своей шубки.
Заметив это, Молния тоже махнул рукой.

-- Поднимайся, -- улыбнулся он девушке. -- Если в твоих планах на
сегодня, конечно, не числится сон под проливным дождем.

Нора улыбнулась в ответ.

Такого в её планах не было.

\section*{72}\label{72}
\addcontentsline{toc}{section}{72}

\markright{72}

За окнами похоронного бюро уже битый час бушевала гроза, что лишь
усилилась с наступлением темноты. Владелица «Маятника» провела в
блаженном забытьи около восемнадцати часов. Проснулась она вечером, --
часам к десяти -- тем самым практически возобновив свой привычный
распорядок дня. Точное время своего пробуждения Агата назвать не могла:
она опомнилась, стоя на кухне за приготовлением чашки крепкого чая. Как
поднялась и прошла на сюда девушка не помнила.

Владислав сидел за столом, нарезая ингредиенты для какого-то салата с
курицей. За всё время он не проронил ни слова, только периодически
одаривал подругу укоризненными взглядами, до которых последней,
откровенно говоря, не было особого дела.

-- Ты видел, как я поднялась? -- нарушила тишину Агата.

Ласло отрицательно качнул головой.

-- Видел, как ты спала, когда спускался к обеду.

-- В гостиной?

Он кивнул.

-- А потом?

-- Потом обедал.

Эта фраза прозвучала без единой нотки цинизма, так что Агата не сразу
поняла, злиться ей, или смеяться, но слишком усердно пыталась в этом
разобраться, так что боль в висках начала предостерегающе пульсировать.
Девушка прикрыла глаза и перефразировала свой вопрос.

-- Что я делала, когда ты увидел меня в следующий раз?

-- Спускался по своей лестнице, -- не отрываясь от готовки, ответил
Владислав, (тут стоит заметить, что своими он называл все скрытые
тоннели, ходы и лестницы) -- потому что услышал шаги в холе. Увидел, как
ты выходишь из комнат.

-- Из своих апартаментов?

-- Нет. Из закрытых.

Агата открыла глаза и вдумчиво оглядела комнату, в которой находилась.
Она по-прежнему чувствовала себя слегка потерянной, но чай понемногу
начинал окунать её в реальность, возвращая чувство времени.
Солнцезащитные очки девушка сдвинула на лоб и теперь раздумывала, а не
надеть ли их снова.

-- А где Элеонора? -- вдруг вспомнила Агата.

Владислав пожал плечами.

-- Не видел её с прошлого вечера.

Опустившись за стол, Агата закусила губу. Происходящее нравилось ей все
меньше и меньше.

-- Тревожишься? -- спросил Владислав.

Это был первый вопрос, заданный молодым человеком за прошедшие двое, а
то и трое суток. Девушка подняла голову и встретилась взглядом с глазами
друга, но те не сказали ровным счетом ничего нового -- в глазах Ласло
уже давно царила спокойная атмосфера обреченности.

-- Зима подходит к концу, -- объяснила Агата.

Ее собеседник скорбно кивнул.

-- Знаю, -- он подвинул к ней тарелку. -- Вот, поешь салатик.

\section*{73}\label{73}
\addcontentsline{toc}{section}{73}

\markright{73}

Несмотря на недавно охватившее её спокойствие, Элеонору вновь увлекли
тревожные видения, стоило ей лишь провалиться в глубокий сон. Молния
развалился рядом, периодически поглаживая обнаженные плечи девушки. Та
лежала к нему спиной и к тому же частенько ворочалась, так что, обнимая
Нору, мужчина видел лишь часть её лишенного косметики личика, но
выражение лица возлюбленной ему совсем не нравилось.

-- Мне приятно находиться рядом с тобой, -- сонно произнесла Элеонора,
когда страсть обоих поутихла и, расслабившись, девушка отдыхала на плече
Молнии. -- Так\ldots{} спокойно.

Молния знал, что она имеет в виду, пускай и не мог выразить свои чувства
словами. Вместо этого он впервые за долгое время изменил старой привычке
и задымил прямо в постели.

Его также охватило спокойствие. Надежда, думал мужчина, вот что в итоге
причиняет душевную боль. Как бы хорошо ты не знал жизнь, сколько бы раз
не разбивали твоё сердце, стоит только подумать, допустить саму мысль о
том, что можешь обрести счастливое будущее с другим человеком -- и вот
ты попал. Остальным займётся твоё подсознание. Именно это Молния и
сделал, впервые увидев Элеонору.

Девушка ещё не успела переступить порог его квартиры, а Молния уже
загадочно улыбался, разглядывая её кричаще красные глаза.

-- Я думал, травка для Агаты, -- заметил он, заставив Нору смутиться.

Она, вроде как, не рассказывала стрёмных историй, в отличии от своей
сожительницы, но, тем не менее, всю ночь смеялась чужим лошадиным
хохотом, за что теперь и испытывала понятную неловкость.

Однако неловкости этой наши герои лишились с той же скоростью, с какой
часом позже лишились одежды.

*

Заложив руку за голову, Молния лежал на широкой кровати посреди комнаты,
единственным источником света в которой был прикроватный ночник.
Возвышающееся над ним мансардное окно покрылось миллионами дождевых
капель, сквозь которые не проглядывалось ничего, кроме зияющей темноты.
Мужчина затушил сигарету в вышеупомянутом стакане с Кровавой Мери, так
что теперь она стала ещё и никотиновой, и осторожно оторвался от
подушки, стараясь не потревожить свою Линор.

Скрипнула кровать. Девушка простонала что-то во сне, но не проснулась.
Молния сполз со своего места и на носочках направился в дальний угол
комнаты, где, как он помнил, остались его штаны. Тишину нарушила трель
звонящего телефона. Мужчина подскочил от неожиданности.

Нора открыла глаза как раз вовремя, дабы в самом разгаре застать его за
прогулкой с голой задницей.

*

Тело мужчины покрывали «цветы молнии», -- бордовые следы ветвящихся
электрических зарядов, что образуются под воздействием высокого
напряжения -- в медицине называемые фигурами Лихтенберга. Тонкие полосы
иголок шли вдоль спины Молнии, устилая его широкие плечи.

У неё уже имелась прекрасная возможность вблизи ознакомиться с
отметинами своего\ldots{} Нора не была уверена, что может называть его
своим молодым человеком. Так или иначе, она села на кровати и,
прикрывшись одеялом, разглядывала следы на теле Молнии, пока тот,
по-прежнему обнаженный, замер с трубкой у уха, объясняя какому-то
новоиспеченному джанки, что больше не занимается транспортировкой некого
Виктора Палыча.

-- Нет, Гертруды тоже нет, -- настаивал Молния. -- Говорю же тебе, кроме
пыли только джараш и диски, -- он замолчал, вероятно, случая реплику
собеседника. -- Звони, если хочешь, но уверен, у него одна солома.

Мужчина выслушал ещё какую-то, по всей видимости, не особо лестную
фразу, после чего отключился и устало потёр лоб. Он виновато взглянул на
возлюбленную.

-- Прости, -- произнес Молния. -- Не стоило тебе это слышать. У нас на
Леннона припрятана упаковка ампул, а у торчков вроде этого просто нюх на
такие вещи, -- и он поспешно добавил, как бы оправдываясь: -- Но я не
кололся. Ни разу. Драг на крайний случай.

Нора удивленно взглянула на собеседника, забыв о том, что держит одеяло,
так что то выскользнуло из руки девушки, обнажив её левую грудь.

Молния улыбнулся.

-- Не бери в голову, -- произнес он, приложив немалые усилия воли к
тому, чтобы не отводить взгляд от глаз Элеоноры, и прежде, чем та успела
что либо возразить, спросил: -- Ты голодна?

После недолгих раздумий девушка кивнула.

-- Мне нужно отлучиться минут на десять, максимум пятнадцать, -- он уже
натягивал штаны. -- Только передам несколько кораблей старому клиенту, а
потом сможем выбраться в город, если хочешь.

Прогулки под дождём -- вполне себе романтичная вещь, но только не под
проливным и не в феврале, когда на улице минус. К тому же, города Норе
на сегодня хватило с лихвой.

-- Давай просто закажем пиццу, -- ответила она.

Мужчина улыбнулся.

-- Две.

Он надел рубашку и, вернувшись к кровати, склонился, чтобы поцеловать
девушку. Та поцеловала его в ответ, и Молния заключил Элеонору в крепких
объятьях. Затем, не выпуская её ладоней из своих рук, мужчина слегка
отстранился и мечтательно взглянул на девушку.

-- О чём ты думаешь? -- спросила она.

-- Тебе не понравится, -- тут же отозвался Молния.

Нора нахмурилась.

-- Скажи мне, и мы узнаем наверняка.

На лице Молнии скользнула меланхолическая улыбка. Он произнес фразу, от
которой наша героиня на мгновенье усомнилась в своей трезвости.

-- Я думаю, что люблю тебя, -- сказал Молния.

Вопрос о молодом человеке отпал как-то сам по себе.

\section*{74}\label{74}
\addcontentsline{toc}{section}{74}

\markright{74}

В «Маятник» Нора вернулась лишь следующим утром. Прогуливаясь среди
могил, девушка боялась, что вот-вот позабудет об уважении к усопшим и
пустится в пляс прямиком на кладбище -- так она была счастлива.

Молния сказал, что любит её. Нора поняла, что он собирается сказать ещё
до того, как мужчина открыл рот, -- всё и так просматривалось в его
глазах -- хотя до последней минуты не верила в то, что подобный исход
событий имеет место быть в её жизни. Девушка думала, что сможет ответить
Молнии тем же, но не смогла: времени прошло слишком мало.

Вороны кружили над крестовидным памятником, крикливо сражаясь за лучшие
места. Тем утром их склока, сопровождаемая шумом дождя, казалась девушке
самой сладкой песней на всём белом свете.

Свернув за угол, наша героиня увидела хозяйку «Маятника», глядящую в
окно со сложенными на груди руками, и про себя отметила, что подруга,
умышленно, или нет, но определенно вернула себе траурный вид. За спиной
Агаты стоял худощавый молодой человек с невероятно бледной кожей и
светлыми волосами. Лишь одно короткое мгновенье девушке удалось
созерцать Владислава. Тот поспешно отступил назад и исчез в темноте
похоронного бюро.

Элеонора приветственно махнула подруге. Та не ответила. Она тоже
отступила вглубь комнаты и принялась спускаться по лестнице, судя по
передвижениям темного пятна, мелькающего в окнах.

Когда раскрасневшаяся от холода и любви Нора закрыла за собой дверь,
Агата уже стояла у края ведущей в спальни лестницы. Её лицо не выражало
никаких эмоций.

Нора же почувствовала иглы ноющей боли где-то в лицевых мышцах и поняла,
что уже долгое время улыбается во все зубы.

-- Где ты была? -- спросила дама в чёрном низким, пугающе загробным
голосом.

Девушка взвесила все «за» и «против» в течении каких-то пары секунд, а
затем, так и не покидая холл, изложила события, случившиеся с ней за
последние сутки, опустив, разве что, подробности полового акта.

Соблюдая каменное лицо, Агата слушала молча.

-- \ldots{} и единственное, что меня смутило, -- рассказывала Нора, --
так это его поспешное откровение. Я имею в виду, ну кто признается в
любви во второй день знакомства, понимаешь? Пожалуй, я верю в то, что он
и впрямь это чувствует, но, всё равно, это как-то дико.

Закончив свой местами даже слишком трогательный монолог, Элеонора в
ожидании взглянула на подругу, чьё лицо в его застывшей
невыразительности едва ли внушало воодушевление: прежде девушка видела
такие лица лишь у клиентов похоронного бюро и то до того, как над ними
поработал Владислав.

-- Скажешь что-нибудь? -- поникшим голосом произнесла Нора.

Девушка полагала, что её подруга была слишком демократична, дабы
осуждать случившееся и, вероятно, слишком похуистична, чтобы в полной
мере понять ту палитру чувств, что она теперь испытывала, так что Нора
не надеялась на бурную, а уж тем более радостную реакцию, но ей, всё же,
хотелось услышать хоть какое-нибудь подобие комментариев.

-- Сколько раз вы занимались сексом? -- безучастно спросила Агата.

Таким тоном обычно говорили о погоде, или до чёртиков наскучившей песне,
что в который раз транслируется по радио.

-- Четыре, -- краснея, ответила Нора.

-- А сколько раз говорили? Всерьёз, об отношениях, например.

Нора удивленно подняла брови. Одновременно с этим действием в стенах
здания раздался глухой удар, с помощью которого Ласло обычно оповещал
других обитателей «Маятника».о своём недовольстве

-- Каких отношениях, Агата? Мы знакомы-то третий день.

Владелица «Маятника» пропустила это мимо ушей.

-- Значит, не говорили?

Девушка покачала головой.

Агата кивнула, словно ждала именно такого ответа. Развернувшись, она
начала молча подниматься по ступеням и вскоре исчезла из поля зрения
подруги.

\section*{75}\label{75}
\addcontentsline{toc}{section}{75}

\markright{75}

Сменив мокрую одежду на пижаму, Элеонора выплыла из сладостных грез о
Молнии и вспомнила, что до сих пор злится на то ли грубость, то ли
дерзость своей подруги. Филантропия филантропией, но такое явное
безразличие выглядело чересчур по-хамски, даже для Агаты, особенно после
душевного рассказа и -- главное -- описания собственных чувств, которыми
для Норы было и без того непросто с кем-либо поделиться\ldots{}

Короче говоря, девушка решила, что заварит себе зеленого чаю, а уж потом
непременно разберется с этим дерьмом.

Однако, долго ждать не пришлось: закинув ногу на ногу, Агата и так
сидела на кухне, только пила она совсем не чай.

-- Выкладывай, -- с порога потребовала Нора.

Хозяйка похоронного бюро вопросительно подняла бровь, но особого
интереса к диалогу не проявила. С зажатой в зубах сигаретой, она
бездумно водила пальцем по ободку винного бокала.

Позабыв о чае, Элеонора опустилась за стол и теперь сидела напротив
Агаты, чувствуя нарастающее раздражение в сторону поведения последней.

-- Мать твою, Агата! -- впервые за долгое время Нора повысила голос,
пытаясь привлечь внимание подруги. -- Ты странно себя ведёшь! Объясни
мне, наконец, что, чёрт возьми, вообще здесь происходит?

Ответа не последовало.

-- Почему ты ведёшь себя так странно? -- настаивала Нора. -- И почему у
тебя лицо как у жертвы инфаркта.

-- Ты заблуждаешься, -- вдруг произнесла дамочка в чёрном. -- Однажды я
перенесла инфаркт и, тебе придется поверить мне на слово, но в тот
момент у меня было совершенно другое лицо.

Элеонора помолчала, обдумывая услышанное.

-- Выкладывай, -- повторила она, на этот раз с большим спокойствием. --
Почему ты себя так ведёшь?

-- Зима близится к концу, -- Агата нервно затянулась сигаретным дымом.
-- Только тебе это мало о чём скажет, не так ли?

Девушка смерила так-себе-собеседницу подозрительным взглядом.

-- Ах\ldots{} -- улыбаясь, протянула Нора. -- Как же я сразу не
догадалась: ты снова курила свою дурь?

Лицо Агаты оставалось серьёзным.

-- Я курю только по четвергам, -- напомнила она.

Затушив сигарету, Агата в один заход осушила бокал и поднялась со своего
места.

-- Подумай о том, что я тебе сказала, -- бросила она, выходя из кухни.

*

Нора так и сделала. Ей понадобилось секунды четыре, дабы обдумать и
окончательно осознать полнейшую нелепость услышанного, а также придти к
выводу о том, что подобный ответ её совсем не устраивает.

Смачно выругавшись, наша героиня подскочила из-за стола, намереваясь
догнать вторую девушку, но надобность в этом сразу же отпала: Агата
вновь ожидала её, стоя у края ступеней. Причем выглядела так, словно
зависает здесь уже не первый час, а вот разогнавшейся Норе пришлось
резко притормозить, чтобы не столкнуть подругу с лестницы и самой,
вдобавок, не полететь следом.

-- Всё в порядке? -- поинтересовалась Агата, глядя на запыхавшуюся
девушку.

Нору такой вопрос чуть более чем просто высадил.

-- У меня такое чувство, что не всё\ldots{} -- переведя дыхание,
ответила Элеонора. -- Причём уже не первый день.

-- Так?

-- Тебя давно что-то тревожит, но ты, почему-то, не горишь желанием
посвящать меня в причины своего поведения. И вот сейчас, когда я
практически счастлива, ты поверх всего ещё умудряешься
игнорировать\ldots{}

Где-то за зеркальными стеклами Агата в нетерпении закатила глаза. Нора
это учуяла: когда начинаешь узнавать человека, тебе не обязательно
видеть всю картину, чтобы понять, что он именно делает в момент речи.
Так в темноте можно увидеть улыбку друга -- услышать по изменившимся
ноткам знакомого голоса, или, в случае с Агатой, понять, что та
закатывает глаза, благодаря слегка вытянувшемуся лицу и, скажем, по
приоткрытым губам, громко выдыхающим воздух.

Заметив нетерпение хозяйки «Маятника», которое уже в конец переполняло и
её саму, Нора решила перейти к сути дела.

-- С Молнией? -- спросила она. -- Между вам что-то было\ldots{} Ты
поэтому за меня не рада?

Агата изумленно приоткрыла рот.

-- О, дорогуша\ldots{} -- начала она и тут же печально рассмеялась.

Словно аккомпанируя раскатам грома, а стенах похоронного бюро раздался
стук, от которого Агата тут же отмахнулась.

-- Пойдем, -- позвала она и вскоре исчезла в темноте, увлекая Элеонору
за собой.

В закрытые апартаменты.

\section*{76}\label{76}
\addcontentsline{toc}{section}{76}

\markright{76}

По мере их приближения к единственной в «Маятнике» запертой двери, стук
в стенах нарастал, превращаясь в настоящий гул.

-- Да уйми ты свою истерию! -- гневно крикнула Агата. -- Хуже уже не
будет!

Ответив на это заявление ещё парочкой глухих постукиваний, Владислав в
итоге затих. Довольствуясь тишиной, Агата окунула руку в карман, чтобы
извлечь на свет немалых размеров ключ. Старомодный, с витиеватым кованым
наконечником, предмет этот походил на те, что обычно фигурируют в
сказках, только был довольно ржавым и не смог отпереть дверь с должной
скоростью, что тут же обламывало всё волшебство.

Со второй попытки замок-таки поддался и, толкнув преклонного вида дверь,
владелица «Маятника» жестом пригласила подругу внутрь.

Щелкнул выключатель. В первую очередь, Элеонора увидела, что перед ней
не комнаты, а одно сплошное помещение, являвшее собой небывалых размеров
спальню. Во-вторых, окружающая её обстановка оказалась завораживающей и
ввиду своей изысканности могла бы послужить достойными декорациями к
фильму о Дракуле. Гробов, конечно, не было, но ведь они находились всего
двумя этажами ниже.

Норе понадобилось какое-то время, чтобы осмотреться и справиться с
восхищением, а заодно и настойчивым желанием сделать как можно больше
снимков окружающих её предметов.

-- Что же ты сразу не сказала, что у тебя есть люкс? -- весело спросила
девушка.

От этих слов выражение лица Агаты обрело своё привычное ироническое
спокойствие. Дамочка подошла к резному гардеробу и, замешкавшись лишь на
мгновенье, отворила обе дверцы.

-- В этой комнате я храню воспоминания, -- произнесла Агата, разглядывая
что-то, лежащее в шкафу. Плавным жестом руки она поманила подругу к
себе. -- Подойди и взгляни.

Одолеваемая любопытством Элеонора была уже на половине пути, когда
увидела, на что смотрит Агата. Заинтересованность на лице девушки
сменилась удивлением, за которым последовал истинный шок, приправленный
едва уловимой ноткой понимания. На одной из полок, среди стопок
картонных ящиков, виднелись три портрета. Они не были идентичными, как
не бывает идентичной подпись, но, неизменно повторяя друг друга,
определенно носили одно авторство. Каждая из картин была улучшенным
вариантом предыдущей, -- результатом мастерства набитой руки -- и вместе
с тем негласной копией новогоднего подарка Агате -- портрета, который
уже второй месяц весел в гостиной второго этажа.

Девушка открыла рот, но так и не смогла ничего сказать. Встретившись с
изумлением в её взгляде, Агата поняла, что ждать дальше бессмысленно.

-- Ты уже в четвёртый раз приходишь в «Маятник» и каждый раз являешься с
первым снегом, -- вздохнув, произнесла Агата. -- Я не давала объявления
о сдаче комнат с октября две тысячи двенадцатого, но ты всё равно
приходишь в опозданием на час, показываешь мне ту старую заметку и
говоришь, что ожидала увидеть памятники вместо могил, а я отвечаю, что в
сущности это одно и то же, потому как обычно людям не возводят памятники
при жизни.

\bookmarksetup{startatroot}

\chapter{Часть III. Переломный Момент
\{-\#chapter-3=``\,``\}}\label{ux447ux430ux441ux442ux44c-iii.-ux43fux435ux440ux435ux43bux43eux43cux43dux44bux439-ux43cux43eux43cux435ux43dux442--chapter-3}

\emph{\ldots и темнота вокруг нее шумела, }

\emph{словно прибой сумрачного моря.}

\emph{--- Эрих Мария Ремарк}

\section*{77}\label{77}
\addcontentsline{toc}{section}{77}

\markright{77}

Молния ударила в окна пустующих апартаментов, словно желая усилить и без
того гнетущий драматизм наступившего момента. Над похоронным бюро
послышались мощные раскаты грома, укрепляющую атмосферу обречённости,
что теперь царила в душе Элеоноры.

К счастью, электрическое снабжение пока не спешило покидать «Маятник»,
и, пытаясь придти в себя, Нора пристально разглядывала окружающую её
обстановку. Агата не торопила подругу с вынесением каких бы то ни было
умозаключений, за что последняя была весьма благодарна.

-- Войдя сюда, я ко многому была готова, -- после длительного молчания
выдохнула Нора. -- Ожидала, что ты покажешь мне один из своих стеклянных
гробов, или ещё что-нибудь в таком роде.

Она слабо улыбнулась и, завидев выражение лица так-себе-собеседницы,
поспешно добавила:

-- Только не говори, что я уже выдавала тебе эту шутку, когда\ldots{}

-- Трижды.

Тщётно силясь справиться с безысходностью во взгляде, девушка запустила
руку в волосы, чувствуя, что на глаза вот-вот навернутся слёзы. Первая
мысль, посетившая голову Норы, казалась ей до ужаса нелепой, но вместе с
тем пугающе возможной. В итоге, наша героиня отважилась озвучить
собственные опасения.

-- Я что\ldots{}

-- Нет, -- прервала её Агата. -- Ты не умерла.

Услышав это, Нора издала вздох облегчения и даже сумела выдавить из себя
некое подобие улыбки, но следом в её голову пришла ещё одна, едва ли
воодушевляющая мысль.

-- Ты знаешь всё, что я скажу? -- печально спросила девушка.

-- Не всё, -- тут же ответила Агата, заставив Нору склоняться к тому,
что и этот вопрос не был для её подруги неожиданным.

Углубившись в размышления, Элеонора приуныла ещё сильнее. Заметив это,
Агата молча взяла девушку под руку и усадила на огромных размеров ложе.

-- Ты не призрак, -- повторила Агата, опускаясь рядом. -- Мы вместе
проверяли твои документы. С возрастом вышла небольшая несостыковка, но в
целом всё в порядке, -- заверила она.

-- Несостыковка?

Агата помедлила.

-- С каждым годом говорить это становится всё труднее, -- заметила она,
-- но тебе уже далеко не двадцать.

Элеонора ошарашено моргнула, и ощутила, как по разгоряченному от
волнения лицу скатилась одинокая слеза. Девушка не знала, смеяться ей,
или плакать, -- дурной знак -- но старательно подавляла в себе оба этих
импульса.

-- Сколько же мне на самом деле? -- боязно поинтересовалась Нора, то ли
от волнения, то ли в поисках морщин исследуя пальцами свою шею.

-- Расслабься, -- наблюдая за ней, ухмыльнулась Агата. -- Всего лишь
двадцать четыре.

-- А тебе сколько?

-- Двадцать пять.

Нора собралась с мыслями, пытаясь переварить полученную информацию. С
таким же успехом она могла бы выкурить вчерашнюю траву и отбить чечётку,
ни разу при этом не рассмеявшись.

-- Ты ведёшь себя старше, -- решила Нора, -- а выглядишь моложе. Иногда
это сбивает с толку.

Агата согласно кивнула.

-- В прошлом месяце мне не продали сигареты, когда я отказалась снимать
очки, -- вспомнила она.

-- А почему ты отказалась?

Владелица похоронного бюро вздохнула и издала протяжный стон, который её
подруга идентифицировала как выражение усталости, смешанное с явным
нежеланием касаться темы таинственных очков.

-- Вернемся к этому позже, -- пообещала Агата. -- Когда решим твою
проблему.

-- Ты говоришь это каждый раз? -- догадалась Нора. -- Только моя
проблема никак не решается\ldots{}

-- Ну, если честно, да, -- впервые (по крайней мере, за то время,
которое помнила Элеонора) Агата улыбнулась как нормальный человек,
показывая ровные зубы. -- И получается, что я вроде как не лгу.

-- И всё же?

-- Давай для начала справимся с одной историей, -- вновь вздохнула
Агата. -- В моей же нет ничего таинственного. Ничего такого, чего не
следовало бы ожидать. Я не вампир, -- при этих словах её губы скривились
в саркастической ухмылке, -- просто немного странная. Поверь, ты отнюдь
не удивишься, получив ответ на свой вопрос.

И Нора поверила. За неимением других вариантов.

\section*{78}\label{78}
\addcontentsline{toc}{section}{78}

\markright{78}

Владислав заскучал, подслушивая диалог наших героинь. Растянувшись на
полу одного из своих тоннелей, он лениво разглядывал потолок, жалея, что
не прихватил с собой книгу, или хотя бы чашку чая.

Говорить с Норой -- даже после того, как девушка узнала часть правды --
ему не хотелось: как и хозяйка бюро ритуальных услуг, Ласло тщательно
подбирал круг общения и болезненно переживал уход близких.

А девушка, в растерянности сидящая всего в нескольких метрах от него,
всегда уходила. Она исчезала с наступлением весны, покидая «Маятник»,
друзей\ldots{} и не только. Теперь она могла уйти, так и не попрощавшись
с Молнией, и для последнего это вылилось бы жутким ударом, тогда как
мужчина едва ли оправился от предыдущего. Конечно, он не знал о недуге
возлюбленной, но проблема, как раз таки, и заключалась в том, что никто
до конца не мог понять, что происходит с Элеонорой.

Сделать это был под силу только самой девушке.

-- Я какая-то ненормальная? -- тем временем с горестью вопрошала Нора.
-- Сумасшедшая, бежавшая из психушки? Скажи мне, если это так! Я должна
знать, -- и она притихла в ожидании.

-- Не думаю, что ты сумасшедшая в традиционном понимании этого слова, --
успокаивающим тоном ответила Агата. -- Правда в том, что, так или иначе,
все мы здесь душевно травмированы обществом, только сказывается это
по-разному. Ласло ведёт затворнический образ жизни, я же, как ты могла
заметить, одержима смертью, не желанием, а значением, идеей самой
смерти, ну а ты\ldots{} -- она замолчала, подбирая слова. -- Ты
вынуждена снова и снова переживать один и тот же фрагмент.

-- То письмо, -- вдруг вспомнила девушка. -- Это ведь я его написала?

-- Боюсь, что так.

-- Но почему я этого не помню?

Знатно отлежав спину, Владислав поднялся на ноги, от нечего делать,
прислонился к стене и, отодвинув затвор, теперь наблюдал за девушками
через пустые глазницы портрета неизвестного музыканта.

-- Вероятно, у тебя случаются моменты просветления, -- говорила Агата; в
знак поддержки она держала подругу за руку, -- которые, вместе с тем,
можно назвать и моментами помутнения.

Элеонора слушала её, периодически вытирая струящиеся по щекам слёзы.

-- Как-то ты поинтересовалась, сколько человек нужно, чтобы нести гроб,
-- продолжала дамочка в чёрном. -- Ума не приложу, зачем тебе
потребовались такие знания, но, услышав ответ, ты начала вести себя
странно.

-- Насколько странно?

-- Что ж, весьма странно: подвисла на долю секунды, а потом на лице
отразился испуг и ты вдруг разрыдалась, выкрикивая несвязные фразы. Ты
размахивала руками и повторяла, что «они идут», как обычно делают
сумасшедшие в фильмах\ldots{} -- она запнулась. -- Чёрт, прости за такое
сравнение, я не нарочно.

Улыбнувшись, Нора отмахнулась и попросила продолжить рассказ, лишь
сильнее сжав руку своей собеседницы.

-- Мне кажется, тут как с гипнозом. Существуют какие-то слова, или
ключевые фразы, что напоминают тебе о пережитом. Из-за них на тебя резко
обрушивается груз прошлого, но затем сознание вновь блокирует все
нежелательные мысли. В тот раз ты уснула. Отключилась прямо на кухне, а
с наступлением утра уже ничего не помнила.

Нора по-детски всхлипнула.

«Это было бы мило, если бы не было так печально», -- отметил про себя
Владислав.

-- Вот почему меня так напугало твоё последнее письмо, -- продолжала
Агата. -- К тому же, есть ещё кое-что.

-- Да? -- горестно отозвалась Нора.

-- Твой дом. Его больше нет.

Девушка на мгновенье забыла о том, что плачет, и с недоверием взглянула
на подругу.

-- Хочешь сказать, он продан? -- предположила Элеонора.

-- Не совсем так, но об этом тебе предстоит узнать не от меня.

Поднявшись со своего места, Агата направилась к шкафу, -- тому самому,
где хранились художественные творения её подруги -- и вернулась с
небольшой записной книжкой, покрытой тонким слоем пыли.

-- Порой ты успокаивалась и успевала кое-что записать, -- произнесла
хозяйка «Маятника», протягивая блокнот Элеоноре. -- Прочти это и найдешь
ответы на многие из возникших вопросов.

Девушка повертела в руках книгу, чувствуя вполне объяснимый страх перед
содержащимся в ней текстом.

-- Так сколько нужно людей для того, чтобы нести гроб? -- напомнила
Нора, стараясь разрядить обстановку.

-- Шесть, -- ответила Агата. -- Иногда восемь, если покойный был
чрезмерно склонен к полноте, ну, и ещё один, чтобы придерживать дверь,
-- она ухмыльнулась, -- так что, мне кажется, девять.

\section*{79}\label{79}
\addcontentsline{toc}{section}{79}

\markright{79}

-- Что это там у тебя? -- поинтересовалась Элеонора.

Она застала Агату в гостиной. Та без особых эмоций перебирала утренние
письма, о чём, собственно, так и сказала.

-- Есть что-нибудь для меня?

-- Одно, -- Агата протянула подруге алый конверт. -- Должно быть,
очередное любовное послание от Молнии.

-- Да будет тебе, -- улыбнулась девушка.

-- Не будет. Я узнаю его почерк.

Краснеющая Нора прижала к груди заветное письмо и в предвкушении
грядущего чтения осмотрела журнальный столик Агаты, на котором виднелся
ещё без малого десяток писем.

-- А остальные?

-- Ну, половина -- это коммунальные счета, -- не отрываясь от дела,
ответила Агата. -- Кое-что -- ответы с аукциона и галереи. Я там картины
заказываю. Ещё одно от поставщиков гробов. А это, -- она потрясла в
воздухе небольшим конвертиком, -- неугомонные предложения товарища
Овсянникова.

-- Что ещё за товарищ Овсянников? -- спросила Нора.

*

Товарищ Овсянников был пухлым человеком среднего возраста, чей и без
того невеликий рост подрезал радикулит, так что вышеупомянутая личность
едва ли доставала до хрупкого плеча нашей траурной героини. Носил он
исключительно лакированные туфли и парадно-выходной костюм, оставшийся,
казалось, ещё со времен Екатерины II. Всё это, в совокупности со щедро
сдобренными гелем волосами, коих было не так уж много, и болезненно
жёлтым цветом лица придавали товарищу Овсянникову вид покойника,
внезапно решившего прогуляться комнатами похоронного бюро.

Именно таким Элеонора нашла товарища Овсянникова, когда одним из
последующих вечеров тот покидал кабинет Агаты. Представлялся, кстати,
этот человек только как товарищ Овсянников. Подписывался также, а потому
имени-отчества его уже давно никто не помнил.

-- Боевая, однако-с! -- говоря сам с собой, произнес товарищ Овсянников,
а завидев в холле Элеонору, соизволил объясниться. -- Живым велела не
возвращаться.

-- Это очень кстати, -- заметила девушка, дивуясь внезапно появившемуся
красноречию. -- В этом месяце у нас мало клиентов.

Так вот, товарищ Овсянников показал себя личностью прагматичной и крайне
настойчивой, а предложения его и впрямь были неугомонными. Мужчина писал
каждый месяц, (иногда сразу по несколько писем) а также под конец
каждого сезона являлся в «Маятник», желая как бы лично
засвидетельствовать выраженные намерения.

История эта длилась уже третий год.

Несмотря на откровенно странное поведение, предложение товарища
Овсянникова не несло в себе ничего романтического. Он не просил ни руки,
ни сердца Агаты, ни других составляющих супового набора. Желание пухлого
человечка имели чисто деловой характер: он хотел купить «Маятник».
Товарищ Овсянников спал и видел как заполучает землю, на которой
возвышалось бюро ритуальных услуг. Он работал агентом по недвижимости,
но в один прекрасный вечер переквалифицировался в похоронных дел
курьера.

Это случилось, когда один из его престарелых клиентов изъявил желание
быть погребенным именно на том месте, где уже не одну сотню лет
располагалось крыльцо «Маятника». По известным одному господу богу
причинам старик хотел возвести свой посмертный мемориал аккурат на этом
куске кладбищенской земли и, не имея наследников, был готов выложить за
свой внезапный каприз целое состояние, пятнадцать процентов которого
обещались товарищу Овсянникову, если тот сумеет всё устроить.

И вот погребальный агент забрасывал Агату настойчивыми предложениями,
каждый раз извергая всё больше аргументов и увеличивая обещанный
гонорар. Сумму он обещал большую.

-- Насколько большую? -- уточнила Нора.

-- Неприлично большую, -- безразлично ответила Агата.

\section*{80}\label{80}
\addcontentsline{toc}{section}{80}

\markright{80}

Наступившим утром двери похоронного бюро оставались открытыми для
проведения очередной прощальной церемонии, а это значило, что события
последующего часа в стенах «Маятника» будут происходить по старому
сценарию: тоскливая музыка, плачущие над усопшим люди и Агата,
прикуривающая около проигрывателя. Тогда-то Элеонора и поинтересовалась,
кому принадлежали пустующие апартаменты.

*

Над городом сгустились поздние сумерки, что, однако, практически не
сказывалось на погоде. Последний день лета подходил к концу, и когда
остатки солнечного света коралловыми лучами проглядывались сквозь стены
Старого города, температура воздуха превышала тридцать пять градусов.

Ввиду подвального расположения, в баре было заметно прохладней. Здесь
работали до последнего клиента, и стоявшая за стойкой девушка довольно
наблюдала толпу, стремительно заполняющую зал, в душе надеясь, что
закроет бар не ранее чётырех утра: в тому времени за стенами должно
стать хоть немного прохладней. Она ненавидела лето сколько себя помнила
-- просто физически не могла находиться в духоте, в этих краях неизменно
сопутствующей жаркую пору года.

Щегловатого вида официантик, пыхтя, взгромоздил на стойку новый поднос с
посудой, которую девушка тут же принялась расставлять по местам. Она
делала это автоматически, мыслями находясь где-то в просторах Исландии и
изредка бросая взгляд на пьющую публику, чтобы прикинуть, когда ждать
следующих заказов.

Пили в основном водку и пиво. Её раздражало и даже слегка задевало, что
из всех возможных вариантов люди предпочитали столь банальные вещи, да
ещё и в чистом виде. Если не считать молчаливого иностранца, который
каждый вечер заходил сюда, чтобы пропустить несколько стаканчиков
бренди-колы, за смену девушка продала от силы тринадцать коктейлей, семь
из которых были Лонг-Айлендами.

Кто-то попросил убавить холод кондишина, но барменша сделала вид, что не
расслышала это непристойное предложение.

Без пяти одиннадцать (она знала это наверняка потому, что по возможности
отлучалась покурить в завершении каждого часа) в зал спустился молодой
человек в круглых солнцезащитных очках. Он тут же направился к барной
стойке, и девушка раздосадовано швырнула сигаретную пачку обратно в
ящик.

Вошедший был не на много выше самой девушки, густые каштановые волосы
небрежно ложились на плечи, а бровь украшала стальная штанга; вдобавок к
очкам на нём были красные джинсы и рубаха, надетая поверх рваной
футболки, ну, и, конечно же, борода -- подобная картина превращала
молодого человека в представителя андерграундного поколения прошлого
века.

Он умостился за стойкой и барменша, вопреки упорным просьбам начальства,
всегда игнорирующая приветствия, вопросительно подняла бровь.

-- Плесни мне Джека, детка, -- голос был мягким.

-- Ты часом не рок-н-ролльщик конца семидесятых? -- спросила девушка, на
этот раз недовольно вскидывая обе брови.

-- Не совсем\ldots{}

-- Тогда не стоит звать меня деткой, чувак.

-- Я тебе не чувак, -- обиженно отозвался молодой человек.

Барменша приоткрыла рот, но тут же закрыла его, вспомнив, где находится.
Она наполнила стакан льдом и потянулась за виски.

-- И колу в стекле.

Девушка кивнула.

-- Можно мне пепельницу? -- вновь напомнил о себе клиент.

-- У нас не курят, -- отрезала барменша.

Возвратившись с перекура, она увидела, как посетитель растерянно вертит
в руках свои очки. Глаза его, хоть и были покрасневшими, чертовски
напомнили девушке её собственные. Заметив приближение барменши, молодой
человек поспешно водрузил предмет на место.

Его стакан пустовал.

-- Повторить? -- спросила девушка и, не дожидаясь ответа, потянулась за
бутылкой: вид молодого человека говорил о том, что это будет далеко не
последний бокал за сегодняшний вечер.

Гость кивнул.

Подрагивающая рука, красные глаза, подведенные красноречивыми синяками,
и скользящая улыбка, то и дело сменявшаяся растерянностью -- всё это
было ей знакомо.

-- Переборщил с дурью? -- спросила барменша, протягивая собеседнику
новую порцию виски. -- Хлебни крепкого чаю, это должно\ldots{}

-- Мой отец умер, -- вдруг произнес молодой человек.

Девушка молча протянула ему пепельницу.

*

-- Снял бы ты очки, -- заметила барменша тремя часами позже, когда
подливающий себе колы мужчина в очередной раз промахнулся мимо стакана.
-- У нас и без того слабое освещение.

Адам, так его звали, и оказался тем самым последним клиентом. Между тем,
вскоре предстояло выяснить, что он вообще оказался тем самым.

-- Не хочу, чтобы кто-то это видел, -- ответил молодой человек, указывая
на свои глаза. -- Никто не должен видеть твою боль, понимаешь? Лучше уж
сойти за торчка.

-- Здесь никого нет.

Адам удивленно обернулся, разглядывая тёмный зал. Помещение пустовало:
официанты, на пару с работниками кухни, покинули его более часа назад.
Тусклый свет озарял лишь барную стойку.

-- Не уверен в том, что я чувствую, -- произнес Адам; речь его была
поразительно чёткой как для человека, опустошившего с дюжину бокалов. --
Сегодня я возвращался с кладбища и видел парней с лопатами. Думаю, они
шли закапывать отца. И они говорили о том, что сейчас идет в кино. Один
из них смеялся, делясь впечатлениями от просмотра «Ученика
чародея»\ldots{} когда шёл закапывать моего отца, ты понимаешь?

Девушка кивнула. Она приготовила себе выпить и закурила, опустившись
напротив собеседника. Тот снял очки, и барменша вновь поразилась
заметному сходству.

-- Это было дико, но вместе с тем так естественно, -- продолжал Адам. --
Я хочу сказать, они ведь даже не знали его. Жизнь продолжается -- так
все говорят. На улицах сновали толпы. Все куда-то спешили. А я ждал
трамвай и надеялся, что он никогда не приедет, потому как мне некуда
идти. И эта музыка\ldots{} Весёлые песни о радостях и летней любви,
звучащие отовсюду -- меня они просто сводили с ума. А затем я увидел как
солнце исчезает за небосводом. Небо окрасилось пастельными тонами, как
на той картине Ван Гога, с оранжевой рощей, и я подумал, что
происходящее мне только снится. Это был самый прекрасный закат, какой я
только видел, но ведь подобной красоте просто не место в такой день,
рядом со смертью\ldots{} И эта чёртова музыка! Думаешь, это когда-нибудь
прекратится?

-- Думаю, музыка никогда не перестанет играть, -- сказала девушка,
которая спустя месяц стала обладательницей похоронного бюро.

\section*{81}\label{81}
\addcontentsline{toc}{section}{81}

\markright{81}

До встречи с Адамом Агата и представить не могла, как много способно
почувствовать человеческое сердце. Жизнь нашей героини сложилась далеко
не счастливо, но в свои девятнадцать она не была особенно грустной, как,
в общем-то, и радостной. Ей было просто\ldots{} нормально.

Не нужно быть экспертом в любовных делах, дабы понять, как больно
бывает, однажды поднявшись на небеса, свалиться обратно. Прямиком на
дно. Но прежде, чем мы дойдем до этого\ldots{}

Первой вещью, поразившей девушку, стало удивительное состояние душевного
спокойствия. Позже, когда с улиц Львова сошел последний снег, Агата
перечитывала письма, которые писала своей дальней подруге в те
счастливые дни, но больше не узнавала себя. Откровенно говоря, она с
трудом верила мысли о том, что в принципе могла быть \emph{такой}.

\begin{quote}
\emph{Этой ночью я практически не спала, волнение было столь сильным,
что у меня (впервые в жизни!) тряслись коленки. Однако, утром меня
охватило странное спокойствие и сейчас, сидя за этим письмом, я всё так
же спокойна.}
\end{quote}

\begin{quote}
\emph{Знаешь, что я люблю больше всего? Наши разговоры. Когда мы
добираемся, скажем, до магазина, но так и продолжаем сидеть в машине --
просто не можем наговориться. Адам рассказывает о вещах, которые,
кажется, всё эти время таились на самой поверхности моего подсознания.
Невозможно описать, что я при этом чувствую. Серьёзно! Это так чудесно
-- понимать, что рядом сидит человек, который знает содержимое твоей
головы лучше тебя самой, просто знать, что ты больше не одна во
Вселенной.}
\end{quote}

\begin{quote}
\emph{Однако, как ни странно, он ничего не говорил, когда я поняла это.
Однажды Адам вернулся в «Маятник», промокший от вечернего дождя, он
вошёл в спальню, взглянул на меня этими карими глазами, которые я люблю
больше всего на свете, и молча поднес с губам мою руку. Никогда бы не
подумала, что такой шаблонный жест сделается столь трогательным и
дорогим моему сердцу. В этот самый момент я в полной мере осознала
собственные чувства. Думаю, это всегда будет моим самым счастливым
воспоминанием. }
\end{quote}

\begin{quote}
\emph{Это далеко не всё, что меня поражает. Естественно, я помню твой
скептицизм в данном вопросе, но, чёрт возьми! В сентябре я начала
работать над картиной, о которой не рассказывала даже тебе. Она
странная. Я имею в виду, странная для меня, потому что так не похожа на
то, что я обычно пишу. Да, ты и сама сейчас это поймешь. В общем, мною
задумывалось изображение открытого космоса; в центре просторы Вселенной
пересекал горизонтальный луч лилового цвета, на котором виднелись
крохотные домики пастельных цветов. Такие, знаешь, из 50-х. И все это на
фоне невозможно близко расположившейся Луны.}
\end{quote}

\begin{quote}
\emph{Как это всегда бывает, финальный образ тут же засел в моей голове,
но на самом деле до завершения работы было ещё далеко. Более того, в
день, о котором я говорю, не было даже черновика. Только изображение в
моей голове, о котором я непрестанно думала. Так вот, следующим утром
Адам рассказал мне свой сон. Ты догадываешься, к чему я клоню? Он
детально описал всё то, что я только собиралась нарисовать. Все аспекты
картины, включая цвета и их оттенки.}
\end{quote}

\section*{82}\label{82}
\addcontentsline{toc}{section}{82}

\markright{82}

\begin{quote}
\emph{Только что осознала, что, зайдя в вагон, почувствовала, что
вернулась домой. Все поезда для меня одинаковы и в каждом я ощущаю себя
дома. Подумать только, сколько эмоций я пережила за эти годы, находясь в
дороге. У меня так много воспоминаний, но я знаю одно: я никогда ещё не
была так спокойна и счастлива как сейчас.}
\end{quote}

\begin{quote}
\emph{В моем городе невыносимый холод и совершенно отсутствует золотая
осень, чего не скажешь о местах, которые видны из окна. Слева замерзшее
поле, а по правую сторону раскинулся лес, он невероятно прекрасен.
Кое-где мелькают небольшие домики -- это то, о чём я мечтаю, милая. Все
эти осенние цвета, я так влюблена в них, приятный холод и шуршание
листьев\ldots{} }
\end{quote}

\begin{quote}
\emph{Вся моя жизнь прошла бы осенью, будь это возможно.}
\end{quote}

\begin{quote}
\emph{Я почему-то вспомнила Джоуэла и Клементину\footnote{Главные
  персонажи романтической драмы «Вечное сияние чистого разума»,
  режиссера Мишеля Гондри.}. Ту сцене в осеннем лесу. Снова-таки,
впервые в жизни, я чувствую, что встретила того самого и это изменило
меня, сделала той, кем мне всегда стоило бы быть. }
\end{quote}

\begin{quote}
\emph{Подумываю сделать себе кофе и закурить. Последнее, правда, вряд ли
возможно, кажется, курение в поездах запретили окончательно, хотя когда
меня это, собственно, останавливало?}
\end{quote}

\begin{quote}
\emph{В моих наушниках любимая музыка и я думаю о тебе, о нашей дружбе и
о том, что, наконец, увижу тебя, ведь ты мой самый близкий друг и я
безумно тебя люблю, не забывай об этом.}
\end{quote}

\begin{quote}
\emph{Конечно, я к тому же думаю о своем мужчине, но, для твоей же
пользы, не стану вдаваться в подробности касательно моих мыслей, тебе ли
не знать, какой я похотливый романтик?}
\end{quote}

\begin{quote}
\emph{Только что выяснила, что курить в поезде уже совсем нельзя, очень
жаль, ведь, зная свою удачу, я обещала Адаму не сходить на перрон до
конечной остановки. Увидев мой трагический взгляд, проводник дала мне
вафель. Чёрт возьми, я перечитала предыдущее предложение и просто не
могу остановить смех! Шикарно! Я даже не буду его убирать. Надеюсь, ты
улыбнешься, прочитав «Увидев мой трагический взгляд, проводник дала мне
вафель» -- хорошее предложение. Такое можно диктовать деткам в школе,
как пример сложноподчиненного, чертовски смешно.}
\end{quote}

\begin{quote}
\emph{Знаю, тебе сложно в это поверить, но я ни разу не чувствовала
того, что чувствую сейчас, ни к чему и ни к кому. И это чудесно.}
\end{quote}

*

\begin{quote}
\emph{Ох, я, всё же, не удержалась и пару часов назад вышла прогуляться.
Удивительное место, названия которого я не запомнила. Станция находится
прямиком около леса. Восхитительная осенняя погода: свежий ветер,
нежаркое солнце, а вокруг -- сплошная осень. Сейчас я проезжаю места,
подобные тому, но погода здесь уже не так дружелюбна. Представляю, что
ждет меня дома, но главное ведь, кто ждет меня там, правда?}
\end{quote}

\begin{quote}
\emph{B вагоне шумно, а я так устала, чувствую себя жутко вымотанной,
это и не удивительно. К слову, я дочитала последнюю книгу Кристи,
которая у меня была. Думаю, перейду к Ф. К. Дику -- никогда не могла
оценить его по общепризнанному достоинству. Почитаю с часик в наушниках
и постараюсь уснуть.}
\end{quote}

\begin{quote}
\emph{Зря я не взяла те таблетки.}
\end{quote}

\begin{quote}
\emph{И всё же, меня весьма удивляет, что я совсем не волнуюсь, думаю, я
начну переживать в самый последний момент, хотя я знаю, всё в порядке,
всё в абсолютном полном порядке.}
\end{quote}

\begin{quote}
\emph{Переключаю своё внимание на «Доктора Бладмани, или как мы стали
жить после бомбы». Не стану окончательно с тобой прощаться, думаю, я ещё
напишу пару строк, прежде чем вручить тебе это письмо. К тому же, мы с
тобой совсем скоро встретимся.}
\end{quote}

\begin{quote}
\emph{Подумать только\ldots{} В общем, ты и сама знаешь.}
\end{quote}

\begin{quote}
\emph{И это делает меня счастливой.}
\end{quote}

\begin{quote}
\emph{С любовью,}
\end{quote}

\begin{quote}
\emph{Агата}
\end{quote}

Вышеуказанному спокойствию понадобилось менее трёх месяцев, дабы вместе
с Адамом навсегда исчезнуть из жизни нашей героини. Осень закончилась, и
Агате казалось, что и сама её жизнь подошла к концу.

Было непросто поверить, что девушка, писавшая столь воодушевляющие
строки и та, что читала их следующей весной, допивая n-ный бокал вина,
являлись одним и тем же лицом.

«Я ведь даже имя свое теперь полностью не пишу» -- мысленно отметила
Агата, и откупорила следующую бутылку.

Осень закончилась, и её сознание напрочь пропиталось грустными
британскими песнями.

\section*{83}\label{83}
\addcontentsline{toc}{section}{83}

\markright{83}

-- Агата? -- взволнованно позвала Нора, возвращая подругу в «здесь и
сейчас».

-- Элеонора? -- медленно отозвалась Агата.

-- Отойди в сторону! Гроб же несут!

Владелица похоронного бюро, временно позабывшая о собственных
обязанностях, подняла голову, и увидела с полсотни пар глаз, что
удивленно уставились на неё, на мгновенье откинув пелену скорби. С
особенным колоритом из толпы выделялись нетерпеливые лица носильщиков.

-- Соболезную вашей утрате, -- глухо произнесла Агата, обращаясь к
заплаканной женщине, что стояла ближе остальных. -- Красивое платье.

Она кивнула толпе и вышла в холл.

*

-- Там от силы десять листов, -- заметила Нора, с надеждой взглянув на
подругу. -- Быть может, мне удастся это обойти?

Содержимое записной книжки по-прежнему вселяло в девушку ужас, пугая
своей неизвестностью. Её не покидало гнетущее ощущение неминуемой беды.
Нора понятия не имела, что скрывается на этих злополучных страницах.

По всей видимости, она не раз возвращалась к собственным тайнам,
переживать которые приходилось снова и снова. И пускай она не могла
вспомнить случившегося, на каком-то уровне Элеонора помнила о душевной
боли, следующей за проникновением в эти самые тайны.

-- Не выйдет. Для того чтоб вернуться к жизни, нужно принять её, ясно
осознать прошлое, и только тогда идти дальше, -- заметила владелица
«Маятника». -- Иначе, встретимся следующей зимой.

-- А что, если я не смогу принять прошлое?

-- Хуже уже не будет, -- Агата пожала плечами.

Это было за месяц до того, как она решила затянуть петлю вокруг шеи.

\section*{84}\label{84}
\addcontentsline{toc}{section}{84}

\markright{84}

Времени оставалось всего ничего.

Радости это, как можно догадаться, обескураженной Элеоноре не добавляло,
да ещё и Агата не менее двух раз в день атаковала её своими «Ну что, ты
прочла?», делая это с безразличной настойчивостью, свойственной
исключительно молодым авторам, вручившим друзьям драгоценную рукопись.

Гонимая попытками подбодрить себя, Нора выбралась в город. Она
собиралась наведаться к Геку, дабы выпить горячего шоколада и, возможно,
послушать какой-нибудь новый (а так ли это?) отрывок его приключений.
Девушка также пыталась вытащить в свет Агату, но та упорно пустила
корни, ссылаясь на предстоящую встречу с клиентами и подготовку к тем
своим странным курсам.

Агата то ли не смогла, то ли не захотела рассказывать подруге о том, кто
являлся единственным (за годы, что она провела здесь) обладателем
закрытых апартаментов. Вместо этого хозяйка «Маятника» снабдила Нору
стопкой мятых писем и ретировалась в свой кабинет ещё до того, как
девушка успела прочесть хотя бы строчку.

Для Норы же чтение писем началось бессонной ночью и закончилось
проливными слезами -- слишком уж она была чуткой.

-- Это и есть Адам? -- поинтересовалась Нора, указывая на одну из
фотографий, висевших над письменным столом, в дальней комнате
«Ретроспективы».

Гек, который уже вовсю хлопотал над излюбленным напитком, прервался и
заинтересованно взглянул на девушку.

-- О-о-о, -- протянул малец. -- Так вона тобі нарешті
розповіла?\footnote{Так она тебе, наконец, рассказала?}

Элеонора водила взглядом по снимкам.

-- Да, -- ответила девушка. -- И мне хотелось бы знать, в который раз?

-- Што в который раз? -- переспросил Гек, чьё чересчур изумленное
выражение лица выдавало тот факт, что он отлично понял суть вопроса.

-- Я все помню, Гек, -- мягко улыбнулась Элеонора. -- Точнее, я
ничегошеньки не помню, но меня тут просветили, что я довольно частый
гость в «Маятнике».

Малец нервно проглотил стоявший в горле ком.

-- Ну так? -- напомнила о себе девушка.

Гек решил, что претворяться дурачком бесполезно.

-- Она рассказала тебе это впервой, -- возвращаясь к приготовлению
шоколада, ответил мальчик. -- Мои поздравления. Такой себе
левел-ап.\footnote{Level-up (англ.) -- повышение уровня.}

Нора расхохоталась.

-- А я знала, что ты говоришь по-русски?

-- Не-е-е, -- Гек ухмыльнулся, довольный собой. -- Это чисто для
праздничных нагод\footnote{Нагода (укр.) -- случай.}.

Лицо Норы вытянулось в непонимании.

-- Что \emph{на год}?

Теперь пришла очередь Гека смеяться.

-- Та, в смысле случаев, -- сквозь хохот ответил малец, наблюдая
недоумение собеседницы.

*

Нора с удовольствием выслушала ещё один отрывок о путешествии Гека,
после чего мальчик традиционно поинтересовался ее мнением.

-- Увлекательно, а главное -- каждый раз как новый, -- с печальной
улыбкой ответила девушка.

Время было позднее, за покрытыми изморозью окнами царил бешенный холод
-- последняя попытка зимы продлить собственное существование. Улицы
пустовали, так что тем вечером Элеонора стала единственным в
«Ретроспективе» покупателем.

Вернее, могла бы стать, но малец отказался от денег, и настойчиво
выпроводил нашу героиню прежде, чем та успела оспорить его решение.

Прогуливаясь магазинчиком, Нора вспомнила о пишущей машинке, которую
присмотрела во время первого визита в «Ретроспективу». С того дня прошло
значительное количество времени, но волею судьбы предмет оказался на
месте. Чёрный с позолотой корпус, утонченные кругляши клавиш -- живя в
двадцать первом веке, девушка осознавала непрактичность своего выбора,
но просто влюбилась в эту вещицу.

-- Щитай это подарком к прошедшим праздникам, -- прокомментировал свой
жест Гек.

Элеонора была вовсе не против. Под Новый год у неё бывали и более
неожиданные подарки, например, полный пакет ритуальных услуг от Агаты, а
также предварительная бронь места на кладбище.

«Только выбери сама, какое больше понравится» -- предложила тогда Агата.

Относительно же машинки у Норы были большие планы. Она оказалась
красивой, думала девушка, такие, должно быть, были у Ремарка и Хема. К
тому же, ей непременно захочется провести за печатью какое-то время, так
всегда бывает с полюбившимися вещами, а это должно поспособствовать
творчеству.

Радуясь внезапному приобретению, Элеонора вернулась в «Маятник».

*

На самом деле, ей хотелось вовсе не этого, ведь, несмотря на сложившуюся
ситуацию, мысли девушки в основном занимал Молния. Пока Норе удавалось
избегать встреч с возлюбленным, ссылаясь на вымышленную простуду,
которую она, якобы, подхватила следующим после их ночи утром, но, как и
страсть, болезнь не может длиться вечно. Вскоре они увидятся, и тогда
Молния непременно затронет тему отношений, задаст вопросы, ответов на
которые Элеонора не знала.

Волны тепла окутывали сердце нашей героини, когда она думала о нём. Этим
вечером Норе не хотелось торчать в «Маятнике», не хотелось копаться в
пугающем прошлом. Единственным искренним желанием Элеоноры оказалась
жажда новых встреч с возлюбленным. Не терпелось вновь пережить те самые
волнительные чувства, ощутить эмоции, которые, по всей видимости,
настигали её уже не в первый раз. Она знала, где именно сейчас желает
оказаться, но вместе с тем, понимала, что нельзя пускать всё на самотек.
Девушка всей душой стремилась к Молнии, но не могла признаться ему в
собственном безумии.

Более того, сама она всё ещё не была в силах это осознать.

Как бы там ни было, вспоминать прошедшие месяцы для Норы было сродни
повторному просмотру фильма с неожиданной развязкой, когда большая часть
предшествующих финалу событий кажется странной в своей монотонности, а
затем -- БАЦ! -- ты начинаешь по-новому понимать действия и диалоги
ключевых персонажей, и всё невещественное становится вдруг существенным.

Элеонора вздохнула. Она вновь подумала о Молнии, -- мальчике с глазами
из самого синего льда\footnote{Аллюзия на песню Сплин «Выхода нет».} --
водрузила на колени пишущую машинку, подле которой уложила свою старую
тетрадь.

До наступления весны оставалось менее трёх часов, когда девушка взялась
за чтение.

*

Читать собственные записи оказалось непросто. Местами даже слишком
сложно: почерк её тянулся вверх, или вниз, а кое-где просто прыгал,
меняя размер и наклон, но наибольшую трудность являл, конечно же, сам
текст. Норе не единожды приходилось его перечитывать, подолгу
задерживаться на одной и той же странице в попытках вместить в голове
увиденное.

Откровенно говоря, девушка ожидала от себя совершенно другой реакции.
Думала, что будет плакать, может быть, даже истерически кричать, да с
такой силой, что в её спальню ворвутся встревоженные обитатели
«Маятника»\ldots{}

Но Нора молчала.

Её сковал ступор, и девушка просто механически перечитывала содержимое
тетради.

В какой-то момент, правда, Нора всё же тихонько вскрикнула, прикрывая
рот рукой. Её зрачки расширились, как это бывает при выбросе окситоцина,
или контакте с психотропными веществами, и девушка резко подскочила с
кровати, заставив ни в чём не повинную машинку свалиться на пол. Та
перевернулась, открывая Норе вкладыш, приклеенный к обратной стороне. На
бумажке был перечень имен -- вероятно, бывших владельцев вещицы.

Что ж, наша героиня едва ли изумилась, увидев последним в списке
собственное имя.

Пару дней назад Элеонора всё-таки не выдержала и ответила на бесконечные
просьбы возлюбленного о встрече. Это был прекрасный вечер: счастливые,
они укутались в наушники и слонялись по ярмаркам, раздражая окружающих
своим влюбленным видом, а позже, вернувшись в спальню мужчины, Нора
заметила небольшое изменение в обстановке -- на прикроватном столике
появилась фотография, запечатлевшая её саму, смеющуюся в объятьях
Молнии.

Снимок момента, о котором Нора ничегошеньки не помнила. Тогда-то она и
поняла, что с дальнейшим развитием отношений стоит повременить до
лучших, более светлых для её памяти времен.

\section*{85}\label{85}
\addcontentsline{toc}{section}{85}

\markright{85}

Группка усталых мужчин самоотверженно долбила застывшую землю,
периодически прерываясь на перекур. До полудня оставались считанные
секунды, а яма была готова лишь наполовину.

К бюро ритуальных услуг уже начинали подтягиваться люди.

-- Бесполезно! -- крикнул один из могильщиков, с яростью отшвыривая своё
орудие. -- Мы всё равно не успеем!

Лопата ударилась о замерзшую почву, издавая звенящий стук.

-- Підітри шмарклі\footnote{Утри сопли.}, -- усмехаясь, ответил второй
могильщик.

Он был слишком стар, чтобы кидаться лопатами, но тоже остановился и
облокотился о многострадальный инструмент.

-- Не зверху ж його ховати\footnote{Не поверх же его хоронить.}.

Третий мужчина ничего не сказал: он предпочитал нытью усердную работу и,
к тому же, был немым.

Могильщики продолжили старое доброе дело, которому сегодняшняя погода
никак не желала способствовать.

Шел первый день марта.

*

Стоя у окна в холле третьего этажа, Агата с тоской наблюдала обстановку
внизу: траурная процессия уже заполняла крыльцо «Маятника», неся с собой
букеты цветов и убогие венки, которые дамочка терпеть не могла, а вот
могила ещё явно не была готова. Обычно копателей было четверо, --
Евгений, Борис, Леонид и Яков -- но Лёня подхватил какой-то вирус, а эти
трое явно не справлялись с работой.

Евгений -- первоклассный нытик -- бросил лопату и закурил, в четвертый
раз за минувший час, так что лишь с помощью неземного усилия воли Агата
смогла подавить в себе настойчивое желание запустить в идиота одной из
пустых винных бутылок, которые с вечера так никто и не убрал.

Двое могильщиков, вероятно, будут работать ещё медленнее, думала Агата.
Придется потерпеть.

Поначалу она намеревалась провести церемонию по стандартному расписанию,
но положение вещей изменилось, и прощанье с усопшим пришлось продлить. В
поисках чего-нибудь воистину трогательного, Агата мысленно перебирала
названия музыкальных исполнителей пока, наконец, не остановилась
на\ldots{}

*

Бон Ивер сделал своё дело.

Содрогающая стены толпа горожан ревела не один час, оглушая своим плачем
Владислава, удручая Элеонору и раздражая Агату, и успешно продолжала бы
реветь дальше, если бы владелица похоронного бюро не свернула лавочку.

Могильщики, в конце концов, покончили с работой, и вскоре шестеро
галантно разодетых мужчин по команде Агаты подняли гроб и медленно
зашагали к выходу. Люди, казалось, вовсе не собирались прерывать своих
рыданий, но, всё же, были вынуждены потянуться за покойником.

-- Подбери что-нибудь динамичное, -- попросила Агата, когда помещение
опустело. Она обвела рукой церемониальный раз. -- Без метёлки тут явно
не обойтись.

И она отправилась в кладовую -- прощанье оказалось слишком плаксивым,
так что пол был усыпан использованными бумажными салфетками,
пластиковыми стаканчиками, в которых раньше были успокаивающие пилюли, и
прочими признаками истерии.

Покопавшись среди пластинок, Элеонора сделала свой выбор и спустя
мгновенье в стенах «Маятника» Фаллон спрашивал, какую же песню сыграют,
когда мы умрем?

\section*{86}\label{86}
\addcontentsline{toc}{section}{86}

\markright{86}

Небеса затянуло серым цветом. Витражи похоронного бюро мерно целовал
ледяной дождь. Стояла последняя суббота ноября и в «Маятнике» впервые
проводили прощальную церемонию с наступлением сумерек.

Зал переполняли скорбные лица, а воздух был пропитан запахом роз. Никто
не перешептывался, не рвался произносить речь, казалось, люди были
слишком поражены случившимся, и только Фаллон продолжал задавать свои
бесконечные вопросы.

\emph{Well, I wonder which song they're gonna play when we go.}

\emph{I hope it's something quiet and minor and peaceful and slow.}

\emph{When we float out into the ether, into the Everlasting
Arms\ldots{}}\footnote{Текст песни «The 59' Sound» группы The Gaslight
  Anthem: (англ.) что ж, мне интересно, какую песню сыграют, когда мы
  умрём? Надеюсь, это что-то тихое, грустное, спокойное и медленное.
  Когда мы выходим в эфир, в Вечные Объятья\ldots{}}

Пластинка вращалась без остановки. Песня уже в который раз начиналась с
начала, но едва ли кто-нибудь обратил на это внимание. Агата стояла у
самого проигрывателя, разглядывая пластинку, но не решалась её сменить
-- руки девушки непрерывно дрожали как, собственно, и она сама.

Слёзы уже вторые сутки бесшумно струились по лицу нашей героини, и
вокруг её зрачков начинала проступать кровавая сетка лопнувших сосудов.
Агата плакала без звука, как в старых фильмах. Дела и при желании не
могли бы обстоять иначе: девушка потеряла голос. Она открывала рот,
шевелила губами, но слова застревали в горле. Где-то в глубине сознания
скользнула мысль о том, что такое бывает при резких душевных травмах, и
при хорошем лечении голос может вернуться, но сейчас это отнюдь не имело
никакого значения.

Церемонией руководил брат покойного, так что, Агата просто стояла у
выхода, периодически переводя застывший взгляд с пластинки на лицо
усопшего.

«При жизни он не был таким\ldots{} Бледным\ldots» -- пронеслось в голове
Агаты. -- «Это как-то\ldots{} Неправильно\ldots Ласло должен был\ldots{}
Больше тона\ldots»

Она осознала, что мыслит отрывочно.

\emph{And I wonder were you scared when the metal hit the glass?}

\emph{See I was playing a show down the road}

\emph{When your spirit left your body.}\footnote{И мне интересно: ты
  испугался, когда металл ударил по стеклу? Видишь ли, я разыгрывал шоу
  на дороге, когда дух покидал твоё тело.}

Тоскливые лица участников толпы сделались для девушки одинаковыми. Они
сливались с чёрными одеждами и, подобно пластинке, начинали вращаться по
кругу, стоило Агате поднять голову. Причиной этому вряд ли была одна
лишь пелена слёз, устлавшая ее глаза.

И только образ покойного оставался чётким.

\ldots хотя, Агата допускала вероятность того, что на самом деле видит
его исключительно в своей голове. Тем вечером владелица «Маятника»
поняла, что внутри неё что-то оборвалось. Прежде она воспринимала слова
о разбитом сердце как красивые метафоры, но в этот самый миг, в
одиночестве стоя у разрывающегося проигрывателя, Агата в полной мере
чувствовала давящую боль под ребрами -- буквально ощущала тяжесть
зияющей в груди дыры и знала, что вне зависимости от того, как в
дальнейшем сложится её жизнь, она, Агата, уже никогда не будет прежней.

\emph{Young boys, young girls,}

\emph{Young boys, young girls,}

\emph{Young boys, young girls,}

\emph{Young boys, young girls,}\footnote{Юноши и девушки\ldots{}}

Она знала, что боль не отступит.

Так и произошло.

\emph{Ain't supposed to die on a Saturday night,}

\emph{Ain't supposed to die on a Saturday night,}

\emph{Ain't supposed to die on a Saturday night,}

\emph{Ain't supposed to die on a Saturday night.}\footnote{\ldots{} не
  собирались умирать субботним вечером.}

Мотив в сотый раз принялся набирать обороты. Внезапно чёрно-белая толпа
в молчании расступилась. Бледные руки подняли гроб, чтобы плавно нести
его к выходу и дальше -- в сырую землю.

Тогда внутри Агаты вновь что-то оборвалось. А затем ещё раз. И ещ ё.
Словно её грудь поразило тонкое лезвие невидимого кинжала. Дышать
становилось всё сложнее.

Девушка вздрогнула от неожиданности, почувствовав на своей талии чью-то
теплую руку.

-- Надень их, -- ласково произнес Молния.

Не оборачиваясь, Агата узнала его по голосу. Молния, который сам только
что потерял брата, приобнял подругу, с безмолвной скорбью глядя в её
истерзанные слезами глаза. Это было очень кстати: девушку затрясло
сильнее, и она чувствовала, что вот-вот упадет.

Тем временем Молния протягивал ей солнцезащитные очки.

«Никто не должен видеть твою боль\ldots» -- вспомнила Агата, и новые
ручейки слёз оросили её щёки.

Агата Рахманинова надела очки как раз в тот момент, когда мимо неё
проносили гроб с преждевременно усопшим.

\emph{Его очки.}

\section*{87}\label{87}
\addcontentsline{toc}{section}{87}

\markright{87}

Минуло пять с лишним лет. Держа в руках инструменты для уборки,
владелица «Маятника» вновь стояла у входа в церемониальный зал. Комната
была пуста, если не считать Элеонору, которая деловито расставляла по
местами хаотично разбросанные стулья. Агата с усердием пыталась
заставить себя уяснить, что это в действительности так, но больное
сознание диктовало ей иную сцену.

Изображения накладывались друг на друга: безлюдная комната и толпа,
плывущая за гробом.

Виной всему была музыка.

\emph{«Ain't supposed to die on a Saturday night!»} -- настаивал на
своем Фаллен.

В этот момент Агата могла бы поклясться, что уловила в воздухе запах
роз, смешавшийся с формалином, хотя последний был здесь вполне ожидаем.

Элеонора не была из тех, кто мог с лёгкостью ощутить чужое присутствие,
особенно если то было бесшумным. Она никогда не чьего-либо на спине
чужого взгляда, а потому, обернувшись, подпрыгнула от неожиданности --
девушка застала подругу за тем самым взором в бесконечность.

-- В чем дело? -- встревожилась Нора: лицо Агаты сделалось бледнее
обычного.

Девушка чуть было не добавила: «Кто-нибудь умер?», но вовремя
спохватилась.

Тем временем Агата, казалось, приоткрыла рот, но вместо ответа с её губ
сорвался лишь усталый выдох.

-- Что стряслось? -- на этот раз громче спросила Нора.

Она уже направлялась к подруге.

-- Ничего нового, -- ровным голосом произнесла Агата. -- Всего лишь
воспоминания.

Элеонора вопросительно взглянула на хозяйку похоронного бюро.

-- Под эту музыку хоронили Адама, -- объяснила Агата, вручая девушке
метёлку. -- А теперь давай разгребем всё это дерьмо.

\section*{88}\label{88}
\addcontentsline{toc}{section}{88}

\markright{88}

Официально шёл двадцать первый день марта, но истинная весна укрыла
город лишь с наступлением следующего месяца, а пока крыльцо похоронного
бюро, как и окружающие его надгробия, купалось в белоснежных хлопьях.
Несмотря на метель, особого холода не ощущалось, и вороны по обыкновению
кружили по дворику, изредка покидая его пределы.

Густой дым струился сразу из нескольких стальных дымоходов. Ввиду своего
мрачного вида и громоздкой, слегка зауженной верхушки, издали «Маятник»
напоминал траурный орган. Только вместо музыки из его труб лился пар.

Именно об органной мелодии подумал Молния, приближаясь к похоронному
бюро. Он едва ли успел осознать эту мысль: ассоциативный ряд вступил в
действие и в голове мужчины тут же заиграла старая митлофовская
мелодия.\footnote{I'd do anything for love -- (с анг. \emph{Я бы сделал
  все ради любви}) песня Meat Loaf.} В руках его был утонченный букет
фиолетовых лилий, в сердце -- любовь и огонь воспламеняющейся надежды.

За прошедшее время он видел свою Линор не менее дюжины раз. Последние
зимние дни насторожили Молнию: казалось, девушка опять спасовала, но
прошло какое-то время, и если Нора и впрямь в чём-то сомневалась, то, по
всей видимости, разрешила этот вопрос, потому как в дальнейшем вела себя
естественно.

\ldots ну, насколько это вообще возможно, для влюбленных, ещё
находящихся на той стадии отношений, когда принято не портить друг при
друге воздух и делать вид, что на некоторых участках тела волосы и вовсе
никогда не росли.

Однако, говоря о бюро ритуальных услуг, этот визит стал для Молнии
первым за долгое время. Мужчина ударил по дереву увесистой ручкой, тем
самым чуть не изувечив собственные пальцы.

Безрезультатно.

Дверь как всегда оказалась открытой: вряд ли кто-то в здравом уме
пожелает проникнуть в эту обитель смерти. Ну, а клиентам тут всегда
рады.

Внутри также мало что изменилось. Окликнув Агату, мужчина несколько
минут провел в холле, прислушиваясь к тишине здания в надежде уловить в
ней шаги, голоса, или, быть может, отдаленные звуки музыки, но ничего из
этого не последовало\ldots{}

(Разумеется, живя в веке социальных сетей, он мог просто позвонить Норе,
или скинуть парочку сэлфи, выходя из дому, но так уж вышло, что наш
герой был любителем сюрпризов.)

\ldots{} и Молния поднялся в единственную хорошо знакомую ему комнату.

Впервые за долгое время та оказалось незапертой, но Молния, конечно, не
мог этого знать.

\section*{89}\label{89}
\addcontentsline{toc}{section}{89}

\markright{89}

В те дни, повествовал Гек, хоть и стояла холодная весна, погода заметно
отличалась от нынешней. Снегопад обратился проливным дождем и дорога, по
которой плёлся дрожащий мальчик, представляла собой дряблую кашицу из
грязи, небесной воды и талого снега.

Температура вновь поднялась, и пускай в душе нашего персонажа зародился
относительный покой, (он наконец-то осел во Львове) Гек изнывал от
хорошо знакомого ему недуга. Первое время он ночевал где придётся, пару
раз даже снимал комнату в хостеле, а затем начал подрабатывать то тут то
там: расклеивал листовки, подметал парадную и разносил газеты. Последней
работой стала уборка подземного барчика, где Гек драил полы,
переворачивал стулья, чтобы отлепить от них застывшие жвачки, выносил
мусор и выполнял прочие пыльные поручения. Платили хорошо, в сравнении с
предыдущими вакансиями, и чаще всего мальчику удавалось перекусить
бесплатно, но полученные деньги уходили на оплату жилья, (малец
арендовал плохо меблированную комнату у одной барыжащей самогоном
старушки, которую встретил на вокзале; та не задавала лишних вопросов
взамен на молчание мальчика) а он вдруг возьми и сляг с простудой недели
на две.

Трудиться в таком состоянии ему не позволяли, особенно в питейном
заведении. На бирже труда Гек не состоял ввиду возраста, а акциями не
владел ввиду бедности, словом, средства его отнюдь не прибавлялись, так
что наступившим утром наш герой очутился на улице.

Не имея багажа и определенных жизненных перспектив, изнуренный тоской и
болезнью, он бродил серыми улицами, не зная куда себя деть, пока наконец
не остановился у единственного места, которое счел подходящим убежищем
-- всё у того же бара. Работа при высокой температуре и впрямь виделась
ему малоприятным времяпрепровождением, однако особых вариантов всё равно
не было, но ребёнок был рад любой возможности побыть в тепле.

Бар, по всей видимости, не разделял его чувств, потому как дверь была
заперта.

«Слишком рано» -- с грустью подумал мальчик.

Он всхлипнул и, убедившись, что поблизости никого нет, позволил соплям
смешаться со слезами.

*

К обеду малец намотал ещё несколько кругов, ошиваясь около Старого
города, от чего ему стало только хуже. Заручившись поддержкой бешеной
усталости, сон приложил все усилия, дабы поймать Гека в свои объятья, но
тот понимал, что нельзя просто упасть посреди грязной улицы: он мог
замерзнуть насмерть, или того хуже -- быть пойманным блюстителями
порядка.

Пришлось взять себя в руки. Темнота близилась, а путь до вокзала казался
слишком долгим, так что мальчик не придумал ничего лучше кроме
как\ldots{}

Прямиком через улицу от всё ещё закрытого бара находилась антикварная
лавка. Посетителей там было немного, а содержимого -- как раз наоборот,
-- Гек не раз прогуливался мимо, разглядывая сквозь витрины сгрудившийся
на старой мебели хлам -- так что вскоре в мальчишеской голове созрел не
больно-то и хитрый план и, дождавшись, пока владелец «Ретроспективы»
отправится обедать в дальнюю комнату, Гек пробрался внутрь.

В магазинчике из каждого уголка веяло стариной. Здесь было невероятно
уютно, а главное -- тепло. Играла музыка, но на входе Гек всё же с
ловкостью опытного воришки придержал хрустальные колокольчики,
подвешенные над дверью, опасаясь, что те могут выдать его присутствие.
Он прокрался в укромный уголок образованный антресолью и парочкой
стеллажей, которые оказались напрочь заставлены диковинными вещами, чьи
владельцы (а не исключено, что и их дети) уже давно почили в земле. По
дороге мальцу встретился целый сундук тряпья, поверх которого лежали
пледы ручной работы. Одолжив парочку, он умостился на полу.

Синатра предлагал начинать распускать сплетни, когда Гек благополучно
погрузился в сон.\footnote{Start spreading the news -- первые строки
  песни Фрэнка Синатры под названием «New York, New York».}

*

Накануне малец прикончил последний пузырек таблеток, но, очнувшись от
глубокого сна, почувствовал улучшение и в кои-то веки нашёл себя
отдохнувшим. Он тут же зажал рот рукой, пытаясь скрыть рвущийся наружу
кашель, но сделал это слишком поздно.

-- Значит, проснулся уже\ldots{} -- донеслось откуда-то из-за стеллажа.

Малец испуганно подскочил\ldots{}

(стыдясь того, что влез сюда, своего внешнего вида, положения, да и
вообще всего, что только может быть на этом свете)

\ldots{} но запутался в одеялах и плюхнулся обратно на задницу.

-- Не мельтеши, -- прокомментировал сие действие неизвестный голос.

-- Как вы меня нашли? -- тихо спросил мальчик, стараясь выиграть время.

Он вспоминал планировку помещения и мысленно уже начинал прокладывать
путь к отступлению. Из-за стеллажа, наконец, показался обладатель на
удивление спокойного голоса.

-- Ты оставил грязные следы -- это раз, -- с теплой улыбкой ответил
Валерий, -- не просто кашлял, ревел во сне -- это два. К тому же, и это
три, я прожил здесь не одно десятилетие, -- он указал на перепачканные
мальчиком пледы, что валялись на полу. -- Ты и впрямь думал, что я не
замечу пропажи?

Гек не нашелся с ответом. Вместо этого он предпринял новую попытку
подняться.

-- Да не мельтеши ты, -- мужчина махнул рукой.

При других обстоятельствах Гек едва ли прислушался бы к такому совету,
(тогда он совершенно не доверял взрослым) но глаза его собеседника
лучились добром и вдобавок ко всему совсем не казались взрослыми, они
скорее выдавали мальчика, по каким-то страшным причинам заключенного в
теле средних лет мужчины.

Тем временем хозяин антикварной лавки окинул взглядом незваного гостя.

-- Не мельтеши, -- в третий раз повторил Валерий. -- Я приготовлю тебе
горячего шоколада.

\section*{90}\label{90}
\addcontentsline{toc}{section}{90}

\markright{90}

Стрыйский парк Элеонора нашла очаровательным и полагала, что он
оставался таковым в любое время года. Место оказалось старое, в
правильном понимании этого слова: вековые деревья, широкие заснеженные
аллеи, заброшенные тропинки и водоёмы прекрасно гармонировали со
всевозможной архитектурой, и Нора с восхищением разглядывала каждый
уголок парка.

Местами, разгуливая укрытыми снегом дебрями, девушка чувствовала себя
так, словно оказалась в настоящем лесу. Агата, вероятно, ощущала то же
самое: вид у неё, в большинстве своем оставался довольный.

Приятно было, наконец, сменить обстановку. Вместо эдгарпошных птиц мимо
наших героев то и дело проносились белки. Одна бесстрашная красотка
прыгнула прямиком на плечо мальчика, а затем без малейших противоречий
умостилась на его ладони, чтобы отправиться в руки Элеоноры.

-- Посунься трошки правіше!\footnote{Подвинься немного правее!} -- Гек
деловито махнул головой.

-- Это обязательно? -- отозвалась Агата, но, всё же, послушно
подвинулась в сторону Норы.

-- Может, хотя бы уберешь сигарету? -- предложила та, обхватывая подругу
за талию.

-- Ну, что за нонсенс! -- отозвалась дамочка в чёрном.

Под конец прогулки они поднялись по пешеходной дорожке, на месте которой
когда-то протекал ручей. Девушки стояли у края верхней террасы. За их
спинами виднелись бесконечные заснеженные просторы.

Нора улыбнулась, и вскоре её инстаграм пополнился ещё одной фотографией.

*

Бюро ритуальных услуг окутывали звуки музыки. Отдаваясь звонким эхом,
трогательные и вместе с тем тоскливые ноты заполняли немногочисленные
комнаты третьего этажа, неся в себе дивную силу -- умение забираться
прямо в глубь человеческой души. Музыка нарастала и потухала чтобы стать
ещё громче и ударить в самое сердце.

Возвратившись в «Маятник» прямиком из морозного дня, Агата Рахманинова,
казалось, сначала почувствовала, а уже затем услышала льющуюся из
пустующих апартаментов музыку. Она узнала мелодию с первых секунд.
Просто не могла не распознать её. Узнала и вздрогнула, но, к счастью,
Элеонора была слишком увлечена шутками Гека, а потому ничего не
заметила.

Оба, мальчик и девушка, весело переговариваясь, чистили обувь от
налипшего на неё снега, когда Агата устремилась вверх по лестнице,
собираясь ни то разрыдаться, ни то отругать Владислава, а, может быть, и
всё сразу. Она убавила шаг лишь достигнув своего этажа. Поняла, что
Ласло нет дома: он ещё вчера уехал на один из этих своих кулинарных
конкурсов.

Приглушенный плотным стеклом свет падал сквозь синие витражи третьего
этажа, создавая призрачную иллюзию тумана. Агата пересекла холл и
медленно следовала по коридору, с благоговейным страхом глядя на слегка
прикрытую дверь.

Сердце девушки обратилось натянутой струной, обещавшей порваться в любую
минуту. По мере её приближения музыка становилась все громче. Сомнений
не возникало. Это была она -- музыка Адама.

-- Призраков не существует, -- шепотом напомнила себе Агата и тут же
добавила с ироничной улыбкой на губах: -- К сожалению\ldots{}

Она вошла в комнату, отказавшись от дальнейших раздумий.

Да, это определенно была музыка Адама, а у изножья широкого ложа сидел
сам Адам. Он расположился лицом к проигрывателю и, опустив глаза, в
задумчивости разглядывая свои пластинки, так что девушке были видны лишь
очертания его профиля. Одет гость был как всегда в чёрное, с высокого
лба струились пряди густых волос, что казались светлее в теплом свете
лампы, и, думая об этом, Агата не совсем поняла, почему до сих пор не
лишилась чувств.

Словно во сне, Агата сделала несколько неуверенных шагов по направлению
к мужчине, но тут же остановилась, чувствуя, что затылок её вот-вот
поцелует половицы. Владелица «Маятника» громко выдохнула. Адам
обернулся, и тогда девушка поняла, кого приняла за покойного
возлюбленного.

Прежде она не замечала их сходства.

«Похожи, но не одинаковы, -- устало подумала Агата. -- Скулы не такие
высокие, да и глаза у Адама были карие».

-- Я напугал тебя, -- с извиняющимся видом произнес Молния.

-- Вовсе нет, -- машинально соврала дама в чёрном, но лицо говорило за
неё.

-- Прости, просто воспоминания вдруг нахлынули\ldots{}

Отнюдь не жаждущая откровенной болтовни Агата вновь предприняла попытку
отвертеться.

-- Ты отпустил бороду.

Но и Молния не сдавал позиций.

-- И волосы, -- безразлично добавил он. -- Я знаю, что так тебя
напугало, ведь я и сам теперь не могу со спокойной душой смотреться в
зеркало.

-- Всего на одно мгновенье\ldots{} -- начала Агата, но так и не нашла в
себе сил окончить фразу.

Они помолчали.

-- Это странно, -- нарушила тишину Агата.

Молния согласно кивнул.

-- Да я и сам никогда не осознавал, как сильно мы похожи, а
сейчас\ldots{}

-- Нет, -- оборвала его девушка. -- Я вдруг поняла, что совсем недавно
стала старше него. Это странно, -- повторила Агата. -- Я имею в виду,
любить мёртвых.

В этот момент на лестнице раздались шаги. За ними послышался голос
Элеоноры. Та звала подругу по имени.

-- Наверное, нашла цветы, -- предположил Молния.

-- Фиалковые лилии?

-- Как всегда, -- ответил мужчина, силясь выдавить из себя улыбку.

-- Иди, -- произнесла Агата, которая также пыталась улыбнуться.

У обоих вышло не ахти.

\section*{91}\label{91}
\addcontentsline{toc}{section}{91}

\markright{91}

Тем вечером в «Маятнике» собралась уже известная нам компания. Их было
лишь пятеро, (Владислав вернулся наступлением темноты) но число это всё
же превышало привычное количество ошивающихся в похоронном бюро людей,
если, конечно, учитывать только живых.

Какое-то время Гек ещё самоотверженно пытался занять Элеонору своими
россказнями, но внимание мечтательной девушки всецело принадлежало
возлюбленному. Застенчивая и вместе с тем любвеобильная парочка
смотрелась так, словно её составляющие знакомы между собой от силы
месяц.

Что ж, в каком-то смысле так оно и было.

Нора стеснялась пить при молодом человеке, зная о том, как быстро
хмелеет, а Молния, по всей видимости, также не собирался крутить косяки:
едва ли это вообще было им нужно. Геку не наливали ввиду возраста, (хотя
разок малец всё же воспользовался тем, что любующимися друг другом
голубки не особо-то обращали внимание на происходящее вокруг, и пригубил
красного полусладкого) так что всё веселье досталось Агате, которая вряд
ли согласилась бы с таким утверждением, потому как большую часть вечера
дамочка провела в своем кресле, в молчании потягивая винишко.

Посиделки вышли незапланированными и, что ещё хуже, какими-то
консервативными. За все месяцы, которые помнила и не помнила Элеонора,
хозяйка похоронного бюро впервые проиграла -- буквально слила -- партию
в шахматы. Норе это совсем не понравилось. Её подруга, конечно,
выглядела тем, что называют absent-minded, (а в нашем случае лучше
сказать `wine-minded')\footnote{Игра слов: absent-minded (англ.) --
  рассеянный. Первое слово созвучно с «absinthe» (абсент), так что,
  ослышавшись, или не зная правописания высказывания, его можно
  буквально перевести как «тот, чей ум поглощен абсентом»; wine -- вино.}
но вместе с тем Агата сильнее других не любила проигрывать, делая это
гораздо ярче остальных. Дамочка была бомбой замедленного действия, и
лучше чем Норе об этом было известно разве что Владиславу. Агата не раз
запускала в подругу колодой карт, делая это со всей мировой яростью, и
однажды разошлась настолько, что девушка в слезах выбежала из
«Маятника», чтобы затем ещё долгое время бродить заснеженными тропинками
кладбища в поисках менее бесноватой компании.

Так вот, Элеонора с недоверием воззрилась на подругу, ожидая от неё
характерных резких движений, взмывающих в воздух шахматных фигур и
предвидя картину раскуроченной доски, летящей в бесконечность.

Однако, ничего такого не последовало. Физиономия мастерицы похоронного
дела не выражала признаков исступления и вскоре Агата уже расставляла по
местам павшие в бою фигурки, предложив гостям сыграть новую партию, но
на сей раз без её участия.

«Сама не своя!» -- мысленно изумилась Нора, но отсутствие
неврастенического припадка подруги безусловно не могло испортить
чьего-либо настроения.

Игра продолжалась.

*

Наступившим утром завтрак между собой разделила лишь часть
вышеупомянутой компании -- её объятая любовью половина. Гек, который
уснул уже после того, как встало солнце, всего на минутку выбрался из
своего диванного убежища. Он иронично хрюкнул при виде целующейся на
кухне парочки и поспешил вернуться в царство сна, прихватив в собой
парочку клубничных круассанов.

Мальчик продолжал спать, когда Молния покинул бюро ритуальных услуг для
встречи с очередным клиентом.

Солнце плавно скрывалось за кладбищем. Его лучи ещё держали в своих
объятьях каменные надгробия, что уже понемногу лишались снежного
покрова. Отдаленно теплый свет укрыл и лицо стоявшей у окна девушки, и
Агата сняла очки, желая в полную силу насладиться закатом. В спальне
раздавались печально вдохновляющие звуки музыки «Облачного атласа».

Впервые за долгие, мучительные годы её лицо выражало воистину
решительное умиротворение. Агата слегка улыбнулась, -- и это непременно
сделало бы её моложе, если бы не сочные синяки, издавна поселившиеся под
глазами -- выкурила шуршащую сигарету, запивая её всё тем же вином, вкус
которого вдруг сделался (хотя, казалось бы, куда ещё?) лучше.

Солнце совсем село. Вскоре за задернутыми шторами раздалось неуверенное
постукивание.

«Дождь пошёл, -- мимоходом подумала Агата. -- И я пойду».\\
Она сделала музыку громче, выкурила ещё одну сигарету, которую также
запила вином, поднялась на древнюю бархатную табуретку о трёх ножках,
продолжая держать винный бокал, свободной рукой накинула на шею петлю,
затянула её, прикрыла глаза и всё с той же легкой улыбкой на лице
шагнула в вечность.

\section*{92}\label{92}
\addcontentsline{toc}{section}{92}

\markright{92}

Наслаждаясь тихой погодой, вечерней чашкой кофе и чарующей музыкой,
доносившейся откуда-то сверху, Элеонора сидела над антикварной машинкой,
перепечатывая записи из собственного дневника, о котором ещё месяц назад
не имела ни малейшего понятия. Она толком не знала, зачем это делает, но
в тайне от себя самой всё же надеялась, что в процессе удастся вспомнить
ещё какие-нибудь детали, события\ldots{} Да что угодно!

Настроение у нашей героини было лежательное. Вовсе не хотелось идти
куда-нибудь, встречаться с людьми, вступать в диалоги, или, боже упаси,
самой вести разговоры.

Однако, тем не предвещавшим беды вечером Норе всё же предстояла одна
неожиданная встреча.

Девушка сладко потянулась и взялась за дело. Хвала небесам, для этого
вовсе не было необходимым покидать пределы постели.

*

\begin{quote}
\emph{С чего мне вообще стоит начать? Я не знаю, но нужно что-нибудь
писать. Хер его знает, как долго может продлиться это чёртово
озарение\ldots{} }
\end{quote}

\begin{quote}
\emph{Лучше бы его и вовсе не было. }
\end{quote}

\begin{quote}
\emph{Мне не так уж давно исполнился двадцать один год, }
\end{quote}

\begin{quote}
\emph{(и я собираюсь начинать с этой фразы все последующие записи, если
таковым суждено появиться на свет, потому как оказалось, что чаще всего
я бываю не в курсе собственного возраста)}
\end{quote}

\begin{quote}
\emph{Сейчас канун Нового года. Я сижу на кухне Агаты, в «Маятнике». Его
хозяйка испуганно носится вокруг меня с коробкой успокоительного. Ещё
одна аптечка лежит на столе. В ней сплошные пилюли для шизанутых и
неврастеников. Замеченные мной препараты можно разделить на три группы:}
\end{quote}

\begin{quote}
\emph{Психотропные: аминезин, голопередом, амитриптилин;}
\end{quote}

\begin{quote}
\emph{Транквилизаторы: феназепам, гидазепам;}
\end{quote}

\begin{quote}
\emph{Седативные: персен, новопасит, фитосет и целая куча валерьяны.}
\end{quote}

\begin{quote}
\emph{Все это -- лишь часть запасов. Интересно, откуда у Агаты такой
широкий выбор? }
\end{quote}

\begin{quote}
\emph{Седативные оказались совершенно бесполезными. Транки, в отличие от
своих предшественников, кое-как заработали. Подавили выброс адреналина,
чем, собственно, и привели меня в чувства. Записывая это, я не ощущаю
особых эмоций и в холодности своей скорее похожу на Агату. Такое
чувство, что мы поменялись местами, ведь она выглядит на удивление
взволнованной, даже чувствительной.}
\end{quote}

\begin{quote}
\emph{Как парадоксально.}
\end{quote}

\begin{quote}
\emph{Ну вот, как я и думала, подруга тычет мне пластинку амитриптилина.
}
\end{quote}

\begin{quote}
\emph{Я не хочу принимать психотропные. Это слишком. Такое дерьмо даже
не продается по рецепту, а выдается исключительно по государственным
бланкам розового цвета и подавляет в человеке всё, что только можно
подавить. Я-то знаю. }
\end{quote}

\begin{quote}
\emph{Я провела в клинике восемь паршивых месяцев, мысля слогами и
ощущая себя как персонаж Николсона после лоботомии.\footnote{Имеется в
  виду Рэндл Макмёрфи -- один из героев романа Кена Кизи под названием
  «Пролетая над гнездом кукушки», в одноименной экранизации которого
  основную роль исполнил Джек Николсон.} Такое медицинское вмешательство
собьёт с толку любого, не позволит думать, а уж тем более записывать
свои думы. Но я должна это сделать. Должна, если не хочу придти сюда
одним декабрьским утром в теле дряхлой старухи, которая думает, что ей
всё ещё двадцать и ровным счетом ничего не помнит.}
\end{quote}

\begin{quote}
\emph{Так что, нахер психотропные! Пора признать это. Я схожу с ума.}
\end{quote}

\begin{quote}
\emph{Я СХОЖУ С УМА.}
\end{quote}

\begin{quote}
\emph{Как там говорилось: «Схожу с ума, иль восхожу к высокой степени
безумства».}\footnote{Цитата из стихотворения Беллы Ахмадулиной
  «Прощание».}
\end{quote}

\begin{quote}
\emph{Удивительно, (ох, если бы только я могла сейчас удивляться) когда
я сняла комнату в «Маятнике» всё шло к тому, что моя домовладелица не
дружит с головой, а оказалось, что это я.}
\end{quote}

\begin{quote}
\emph{Я тронутая.}
\end{quote}

\begin{quote}
\emph{Ещё вчера я унывала из-за того, что я самая обычная и мне не о чем
писать книги, а теперь этот вопрос вдруг отпал сам собой. Разве самые
обычные девушки проводят по три сезона в психиатрической лечебнице,
чтобы затем вернуться в свой старый дом и не помнить об этом? Не помнить
вообще о том, что уже были здесь?}
\end{quote}

\begin{quote}
\emph{Сомневаюсь.}
\end{quote}

\begin{quote}
\emph{Новость о том, что однажды я уже зимовала в «Маятнике» в более
спокойное для своей психики время, стала воистину ошеломительной, но
сегодня я узнала не только это. Сегодня я узнала о так званых
первопричинах, приведших меня в объятья безумия.}
\end{quote}

\begin{quote}
\emph{Впервые я появилась на ступенях похоронного дома в конце ноября
позапрошлого года. Тем днем первый снег выпал раньше обычного. Я была
предельно счастлива, с тремя чемоданами тряпья, желанием обустроиться и
отсутствием видимых горестей.}
\end{quote}

\begin{quote}
\emph{Незадолго до этого я рассталась с молодым человеком, имени
которого упоминать не стоит. Отношения между нами были не иначе как
странными. Они продлились четыре года, но на деле оказались
неэмоциональными и чертовски затяжными. Иначе не скажешь.}
\end{quote}

\begin{quote}
\emph{Нет, возможно в самом начале, будучи школьницей я и любила его той
самой наивной детской любовью, когда говоришь метафорами, помышляешь о
раннем замужестве и ссоришься из-за ерунды, впоследствии чего
разыгрываешь настоящую трагедию. Естественно, всё это делается
неосознанно, но участники таких вот мелодрам почему-то всегда уверены в
собственной правоте и яро убеждены в том, что знают жизнь.}
\end{quote}

\begin{quote}
\emph{Ну, а в итоге заблуждаются оба}.
\end{quote}

\begin{quote}
\emph{Помню как тряслась из-за пустяков. По ним же и страдала, ревела.
Сейчас мне трудно поверить в собственную причастность ко всем тем вещам.
Ещё труднее осознать, что чувства вообще имели место быть\ldots{} Думаю,
так со всеми бывает.}
\end{quote}

\begin{quote}
\emph{Что ж, тот период моей жизни (когда думаешь, что уже вовсю такой
взрослый, а нихера подобного) можно описать всего одной фразой:
типичная, подростковая, но зачем-то слишком затянувшаяся связь.}
\end{quote}

\begin{quote}
\emph{На самом деле первые отношения редко бывают настоящими. Я имею в
виду, что они едва ли связаны с любовью и, главное, тем, что она с собой
несёт, так что счастлив тот, у кого вовремя не сложилось.}
\end{quote}

\begin{quote}
\emph{Став старше, я поняла это, но по ряду причин (включавших в себя
такие штуки как боязнь одиночества и чувство сострадания в сторону
других людей) не могла просто взять и разорвать всё. Тот парень был,
мягко говоря, нелюдимым. Делал покупки в полном молчании, а затем
возвращался домой и красноречиво описывал в своём бложике пожелания
мучительной смерти всем встреченным за день ребятам: от водителя
автобуса до кассирши из супермаркета, но он не был плохим человеком. Он
был никакущим.}
\end{quote}

\begin{quote}
\emph{И я его не любила.}
\end{quote}

\begin{quote}
\emph{Он тоже меня не любил, хотя не забывал ежедневно ломать комедию
перед целым миром, (которому, правда, было глубоко насрать) показательно
выставляя свои ко мне чувства, что выражалось, конечно же, только на
словах. Ему нравилось делать вид, что он меня любит так же сильно, как
нравилось притворяться словно у него уйма злостных врагов, тогда как на
самом деле у него не было даже друзей.}
\end{quote}

\begin{quote}
\emph{Несмотря ни на что, я всегда оставалась верной этому человеку.
Даже в мыслях. Бог свидетель, я терпела многое и не единожды пыталась
помочь его могзам встать на место, но силы мои иссякли, и сколько бы
сопереживания во мне не было, так больше не могло продолжаться. Вдобавок
ко всему вышесказанному и куче других подробностей, которые мне
приводить не хочется не столько из-за экономии времени, сколько от
нежелания вторично вникать во всё это, было здесь кое-что ещё.}
\end{quote}

\begin{quote}
\emph{Пожалуй, самое главное. Я так опасалась одиночества, думала, что
просто не смогу найти того, кто оказался бы лучше, но с ним я не просто
не чувствовала себя любимой -- я в принципе не ощущала, что состою в
каких бы то ни было отношениях, а в таком вопросе нет ничего хуже, чем
быть с человеком, который заставляет тебя чувствовать себя одинокой.}
\end{quote}

\begin{quote}
\emph{Итак, став старше и осознав это, я окончательно утвердилась в
правильности мысли о том, что эти отношения давно себя исчерпали.}
\end{quote}

\begin{quote}
\emph{Наш, с позволения сказать, разрыв состоялся незадолго до моего
двадцатилетия и, по сути, вовсе не был разрывом как таковым. Общение в
очередной раз сошло на «нет», и я просто отошла от привычной модели
поведения, то есть не стала пытаться что-либо изменить. В общем, я
отпраздновала свой двадцатый день рождения и поспешила убраться из и без
того ненавистного города. Наскоро собрала вещи и успела прыгнуть в поезд
ещё до того, как начала волноваться и, вопреки здравому смыслу, побежала
к нему во зло себе самой.}
\end{quote}

\begin{quote}
\emph{Я прекрасно помню своё первое утро во Львове. Снежное и спокойное.
Тогда я ощущала небывалый прилив сил, вызванный ничем иным кроме как
обретением долгожданной свободы. Помнила я это и вчера, ясно и
отчетливо, за тем только исключением, что думала, будто описываемые мной
события произошло не год, а от силы месяц назад. }
\end{quote}

\begin{quote}
\emph{Пожалуй, на сегодня достаточно предыстории.}
\end{quote}

\begin{quote}
\emph{Оба моих родителя родились весной, и если я ещё могла позволить
себе обойти стороной именины отца, то мамины -- ни за что на свете. Я
никогда не пропускала дней её рождения и отнюдь не собиралась начинать.
Указанной датой было третье марта. Я собиралась сесть в поезд первого,
прибыть второго и вечером явиться с сюрпризом. Так и поступила. Выехала
налегке, с одной только дорожной сумкой, что в размерах своих
незначительно превосходит дамскую. Сошла с поёзда следующим днем и,
прежде чем взять такси, (дом мой находился за городской чертой) решилась
зайти ещё в одно место.}
\end{quote}

\begin{quote}
\emph{Вероятно, во многом я просто желала убедиться, что он жив-здоров и
пришёл в норму после нашего расставания, потому как, вопреки всем
законам логики, где-то в глубине души я всё ещё чувствовала себя
виноватой перед этим человеком\ldots{} Себе же я сказала, что
всего-навсего хочу забрать книги, которые он одалживал почитать, но
никогда толком не возвращал (так что за минувшие годы их накопилось не
менее двух десятков).}
\end{quote}

\begin{quote}
\emph{То же сказала и ему, показавшись на пороге захудалой квартирки,
которую арендовал мой бывший молодой человек. За всё время, проведенное
вместе, я была здесь лишь раз -- вот вам ещё один образец странности
наших отношений. Всего раз за целых четыре, мать их, года! Обычно мы
встречались в парках, кафешках и торговых центрах. Короче говоря, на
нейтральной территории. И слава богу! Находиться в его жилище сделалось
задачей не из легких. Представьте себе чердак заброшенного дома, куда
лет эдак десять не проникал никто кроме смачно гадящих голубей.
Представили? А теперь пустите туда пожить семейство наркозависимых
бездомных, и в общих чертах картина готова.}
\end{quote}

\begin{quote}
\emph{К слову, открывший мне дверь человек являл собой всё и сразу. Он
выглядел и как наркозависимый бездомный, и как покинутый грязный чердак,
и, в конце концов, как лоснящийся от жира и пота голубь, этот самый
чердак оскверняющий.}
\end{quote}

\begin{quote}
\emph{К тому же, он, кажись, температурил.}
\end{quote}

\begin{quote}
\emph{Мне было предложено войти. Делать этого совсем не хотелось, но
проклятая вежливость взяла свое, и я перешагнула порог. Невольно
подпрыгнула при хлопке закрывающейся за моей спиной двери, и тут же, с
опозданием на четыре секунды, решила зачем-то добавить, что очень
спешу.}
\end{quote}

\begin{quote}
\emph{«Ты никуда не пойдешь, пока мы не поговорим!» -- внезапно заявил
он,}
\end{quote}

\begin{quote}
\emph{(и это мне совсем не понравилось)}
\end{quote}

\begin{quote}
\emph{а затем принялся запирать дверь на ключ,}
\end{quote}

\begin{quote}
\emph{(это мне понравилось ещё меньше)}
\end{quote}

\begin{quote}
\emph{а потом широко открыл рот и\ldots{} проглотил тот самый ключ.}
\end{quote}

\begin{quote}
\emph{(Это был пиздец.)}
\end{quote}

\begin{quote}
\emph{Как и все героини историй с подобным началом, я повиновалась зову
инстинкта, и сделала несколько молчаливых шагов назад. Батарейка моего
сотового приказала долго жить ещё в поезде, (видимо, вот откуда однажды
во мне появилось настойчивое желание приобрести не одну, а целых три
переносных зарядки) соседей за стеной грязной квартирки давно не было, а
единственным живым существом поблизости, как мне помнилось, числилась
глухая бабка, живущая этажом ниже, о которой совсем не известно, жива ли
она. }
\end{quote}

\begin{quote}
\emph{Савелий, завидев моё поведение, принялся тараторить о том, что
ничего мне не сделает. Никакого вреда, но разве не это обычно убийцы
втолковывают своим жертвам? Я тихонько застонала и продолжила пятиться,
размышляя, что делать дальше. В итоге, спина моя уткнулась в обшарпанную
стену, а подходящих мыслей всё не было.}
\end{quote}

\begin{quote}
\emph{Тогда меня, конечно же, обуял ужас и все его друзья, ведь,
вдобавок ко всему, Савелий выглядел чертовски помешанным,
(щит\footnote{Shit (англ.) -- дерьмо.}, только сейчас поняла, что таки
спалила его имя, но хер с ним, так тому и быть) но не стану лишний раз
тянуть кота за яйца. Он не бил меня и не насиловал. Пальцем не тронул.}
\end{quote}

\begin{quote}
\emph{Савелий сделал кое-что гораздо хуже. Сам того не замышляя, (по
крайней мере мне хотелось бы на это надеяться) он воплотил в жизнь куда
более изощренный план мести\ldots{}}
\end{quote}

В этот момент новообращённую машинистку прервали. За стенами спальни
раздался приглушенный топот, сопровождаемый сильными ударами,
издаваемыми по всей видимости соприкосновением кулаков со внутренней
стороной стены. Шум нарастал. Гремело так, словно коридорами несся не
один человек, а целое стадо ошалелых Владиславов.

Мгновение спустя в комнату влетел белобрысый парнишка, ярко
контрастирующий с тёмного цвета стенами, из которых он и явился. Как и
при их первой (во время нынешнего визита в «Маятник») встрече,
таинственный таксидермист застал Элеонору чутка растрепанной, в одной
футболке возлежавшую на шёлковых простынях.

-- Влад\ldots{} -- начала девушка.

Глаза её в изумлении расширились.

-- Быстрее! -- воскликнул внезапный гость, в нетерпении размахивая
руками.

Ошарашенная Нора отреагировала не сразу. В уголок её подсознания уже
закралась тревожная мысль о том, что Владислава так просто из убежища не
выманишь, разве что стряслась какая-то беда\ldots{}

Но девушка ещё несколько секунд зачарованно разглядывала пришельца.
Особое внимание нашей героини приковали большие и на редкость светлые
глаза, окаймлённые такого же цвета ресницами.

-- Скорее! -- настаивал Владислав. -- Давай за мной!

Его переполненный испуга голос привел девушку в чувства и мгновенье
спустя оба уже неслись вверх по лестницам тёмных стенных коридоров.

\section*{93}\label{93}
\addcontentsline{toc}{section}{93}

\markright{93}

Тем временем этажом выше медленно умирала Агата. Но делала она это
духовно, угасала мнимой, как Остап Бендер\footnote{Остап Бендер --
  персонаж книг Ильи Ильфа и Евгения Петрова, который был однозначно
  убит в конце «Двенадцати стульев», но необъяснимо продолжил жить на
  страницах «Золотого телёнка», следующей работы писателей.}, смертью,
отделавшись лишь несколькими царапинами, тогда как кое-каким предметам
окружения мрачной дамочки посчастливилось куда меньше.

Так перед глазами Элеоноры, ворвавшейся в покои подруги, предстали
останки хрустального бокала и бархатный табурет, обезглавленный жестокой
случайностью. Завершал картину треснувший надвое карниз, что ввиду
преклонного возраста оказался слишком ветхим, а посему свершено
непригодным для самоубийства.

Виновница торжества тёмным пятном растянулась на полу. Петля, уже кем-то
заботливо ослабленная, по-прежнему обрамляла фарфоровую шейку. Агата
лежала лицом вниз, и это напугало Нору больше всего остального, но,
услышав топот друзей, хозяйка похоронного бюро, вероятно чересчур
преданная своему делу, медленно подняла голову. Передние пряди волос
хаотично устлали бледное лицо. Спешащая к подруге Нора вдруг замерла и
мимо воли отпрянула, когда Агата приподнялась на локтях. Её волосы
отхлынули, и взгляды девушек впервые встретились без стеклянной
преграды.

Это были глаза поэта, с отчаяньем отшлифованные болью и глаукомой.
Последняя всё же преуспела меньше и была всего лишь производным
последствием первой. Нора как-то читала, что заболевание это получило
своё название в честь лазурной воды, но пока ничего подобного на
физиономии Агаты не проглядывалось. А было вот что: правый глаз девушки
испещряла алая сетка лопнувших сосудов, не выдержавших злостной судьбы,
повышенного давления и моря пролитых слёз; левый же (если не учитывать
зрачка и карей радужки) и вовсе был красным, без видимых признаков
уцелевшего белка.

-- Отдыхаешь? -- поинтересовался Владислав.

Агате такой подход к сложившейся ситуации более чем пришелся по душе.
Она попыталась рассмеяться, но вместо этого из горла чуть было не
окочурившейся девушки вырвался лишь хриплый стон, скорее походящий на
карканье какого-нибудь издыхающего ворона.

Этот жуткий звук вернул Нору к реальности. Она опустилась на колени
рядом с подругой, пытаясь одновременно обнять несчастную и помочь ей
встать на ноги.

-- Мои очки\ldots{} -- попросила Агата пугающе потусторонним голосом.

Не видя теперь в этом особого смысла, Элеонора всё же протянула
суициднице запрашиваемый предмет. За её спиной Ласло осмотрел место
происшествия и, уверившись в том, что цель достигнута, с наивысшей
степенью, увы, никем не замеченного артистизма скрылся в стене.

Девушке без особых усилий удалось поднять на ноги Агату. Ведя исхудавшую
подругу к кровати, Элеонора ощутила вязкую влагу на собственных пальцах,
но она не чувствовала боли, а потому тут же перевела взгляд на Агату.
Вдоль запястья второй девушки хлестала кровь.

-- Ты поранилась! -- охнула Элеонора. -- Здесь есть аптечка?

-- Пустяки, -- просипела Агата. Вышло что-то вроде «ус\ldots м\ldots к».

-- Что? -- с быстротой медсестёр военного госпиталя Элеонора уже
выдвигала ящики комода в поисках заветной коробки.

-- Пустяки\ldots{} По сравнению с тем\ldots{} Что я собиралась\ldots{}
Сделать\ldots{} -- медленно и с придыханием объяснила мастерица
похоронного дела, усаживаясь на незаправленную постель.

-- Ну, об этом позже, -- бросила Элеонора.

Аптечка нашлась под кроватью. Однако, внутри не оказалось ничего, хоть
издали напоминавшего бинт. Сплошные седативные, транки и хорошо знакомые
Норе психотропные.

*

Взмыленные постояльцы «Маятника» напрочь забыли о Геке. Его разбудил
грохот, издаваемый падением Агаты, которое, в свою очередь, продлилось
топотом Владислава. Наш герой поспешил наверх, но представший перед ним
таксидермист строго-настрого запретил мальчику проникать в эпицентр
внезапной суеты. Будучи личностью упрямой и чертовски изворотливой, Гек,
всё же, озадачился появлением таинственного Владислава, да настолько,
что так и не решился его ослушаться.

Какое-то время мальчик обиженно бродил по дому, разглядывая гробовые
обивки и сокращая запас провианта до минимума, после чего занятие это
ему наскучило и малец удалился, прихватив с собой бутылочку всё того же
красного полусладкого.

\section*{94}\label{94}
\addcontentsline{toc}{section}{94}

\markright{94}

За неудавшимся самоубийством последовало новое утро. Туманное, голубое и
очень красивое.

Агата Рахманинова очнулась от глубокого, лишенного видений сна, когда
стрелка старинных напольных часов приближалась к отметке с цифрой
девять. Утонувшая в полумраке спальня оказалась чистой, без виденных
признаков битого стекла, или карниза, погибшего в неравном бою с
гравитацией. Руку ей забинтовали минувшим вечером, петлю с шеи сняли и,
досыта накормив голопередоном, отправили в длительное плаванье по
царству забвения.

Дамочка не имела ни малейшего понятия, в котором часу она в последний
раз принимала психотропное. Она с тоской взглянула на бутылку, хранившую
в себе останки вчерашнего вина, но так и не отважилась её опустошить.

И оказалась права. Эффект от пилюль ещё не прошел полностью: пытаясь
читать, Агата то и дело проваливалась в сон.

Нора показалась в начале одиннадцатого. Она принесла завтрак, к которому
так никто и не притронулся, а также печатную машинку и старый дневник,
над которыми принялась трудиться, то и дело одаривая подругу жалостливым
взглядом.

-- Ты прочитала? -- наконец, не выдержала Агата.

Она задала этот вопрос рефлекторно, просто потому, что привыкла, и вовсе
не ожидала положительного ответа, так что глаза хозяйки «Маятника»
комично расширились, а лицо удивленно вытянулось, когда Элеонора
произнесла:

-- Прочитала.

Агата несколько раз моргнула, с недоверием воззрившись на собеседницу.
На мгновенье она забыла и о своем разбитом сердце, и о широкой синей
полоске, которая теперь обрамляла её тонкую шею, подобно уже успевшему
всем надоесть чокеру. Словом, наша траурная дамочка оживилась.

-- Серьёзно?

Элеонора спокойно кивнула.

-- Ты \emph{всё} прочла? -- допытывалась Агата.

-- Всё.

По-прежнему не веря своим ушам, Агата приподнялась на локтях и в
недоумении взглянула на подругу.

-- Но почему ты всё ещё здесь? -- тихо спросила она.

-- Ну, как я могла уйти? -- ласково произнесла девушка.

-- Ты\ldots{} Ты осталась из-за меня?

Элеонора отложила машинку. Её лицо осветила широкая улыбка.

-- Конечно, я осталась из-за тебя. Как бы мне не хотелось вернуться в
дурку\ldots{} Нет, на самом деле мне совершенно туда не хочется, но я ни
за что не стала бы оставлять тебя в таком состоянии. Я же уже не первый
день вижу, что тебе становится только хуже, -- она вздохнула и вновь
улыбнулась. -- Я не виню тебя за такую реакцию. Раньше я всегда уходила,
узнав о собственном безумии. Вроде как жертвовала собой во благо других
-- так я, по крайней мере, думала. Но, как оказалось, я здесь не
единственная, у кого проблемы с головой, так что нельзя и дальше думать
только о себе. В конце концов, мне меньше всего хочется уходить от тебя.

-- И от Молнии, -- напомнила Агата.

-- И от Молнии, -- согласилась Нора, застенчиво закусывая нижнюю губу.
-- Знаешь, думаю, на этот раз у нас всё может сложиться\ldots{}

-- А если нет, у тебя всегда есть шанс начать с чистого листа, --
прохрипела владелица похоронного бюро.

Воцарилось секундное молчание, ну а далее\ldots{}

Обе девушка заливисто рассмеялись.

\section*{95}\label{95}
\addcontentsline{toc}{section}{95}

\markright{95}

-- Можно мне взглянуть? -- спросила Агата.

Она указала на стопку отпечатанных страниц, лежавших около Элеоноры. Та
согласилась без лишних раздумий, и дамочка в чёрном халате тут же
углубилась в чтение небольшой истории.

Здесь же, во избежание повторения, приводится только вторая часть
записей Элеоноры.

*

\begin{quote}
\emph{Он не бил меня и не насиловал. Пальцем не тронул.}
\end{quote}

\begin{quote}
\emph{Савелий сделал кое-что гораздо хуже. Сам того не замышляя, (по
крайней мере мне хотелось бы на это надеяться) он воплотил в жизнь куда
более изощренный план мести\ldots{}}
\end{quote}

\begin{quote}
\emph{Роста в нём было где-то сто восемьдесят, хотя сам Савелий зачем-то
любил говаривать, что там не меньше ста девяноста двух. Веса своего он
наоборот не разглашал. В этом, в принципе, и так не было нужды. Мой
бывший был необычайно тучным человеком, а за три с лишним месяца, что мы
не виделись, сделался ещё шире. Короче, спорить я с ним не собиралась и
ни о каком побеге не помышляла. Закрытая дверь, девятый этаж.}
\end{quote}

\begin{quote}
\emph{Квартира та была двухкомнатной. Он проводил меня в дальнюю
спальню, а сам замер в дверном проёме, без преувеличений перекрывая
собой путь к возможному отступлению. В совокупности с его потрепанным
видом и проглоченным (!!!) ключом, забыть о котором было бы сложно, это
не сулило ничего хорошего. Я опустилась на залитый бог знает чем диван,}
\end{quote}

\begin{quote}
\emph{(постели на нем не было, но плешивый плед и лишенная наволочки
подушка -- с пятнами в тон диванных -- свидетельствовали о том, что
именно здесь спал мой бывший)}
\end{quote}

\begin{quote}
\emph{и приготовилась к мучительно долгому разговору. Возможно, даже
скандалу.}
\end{quote}

\begin{quote}
\emph{Его не последовало.}
\end{quote}

\begin{quote}
\emph{Как я уже сказала, Савелий меня и пальцем не тронул.}
\end{quote}

\begin{quote}
\emph{-- Ты вообще когда-нибудь любила меня? -- начал он.}
\end{quote}

\begin{quote}
\emph{Я не поднимала глаз. Промямлила что-то о том, что всё слишком
затянулось, мы были совсем детьми и только думали, что знаем друг друга,
мол, я старалась понять его, но он никогда не шёл навстречу, что,
кстати, так и было -- это и ещё кучку невнятной чепухи.}
\end{quote}

\begin{quote}
\emph{Савелий прервал меня на полуслове. Он ударил по дверному косяку
так, что тот дал трещину, и повторил свой вопрос, на этот раз срываясь
на поразительно высокий для его внешности крик.}
\end{quote}

\begin{quote}
\emph{Я испугалась. Даже не так. В моей душе ожил первобытный страх.
Перед глазами пронеслись все виденные раннее киноленты о маньяках. Я
подумала, что должна подыграть. Солгать, чтобы успокоить его. Логично.
Но затем я вспомнила, кто стоит передо мной. Человек, с которым я
провела четыре года, за которого так волновалась, и о ком, наверное,
когда-то так тосковала, чьи губы не раз целовала в покрытых сумерками
парках\ldots{} }
\end{quote}

\begin{quote}
\emph{Подумала и сказала правду.}
\end{quote}

\begin{quote}
\emph{Я сказала «нет».}
\end{quote}

\begin{quote}
\emph{Савелий не ответил. Никогда. Я подняла голову, и увидела, что он
тоже смотрит на меня. Глаза его расширились. Слово, которое, кажется,
должно было начинаться на «с», застыло в горле. Он посмотрел куда-то
сквозь невидимую пелену, сделал неуверенный полушаг в мою сторону и
оглушительным громом рухнул лицом вниз.}
\end{quote}

\begin{quote}
\emph{По природе своей я та ещё впечатлительная трусишка. Однако, в
по-настоящему стрессовых ситуациях не начинаю орать, несясь сквозь
пространство и время. Обычно, я просто впадаю в ступор.}
\end{quote}

\begin{quote}
\emph{На этот раз реакция моя была всё той же. Мне понадобилась добрая
четверть часа дабы сообразить, что нужно проверить его пульс. Тяжело
дыша, я склонилась над распростертым на полу телом. Рука его всё ещё
оставалась тёплой, но пульс отсутствовал.}
\end{quote}

\begin{quote}
\emph{Как и дыхание.}
\end{quote}

\begin{quote}
\emph{Я убедилась в том, что и так уже знала.}
\end{quote}

\begin{quote}
\emph{Нужно было размышлять о том, как выбраться из помещения, но я
беззвучно отползла от трупа. В голове как загнанные лошади бились
безрадостные мысли. Меня одолевало чувство вины и отвращения, в равной
степени смешавшихся. Страшило осознание того, что человек, такой же как
и я по факту, просто взял и превратился в гору тухнущего мяса. Я имею в
виду, он технически был разумным существом, вне зависимости от того,
нравился мне его разум, или нет. }
\end{quote}

\begin{quote}
\emph{А теперь стал мясом. Его могут жрать черви, крысы,
каннибалы\ldots{} Пройдет какое-то время, (совсем немного, учитывая силу
отопления) и можно будет отрывать куски. Сознания-то в нём нет. Больше
нет. }
\end{quote}

\begin{quote}
\emph{Естественно, в итоге эти мысли свелись к тому, что рано или поздно
так же будет и со мной. Со всеми, кого я люблю, и теми, кого едва знаю.}
\end{quote}

\begin{quote}
\emph{«Не стану же я его есть!» -- пронеслась истерическая мысль. Я
разразилась чужим, холодящим кровь смехом. Это заставило меня
разрыдаться. }
\end{quote}

На этой неопределенной ноте запись обрывалась, но тут же начиналась
следующая.

\section*{96}\label{96}
\addcontentsline{toc}{section}{96}

\markright{96}

\begin{quote}
\emph{Прошлой осенью мне исполнилось двадцать два. Сейчас двадцать
девятое января и я вновь сижу на кухне в «Маятнике», где меня
по-прежнему отпаивают успокоительным. Всё по тем же причинам.}
\end{quote}

\begin{quote}
\emph{Боюсь, эта запись будет совсем короткой, ибо рука моя дрожит, а
под слипающиеся веки впору вставлять спички, как в тех старых
диснейевских мультфильмах. В общем, дела у меня складываются не очень.
Ну, а теперь о том, что повергло меня в это состояние.}
\end{quote}

\begin{quote}
\emph{Сегодняшним вечером мне явился очередной из недостающих пазлов.
Новый отрывок описанной выше истории.}
\end{quote}

\begin{quote}
\emph{За проведённое во Львове время я перевидала уйму прощальных
церемоний, но за ужином вдруг поняла, что никогда не обращала внимания
на то, сколько человек обычно несут гроб. Не знаю, почему мои мысли
вообще коснулись подобной темы, неужели мне было настолько скучно?
Энивей,\footnote{Anyway (англ.) -- в любом случае.} путём несложных
логических умозаключений я пришла к выводу о том, что носильщиков должно
быть парное количество, но сколько именно не помнила. Вот и спросила у
той, кто не мог не знать ответ. Агата сказала, что при желании это могут
быть и четыре человека, но обычно гроб несут шесть мужчин, ну, или
восемь, если покойник оказался особенно крупных размеров.}
\end{quote}

\begin{quote}
\emph{И тогда я вспомнила.}
\end{quote}

\begin{quote}
*Вскрытия не проводили. Мне это показалось как минимум странным, ведь
\end{quote}

\begin{quote}
\emph{А) Савелий просто упал замертво, без каких-либо предупреждений;}
\end{quote}

\begin{quote}
\emph{Б) Я была в его квартире в тот самый момент и после, а брюшная
полость покойного в итоге оказалась\ldots{} Не хочу об этом думать;}
\end{quote}

\begin{quote}
\emph{В) По украинским законам вскрытие обязательно даже для смертельно
больных, стоявших на учёте у онколога.}
\end{quote}

\begin{quote}
\emph{Короче, всё это было и остаётся необъяснимо непрофессиональным
шагом со стороны полиции. Сложилось впечатление, что и патологам, и
судмедэкспертам было попросту лень заниматься установлением причин
смерти этого несчастного. Как было, так и запихнули в землю. Я нашла это
ужасным, но не настолько, чтобы бить тревогу. Ужасов мне к тому времени
уже хватало, хотя, как оказалось следующим днём, они и не собирались
заканчиваться.}
\end{quote}

\begin{quote}
\emph{Полиции, как и мне, по всей видимости, хотелось поскорей
расправиться с этой историей.}
\end{quote}

\begin{quote}
\emph{Хоронили Савелия в закрытом гробу. Вечером того же дня, которым
его обнаружили. Но это едва ли помогло. Проститься с ним пришло человек
двадцать. Все бывшие коллеги, которым происходящее, по-моему, было
безразлично. Их пригнал большой босс, преданный идее
team-building-а.}\footnote{Team-building (с англ. team -- команда,
  building -- строительство) -- сплочение рабочего коллектива.}
\end{quote}

\begin{quote}
\emph{Похоронное бюро оказалось крохотным, а зал прощаний -- так вообще
стесняющим в размерах. Приходилось стоять слишком близко к гробу. Тот
был закрыт, но запахи разложения всё равно умудрялись просочиться
наружу. Священник, одетый по всем канонам современной церковной моды,
держал в руках потрёпанный томик и бубнил что-то своём, о вечном.
Местами ему вторила юная бабонька, внешний вид который говорил о том,
что она не первый год страдает ПГМ.\footnote{Православие головного
  мозга.} На протяжении всего времени действия я трижды выходила
проветриться, а, возвратившись, видела, что эти ребята и не думают
заканчивать. }
\end{quote}

\begin{quote}
\emph{И вот, когда я в четвёртый раз намеревалась покинуть тесное
помещение, в дверях возникло восемь молодых людей, одетых во всё чёрное,
разумеется. Поднять гроб им удалось не с первой попытки, но, в конце
концов, мужчины водрузили на плечи свою тягостную ношу и тут\ldots{}}
\end{quote}

\begin{quote}
\emph{Я увидела, как преобразились их лица. Отсутствующее выражение
сменилось отвращением, к которому примешивалась просьба избежать
происходящего. В ноздри носильщиков с новыми силами ударило трупное
зловоние, ведь те находились совсем близко к его источнику. Я смотрела в
глаза бедолагам и понимала их лучше, чем кто-либо другой, а они лишь
продолжали идти, несмотря на запах, влекущий за собой тошноту,
головокружение и страх перед тем днём, когда и они сами превратятся в
нечто подобное.}
\end{quote}

\begin{quote}
\emph{Думаю, в описываемый момент, я была не единственной, кто осознал,
что предпочитает быть кремированным.}
\end{quote}

Остаток страницы был пуст, а под конец записи почерк Элеоноры сделался
столь размашистым, что читавшей всё это дело Агате не единожды пришлось
прибегнуть к расшифровкам самого автора.

К облегчению нашей траурной героини, последующая запись выглядела куда
аккуратней. По крайней мере, в начале.

\begin{quote}
\emph{Мне не так уж давно исполнилось двадцать три. Только до недавнего
времени я думала, что двадцать.}
\end{quote}

\begin{quote}
\emph{Зима близится к концу. Я лежу в своей постели, в «Маятнике».
Закинулась всё теми же упомянутыми выше пилюлями. Прочитав предыдущую
запись. Всё так же отказываюсь принимать психотрпные, чтобы иметь
возможность её продолжить. Должно быть, это удобно: львиная доля моей
истории уже записана, а я даже не помню, как это делала.}
\end{quote}

\begin{quote}
\emph{Энивей, я чувствую себя так, будто описываемые события произошли
совсем на днях, так что могу продолжить. Воспоминания о тех днях подобно
раковой опухоли засели в моем мозгу.}
\end{quote}

\begin{quote}
\emph{Первое время я боялась\ldots{} Просто не могла заставить себя
перешагнуть его\ldots{} Его тело. Подтянув к подбородку колени, я сидела
на заляпанном диване. То роняла в отчаянье голову, то разглядывала
распластавшегося на ковре мужчину.}
\end{quote}

\begin{quote}
\emph{Савелий свалился параллельно дверному проёму. В какой-то мере мне
даже повезло: -- если подобное слово может быть применимо в такой
ситуации -- часть его головы скрывалась под антресолью, так что лица
совсем не было видно. Как я уже говорила, при жизни Савелий был
неимоверно толстым человеком, и смерть ничего не изменила, посему вскоре
я столкнулась с вполне ожидаемой проблемой. Я подступалась со всех
сторон и, хотя рост мой довольно сложно назвать маленьким, перешагнуть
покойника никак не выходило. Во время очередной попытки голова моя
закружилась от страха, что ни на мгновенье не ослаблял своих оков, и
тошнотворного запаха, которые уже вовсю начинали источать исходившие от
трупа испарения. Это значило, что прошло много времени. Гораздо больше,
чем мне казалось. Я едва не упала, балансируя на одной ноге, но в
последний момент рука моя успела опереться о шкаф, и я устояла на
ногах.}
\end{quote}

\begin{quote}
\emph{Само собой разумеется, я кричала. Не от ужаса, но желания позвать
на помощь. Вопила что было мочи, барабанила по батарее и избивала
стены.}
\end{quote}

\begin{quote}
\emph{Безрезультатно.}
\end{quote}

\begin{quote}
\emph{Думать о нём как о теле -- гниющей груде мяса -- всё ещё было
сложно. Не знаю, как долго я пробыла в той комнате, всё происходило как
во сне, но, видимо, я провела там немало времени, потому как с каждым
моим вдохом трупные испарения становились всё яростней. Я перерыла всю
комнату в поисках ключа, или мобильника -- чего угодно, что могло бы
пригодиться.}
\end{quote}

\begin{quote}
\emph{Нихера.}
\end{quote}

\begin{quote}
\emph{Запасной ключ. Он должен был где-нибудь быть. Я попыталась
набраться сил. Силы никак не приходили, но ждать с моря погоды ещё
дольше виделось мне бессмысленным, и, превозмогая рвотные порывы, я
наступила на тело. Рассчитывала, что оно будет твёрдым, однако трупное
окоченение уже отступило. Почва под моей ногой оказалась дряблой и
какой-то липкой. Отвратительно, неестественно мягкой. Новый ком ужаса
подобрался к моему горлу, и, неистово взмахнув руками, я вновь
забалансировала на одной ноге. В итоге, всё равно упала лицом вниз, и
теперь лежала в той же позе, что и покойник.}
\end{quote}

\begin{quote}
\emph{Я обшарила каждый уголок пыльной квартирки. Где-нибудь непременно
должен был найтись второй ключ. Но его не было. Нигде.}
\end{quote}

\begin{quote}
\emph{Как и стационарного телефона. Его так вообще никогда и в помине
здесь не устанавливали. За интернет хозяин квартиры также давненько не
платил, посему от компьютера не было никакого толку. Я зажгла во всех
комнатах свет и тщетно пыталась распахнуть окна. Старые деревянные
ублюдки оказались наглухо заклеенными на зиму.}
\end{quote}

\begin{quote}
\emph{Изголодавшись по свежему воздуху, я схватила первый угодивший под
руку предмет, (им оказался доисторический стул) и яростно молотила по
кухонному окну до тех пор, пока практически всё стекло не оказалось у
моих ног. То же я проделала и с другими окнами. Швыряла в них всё
подряд, выкрикивая ругательства и мольбы о помощи, в ответ на которые
какой-то старый скряга, наконец, пригрозил вызвать полицию.}
\end{quote}

\begin{quote}
\emph{Однако, за окнами сгущалась ночь, а стражей порядка так никто и не
потревожил. Думаю, район, в котором находилось вышеупомянутое жилище,
слыл слишком неблагополучным, чтобы хоть кто-нибудь обратил внимание на
одинокие крики и мебель, галантно выбрасываемую из окон.}
\end{quote}

\begin{quote}
\emph{Несмотря на разбитые стекла, здесь по-прежнему царила духота.
Дышать было трудно. Стоявший в помещении дурман не раз вынуждал мой
желудок опорожниться. Котелок вскипал от боли, а я всё думала и думала,
но оставим позади глупые россказни о смельчаках, что в таких ситуациях
сплетают тросы из штор и простыней, чтобы выбросить из окна и спуститься
по ним на волю.}
\end{quote}

\begin{quote}
\emph{Выход был только один. Я знала так с самого начала. Потому так
испугалась. Знала, что Савелий был слишком ленив и неосмотрителен, чтобы
иметь проклятый запасной ключ.}
\end{quote}

\begin{quote}
\emph{«Иголка в яйце, яйцо в утке, утка в зайце, заяц где-то там
ещё\ldots» -- бездумно прошептала я, подбирая на кухне наименее тупой
нож».}
\end{quote}

\section*{97}\label{97}
\addcontentsline{toc}{section}{97}

\markright{97}

-- Как много деталей, -- произнесла Агата. -- Транки сделали тебя
поразительно хладнокровной.

-- Да прекрати, -- отмахнулась Нора так, словно ей отвесили комплимент.

-- Нет, я серьёзно, -- владелица похоронного бюро ткнула ногтем в
открытую страницу старого блокнота, оставляя в ней вмятину. -- Это --
достойное чтиво. Ты не знаешь прозекторского дела и об анатомии, видно,
располагаешь лишь общими сведеньями\ldots{} Словом, ты не штудировала
учебников, дабы это написать. Ты сама свой учебник, и от этого
написанное становится реальным.

-- Потому что это и впрямь реально.

-- Для тебя -- да. А теперь становится реальным и для читателя.

Нора задумалась. Неужто Агата вела к тому, что даже сложившаяся ситуация
имеет положительные стороны. Не очень-то похоже на Агату. Особенно
сегодняшнюю. Но, в конце концов, разве не этого она -- Нора -- так долго
жаждала: писать достойные вещи?

-- Не вижу ничего о твоём доме, -- заметила тем временем Агата, листая
блокнот. -- Почему? Ты не вспомнила?

-- Я вспомнила, -- отозвалась девушка. -- Но только вчера. -- Она
указала на стопку отпечатанных листов. -- Всё здесь.

-- И что там?

-- Все сгорело. Когда я, наконец, сбежала с похорон и добралась до дома,
всё сгорело, -- она помолчала, а затем с горечью добавила. -- И все
сгорели.

-- Хорошо\ldots{}

-- Хорошо?

-- Хорошо, что ты это вспомнила, -- объяснила Агата.

-- Жаль, я не могу быть в этом столь уверенна.

Пауза.

-- Давай сюда продолжение.

\section*{98}\label{98}
\addcontentsline{toc}{section}{98}

\markright{98}

-- Ну, а теперь касательно тебя, -- уверенно начала Нора.

Агата к тому моменту уже покончила с чтением старого дневника подруги и
его отпечатанной версии. Особого впечатления на неё записи не произвели:
Агата и без того предполагала что-то подобное, хотя тем, как излагались
события, дамочка не могла не гордиться. Дневник, безусловно, прояснял
многое, но, расправившись с ним, владелица похоронного бюро вновь
раскисла. Она откинулась назад и теперь безмолвно лежала, зарывшись
лицом в подушки, и, если бы Элеонора заботливо поинтересовалась, что ей
принести, Агата непременно заказала бы новую жизнь.

Но Нора сказала вовсе не это.

-- Эй ты, страдалица! -- позвала девушка. -- Подними-ка свою тощую
задницу и откушай немного чаю!

Не привыкшая к такому обращению Агата медленно поднялась и с тоской
посмотрела на подругу своими измученными глазами. Довольная
произведенным эффектом Нора продолжила излагать речь.

-- Мне понятны твои чувства, -- девушка взяла подругу за руку. -- Больше
чем ты думаешь.

Агата приоткрыла рот, но так ничего и не сказала.

-- У тебя случаются периоды эмоциональных срывов, -- говорила Нора. -- У
нас обеих. По понятным причинам. Я не виню тебя, не приуменьшаю твоей
боли, но Агата\ldots{} -- она помолчала, размеренно взвешивая слова. --
В такие моменты, правильно говорят, ты должна вспоминать счастливые
моменты. Я имею в виду, общие\ldots{}

Агата в нетерпении махнула рукой.

-- У меня больше нет счастливых воспоминаний. Когда он ушёл\ldots{}

-- Смерть не отменяет того, что он сделал при жизни, -- ласково заметила
Нора.

-- Я не сказала «умер». Я сказала «ушел», -- мрачно объяснила хозяйка
похоронного бюро. -- Он ушел от меня за несколько недель до собственной
смерти. Так вот, он ушёл, и это стерло все счастливые воспоминания. Ты
понимаешь?

Элеонора кивнула.

-- Думаю, да.

Воцарилось молчание.

-- Но почему? -- наконец, решилась Нора. -- Мне думалось, вы были
счастливы.

-- Мне тоже так думалось. Я уж точно была, да и он, наверное, какое-то
время. По крайней мере, мне бы очень хотелось думать, что был.

-- Тогда почему он ушёл?

Агата не стала спешить с ответом. Она отпила успевшего остыть чаю,
подкурила сигарету и только потом заговорила.

-- Что ж, версий оказалось много. Адам был личностью\ldots{} М-м-м,
переменчивой, и каждый раз менял свой ответ. Он говорил банальную
ерунду, вроде той, что обычно вскармливают пустоголовым девицам, мол, у
нас разные цели в жизни, все дела. Говорил, что ушёл из-за того, что я
не хочу детей и потому, что я слишком мрачная. Хотя ничего из этого, как
бы, не было новостью, а последним я так вообще стала из-за него. Ещё он
говорил, что я слишком сильно выражала свои чувства, а затем что он
бросил меня, потому как боялся, что в один прекрасный день попросту мне
изменит\ldots{} -- она тяжело вздохнула. -- Короче, он одарил меня кучей
безобразной чепухи, но, знаешь, кое-что из его слов всё-таки было
правдой.

Элеонора вопросительно воззрилась на подругу.

-- Он сказал, что никогда не любил меня, -- просто сказала Агата.

\section*{99}\label{99}
\addcontentsline{toc}{section}{99}

\markright{99}

Солнце садилось и поднималось над кладбищем, а заставить Агату
Рахманинову встать на ноги всё никак не представлялось возможным.
Новообращенная самоубийца продолжала отказываться от завтраков и даже
вина, которые Элеонора настойчиво носила в спальню подруги.

Очередной раз с подносом в руках пересекая коридор третьего этажа,
Элеонора машинально рассматривала украшающие его стены картины. Они не
были авторства Агаты, но всё же вряд ли принадлежали предыдущему
владельцу похоронного бюро.

Хотя\ldots{} Как знать, думала девушка.

Она без стука проникла в укрытые мраком покои, и с порога заявила:

-- Ты и впрямь слишком мрачная!

Ответом ей была лишь тишина.

Девушка зажгла свет, желая убедиться, что её траурная подруга не успела
покончить с собой за те четыре часа, что они не виделись.
Удостоверившись, что Агата на месте, наша героиня завела старую волынку.

-- Взгляни на все это, -- Нора обвела рукой комнату. -- Круг твоих
интересов сводится к смерти во всех её проявлениях.

-- К чему ты ведешь? -- безрадостно спросила Агата, чей тон как бы
подтверждал произнесенное Элеонорой утверждение.

-- Помнишь классический разбор произведения?

-- Какой к чёрту разбор? -- переспросила Агата, падая обратно на
подушки. -- Женщина, дай мне умереть спокойно.

-- Нет, не дам, -- чуть ли не кокетливо возразила Нора, и тут же
продолжила свою тираду. -- Я говорю об обычном разборе произведения.
Такой тысячу раз расставляют делать в школке. Ну, знаешь, там цель,
тема, объект, субъект, что всем этим хотел сказать автор\ldots{}

-- Ну?

-- Вчера я сортировала скопившиеся в ящике письма. Кстати, для тебя там
ещё штуки три от товарища Овсянникова. Так вот, сортировала я письма, --
говорила девушка, -- и забавы ради решила разобрать наши собственные
жизни по этому самому плану.

-- Это ещё зачем? -- Агата презрительно хмыкнула.

-- Говорят же тебе: забавы ради. Люблю, знаешь ли, поиграть в
филологическую деву.

Сие заявление осталось без ответа.

И слава богу.

-- Энивей, в итоге я добралась до скрытых образов, и, знаешь, что
поняла?

-- Удиви меня, -- бросила Агата в манере, говорившей о том, что её уже
едва ли что может удивить.

-- Все очень просто, -- ответила Нора, которая, кажется, не уловила
сарказма в тоне собеседницы. -- Я поняла, что «Маятник», со всем свои
темным величием -- это образ горя, тоски и печали, которые, подобно
старому зданию, нерушимо возвышаются над жизнями мрачных героев. Моей,
твоей и даже Владислава.

-- Матерь божья, -- зевнула Агата. -- Вот так и напишешь в своей книге.
А потом добавишь: услышав это слова, героиня номер два зевнула и
отвернулась к стене.

И она отвернулась к стене.

-- Нет, я не шучу, -- настаивала Нора, а затем сыграла запретной картой.
-- Вспомни, чего ты хотела до встречи с Адамом?

Агата села на кровати. Впервые за долгое-долгое время на лице дамочки
отразилась неуверенность.

-- Присмотрись, -- не сдавалась Нора. -- Бюро -- не просто символ
смерти. Для тебя это ещё и образ скорби по человеку, который не просто
ушёл, а ещё и позволил себе заявить, что, несмотря на твои глубокие
чувства, никогда не любил тебя. Этим он забрал последнее, разве ты не
видишь? Игра не стоит свеч, моя дорогая.

Агата молчала, тогда как глаза её делались всё более печальными.

-- Так чего ты хотела до встречи с ним? -- повторила свой вопрос Нора.

-- Мне было всего девятнадцать\ldots{}

-- Ну, а сейчас тебе всего двадцать пять.

Вновь никакого ответа.

-- Как бы там ни было, -- подытожила Нора, -- не думаю, что ты хотела
всего этого. Оглянись! Тебя сплошь и рядом окружает смерть. Ты знаешь
это, но продолжаешь лелеять её как Владислав свою социофобию. Твоя
работа, книги, картины, даже твой внешний вид -- всё пропитано духом
смерти!

-- Не правда, -- сонно возразила Агата. -- У меня полно пейзажей!

-- Это пейзажи \emph{кладбищ}, -- напомнила Нора и, вопреки протестам
собеседницы, раздвинула шторы, карниз для которых уже успели заменить.

Та закатила глаза. Учитывая их плачевное состояние, зрелище вышло
воистину жутким.

Пытаясь не выдать охватившей её к этому жесту неприязни, Элеонора
перебирала в голове всё картины, имеющиеся в «Маятнике», ну, или,
вернее, их большую часть -- ту, что смогла вспомнить.

-- Хотя, всё же есть одна\ldots{} -- припомнила Нора. -- Висит в самом
конце твоего коридора. У двери в гостиную.

-- Да?

-- На ней нет ни кладбищ, ни мертвых людей. Что это? -- спросила Нора,
протягивая подруге стопку конвертов.

Агата взяла её и без особого интереса изучала письма, пока не добралась
до последнего. При взгляде на небольших размеров бумажку глаза владелицы
похоронного бюро удивленно расширились.

-- Thorsmörk.

-- По-русски, если не затруднит.

-- Исландия, -- зачарованно ответила Агата.

Тут-то всё и решилось.

\section*{100}\label{100}
\addcontentsline{toc}{section}{100}

\markright{100}

Близился полдень восемнадцатого мая.

Агата обосновалась прямиком перед входом во дворик бюро ритуальных
услуг, в котором провела последние пять с лишним лет. Она сидела на
одном из безымянных надгробий, зажав между пальцев сигарету и изредка
затягиваясь. Несмотря на стремительно приближавшееся лето, в воздухе ещё
витала приятная свежесть, оставшаяся после ночного дождя. Термометр
показывал всего семнадцать градусов тепла -- любимая погода нашей
траурной героини.

Слегка опустив очки, она взглянула на здание, возвышающееся впереди,
среди тонких ветвей и старых памятников. Многие из последних были
настолько прекрасны, что Агате не составляло труда вообразить каждое из
них, закрыв глаза: (пара примеров). Сквозь покрытые листвой деревья то
тут то там пробивались жаркие лучи солнца, а трава к тому времени уже
полностью позеленела, что не просто завершало картину -- делало её
идеальной.

Затянувшись, Агата медленно выдохнула табачный дым и ещё какое-то время
наблюдала за тем, как он растворяется в пастельных красках поздней
весны. Сейчас ей определенно нравилось это время года, гораздо больше
обычного. Хотя бы потому, что товарищ Овсянников больше не донимал
обитателей «Маятника» своими письменами.

Вскоре Агата услышала за спиной шаги, слегка приглушенные мокрой травой,
но не стала оборачиваться, прекрасно зная, кто стоит позади.

-- Пора, -- сказала Элеонора, ненавязчиво касаясь плеча подруги.

-- Ещё один момент, -- отозвалась та.

Ленивым, но вполне уверенным движением, Агата стянула с лица
солнцезащитные очки, к которым уже давно успела более чем привыкнуть,
после чего швырнула их на землю и раздавила одним из своих тяжелых
ботинок. Сделав это, она почувствовала нарастающую тревогу, но вместе с
тем теплом расплывающееся по телу спокойствие, о котором так много
слышала за последние дни.

Наконец обернувшись, Агата встретилась с удивленным взглядом зелёных
глаз.

-- Говорят, надо начинать с мелочей, -- с намеком на веселье в голосе
произнесла бывшая владелица похоронного бюро, и зеленые глаза
улыбнулась.

Агата тоже улыбнулась краем губ. В её голове вдруг заиграл задорный
мотивчик, о котором она не вспоминала на протяжении пары лет.

-- Такое чувство\ldots{} -- начала Нора, но затормозила, пытаясь
правильно подобрать слова. -- Чувствуется словно\ldots{}

-- Чувствуется, словно начало новой эры, -- закончила Агата.

Элеонора удовлетворительно кивнула. В мыслях девушки, как оказалось,
играла та же что и у подруги же песня.

А потом началось.

*

Владислав притаился за широким каменным постаментом в виде почившего
льва, опрометчиво надеясь остаться незамеченным. Всё бы ничего, но и без
того высокий юноша периодически выглядывал из-за надгробия. Он наблюдал
за девушками. Обнявшись, те стояли в нескольких метрах от «Маятника».

Агата сказала что-то о наступлении новой эры. Несмотря на теплую погоду,
от этих слов по телу Владислава скользнул приятный холодок. Бывшие
обитатели похоронного бюро -- все трое -- мысленно напевали одну и ту же
песню, даже не догадываясь об этом.

Тем временем Элеонора махнула рукой -- подала знак рабочим, нанятым
товарищем Овсянниковым, чей богатый клиент накануне дал дуба.

«Feels like\ldots{} Feels like it's coming\ldots»\footnote{Текст песни
  группы Foster the People под названием Coming of Age: чувствуется,
  словно это наступает\ldots{}} -- пронеслось в голове Ласло.

Словно в такт этим обнадеживающим строкам, подъёмный кран, все утро
дожидающийся своей очереди выйти на сцену, загудел, поднимая в воздух
чугунный дробильный шар. Подобно маятнику, тот заколебался в
пространстве и принялся ходить по кругу, набирая обороты, а затем нанёс
сокрушительный удар по тёмному кирпичу бюро ритуальных услуг. Как ни
странно, первой комнатой, одержавшей поражение, стали закрытые
апартаменты.

За этим ударом последовал ещё один, и вот с лица Земли исчезла гостиная
Агаты Рахманиновой, которая по-прежнему думала о себе исключительно как
о владелице похоронного дома, но уже начинала осознавать, что вскоре
таким мыслям придётся покинуть её сознание, вне зависимости от того,
хочется ей этого, или нет.

А ей, кажется, очень хотелось.

-- Feels like\ldots{} feels like it's coming, -- пропела дамочка в
черной шляпе. -- It feels like\ldots{} It's like a coming of
age.\footnote{Чувствуется, словно это начало новой эры.}



\end{document}
